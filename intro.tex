\chapter{Introduction}
\section{Formal Verification of Cryptographic Protocols}
\subsection{Background}
Cryptographic protocols are essential elements of the infrastructure
to ensure secure communication and information processing. However,
security proofs of such protocols tend to be complex and
difficult to verify, which is recognized by researchers 
\cite{Shoup2004, Halevi2005, Canetti2006}.
Indeed, flaws of designs \cite{Lowe1996,
Bleichenbacher1998, Cervesato2008}
and security proofs \cite{Shoup2001oaep, Galindo2005} of
cryptographic protocols
were found years after they had been presented.
The difficulty of verification of cryptographic protocols 
is considered to come from the following points \cite{Ouyousuuri2010}.
\begin{itemize}
 \item To define and formulate security properties, which depend on 
       the functionality of each protocol, is difficult.
       Indeed, the definitions are often reconsidered
       \cite{GoldwasserMicali1984,
       DolevDworkNaor2003, RackoffSimon1992,
       Benalohetal1994, Juelsetal2005}.
 \item Execution models of cryptographic protocols
       are complex. In general, principals in a cryptographic
       protocol, including adversaries,
       run in parallel. Since each principal runs
       nondeterministically and probabilistically,
       execution models and security properties must be
       defined on the basis of parallelism, nondeterminacy, and
       probability.
 \item Methods to prove that cryptographic protocols satisfy defined
       security properties are not self-evident.
       Security proofs performed on the basis of such execution
       models tend to be quite complex.
\end{itemize}
{\it Formal methods} have been applied
to model, analyze, and verify cryptographic protocols
\cite{Blanchet2001, Armando2005, Blanchet2008cryptoverif,
Barthe2009certi, Barthe2011}.
They are based on formal frameworks, including formal
languages and systems to prove security properties such as
inference rules.
The languages are used to formalize cryptographic protocols and
security properties, and the inference rules are used to perform 
formal proofs.

Advantages of formal methods are as follows.
\begin{itemize}
 \item The use of formal languages precisely prevents ambiguity.
       Although mathematical proofs in natural languages are rigorous,
       ones in formal languages can be 
       more in terms of both description and interpretation,
       because the syntax and
       semantics of them are defined mathematically.
 \item That all inferences in a proof obey
       pre-defined inference rules is made to be explicit.
       On the other hand in ordinary proofs,
       it is not always the case. This is possibly because
       some inferences are apparently too obvious to write down
       explicitly. However, writing such inferences is still valuable
       when we put priority on rigor.
 \item Verification and proof can be partially automated.
       It reduces human costs as well as prevents errors.
       Parts that are automated depend on software tools.
       We introduce some of the tools in the next subsection.
\end{itemize}

\subsection{Existing Verification Tools}
\label{intro:tools}
$\mathsf{CertiCrypt}$ \cite{Barthe2009certi}
is a framework where we interactively construct
game-based cryptographic proofs in a proof assistant
$\mathsf{Coq}$ \cite{coq2010}. It has been 
successfully applied to
verify the security of prominent cryptographic protocols such as 
FDH \cite{Zanella2009FDH} and OAEP \cite{Barthe2011OAEP}.
The effort one needs to
build the machine-checked proofs in $\mathsf{CertiCrypt}$ 
is thought to be more than moderate \cite{Barthe2011}, 
contrasted to the high guarantees
of security. In spite of the effort to construct the proofs,
ones who read them have
only to understand the correctness of the formalization of
the target cryptographic protocol and the statement of the security to
achieve.
$\mathsf{EasyCrypt}$ \cite{Barthe2011} further
reduces users' effort. It generates
a whole machine-checked proof from {\it proof sketches}, which are
representation of essence of a security proof. It was 
applied to verify the security of Cramer-Shoup public key encryption
scheme \cite{CramerShoup1998}, to which $\mathsf{CertiCrypt}$ had
never been applied. Those approaches are
computer-aided verification of cryptographic {\it proofs},
ideas of which are in general considered by men.

$\mathsf{AVISPA}$ \cite{Armando2005} and 
$\mathsf{ProVerif}$ \cite{Blanchet2001} are frameworks to analyze
security
protocols in abstract models, where cryptographic primitives are 
idealized. For example, ciphertexts encrypted using a public key are
decrypted only using the secret key which corresponds
to the public key. In $\mathsf{AVISPA}$ framework, a protocol designer
describe a protocol paired with expected security properties
in a formal language HLPSL. The scripts are 
translated into the intermediate format and passed
to the backends including model checkers. One of them for example
tries to find attacks by
exploring the transition system specified by the intermediate format.
Those approaches use a computer to exhaust execution states of
protocols, whose number is possibly too large to do by hand,
rather than obtaining cryptographic proofs.

% Program transformation techniques is CryptoVerif
% \cite{Blanchet2008cryptoverif}.
% For another example,
% as formal languages can be interpreted by a computer,
% protocols can be simulated as programs or their execution states
% can be exhausted \cite{Armando2005, Blanchet2001}.***
% ***

\section{Formal Methods for Quantum Cryptography}
We may call {\it classical} cryptography, formal framework, process
calculus, and so on for {\it non-quantum} ones.

\subsection{Security of Quantum Key Distribution Protocols}
Security proofs are also complex in quantum cryptography,
where we must also consider attacks using entanglements.
BB84 \cite{BennetBrassard1984}
is a prominent quantum key distribution (QKD)
protocol, which allows two remote principals to share a secret
key. Let us call the two principals and an adversary Alice, Bob, and Eve 
respectively. A key of the security of BB84 and other QKD protocols
\cite{LoChau1999, TamakiKoashiImoto2003} is 
that Alice and Bob can estimate the amount of information leakage to
Eve. If Eve tries to obtain
information from quantum systems passing through a quantum channel,
she must measure some physical values of them.
It possibly disturbs the states of quantum
systems. To check the disturbance, Alice sends to Bob 
additional quantum systems
other than those which are sources of the shared secret key.
If the information leakage is judged to be too large, they
abort the protocol.
BB84 provides {\it unconditional security} that
the mutual information of Alice's and Eve's key is negligible
with respect to the number of quantum bits if Alice and Bob have not
aborted the protocol. The advantage of QKD protocols is that
the security does not depend on adversary's computational resources, 
while the security of classical key exchange protocols
are ensured on the basis of
conjectured difficulty of computing certain functions
\cite{DiffieHellman1976}.

The first security proof of BB84 is presented by Mayers 
\cite{Mayers1998}. It is about 50 pages long and complex.
Lo and Chau proposed another QKD protocol \cite{LoChau1999} whose
security proof is simple. It is based
on an entanglement distillation protocol (EDP).
It has the drawback that it requires quantum
computers, while BB84 only needs apparatus for state preparation and
quantum measurement.
Shor and Preskill presented a simple proof of BB84
\cite{ShorPreskill2000}. They showed that the security of BB84 is
equivalent to that of a modified version of Lo and Chau's protocol
(modified Lo-Chau protocol, the EDP-based protocol) \cite{LoChau1999},
whose security proof is also simple.
Our target of formal verification in this thesis 
is Shor and Preskill's proof.

\subsection{Motivation to Apply Formal Methods to Quantum\\
Cryptographic Protocols}
QKD protocols have advantage not to depend on conjectured difficulty
of computing certain functions.
They are ones of the closest application to practice in the quantum
information field.
Actually, several companies such as Id Quantique,
MagiQtechnologies, Toshiba, and NEC are developing 
commercial quantum cryptographic systems.
It is also possible that more complex quantum protocols
be presented in the future.
Therefore, it is important to develop 
formal frameworks to verify quantum protocols'
security and also make the security proofs machine-checkable.

\subsection{Process Calculi for Quantum Protocols}
Process calculi \cite{Milner1999, AbadiFournet2001, 
Blanchet2008cryptoverif}
are formal frameworks that are
suitable to verify properties of parallel systems.
They have successfully applied to verification of a number of
classical cryptographic protocols such as Diffie-Hellman
key agreement \cite{AbadiFournet2001}, 
Needham Shoroeder shared and public key protocols
\cite{Blanchet2008cryptoverif},
FDH \cite{Blanchet2006FDH}, and Kerberos \cite{Blanchet2008Kerberos}.
To clone the success in quantum information fields,
several quantum process calculi have been proposed
\cite{NagarajanPapanikolaouBowenGay2005, Lalire2006, Adao2007,
FengDuanYing2011}. Feng et al. defined a process calculus qCCS
\cite{FengDuanJiYing2007, Ying2009, FengDuanYing2011, DengFeng2012,
FengDengYing2012}.
In qCCS, a quantum protocol is formalized as a
configuration $\con{P}{\rho}$\footnote{In \cite{FengDuanJiYing2007,
Ying2009, FengDuanYing2011, DengFeng2012,
FengDengYing2012}, a configuration is written of the form
$\langle P, \rho \rangle$ using angle brackets. In this thesis,
we write $\con{P}{\rho}$ since we frequently write density operators using
bra-ket notation.}.
which is a pair of a process $P$ and a collective quantum mixed state
$\rho$
that is referred by the variables in $P$.
Gay et al. defined a quantum process calculus CQP
\cite{NagarajanPapanikolaouBowenGay2005, Davidson-etal2012}, which is
based on pi-calculus. In the later version of
CQP \cite{Davidson-etal2012}, a configuration is defined as a triple
$(\sigma;\tilde q;P)$ consisting of a collective
quantum pure state $\sigma$,
an indicator of $P$'s ownership of variables $\tilde q$,
and a process $P$.

An important notion in process calculi is 
{\it weak bisimulation} relation on processes
\cite{Milner1999, AbadiFournet2001}.
Processes in weak bisimulation relation behave equivalently:
they perform identical actions that are
visible from the outside up to invisible ones.
For example, visible actions are communications of processes
via public channels, and invisible ones are 
communications via private channels.
Usage of the relation is for example as follows.
If we formalize some protocol and its 
specification as processes and prove that they 
are in bisimulation relation,
then we have proved the protocol satisfies the specification.

In qCCS, weak bisimulation relation $\approx$
on configurations is defined.
Their definition is successful in that 
the relation is closed by application of parallel composition
of processes: if $\con{P}{\rho} \approx \con{Q}{\sigma}$ holds,
then $\con{P||R}{\rho} \approx \con{Q||R}{\sigma}$ holds for all
process $R$ with which $P||R$ and $Q||R$ are defined.
$P||R$ means that $P$ and $R$ run in parallel.
This property of $\approx$ is called congruence.
Similarly in CQP, weak bisimulation relation is defined and proven to
be congruent. The congruence property is significant when we account on 
compositional behavioural equivalence.
% The facts that the weak bisimulation relations are congruent
% support the claim that the definitions capture the intuition of
% behavioural equivalence. It is also useful practically.

\subsubsection{Application of Quantum Process Calculi}
In qCCS, quantum teleportation, super dense coding protocols
\cite{FengDuanYing2011}, and simplified version of BB84
\cite{DengFeng2012} are formalized. That they satisfy their
specifications is also verified using weak bisimulation relation.
In CQP, quantum coin-flipping game \cite{Meyer1999},
quantum teleportation protocol,
and quantum bit-commitment protocol
are formalized and their execution
are modeled \cite{NagarajanPapanikolaouBowenGay2005}.
Three fold repetition quantum error correcting code and
its specification are formalized, and they are proven to
be weakly bisimilar \cite{Davidson-etal2012}.

To extend application of process calculi to security proofs, we
applied qCCS to Shor and Preskill's
security proof of BB84 \cite{ShorPreskill2000}.
We previously formalized BB84 and the modified Lo and Chau's protocol
as configurations of qCCS and proved their bisimilarity by hand
\cite{Kubota2012}.

\section{Contributions}
We addressed the following three limitations of previous work.
\begin{itemize}
 \item Automated verification techniques have never been applied to
       verify that a quantum cryptographic protocol satisfies
       certain security criteria that are accepted in the field of
       quantum cryptography.
       In previous work \cite{NagarajanPapanikolaouBowenGay2005},
       security of BB84 was analyzed
       automatically but the strategy of the adversary Eve was
       limited: she only intercepts each qubit through the
       quantum channel, chooses either $\{\ket{0},\ket{1}\}$ or 
       $\{\ket{+},\ket{-}\}$ basis randomly,
       measures the qubit in the basis, and resends it through the
       quantum channel. She thus cannot make her qubits entangled with
       Bob's.
       On the other hand, Eve is assumed to perform arbitrary
       quantum operations in quantum cryptographic proofs
       \cite{Mayers1998, ShorPreskill2000}.
 \item Verification of weak bisimilarity in 
       any quantum process calculi
       has not been automated although by-hand verification of it
       is often hard when objective configurations have
       many long branches in their transition trees.
 \item Methodology to prove quantum cryptographic security using
       process calculi has not been established yet
       while considering weak bisimilarity is useful to verify
       equivalence of protocols. As for
       Shor and Preskill's security proof, equivalence of
       BB84 and modified Lo and Chau's protocol is stated as
       weak bisimilarity of configurations formalizing them.
       However, the way to state the security of
       the latter has not been obvious.
\end{itemize}
Specifically, the contributions of this thesis are as follows.
\subsection{A Software Tool to Verify Weak Bisimilarity of qCCS Configurations}
We implemented a software tool, which
we call {\it Verifier1}, that formally verifies
weak bisimilarity of qCCS configurations
without recursive structures.
The overview is as follows.
 \begin{itemize}
  \item Verifier1 adopts a simplified formal framework based on qCCS,
	which we may call {\it nondeterministic qCCS}.
	\begin{enumerate}
	 \item In the original syntax, there are constructors of a quantum
	       measurement $M[\tilde q;x].P$
	       and application of a quantum operator $\mathit{op}[\tilde
	       q].P$. Since we can also formalize
	       a measurement as a special quantum operation,
	       we always have two ways to
	       formalize one. We considered criteria
	       to select one way from the two, which is
	       reported in \cite{Kubota2012}. 
	       We simplified the syntax reflecting the
	       criteria so that
	       a user who verifies equivalence of
	       quantum cryptographic protocols using weak bisimulation
	       can select the
	       feasible way to formalize a quantum measurement.
	 \item qCCS's transition system
	       is probabilistic
	       and nondeterministic. A configuration is of the form
	       $\con{P}{\rho}$ and $\tr{}{\rho} = 1$. Suppose a 
	       probability-weighted configuration $\frac{1}{2} \bullet
	       \con{P}{\rho}$, which is interpreted to that we have
	       the configuration $\con{P}{\rho}$ with probability
	       $\frac{1}{2}$.
	       Instead of considering a
	       probability-weighted configuration, we
	       allow $\rho$ satisfying $0 < \tr{}{\rho} \le
	       1$ and
	       interpret $\tr{}{\rho}$ as the probability to reach the
	       configuration. For instance, we consider
	       $\con{P}{\frac{1}{2}\rho}$ in
	       the simplified transition system instead of $\frac{1}{2}
	       \bullet \con{P}{\rho}$ in the original one.
	       The simplified one is only nondeterministic.
	       To verify behavioural equivalence of
	       configurations has become easier whether
	       it is done by hand or tool.
	\end{enumerate}
  \item Verifier1 handles quantum states symbolically and
	it can be applied to security proofs.
	In security proofs,
	the dimensions of quantum states
	are generally unfixed, because they
	depend on security parameters such as the number of
	qubits which Alice sends to Bob. 
	Therefore, the way to represent states in
	a software tool is not self-evident.
	In our verifier, security parameters and quantum
	states are represented as symbols.
	A user is
	supposed to define in Verifier1 symbolic 
	representations of quantum states and 
	equations on them. To compare symbolic representations,
	Verifier1 applies such user-defined
	equations to them and simplifies them.
  \item Although Verifier1 adopts the simplified
	syntax and operational semantics, it is sound with respect
	to qCCS. If Verifier1 returns $\mathit{true}$
	with two configurations and a set of
	valid user-defined equations as input,
	then the two converted configurations
	in qCCS are weakly bisimilar.
 \end{itemize}
The design of Verifier1 and the proof of
soundness are described in Chapter \ref{symqccs}.

\subsection{Approximate Bisimulation for Quantum
Processes}
In formal verification using process calculi,
notions of {\it approximate bisimulation} relation are useful:
configurations in the relation behave equivalently up to
negligible probability. This is a possible usage of the notions.
To evaluate security of a system, we first consider an ideal
system that is always secure. We next prove the system and
the ideal one are approximately bisimilar. 
This proves that the system is secure except for
negligible probability.

We defined two kinds of approximate bisimulation relations
in our formal framework (nondeterministic qCCS),
and studied properties of them.
As described in the next subsection, we applied one of the notion
to verify formally the last step of Shor and Preskill's security proof.

Originally, $\con{P}{\rho} \approx \con{Q}{\sigma}$ means 
  \begin{itemize}
   \item $\tr{\qv{P}}{\rho} = \tr{\qv{Q}}{\sigma}$, and
   \item whenever one of $\con{P}{\rho}$ or $\con{Q}{\sigma}$
	 can perform an action,
	 the other can perform the same action up to invisible ones.
  \end{itemize}
The set of quantum variables occurring in $P$ 
is denoted by $\qv{P}$. A state space corresponds to each quantum variable.
When the quantum state of all quantum variables is $\rho$, 
$\tr{\qv{P}}{\rho}$ is the quantum state that one who does not have
variables in $\qv{P}$ can access.

We relaxed the conditions up to gaps of probabilities of
configurations' performing actions and {\it trace distance}
$d(\tr{\qv{P}}{\rho}, \tr{\qv{Q}}{\sigma})$.
When we measure an arbitrary observable (a physical value)
of quantum states with small trace distance,
we obtain identical results with close probability.
\begin{enumerate}
 \item  The first relation $\sim_{\zeta, \eta}$ is
	parametrized with $\zeta, \eta$ with $0 \le 
	\eta, \zeta \le 1$. Roughly speaking,
	$\con{P}{\rho} \sim_{\zeta, \eta} \con{Q}{\sigma}$
	means
	\begin{itemize}
	 \item $d(\tr{\qv{P}}{\rho}, \tr{\qv{Q}}{\sigma}) \le
	       \zeta$, and
	 \item whenever one of $\con{P}{\rho}$ or $\con{Q}{\sigma}$
	       can perform an action with probability greater than
	       $\eta$,
	       the other can perform the same action up to
	       invisible ones.
	\end{itemize}
	The relation is not transitive but
	if $\con{P}{\rho} \sim_{\zeta, \eta} \con{Q}{\sigma}$
	and $\con{Q}{\sigma} \sim_{\zeta', \eta'} \con{R}{\theta}$
	hold, then $\con{P}{\rho}
	\sim_{\zeta + \zeta', \max\{\eta, \eta'\} + 2(\zeta + \zeta')}
	\con{Q}{\sigma}$
	holds.
	We proved that if $\con{P}{\rho} \sim_{\zeta, \eta}
	\con{Q}{\sigma}$, then 
	$\con{P||R}{\rho} \sim_{\zeta, \eta} \con{Q||R}{\sigma}$
	for an arbitrary process $R$,
	which suggests sanity of the
	definition.
 \item The second relation $\sim$ is defined when
       quantum states depend on security
       parameters, with which the notion of {\it negligibility}
       makes sense. Roughly speaking,
       $\con{P}{\rho} \sim \con{Q}{\sigma}$ means
       \begin{itemize}
	\item $d(\tr{\qv{P}}{\rho}, \tr{\qv{Q}}{\sigma})$
	      is negligible, and
	\item whenever one of $\con{P}{\rho}$ or $\con{Q}{\sigma}$
	      can perform an action with
	      non-negligible probability,
	      the other can perform the same action up to invisible
	      ones.
       \end{itemize}
       The relation is transitive.
       We proved that if $\con{P}{\rho} \sim
       \con{Q}{\sigma}$ holds, then \\
       $\con{P||R}{\rho} \sim \con{Q||R}{\sigma}$ holds
       for an arbitrary process $R$. 
       Eventually, the relation $\sim$ is an equivalence 
       relation and closed under
       parallel composition, and thus we say it is {\it congruent}.
       The property suggests sanity of the definition as
       well as feasibility in practice.
\end{enumerate}
       We finally extended Verifier1
       to verify a subset of the second approximate bisimulation
       relation. The extended verifier is called {\it Verifier2}.
       It uses
       user-defined rewriting rules of the form $\rho = \sigma$
       for symbolic representations $\rho, \sigma$ of quantum states.
       Each rule is expected to satisfy that
       $d(\braw{\rho}, \braw{\sigma})$ is negligible, where
       $\braw{\rho}$ is the quantum state represented by the symbol 
       $\rho$. The definitions of approximate bisimulation
       relations, the proofs of their properties, and
       the extension of Verifier1 are described in
       Chapter \ref{prob_bisim}.
\subsection{Application of the Verifiers to Shor and Preskill's 
Security Proof of BB84}
We applied Verifier1 and Verifier2 to Shor and Preskill's
security proof of BB84 \cite{ShorPreskill2000}.
Our formal verification consists of the following two steps.
 \begin{itemize}
  \item In the first step of Shor and Preskill's security proof,
	equivalence of BB84 and an EDP-based protocol is proven.
	The latter protocol is a modification of Lo and Chau's
	protocol \cite{LoChau1999}.
	We first verified the equivalence using Verifier1.
	We formalized them
	as configurations
	based on our previous work
	\cite{Kubota2012}.
	We then defined equations to verify
	equivalence of the two protocols.
	They are obtained from properties of
	error-correcting codes
	discussed in the original proof \cite{ShorPreskill2000}
	and basic facts about measurement of halves of EPR pairs.
	The input is the equations and configurations of
	BB84 and of the EDP-based protocol.
	Verifier1 returns $\mathit{true}$ with the input.
  \item Second, we verified the security of the
	EDP-based protocol.
	We defined an {\it ideal} protocol, where Alice and Bob
	can create a shared key leaking no information to Eve.
	We formalized it as a configuration in the extended
	verifier. 
	We then defined rewriting rules to verify
	approximate equivalence of the
	two protocols. They
	are obtained from the second step of the 
	original proof \cite{ShorPreskill2000}
	to show the security of the EDP-based protocol.
	The input is the rewriting rules and configurations of
	the EDP-based protocol and of the ideal protocol.
	Verifier1 returns $\mathit{true}$ with the input.
 \end{itemize}
Formalization techniques, scripts, and experimental results
are described in Chapter \ref{formaliz}.
The package of Verifier1 and Verifier2 is available from\\
\texttt{http://hagi.is.s.u-tokyo.ac.jp/\~{}tk/qccs}\texttt{verifier.tar.gz}.
It includes a user manual and example scripts in the directories
\texttt{doc} and \texttt{scripts}.

\section{Related Work}
\subsection{Formal Approaches to Security of BB84}
\subsubsection{Automated Analysis using Probabilistic Model Checking}
Model checking methods have been applied to analyze security 
of QKD protocols.
Nagarajan et al.\ applied the probabilistic model checker
PRISM 2.0 \cite{KwiatkowskaNormanParker2004}
to analyze BB84 \cite{NagarajanPapanikolaouBowenGay2005}
by calculating
the probability of eavesdropping detection. They assumed a
restricted adversary who only performs intercept and resend attack
through the noiseless quantum channel, and left a full security 
proof in a formal framework for future work.
In contrast, we target 
formalization of security proofs such as Shor and Preskill's
\cite{ShorPreskill2000}, where the quantum channel is assumed to be
noisy and Eve performs arbitrary quantum operations.

\subsubsection{Verification Using a Sequential Quantum Programming Language}
In our previous work \cite{KubotaKakutaniKatoKawono2011}, we applied
program transformation methods and Hoare logic to
Shor and Preskill's security proof of BB84.
We formalized BB84 and the EDP-based protocol using a
Selinger's QPL \cite{Selinger2004}.
We then formalized their inferences
as rewriting rules of programs. Soundness of each rule was
proved on the basis of the semantics of QPL.
BB84 is transformed to the EDP-based protocol by the rewriting by
the rules.
We finally verified the security of the EDP-based protocol formally
using Kakutani's quantum Hoare logic \cite{Kakutani2009}.

When we formally verify cryptographic protocols,
an advantage of process calculi to sequential programs is that
communications and nondeterminacy are explicitly written.

\subsection{Quantum Process Calculi}
\subsubsection{CQP}
In CQP's transition system, a state is a triple called a configuration
that consists of a map from a quantum variable to a quantum pure
state, a set of quantum variables, and a process. 
A configuration transits its state interacting with the outsider
similarly to qCCS.
An example of the configuration is as follows.
\[
A \defequiv ([q,r \mapsto
 \frac{1}{\sqrt{2}}(\ket{00}+\ket{11})];r;d![\mathsf{measure}~r].Q)
\]
There are quantum variables $q$ and $r$ in this configuration.
The first component means the state of 2-qubit system indicated by 
the variables $q$ and $r$ is $\frac{1}{\sqrt{2}}(\ket{00}+\ket{11})$.
The second component $r$ means that the process 
$d![\mathsf{measure}~r].Q$ has
access to $r$, but not to $q$. Instead, the outsider has access to $q$.
The third component $d![\mathsf{measure}~r].Q$ is a process
that sends the measurement result (i.e. classical data) through the 
channel $d$, and executes $Q$.

The first state transition of the above configuration is as follows.
The right-hand side is called a mixed configuration.
\[
 A
 \xrightarrow{\tau}
 \frac{1}{2}([q,r \mapsto
 \ket{00}];r; d![0].Q) \oplus
 \frac{1}{2}([q,r \mapsto
 \ket{11}];r; d![1].Q) \defequiv B
\]
Next, it sends the value $0$ or $1$, which means
that it reveals the measurement result to the outsider.
A probability distribution of configurations,
which is called an intermediate configuration, is as follows.
\[
 B
 \xrightarrow{\tau}
 \frac{1}{2}([q,r \mapsto
 \ket{00}];r; d![0].Q) \boxplus
 \frac{1}{2}([q,r \mapsto
 \ket{11}];r; d![1].Q) \defequiv C
\]
The intermediate configuration probabilistically
performs one of 
the following transitions to become a configuration.
\begin{align*}
 & C
 \xgizarrow{\frac{1}{2}}
 ([q,r \mapsto
 \ket{00}];r; d![0].Q)
&&C
 \xgizarrow{\frac{1}{2}}
 ([q,r \mapsto
 \ket{11}];r; d![1].Q)
\end{align*}

CQP is a successful formal framework with
the definition of congruent bisimulation.
It is future work to study automated verification of
security proofs in CQP.
%  but
% there are two points that we did not adopt CQP to formalize
% QKD protocols.
% \begin{enumerate}
%  \item We thought the constructors of CQP, unitary transformation
%        and measurement, are too primitive for our purpose.
%        As we can see in Davidson's application \cite{Davidson-etal2012},
%        to perform a randomly chosen unitary transformation,
%        the source of randomness must be described as a
%        process. Prepared randomnesses are passed
%        to the process that chooses the unitary transformation.
%        When verifying quantum error correcting codes,
%        it is quite valuable to trace all cases of executions for
%        each concrete value of randomness.
%        For our targets of formalization,
%        concrete values of randomnesses are not
%        essential and actually cannot be fixed, because the lengths
%        of the randomnesses depend on the security parameter.
%        Therefore, we decided to represent probabilistic
%        local quantum operations as TPCP maps in bulk.
%  \item We did not come up with an efficient way to automate
%        the verification of bisimilarity of the CQP configurations.
%        To be a bismulation relation, a relation on
%        CQP configurations is required to be an equivalence relation.
%        If it were required to be only symmetric, 
%        we could design a verfication algorithm that
%        runs symmetrically with respect to the arguments,
%        which might be two configurations to judge the bisimilarity.
% \end{enumerate}

\subsubsection{Symbolic Bisimulation in qCCS}
The authors of qCCS presented the notion of symbolic
bisimulation for quantum processes \cite{FengDengYing2012}.
A purpose is to
verify bisimilarity algorithmically. They proved the
strong symbolic bisimilarity (internal actions must be simulated) is
equivalent to the strong open
bisimilarity, and actually presented an algorithm to verify symbolic ground
bisimilarity (outsiders do not perform quantum operations adaptively).
 Since our purpose is to apply a process calculus to security proofs
 where adversarial interference must be taken into account, we 
implemented a tool that verifies weak open bisimilarity
on the basis of the previous version of qCCS \cite{DengFeng2012}.

\subsection{Approximate Bisimulation}
\subsubsection{Approximate Bisimulation in qCCS}
The authors of qCCS also presented the notion of approximate
strong bisimulation in an earlier version of qCCS \cite{Ying2009}.
The notion is different from ours in this thesis.
In the framework, the transitions of TPCP map application are
defined to be {\it labeled}, namely, not $\tau$ transitions.
Approximate strong bisimulation identifies transitions of TPCP maps
whose {\it diamond distance} is not grater than some parameter.
While this identification is significant,
we did not directly apply this framework to our verification targets.
The first reason was only {\it strong} bisimulation was proposed,
while the protocols that we attempted to verify formally were thought
to be weakly bisimilar but not to be strongly bisimilar.
The second reason was that there was no conditional branch in the syntax,
while aborting of an execution of a QKD protocol must be formalized.

\subsubsection{Approximate Bisimulation in Classical Process Calculi}
Ying et al. introduced a notion of approximate bisimulation in
classical process calculi \cite{Ying2000} with labeled transition
systems. In their framework, the set of \emph{actions} is a metric space.
They applied the notion to verify formally 
approximate correctness of real time
systems such as real time ACP. Our notion of approximate bisimulation is
independent of theirs: distance of quantum states is considered but
that of actions is not.

\subsubsection{Approximate Bisimulation in Labeled Transition Systems
   with Observations}
Girard et al. defined a notion of approximate bisimulation in \emph{labeled
transition systems with observations} \cite{Girard2005}.
In a labeled transition system with observations, there is an
\emph{observation map}
that carries a state $q$ to an \emph{observation}  $\langle\!\langle q
\rangle\!\rangle$, and the set of observations $\Pi$ is a metric space.
The notion of approximate bisimulation is defined based on
the distance $d_{\Pi}(\langle\!\langle q \rangle\!\rangle,
\langle\!\langle q' \rangle\!\rangle)$.
Our notion of approximate bisimulation appears to be similar to
theirs when we substitute trace distance for $d_{\Pi}$ and
partial trace for $\langle\!\langle \cdot \rangle\!\rangle$.
There are three different points between our work and theirs.
First, we considred \emph{weak} bisimulation relation and proved
that it is closed by parallel composition of processes,
which are important peculiarly in process calculi.
Second, since our formal framework involves probability,
transitions with probability less than some threshold $\eta$
(respectively, transitions with negligible probability) is ignored in our
notion. Third, since the relation $\sim$ incorporates with 
the notion of negligibility, it is transitive.

\subsection{Automated Verification Tool for Classical Protocols}
$\mathsf{CryptoVerif}$ \cite{Blanchet2008cryptoverif} is a software tool
to verify security of
\emph{classical} protocols. It has been applied to
both high-level protocols \cite{Blanchet2008Kerberos, Blanchet2012oeke} 
that employ cryptographic primitives and to
cryptographic schemes \cite{Blanchet2006FDH} that are possibly be
used as primitives.
There are two features of $\mathsf{CryptoVerif}$ that are 
related to our verifiers: it is designed on the basis of a probabilistic
process calculus, and it incorporates with negligibility.
Of course, it cannot be directly applied to verify security of QKD
protocols. In $\mathsf{CryptoVerif}$'s framework,
all data are classical and
a process, which may be an adversary,
is bounded in polynomial time, while an adversary against a QKD protocol
is not. 

As a proof technique, $\mathsf{CryptoVerif}$ applies 
\emph{observational equivalence} of processes.
Let $\Pr(P \rightsquigarrow a)$
be the probability
that the process $P$ transits to a process that is ready to send some
data through the channel $a$.
Two processes $P$ and $Q$ are observationally equivalent, written
$P \cong Q$ here, if
$$
|\Pr(C[P] \rightsquigarrow a) - \Pr(C[Q] \rightsquigarrow a)|
$$
is negligible for all evaluation context\footnote{An evaluation context
is composed by a hole, channel restrictions, and
parallel compositions.}
$C[\_]$ that runs in polynomial time
and channel $a$ that is
not restricted. 
The relation $\cong$ is \emph{congruent} by the definition,
namely, if $P \cong Q$ holds, then $C[P] \cong C[Q]$ holds for all
evaluation context $C[\_]$ running in polynomial time.
When we consider $C[\_]$ as a polynomial time
adversary that runs in parallel and interacts with the protocol $P$ or
$Q$,
the observational equivalence of them is intuitively interpreted
as indistinguishability of the protocols from an adversary.

In cryptographic proofs, security of a high-level protocol
is reduced to security of the employed cryptographic primitives.
Similarly, security of a cryptographic scheme is reduced to 
assumed difficulty of computing certain functions.
In $\mathsf{CryptoVerif}$, a user formalizes such assumptions as
observational equivalence of processes.
$\mathsf{CryptoVerif}$ uses such user-defined equivalences as
rewriting rules:
if a target process $P$ is of the form $C[X]$ and there is a user
defined
equivalence $X \cong Y$, then it is rewritten to $C[Y]$.
By congruence, $P \cong C[Y]$ holds.
Given a process formalizing a target protocol and 
user-defined observational equivalences, 
$\mathsf{CryptoVerif}$
rewrites the process repeatedly until it becomes a process
that is obviously secure.

On the other hand, our tools verify bisimilarity by
tracing execution paths of configurations, not by
rewriting processes.
Fortunately, the bisimulation in qCCS is congruent, and
it is thus possible that a verification tool is
designed to verify bisimilarity by rewriting.
With such a verifier, bisimilarity of 
big-sized configurations is derived from that of some small-sized ones.
Especially in proofs of security of QKD protocols,
difficulty of computing certain functions is not assumed.
Therefore, even if a verifier conducts rewriting,
we possibly need to prove bisimilarity of
such small-sized configurations unlike
verification using $\mathsf{CryptoVerif}$.
