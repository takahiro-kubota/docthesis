%#BIBTEX jbibtex main
\documentclass[doctor, 11pt]{iscs-thesis}

\usepackage{amsmath}
\usepackage{amssymb}
\usepackage{amsthm}
\usepackage{bm}
\usepackage{stmaryrd}
\usepackage{braket}
\usepackage{url}
\usepackage{multicol}
\usepackage[dvipdfmx]{graphicx}

%-------------------
\newcommand{\dom}[1]{\mathrm{dom}(#1)}
\newcommand{\cod}[1]{\mathrm{cod}(#1)}
\newcommand{\defequiv}[0]{\overset{\mathrm{def}}{\equiv}}
\newcommand{\defeq}[0]{\overset{\mathrm{def}}{=}}
\newcommand{\strg}[0]{\,\dot\approx\,}
\newcommand{\xgizarrow}[1]{\overset{#1}{\rightsquigarrow}}

\newcommand{\A}[0]{\mathcal{A}}
\newcommand{\C}[0]{\mathcal{C}}
\newcommand{\D}[0]{\mathcal{D}}
\newcommand{\R}[0]{\mathcal{R}}
\newcommand{\E}[0]{\mathcal{E}}
\newcommand{\F}[0]{\mathcal{F}}
\newcommand{\G}[0]{\mathcal{G}}
\renewcommand{\H}[0]{\mathcal{H}}
\newcommand{\I}[0]{\mathbb{I}}
\newcommand{\N}[0]{\mathbb{N}}

\newcommand{\CSS}[0]{\mathrm{CSS}}

\newcommand{\Con}[0]{\mathit{Con}}
\newcommand{\qChan}[0]{\mathit{qChan}}
\newcommand{\Proc}[0]{\mathit{Proc}}
\newcommand{\qVar}[0]{\mathit{qVar}}
\newcommand{\Env}[0]{\mathit{Env}}
\newcommand{\nil}[0]{\mathtt{nil}}
\newcommand{\inds}[0]{\mathit{inds}}
\newcommand{\eqs}[0]{\mathit{eqs}}
\newcommand{\pr}[1]{{\rm pr}(#1)}
\newcommand{\ptr}[2]{\mathtt{Tr}\texttt{[}#1\texttt{]}\texttt{(}#2\texttt{)}}
\newcommand{\ptrtt}[2]{\texttt{Tr[}#1\texttt{](}#2\texttt{)}}
\newcommand{\qv}[1]{\mathrm{qv}(#1)}
\newcommand{\cnv}[1]{\mathrm{cnv}(#1)}
\newcommand{\tr}[2]{{\rm tr}_{#1}(#2)}
\newcommand{\braw}[1]{[\![#1]\!]}
\newcommand{\brat}[1]{\langle #1 \rangle}
\newcommand{\vecb}[1]{{\bf #1}}
\newcommand{\con}[2]{\llparenthesis #1, #2 \rrparenthesis}
%\newcommand{\con}[2]{\langle #1, #2 \rangle}
\newcommand{\sndq}[2]{\mathsf{#1}! #2}
\newcommand{\rcvq}[2]{\mathsf{#1}? #2}
\newcommand{\subst}[2]{\{#1 / #2\}}

\renewcommand{\braket}[2]{\langle #1|#2\rangle}
\renewcommand{\bra}[1]{\langle #1|}
\renewcommand{\ket}[1]{|#1\rangle}
\newcommand{\ceil}[1]{\lceil #1 \rceil}

\newcommand{\discard}[1]{\mathtt{discard(} #1 \mathtt{)}}
\newcommand{\myif}[2]{\mathtt{if}~#1~\mathtt{then}~#2~\mathtt{fi}}
\newcommand{\measure}[2]{{\tt meas}~#1~{\tt then}~#2~{\tt saem}}

\newcommand{\op}[2]{#1[#2]}
\newcommand{\dsym}[2]{#1\texttt{[}#2\texttt{]}}
\newcommand{\opapp}[3]{#1_{#2}(#3)}
\newcommand{\opsym}[3]{#1\texttt{[}#2\texttt{]}\texttt{(}#3\texttt{)}}

\newcommand{\weak}[1]{\xrightarrow{\tau*} \xrightarrow{#1}
\xrightarrow{\tau*}}

\newcommand{\barb}[4]{#1 \Downarrow^{#2} (#3, #4)}

\newcommand{\brac}[1]{\{#1\}}
\newcommand{\EPR}[0]{\mathit{EPR}}
\newcommand{\sfqv}[0]{\mathit{qVar}}
\newcommand{\pair}[2]{(#1, #2)}
%-------------------
\newtheorem{defi}{Definition}[section]
\newtheorem{prf}{Proof}
\newtheorem{prop}[defi]{Proposition}
\newtheorem{thm}[defi]{Theorem}
\newtheorem{lem}[defi]{Lemma}
\newtheorem{clm}[defi]{Claim}
\newtheorem{col}[defi]{Corollary}
\newtheorem{ex}[defi]{Example}
\newtheorem{rem}[defi]{Remark}

%-------------------

\etitle{Application of Formal Methods \\
to Quantum Cryptography}
\jtitle{����Ū��ˡ���̻ҰŹ�ؤα���}
\eauthor{Takahiro Kubota}
\jauthor{������ ����}
\esupervisor{Masami Hagiya}
\jsupervisor{��ë ����}
\supervisortitle{Professor} % Professor, etc.
\date{December 13, 2013}
%-------------------
\begin{document}
\begin{eabstract}
 In general, it is
 difficult to verify security of cryptographic protocols.
 Indeed, flaws of designs and security proofs of some protocols
 were found after they had been presented. In 
 deductive verification using 
 formal methods, protocols and security properties are described in
 formal languages, and correctness of designs and security proofs are
 deduced by inference rules. While a number of formal frameworks and
 verification tools have been developed and applied to classical
 protocols, few formal methods have been applied to security proofs
 of quantum protocols.
 The contributions of this thesis consist of the following three
 results.

 First, we developed a software tool to verify
 bisimilarity
 of configurations (pairs of processes and quantum states) of qCCS, 
 a quantum process calculus presented by Feng et al.
 Bisimilar configurations behave indistinguishably from the outside.
 We designed a formal framework for the verifier, which we call
 nondeterministic qCCS, on the basis of qCCS by Feng et al.
 While the transition system of qCCS is both
 nondeterministic and probabilistic,
 we presented a nondeterministic and 
 non-probabilistic transition system for configurations, extending the
 definition of them.
 This allows the verifier to verify bisimilarity efficiently.
 Next, we designed the verifier to handle security parameters and
 quantum states symbolically. A purpose is to apply it to
 quantum cryptographic protocols, where the dimensions of quantum states
 depend on security parameters.
 When the verifier checks bisimulation relation of
 configurations, it uses user-defined equations on the symbolic
 representations to check the quantum states that the
 outsider can access are always equal.
 Besides, the verifier is sound with respect to
 qCCS, that is, when it runs with two configurations as input and
 returns
 $\mathit{true}$, a symbol representing
 success of the verification, the configurations are in bisimulation
 relation in the qCCS's definition.

 Second, we defined the notion of approximate bisimulation 
 relation of configurations of nondeterministic qCCS.
 Approximately bisimilar configurations behave indistinguishably
 from the outside up to negligible probability.
 We then proved that the approximate bisimulation relation is
 closed under application of evaluation contexts of processes.
 This suggests sanity of our definition as well as 
 feasibility in practice. As a result, we are able to
 formally verify not only that protocols are precisely equivalent but
 also that protocols are equivalent up to negligible probability.
 Moreover, we extended the verifier
 to verify the approximate bisimulation relation of
 configurations.

 Third, we formally verified Shor and Preskill's
 security proof of BB84 quantum key distribution protocol. 
 They first considered another protocol (the EDP-based protocol)
 and proved the security of BB84 and the EDP-based protocol is
 equivalent. They next proved the latter is secure.
 For the first step of their proof, we formalized the two protocols
 as configurations and formally verified that they are bisimilar.
 For the second step, we defined a completely secure protocol 
 (EDPideal)
 and formally verified that the configurations of it and the EDP-based
 protocol are approximately bisimilar. This is the first work
 where a security proof of a quantum cryptographic protocol 
 is mechanically verified using a software tool.
\end{eabstract}

\begin{jabstract}
 �Ź�ץ��ȥ���ΰ������θ��ڤϰ��̤��񤷤�, �º�, �����Ĥ��Υץ��ȥ����
 �߷פ�����������θ��꤬, ��Ƹ�˻�Ŧ�����Ȥ������Ȥ������äƤ���.
 ����Ū��ˡ���Ѥ���Ƴ��Ū�ʸ��ڤǤ�, �ץ��ȥ����������������������ǵ��Ҥ�, 
 �߷פ�������������������§�ˤ������ä�Ƴ�Ф���.
 ��ŵ�Ź�ץ��ȥ�����Ф��Ƥ�,
 ���ڤΤ����¿���η����ηϤ両�ڥġ��뤬��ȯ����Ŭ�Ѥ���Ƥ��뤬,
 �̻ҰŹ�ץ��ȥ���ΰ������������Ф��Ʒ���Ū��ˡ��
 �ۤȤ��Ŭ�Ѥ���Ƥ��ʤ�.
�ܸ���ι׸��ϰʲ���3����η�̤���ʤ�.

 ����ܤη�̤�, Feng����̻ҥץ������׻�qCCS�Υ���ե����졼�����(�ץ�
 ������
 �̻Ҿ��֤���)���������������θ��ڴ������������ȤǤ���.
 ������ʥ���ե����졼����󤿤���, �����鸫��Ʊ���褦�˿�����. 
 �ޤ��桹��, ���ڴ�Τ���η����ηϤǤ��������ŪqCCS��,
 Feng���qCCS�˴�Ť����߷פ���. qCCS�ξ������ܷϤ�
 ��ΨŪ���������Ū�Ǥ��ä���, ������ĥ��������ե����졼������
 �Ф����ΨŪ�Ǥʤ��������ܷϤ���Ƥ�, ���Ѥ���.
 ���Τ��Ȥˤ��, ���ڴ�������������Ψ�褯���ڤ���.
 ����, �桹�ϸ��ڴ��, �������ƥ��ѥ�᥿���̻Ҿ��֤򵭹�Ū��
 �����褦���߷פ���. �̻ҰŹ�ץ��ȥ���Ǥ�, �̻Ҿ��֤μ������������ƥ�
 �ѥ�᥿�˰�¸���뤳�Ȥ����뤬, ���Τ褦�ʥץ��ȥ����Ŭ�Ѥ��뤿��Ǥ�
 ��. ������ط��θ��ڤˤ�����,
 ���ڴ��, �����Ԥ�����������ǽ���̻Ҿ��֤�������������Ȥ��ǧ���뤿��
 ��, �̻Ҿ��֤ε���ɽ�����Ф���桼��������������Ѥ���.
 % �ޤ�, �̻ҰŹ�ץ��ȥ���Ǥ�, �̻Ҿ��֤��������ƥ��ѥ�᥿�˰�¸����
 % ����, ���μ���������ˤʤ뤳�Ȥ�����.
 % ���Τ褦���оݤ�Ŭ�Ѥ��뤿��, �桹�ϸ��ڴ��,
 % �������ƥ��ѥ�᥿���̻Ҿ��֤򵭹�Ū�˰����褦��
 % �߷פ���. ������ط��θ��ڤˤ�����, 
 % ���ڴ��, �����Ԥ�����������ǽ���̻Ҿ��֤�������������Ȥ��ǧ���뤿���,
 % �̻Ҿ��֤ε���ɽ�����Ф���桼��������������Ѥ���.
 �����, ���ڴ��qCCS���Ф��Ʒ����Ǥ���.
 ���ʤ��, ���ڴ�ˤդ��ĤΥ���ե����졼���������Ϥ����Ȥ�,
 ����������ɽ��$\mathit{true}$����Ϥ����ʤ��, ������qCCS������Ǥ�
 ������ط��ˤ���.

 ����ܤη�̤Ȥ���, �桹��, �����ŪqCCS�Υ���ե����졼����󤿤���,
 ̵��Ǥ����Ψ������Ƴ����鸫��Ʊ���褦�˿����񤦤Ȥ����ط��Ǥ���
 ��Ψ������ط����������. 
 �����, ��Ψ������ط����ץ�������ɾ��ʸ̮��Ŭ�Ѥ��Ф����Ĥ��Ƥ��뤳�Ȥ�
 ������. ���Τ��Ȥ�, �������������΢�դ��Ƥ���, �ޤ�, ���Ѿ��ͭ�ѤǤ�
 ��.
 ���Ū��, �����ŪqCCS���Ѥ���, �ץ��ȥ��뤬�����������Ȥ�������
 �����Ǥʤ�, ̵��Ǥ����Ψ������������Ǥ���Ȥ������Ȥ򸡾ڤǤ���
 �褦�ˤʤä�. �ޤ�, ��Ψ������ط��򸡾ڤ���褦���ڴ���ĥ����.

 �����ܤη�̤�, Shor��Preskill�ˤ��BB84�̻Ҹ������ץ��ȥ���ΰ�������
 ����, ���ڴ���Ѥ��Ʒ���Ū�˸��ڤ������ȤǤ���.
 ���ξ����Ǥ�,
 ���˰����������Ϥ��䤹���̤Υץ��ȥ���(EDP�˴�Ť��ץ��ȥ���)���ͤ���
 ��, BB84��EDP�˴�Ť��ץ��ȥ���ΰ������������Ǥ��뤳�Ȥ������줿.
 �����, EDP�˴�Ť��ץ��ȥ��뤬�����Ǥ��뤳�Ȥ������줿.
 �桹��, ���Υ��ƥåפ��Ф��Ƥ�, ������ĤΥץ��ȥ�������������
 ����ե����졼�����������Ǥ��뤳�Ȥ򸡾ڤ���.
 ����Υ��ƥåפ��Ф��Ƥ�, �ޤ�, 
 �����˰����ʸ������ץ��ȥ���(EDPideal)�������, �����
 EDP�˴�Ť����ץ��ȥ���Υ���ե����졼����󤬳�Ψ������Ǥ��뤳�Ȥ�
 ���ڤ���. �̻Ҹ������ץ��ȥ���ο���Ū�ʰ�������,
 ���եȥ��������Ѥ��Ƶ���Ū�˸��ڤ����Τ��ܸ��椬���ƤǤ���.
\end{jabstract}
% ���ڤ��� ���եȥ��������Ѥ����ݾڤ��뤳�Ȥ���������  
\maketitle
\begin{acknowledge}
 I would like to express my gratitude 
 to my supervisor, Prof.\ Masami Hagiya for his advice, encouragement,
 and offering for me valuable opportunities to study.
 I am sincerely grateful to my mentor, Dr.\ Yoshihiko Kakutani
 for his guidance, 
 decisive comments, enlightening, and encouragement during the
 research. He kindly welcomed any of my questions and 
 took a lot of time for discussions with me.
 I thank Prof. Mingsheng Ying for useful comments and 
 encouragement.

 I express my thanks to my co-authors in
 NTT Communication Science Laboratories,
 Dr.\ Go Kato, Dr.\ Yasuhito Kawano,
 and Mr. Hideki Sakurada for advice and
 comments on the basis of their expertise of physics and formal methods.

 I appreciate Dr.\ Ben Smyth and Dr.\ Gergei Bana for
 fruitful discussions.
 I thank Mr. Taku Onodera for helpful comments.
 I was supported by a grant from Graduate School
 of Information Science and Technology, The University of Tokyo.

 I thank members and ex-members of Hagiya
 laboratory especially Dr.\ Yusuke Kawamoto, Dr.\ Yoichi Hirai,
 Mr.\ Kentaro Honda, Dr.\ Masahiro Hamano,
 Dr.\ Tatsuya Abe, and Mr.\ Yukinao Kano for useful comments,
 kind advice, and help.
 Discussing with them, I could make vague ideas concrete and
 improve presentations.
\end{acknowledge}
\frontmatter
\tableofcontents
\listoffigures
%\listoftables
%-------------------
\mainmatter
\chapter{Introduction}
\section{Formal Verification of Cryptographic Protocols}
\subsection{Background}
Cryptographic protocols are essential elements of the infrastructure
to ensure secure communication and information processing. However,
security proofs of such protocols tend to be complex and
difficult to verify, which is recognized by researchers 
\cite{Shoup2004, Halevi2005, Canetti2006}.
Indeed, flaws of designs \cite{Lowe1996,
Bleichenbacher1998, Cervesato2008}
and security proofs \cite{Shoup2001oaep, Galindo2005} of
cryptographic protocols
were found years after they had been presented.
The difficulty of verification of cryptographic protocols 
is considered to come from the following points \cite{Ouyousuuri2010}.
\begin{itemize}
 \item To define and formulate security properties, which depend on 
       the functionality of each protocol, is difficult.
       Indeed, the definitions are often reconsidered
       \cite{GoldwasserMicali1984,
       DolevDworkNaor2003, RackoffSimon1992,
       Benalohetal1994, Juelsetal2005}.
 \item Execution models of cryptographic protocols
       are complex. In general, principals in a cryptographic
       protocol, including adversaries,
       run in parallel. Since each principal runs
       nondeterministically and probabilistically,
       execution models and security properties must be
       defined on the basis of parallelism, nondeterminacy, and
       probability.
 \item Methods to prove that cryptographic protocols satisfy defined
       security properties are not self-evident.
       Security proofs performed on the basis of such execution
       models tend to be quite complex.
\end{itemize}
{\it Formal methods} have been applied
to model, analyze, and verify cryptographic protocols
\cite{Blanchet2001, Armando2005, Blanchet2008cryptoverif,
Barthe2009certi, Barthe2011}.
They are based on formal frameworks, including formal
languages and systems to prove security properties such as
inference rules.
The languages are used to formalize cryptographic protocols and
security properties, and the inference rules are used to perform 
formal proofs.

Advantages of formal methods are as follows.
\begin{itemize}
 \item The use of formal languages precisely prevents ambiguity.
       Although mathematical proofs in natural languages are rigorous,
       ones in formal languages can be 
       more in terms of both description and interpretation,
       because the syntax and
       semantics of them are defined mathematically.
 \item That all inferences in a proof obey
       pre-defined inference rules is made to be explicit.
       On the other hand in ordinary proofs,
       it is not always the case. This is possibly because
       some inferences are apparently too obvious to write down
       explicitly. However, writing such inferences is still valuable
       when we put priority on rigor.
 \item Verification and proof can be partially automated.
       It reduces human costs as well as prevents errors.
       Parts that are automated depend on software tools.
       We introduce some of the tools in the next subsection.
\end{itemize}

\subsection{Existing Verification Tools}
\label{intro:tools}
$\mathsf{CertiCrypt}$ \cite{Barthe2009certi}
is a framework where we interactively construct
game-based cryptographic proofs in a proof assistant
$\mathsf{Coq}$ \cite{coq2010}. It has been 
successfully applied to
verify the security of prominent cryptographic protocols such as 
FDH \cite{Zanella2009FDH} and OAEP \cite{Barthe2011OAEP}.
The effort one needs to
build the machine-checked proofs in $\mathsf{CertiCrypt}$ 
is thought to be more than moderate \cite{Barthe2011}, 
contrasted to the high guarantees
of security. In spite of the effort to construct the proofs,
ones who read them have
only to understand the correctness of the formalization of
the target cryptographic protocol and the statement of the security to
achieve.
$\mathsf{EasyCrypt}$ \cite{Barthe2011} further
reduces users' effort. It generates
a whole machine-checked proof from {\it proof sketches}, which are
representation of essence of a security proof. It was 
applied to verify the security of Cramer-Shoup public key encryption
scheme \cite{CramerShoup1998}, to which $\mathsf{CertiCrypt}$ had
never been applied. Those approaches are
computer-aided verification of cryptographic {\it proofs},
ideas of which are in general considered by men.

$\mathsf{AVISPA}$ \cite{Armando2005} and 
$\mathsf{ProVerif}$ \cite{Blanchet2001} are frameworks to analyze
security
protocols in abstract models, where cryptographic primitives are 
idealized. For example, ciphertexts encrypted using a public key are
decrypted only using the secret key which corresponds
to the public key. In $\mathsf{AVISPA}$ framework, a protocol designer
describe a protocol paired with expected security properties
in a formal language HLPSL. The scripts are 
translated into the intermediate format and passed
to the backends including model checkers. One of them for example
tries to find attacks by
exploring the transition system specified by the intermediate format.
Those approaches use a computer to exhaust execution states of
protocols, whose number is possibly too large to do by hand,
rather than obtaining cryptographic proofs.

% Program transformation techniques is CryptoVerif
% \cite{Blanchet2008cryptoverif}.
% For another example,
% as formal languages can be interpreted by a computer,
% protocols can be simulated as programs or their execution states
% can be exhausted \cite{Armando2005, Blanchet2001}.***
% ***

\section{Formal Methods for Quantum Cryptography}
We may call {\it classical} cryptography, formal framework, process
calculus, and so on for {\it non-quantum} ones.

\subsection{Security of Quantum Key Distribution Protocols}
Security proofs are also complex in quantum cryptography,
where we must also consider attacks using entanglements.
BB84 \cite{BennetBrassard1984}
is a prominent quantum key distribution (QKD)
protocol, which allows two remote principals to share a secret
key. Let us call the two principals and an adversary Alice, Bob, and Eve 
respectively. A key of the security of BB84 and other QKD protocols
\cite{LoChau1999, TamakiKoashiImoto2003} is 
that Alice and Bob can estimate the amount of information leakage to
Eve. If Eve tries to obtain
information from quantum systems passing through a quantum channel,
she must measure some physical values of them.
It possibly disturbs the states of quantum
systems. To check the disturbance, Alice sends to Bob 
additional quantum systems
other than those which are sources of the shared secret key.
If the information leakage is judged to be too large, they
abort the protocol.
BB84 provides {\it unconditional security} that
the mutual information of Alice's and Eve's key is negligible
with respect to the number of quantum bits if Alice and Bob have not
aborted the protocol. The advantage of QKD protocols is that
the security does not depend on adversary's computational resources, 
while the security of classical key exchange protocols
are ensured on the basis of
conjectured difficulty of computing certain functions
\cite{DiffieHellman1976}.

The first security proof of BB84 is presented by Mayers 
\cite{Mayers1998}. It is about 50 pages long and complex.
Lo and Chau proposed another QKD protocol \cite{LoChau1999} whose
security proof is simple. It is based
on an entanglement distillation protocol (EDP).
It has the drawback that it requires quantum
computers, while BB84 only needs apparatus for state preparation and
quantum measurement.
Shor and Preskill presented a simple proof of BB84
\cite{ShorPreskill2000}. They showed that the security of BB84 is
equivalent to that of a modified version of Lo and Chau's protocol
(modified Lo-Chau protocol, the EDP-based protocol) \cite{LoChau1999},
whose security proof is also simple.
Our target of formal verification in this thesis 
is Shor and Preskill's proof.

\subsection{Motivation to Apply Formal Methods to Quantum\\
Cryptographic Protocols}
QKD protocols have advantage not to depend on conjectured difficulty
of computing certain functions.
They are ones of the closest application to practice in the quantum
information field.
Actually, several companies such as Id Quantique,
MagiQtechnologies, Toshiba, and NEC are developing 
commercial quantum cryptographic systems.
It is also possible that more complex quantum protocols
be presented in the future.
Therefore, it is important to develop 
formal frameworks to verify quantum protocols'
security and also make the security proofs machine-checkable.

\subsection{Process Calculi for Quantum Protocols}
Process calculi \cite{Milner1999, AbadiFournet2001, 
Blanchet2008cryptoverif}
are formal frameworks that are
suitable to verify properties of parallel systems.
They have successfully applied to verification of a number of
classical cryptographic protocols such as Diffie-Hellman
key agreement \cite{AbadiFournet2001}, 
Needham Shoroeder shared and public key protocols
\cite{Blanchet2008cryptoverif},
FDH \cite{Blanchet2006FDH}, and Kerberos \cite{Blanchet2008Kerberos}.
To clone the success in quantum information fields,
several quantum process calculi have been proposed
\cite{NagarajanPapanikolaouBowenGay2005, Lalire2006, Adao2007,
FengDuanYing2011}. Feng et al. defined a process calculus qCCS
\cite{FengDuanJiYing2007, Ying2009, FengDuanYing2011, DengFeng2012,
FengDengYing2012}.
In qCCS, a quantum protocol is formalized as a
configuration $\con{P}{\rho}$\footnote{In \cite{FengDuanJiYing2007,
Ying2009, FengDuanYing2011, DengFeng2012,
FengDengYing2012}, a configuration is written of the form
$\langle P, \rho \rangle$ using angle brackets. In this thesis,
we write $\con{P}{\rho}$ since we frequently write density operators using
bra-ket notation.}.
which is a pair of a process $P$ and a collective quantum mixed state
$\rho$
that is referred by the variables in $P$.
Gay et al. defined a quantum process calculus CQP
\cite{NagarajanPapanikolaouBowenGay2005, Davidson-etal2012}, which is
based on pi-calculus. In the later version of
CQP \cite{Davidson-etal2012}, a configuration is defined as a triple
$(\sigma;\tilde q;P)$ consisting of a collective
quantum pure state $\sigma$,
an indicator of $P$'s ownership of variables $\tilde q$,
and a process $P$.

An important notion in process calculi is 
{\it weak bisimulation} relation on processes
\cite{Milner1999, AbadiFournet2001}.
Processes in weak bisimulation relation behave equivalently:
they perform identical actions that are
visible from the outside up to invisible ones.
For example, visible actions are communications of processes
via public channels, and invisible ones are 
communications via private channels.
Usage of the relation is for example as follows.
If we formalize some protocol and its 
specification as processes and prove that they 
are in bisimulation relation,
then we have proved the protocol satisfies the specification.

In qCCS, weak bisimulation relation $\approx$
on configurations is defined.
Their definition is successful in that 
the relation is closed by application of parallel composition
of processes: if $\con{P}{\rho} \approx \con{Q}{\sigma}$ holds,
then $\con{P||R}{\rho} \approx \con{Q||R}{\sigma}$ holds for all
process $R$ with which $P||R$ and $Q||R$ are defined.
$P||R$ means that $P$ and $R$ run in parallel.
This property of $\approx$ is called congruence.
Similarly in CQP, weak bisimulation relation is defined and proven to
be congruent. The congruence property is significant when we account on 
compositional behavioural equivalence.
% The facts that the weak bisimulation relations are congruent
% support the claim that the definitions capture the intuition of
% behavioural equivalence. It is also useful practically.

\subsubsection{Application of Quantum Process Calculi}
In qCCS, quantum teleportation, super dense coding protocols
\cite{FengDuanYing2011}, and simplified version of BB84
\cite{DengFeng2012} are formalized. That they satisfy their
specifications is also verified using weak bisimulation relation.
In CQP, quantum coin-flipping game \cite{Meyer1999},
quantum teleportation protocol,
and quantum bit-commitment protocol
are formalized and their execution
are modeled \cite{NagarajanPapanikolaouBowenGay2005}.
Three fold repetition quantum error correcting code and
its specification are formalized, and they are proven to
be weakly bisimilar \cite{Davidson-etal2012}.

To extend application of process calculi to security proofs, we
applied qCCS to Shor and Preskill's
security proof of BB84 \cite{ShorPreskill2000}.
We previously formalized BB84 and the modified Lo and Chau's protocol
as configurations of qCCS and proved their bisimilarity by hand
\cite{Kubota2012}.

\section{Contributions}
We addressed the following three limitations of previous work.
\begin{itemize}
 \item Automated verification techniques have never been applied to
       verify that a quantum cryptographic protocol satisfies
       certain security criteria that are accepted in the field of
       quantum cryptography.
       In previous work \cite{NagarajanPapanikolaouBowenGay2005},
       security of BB84 was analyzed
       automatically but the strategy of the adversary Eve was
       limited: she only intercepts each qubit through the
       quantum channel, chooses either $\{\ket{0},\ket{1}\}$ or 
       $\{\ket{+},\ket{-}\}$ basis randomly,
       measures the qubit in the basis, and resends it through the
       quantum channel. She thus cannot make her qubits entangled with
       Bob's.
       On the other hand, Eve is assumed to perform arbitrary
       quantum operations in quantum cryptographic proofs
       \cite{Mayers1998, ShorPreskill2000}.
 \item Verification of weak bisimilarity in 
       any quantum process calculi
       has not been automated although by-hand verification of it
       is often hard when objective configurations have
       many long branches in their transition trees.
 \item Methodology to prove quantum cryptographic security using
       process calculi has not been established yet
       while considering weak bisimilarity is useful to verify
       equivalence of protocols. As for
       Shor and Preskill's security proof, equivalence of
       BB84 and modified Lo and Chau's protocol is stated as
       weak bisimilarity of configurations formalizing them.
       However, the way to state the security of
       the latter has not been obvious.
\end{itemize}
Specifically, the contributions of this thesis are as follows.
\subsection{A Software Tool to Verify Weak Bisimilarity of qCCS Configurations}
We implemented a software tool, which
we call {\it Verifier1}, that formally verifies
weak bisimilarity of qCCS configurations
without recursive structures.
The overview is as follows.
 \begin{itemize}
  \item Verifier1 adopts a simplified formal framework based on qCCS,
	which we may call {\it nondeterministic qCCS}.
	\begin{enumerate}
	 \item In the original syntax, there are constructors of a quantum
	       measurement $M[\tilde q;x].P$
	       and application of a quantum operator $\mathit{op}[\tilde
	       q].P$. Since we can also formalize
	       a measurement as a special quantum operation,
	       we always have two ways to
	       formalize one. We considered criteria
	       to select one way from the two, which is
	       reported in \cite{Kubota2012}. 
	       We simplified the syntax reflecting the
	       criteria so that
	       a user who verifies equivalence of
	       quantum cryptographic protocols using weak bisimulation
	       can select the
	       feasible way to formalize a quantum measurement.
	 \item qCCS's transition system
	       is probabilistic
	       and nondeterministic. A configuration is of the form
	       $\con{P}{\rho}$ and $\tr{}{\rho} = 1$. Suppose a 
	       probability-weighted configuration $\frac{1}{2} \bullet
	       \con{P}{\rho}$, which is interpreted to that we have
	       the configuration $\con{P}{\rho}$ with probability
	       $\frac{1}{2}$.
	       Instead of considering a
	       probability-weighted configuration, we
	       allow $\rho$ satisfying $0 < \tr{}{\rho} \le
	       1$ and
	       interpret $\tr{}{\rho}$ as the probability to reach the
	       configuration. For instance, we consider
	       $\con{P}{\frac{1}{2}\rho}$ in
	       the simplified transition system instead of $\frac{1}{2}
	       \bullet \con{P}{\rho}$ in the original one.
	       The simplified one is only nondeterministic.
	       To verify behavioural equivalence of
	       configurations has become easier whether
	       it is done by hand or tool.
	\end{enumerate}
  \item Verifier1 handles quantum states symbolically and
	it can be applied to security proofs.
	In security proofs,
	the dimensions of quantum states
	are generally unfixed, because they
	depend on security parameters such as the number of
	qubits which Alice sends to Bob. 
	Therefore, the way to represent states in
	a software tool is not self-evident.
	In our verifier, security parameters and quantum
	states are represented as symbols.
	A user is
	supposed to define in Verifier1 symbolic 
	representations of quantum states and 
	equations on them. To compare symbolic representations,
	Verifier1 applies such user-defined
	equations to them and simplifies them.
  \item Although Verifier1 adopts the simplified
	syntax and operational semantics, it is sound with respect
	to qCCS. If Verifier1 returns $\mathit{true}$
	with two configurations and a set of
	valid user-defined equations as input,
	then the two converted configurations
	in qCCS are weakly bisimilar.
 \end{itemize}
The design of Verifier1 and the proof of
soundness are described in Chapter \ref{symqccs}.

\subsection{Approximate Bisimulation for Quantum
Processes}
In formal verification using process calculi,
notions of {\it approximate bisimulation} relation are useful:
configurations in the relation behave equivalently up to
negligible probability. This is a possible usage of the notions.
To evaluate security of a system, we first consider an ideal
system that is always secure. We next prove the system and
the ideal one are approximately bisimilar. 
This proves that the system is secure except for
negligible probability.

We defined two kinds of approximate bisimulation relations
in our formal framework (nondeterministic qCCS),
and studied properties of them.
As described in the next subsection, we applied one of the notion
to verify formally the last step of Shor and Preskill's security proof.

Originally, $\con{P}{\rho} \approx \con{Q}{\sigma}$ means 
  \begin{itemize}
   \item $\tr{\qv{P}}{\rho} = \tr{\qv{Q}}{\sigma}$, and
   \item whenever one of $\con{P}{\rho}$ or $\con{Q}{\sigma}$
	 can perform an action,
	 the other can perform the same action up to invisible ones.
  \end{itemize}
The set of quantum variables occurring in $P$ 
is denoted by $\qv{P}$. A state space corresponds to each quantum variable.
When the quantum state of all quantum variables is $\rho$, 
$\tr{\qv{P}}{\rho}$ is the quantum state that one who does not have
variables in $\qv{P}$ can access.

We relaxed the conditions up to gaps of probabilities of
configurations' performing actions and {\it trace distance}
$d(\tr{\qv{P}}{\rho}, \tr{\qv{Q}}{\sigma})$.
When we measure an arbitrary observable (a physical value)
of quantum states with small trace distance,
we obtain identical results with close probability.
\begin{enumerate}
 \item  The first relation $\sim_{\zeta, \eta}$ is
	parametrized with $\zeta, \eta$ with $0 \le 
	\eta, \zeta \le 1$. Roughly speaking,
	$\con{P}{\rho} \sim_{\zeta, \eta} \con{Q}{\sigma}$
	means
	\begin{itemize}
	 \item $d(\tr{\qv{P}}{\rho}, \tr{\qv{Q}}{\sigma}) \le
	       \zeta$, and
	 \item whenever one of $\con{P}{\rho}$ or $\con{Q}{\sigma}$
	       can perform an action with probability greater than
	       $\eta$,
	       the other can perform the same action up to
	       invisible ones.
	\end{itemize}
	The relation is not transitive but
	if $\con{P}{\rho} \sim_{\zeta, \eta} \con{Q}{\sigma}$
	and $\con{Q}{\sigma} \sim_{\zeta', \eta'} \con{R}{\theta}$
	hold, then $\con{P}{\rho}
	\sim_{\zeta + \zeta', \max\{\eta, \eta'\} + 2(\zeta + \zeta')}
	\con{Q}{\sigma}$
	holds.
	We proved that if $\con{P}{\rho} \sim_{\zeta, \eta}
	\con{Q}{\sigma}$, then 
	$\con{P||R}{\rho} \sim_{\zeta, \eta} \con{Q||R}{\sigma}$
	for an arbitrary process $R$,
	which suggests sanity of the
	definition.
 \item The second relation $\sim$ is defined when
       quantum states depend on security
       parameters, with which the notion of {\it negligibility}
       makes sense. Roughly speaking,
       $\con{P}{\rho} \sim \con{Q}{\sigma}$ means
       \begin{itemize}
	\item $d(\tr{\qv{P}}{\rho}, \tr{\qv{Q}}{\sigma})$
	      is negligible, and
	\item whenever one of $\con{P}{\rho}$ or $\con{Q}{\sigma}$
	      can perform an action with
	      non-negligible probability,
	      the other can perform the same action up to invisible
	      ones.
       \end{itemize}
       The relation is transitive.
       We proved that if $\con{P}{\rho} \sim
       \con{Q}{\sigma}$ holds, then \\
       $\con{P||R}{\rho} \sim \con{Q||R}{\sigma}$ holds
       for an arbitrary process $R$. 
       Eventually, the relation $\sim$ is an equivalence 
       relation and closed under
       parallel composition, and thus we say it is {\it congruent}.
       The property suggests sanity of the definition as
       well as feasibility in practice.
\end{enumerate}
       We finally extended Verifier1
       to verify a subset of the second approximate bisimulation
       relation. The extended verifier is called {\it Verifier2}.
       It uses
       user-defined rewriting rules of the form $\rho = \sigma$
       for symbolic representations $\rho, \sigma$ of quantum states.
       Each rule is expected to satisfy that
       $d(\braw{\rho}, \braw{\sigma})$ is negligible, where
       $\braw{\rho}$ is the quantum state represented by the symbol 
       $\rho$. The definitions of approximate bisimulation
       relations, the proofs of their properties, and
       the extension of Verifier1 are described in
       Chapter \ref{prob_bisim}.
\subsection{Application of the Verifiers to Shor and Preskill's 
Security Proof of BB84}
We applied Verifier1 and Verifier2 to Shor and Preskill's
security proof of BB84 \cite{ShorPreskill2000}.
Our formal verification consists of the following two steps.
 \begin{itemize}
  \item In the first step of Shor and Preskill's security proof,
	equivalence of BB84 and an EDP-based protocol is proven.
	The latter protocol is a modification of Lo and Chau's
	protocol \cite{LoChau1999}.
	We first verified the equivalence using Verifier1.
	We formalized them
	as configurations
	based on our previous work
	\cite{Kubota2012}.
	We then defined equations to verify
	equivalence of the two protocols.
	They are obtained from properties of
	error-correcting codes
	discussed in the original proof \cite{ShorPreskill2000}
	and basic facts about measurement of halves of EPR pairs.
	The input is the equations and configurations of
	BB84 and of the EDP-based protocol.
	Verifier1 returns $\mathit{true}$ with the input.
  \item Second, we verified the security of the
	EDP-based protocol.
	We defined an {\it ideal} protocol, where Alice and Bob
	can create a shared key leaking no information to Eve.
	We formalized it as a configuration in the extended
	verifier. 
	We then defined rewriting rules to verify
	approximate equivalence of the
	two protocols. They
	are obtained from the second step of the 
	original proof \cite{ShorPreskill2000}
	to show the security of the EDP-based protocol.
	The input is the rewriting rules and configurations of
	the EDP-based protocol and of the ideal protocol.
	Verifier1 returns $\mathit{true}$ with the input.
 \end{itemize}
Formalization techniques, scripts, and experimental results
are described in Chapter \ref{formaliz}.
The package of Verifier1 and Verifier2 is available from\\
\texttt{http://hagi.is.s.u-tokyo.ac.jp/\~{}tk/qccs}\texttt{verifier.tar.gz}.
It includes a user manual and example scripts in the directories
\texttt{doc} and \texttt{scripts}.

\section{Related Work}
\subsection{Formal Approaches to Security of BB84}
\subsubsection{Automated Analysis using Probabilistic Model Checking}
Model checking methods have been applied to analyze security 
of QKD protocols.
Nagarajan et al.\ applied the probabilistic model checker
PRISM 2.0 \cite{KwiatkowskaNormanParker2004}
to analyze BB84 \cite{NagarajanPapanikolaouBowenGay2005}
by calculating
the probability of eavesdropping detection. They assumed a
restricted adversary who only performs intercept and resend attack
through the noiseless quantum channel, and left a full security 
proof in a formal framework for future work.
In contrast, we target 
formalization of security proofs such as Shor and Preskill's
\cite{ShorPreskill2000}, where the quantum channel is assumed to be
noisy and Eve performs arbitrary quantum operations.

\subsubsection{Verification Using a Sequential Quantum Programming Language}
In our previous work \cite{KubotaKakutaniKatoKawono2011}, we applied
program transformation methods and Hoare logic to
Shor and Preskill's security proof of BB84.
We formalized BB84 and the EDP-based protocol using a
Selinger's QPL \cite{Selinger2004}.
We then formalized their inferences
as rewriting rules of programs. Soundness of each rule was
proved on the basis of the semantics of QPL.
BB84 is transformed to the EDP-based protocol by the rewriting by
the rules.
We finally verified the security of the EDP-based protocol formally
using Kakutani's quantum Hoare logic \cite{Kakutani2009}.

When we formally verify cryptographic protocols,
an advantage of process calculi to sequential programs is that
communications and nondeterminacy are explicitly written.

\subsection{Quantum Process Calculi}
\subsubsection{CQP}
In CQP's transition system, a state is a triple called a configuration
that consists of a map from a quantum variable to a quantum pure
state, a set of quantum variables, and a process. 
A configuration transits its state interacting with the outsider
similarly to qCCS.
An example of the configuration is as follows.
\[
A \defequiv ([q,r \mapsto
 \frac{1}{\sqrt{2}}(\ket{00}+\ket{11})];r;d![\mathsf{measure}~r].Q)
\]
There are quantum variables $q$ and $r$ in this configuration.
The first component means the state of 2-qubit system indicated by 
the variables $q$ and $r$ is $\frac{1}{\sqrt{2}}(\ket{00}+\ket{11})$.
The second component $r$ means that the process 
$d![\mathsf{measure}~r].Q$ has
access to $r$, but not to $q$. Instead, the outsider has access to $q$.
The third component $d![\mathsf{measure}~r].Q$ is a process
that sends the measurement result (i.e. classical data) through the 
channel $d$, and executes $Q$.

The first state transition of the above configuration is as follows.
The right-hand side is called a mixed configuration.
\[
 A
 \xrightarrow{\tau}
 \frac{1}{2}([q,r \mapsto
 \ket{00}];r; d![0].Q) \oplus
 \frac{1}{2}([q,r \mapsto
 \ket{11}];r; d![1].Q) \defequiv B
\]
Next, it sends the value $0$ or $1$, which means
that it reveals the measurement result to the outsider.
A probability distribution of configurations,
which is called an intermediate configuration, is as follows.
\[
 B
 \xrightarrow{\tau}
 \frac{1}{2}([q,r \mapsto
 \ket{00}];r; d![0].Q) \boxplus
 \frac{1}{2}([q,r \mapsto
 \ket{11}];r; d![1].Q) \defequiv C
\]
The intermediate configuration probabilistically
performs one of 
the following transitions to become a configuration.
\begin{align*}
 & C
 \xgizarrow{\frac{1}{2}}
 ([q,r \mapsto
 \ket{00}];r; d![0].Q)
&&C
 \xgizarrow{\frac{1}{2}}
 ([q,r \mapsto
 \ket{11}];r; d![1].Q)
\end{align*}

CQP is a successful formal framework with
the definition of congruent bisimulation.
It is future work to study automated verification of
security proofs in CQP.
%  but
% there are two points that we did not adopt CQP to formalize
% QKD protocols.
% \begin{enumerate}
%  \item We thought the constructors of CQP, unitary transformation
%        and measurement, are too primitive for our purpose.
%        As we can see in Davidson's application \cite{Davidson-etal2012},
%        to perform a randomly chosen unitary transformation,
%        the source of randomness must be described as a
%        process. Prepared randomnesses are passed
%        to the process that chooses the unitary transformation.
%        When verifying quantum error correcting codes,
%        it is quite valuable to trace all cases of executions for
%        each concrete value of randomness.
%        For our targets of formalization,
%        concrete values of randomnesses are not
%        essential and actually cannot be fixed, because the lengths
%        of the randomnesses depend on the security parameter.
%        Therefore, we decided to represent probabilistic
%        local quantum operations as TPCP maps in bulk.
%  \item We did not come up with an efficient way to automate
%        the verification of bisimilarity of the CQP configurations.
%        To be a bismulation relation, a relation on
%        CQP configurations is required to be an equivalence relation.
%        If it were required to be only symmetric, 
%        we could design a verfication algorithm that
%        runs symmetrically with respect to the arguments,
%        which might be two configurations to judge the bisimilarity.
% \end{enumerate}

\subsubsection{Symbolic Bisimulation in qCCS}
The authors of qCCS presented the notion of symbolic
bisimulation for quantum processes \cite{FengDengYing2012}.
A purpose is to
verify bisimilarity algorithmically. They proved the
strong symbolic bisimilarity (internal actions must be simulated) is
equivalent to the strong open
bisimilarity, and actually presented an algorithm to verify symbolic ground
bisimilarity (outsiders do not perform quantum operations adaptively).
 Since our purpose is to apply a process calculus to security proofs
 where adversarial interference must be taken into account, we 
implemented a tool that verifies weak open bisimilarity
on the basis of the previous version of qCCS \cite{DengFeng2012}.

\subsection{Approximate Bisimulation}
\subsubsection{Approximate Bisimulation in qCCS}
The authors of qCCS also presented the notion of approximate
strong bisimulation in an earlier version of qCCS \cite{Ying2009}.
The notion is different from ours in this thesis.
In the framework, the transitions of TPCP map application are
defined to be {\it labeled}, namely, not $\tau$ transitions.
Approximate strong bisimulation identifies transitions of TPCP maps
whose {\it diamond distance} is not grater than some parameter.
While this identification is significant,
we did not directly apply this framework to our verification targets.
The first reason was only {\it strong} bisimulation was proposed,
while the protocols that we attempted to verify formally were thought
to be weakly bisimilar but not to be strongly bisimilar.
The second reason was that there was no conditional branch in the syntax,
while aborting of an execution of a QKD protocol must be formalized.

\subsubsection{Approximate Bisimulation in Classical Process Calculi}
Ying et al. introduced a notion of approximate bisimulation in
classical process calculi \cite{Ying2000} with labeled transition
systems. In their framework, the set of \emph{actions} is a metric space.
They applied the notion to verify formally 
approximate correctness of real time
systems such as real time ACP. Our notion of approximate bisimulation is
independent of theirs: distance of quantum states is considered but
that of actions is not.

\subsubsection{Approximate Bisimulation in Labeled Transition Systems
   with Observations}
Girard et al. defined a notion of approximate bisimulation in \emph{labeled
transition systems with observations} \cite{Girard2005}.
In a labeled transition system with observations, there is an
\emph{observation map}
that carries a state $q$ to an \emph{observation}  $\langle\!\langle q
\rangle\!\rangle$, and the set of observations $\Pi$ is a metric space.
The notion of approximate bisimulation is defined based on
the distance $d_{\Pi}(\langle\!\langle q \rangle\!\rangle,
\langle\!\langle q' \rangle\!\rangle)$.
Our notion of approximate bisimulation appears to be similar to
theirs when we substitute trace distance for $d_{\Pi}$ and
partial trace for $\langle\!\langle \cdot \rangle\!\rangle$.
There are three different points between our work and theirs.
First, we considred \emph{weak} bisimulation relation and proved
that it is closed by parallel composition of processes,
which are important peculiarly in process calculi.
Second, since our formal framework involves probability,
transitions with probability less than some threshold $\eta$
(respectively, transitions with negligible probability) is ignored in our
notion. Third, since the relation $\sim$ incorporates with 
the notion of negligibility, it is transitive.

\subsection{Automated Verification Tool for Classical Protocols}
$\mathsf{CryptoVerif}$ \cite{Blanchet2008cryptoverif} is a software tool
to verify security of
\emph{classical} protocols. It has been applied to
both high-level protocols \cite{Blanchet2008Kerberos, Blanchet2012oeke} 
that employ cryptographic primitives and to
cryptographic schemes \cite{Blanchet2006FDH} that are possibly be
used as primitives.
There are two features of $\mathsf{CryptoVerif}$ that are 
related to our verifiers: it is designed on the basis of a probabilistic
process calculus, and it incorporates with negligibility.
Of course, it cannot be directly applied to verify security of QKD
protocols. In $\mathsf{CryptoVerif}$'s framework,
all data are classical and
a process, which may be an adversary,
is bounded in polynomial time, while an adversary against a QKD protocol
is not. 

As a proof technique, $\mathsf{CryptoVerif}$ applies 
\emph{observational equivalence} of processes.
Let $\Pr(P \rightsquigarrow a)$
be the probability
that the process $P$ transits to a process that is ready to send some
data through the channel $a$.
Two processes $P$ and $Q$ are observationally equivalent, written
$P \cong Q$ here, if
$$
|\Pr(C[P] \rightsquigarrow a) - \Pr(C[Q] \rightsquigarrow a)|
$$
is negligible for all evaluation context\footnote{An evaluation context
is composed by a hole, channel restrictions, and
parallel compositions.}
$C[\_]$ that runs in polynomial time
and channel $a$ that is
not restricted. 
The relation $\cong$ is \emph{congruent} by the definition,
namely, if $P \cong Q$ holds, then $C[P] \cong C[Q]$ holds for all
evaluation context $C[\_]$ running in polynomial time.
When we consider $C[\_]$ as a polynomial time
adversary that runs in parallel and interacts with the protocol $P$ or
$Q$,
the observational equivalence of them is intuitively interpreted
as indistinguishability of the protocols from an adversary.

In cryptographic proofs, security of a high-level protocol
is reduced to security of the employed cryptographic primitives.
Similarly, security of a cryptographic scheme is reduced to 
assumed difficulty of computing certain functions.
In $\mathsf{CryptoVerif}$, a user formalizes such assumptions as
observational equivalence of processes.
$\mathsf{CryptoVerif}$ uses such user-defined equivalences as
rewriting rules:
if a target process $P$ is of the form $C[X]$ and there is a user
defined
equivalence $X \cong Y$, then it is rewritten to $C[Y]$.
By congruence, $P \cong C[Y]$ holds.
Given a process formalizing a target protocol and 
user-defined observational equivalences, 
$\mathsf{CryptoVerif}$
rewrites the process repeatedly until it becomes a process
that is obviously secure.

On the other hand, our tools verify bisimilarity by
tracing execution paths of configurations, not by
rewriting processes.
Fortunately, the bisimulation in qCCS is congruent, and
it is thus possible that a verification tool is
designed to verify bisimilarity by rewriting.
With such a verifier, bisimilarity of 
big-sized configurations is derived from that of some small-sized ones.
Especially in proofs of security of QKD protocols,
difficulty of computing certain functions is not assumed.
Therefore, even if a verifier conducts rewriting,
we possibly need to prove bisimilarity of
such small-sized configurations unlike
verification using $\mathsf{CryptoVerif}$.

\chapter{Preliminaries}
\label{prel}
\section{Notations}
We use the following notations in this thesis.
\begin{itemize}
 \item $\mathbb{N}$, $\mathbb{N}_+$, $\mathbb{R}$, and $\mathbb{C}$ are
       the set of natural numbers, the set of positive natural numbers, 
       real numbers, and complex numbers,
       respectively.
 \item $e$ is the base of the natural logarithm.
 \item For $a \in \mathbb{C}$, $a^\ast$ is the complex conjugate of
       $a$, and $|a| = \sqrt{a^\ast a}$.
 \item For a linear operator $A$, $A^\dagger$ is the adjoint of
       $A$. $I$ and $O$ are the identity operator and the zero operator
       on
       a vector space with the appropriate dimension in a context.
       $A > 0$ means that $A$ is positive.
 \item $[1..n]$, $(0,1]$, and $[0,1]$ are
       $\{1,2,...,n\}$, $\{x \in \mathbb{R}\,|\,0 < x \le 1\}$, and
       $\{x \in \mathbb{R}\,|\,0 \le x \le 1\}$, respectively.
 \item $\R^\ast$ is the reflexive and transitive closure of 
       a binary relation $\R$. $\R^{-1}$ is the inverse relation of
       $\R$.
 \item $\Pr(A)$ is the probability that an event $A$ happens.
 \item $I(X;Y)$ is the mutual information of discrete
       classical random variables $X$ and $Y$.
%\item $A:=B$ means that $A$ is defined as $B$.
\end{itemize}

\section{Basic Quantum Information}
We consulted the textbooks by Nielsen and Chuang 
\cite[Part 1]{NielsenChuang-Kimura2004} and 
by Ishizaka et al.\ \cite{Ishizakaetal2012} in writing this section.
\subsection*{Quantum States and Operators}
A quantum bit (qubit) is a physical system
whose {\it pure state} is described as a unit vector in
a 2-dimensional complex Hilbert space. The space is called
the state space of the qubit.
For a 2-dimensional complex Hilbert space $\H$,
we can fix an orthonormal basis of $\H$ and write one as
$\ket{0}$ and the other as $\ket{1}$. 
For $\ket{\phi} \in \H$,
the adjoint $\ket{\phi}^\dagger$ of a qubit string $\ket{\phi}$ is denoted
by $\bra{\phi}$.
For $\ket{\phi},\ket{\psi} \in \H$,
their inner product is written $\braket{\phi}{\psi}$.
$\mathbb{C}^2$ with the
canonical inner product is an instance of 2-dimensional Hilbert space.
For all $\ket{\psi} \in \mathbb{C}^2$,
there exist $\alpha, \beta \in \mathbb{C}$ satisfying
\begin{align*}
&\ket{\psi} = \alpha \ket{0} + \beta \ket{1} \mbox{ and }|\alpha|^2 + |\beta|^2 = 1,
\mbox{ where }\ket{0} 
:= \begin{bmatrix}
    1\\
    0
  \end{bmatrix},
~\ket{1}
:= \begin{bmatrix}
  0 \\
  1
  \end{bmatrix},
\end{align*}
and for all $\ket{\phi}, \ket{\xi} \in \mathbb{C}^2$, if
$\ket{\phi} = a\ket{0} + b\ket{1}$ and $\ket{\xi} = c\ket{0} +
d\ket{1}$, then 
\[
\braket{\phi}{\psi} = a^\ast c + b^\ast d.
\]
The two pure states $\ket{0}$ and $\ket{1}$ correspond to classical bit
values 0 and 1. Quantum states $\frac{\ket{0}+\ket{1}}{\sqrt{2}}$ and
$\frac{\ket{0}-\ket{1}}{\sqrt{2}}$ are written $\ket{+}$ and $\ket{-}$.

A discrete time evolution of a qubit is a unitary operator on its state
space. For example, the following operators on $\mathbb{C}^2$ are
unitary.
\[
 X := \begin{bmatrix}
  0 & 1\\
  1 & 0
  \end{bmatrix},\,
 Y := \begin{bmatrix}
  0 & -i\\
  i & 0
  \end{bmatrix},\,
 Z := \begin{bmatrix}
  1 & 0\\
  0 & -1
  \end{bmatrix},\,
 H := \frac{1}{\sqrt{2}}
 \begin{bmatrix}
  1 & 1\\
  1 & -1
 \end{bmatrix}.
\]
$X, Y,$ and $Z$ are called Pauli matrices. $H$ is called
an Hadamard transformation. The following equations hold for the matrices.
\begin{align*}
 &X\ket{0} = \ket{1},\, X\ket{1} = \ket{0},\, Z\ket{+} = \ket{-},\,
 Z\ket{-} = \ket{+},\, Y = iXZ,\\
 &H\ket{0} = \ket{+},\,H\ket{1} = \ket{-}.
\end{align*}
$X$ is said to give a bit flip, and
$Z$ is said to give a phase flip.

Let $\H_1$,...,$\H_n$ be 2-dimensional complex Hilbert spaces.
The pure state of a qubit string with bit length $n$ is described as a
unit vector in $\H_1 \otimes \cdots \otimes \H_n$.
We write $\ket{\psi_1...\psi_n}$ as $\ket{\psi_1} \otimes...\otimes
\ket{\psi_n}$.
% Let $\ket{\psi_1},...,\ket{\psi_n}$ be quantum states of qubits.
% The state of the qubit string $\ket{\psi_1...\psi_n}$, whose bit length
% is $n$,
% is defined as the 
% tensor product $\ket{\psi_1} \otimes...\otimes \ket{\psi_n}$.
The set $\{\ket{x_1...x_n}\}_{x_1,...,x_n \in \{0,1\}}$ is 
an orthonormal basis and
called the \emph{computational basis state}.
In this thesis, we may convert
$\ket{\psi} \otimes \ket{\phi} \in \H_1 \otimes \H_2$ and 
$\ket{\phi} \otimes \ket{\psi} \in \H_2 \otimes \H_1$ each other,
where $\H_1$ and
$\H_2$ are Hilbert spaces. This conversion is written as $\simeq$.

Let $\ket{\phi_1},...,\ket{\phi_m}$ be states of qubit strings with the
same bit length and $p_1,...,p_m$ satisfy $\sum_{i=1}^{m}p_i = 1$
and $0 \le p_i \le 1$ for all $i$. The quantum {\it mixed state} where the
state is $\ket{\phi_i}$ with probability $p_i$ for all $i$ is
denoted by the {\it density operator} $\sum_{i=1}^m
p_i\ket{\phi_i}\bra{\phi_i}$. The set of density operators on 
a Hilbert space $\H$ is written $\D(\H)$.
We may omit scalar multiplication when it is trivial.
A local quantum 
operation acting on a mixed state is represented by a 
\emph{trace preserving and completely-positive (TPCP) map}.
A map $\E$ is positive if it maps a positive operator to a
positive operator.
A map $\E$ is CP if $\E \otimes I$ is positive
for all $n \in \mathbb{N}$, which is the dimension of the domain of $I$.
For each TPCP map $\E$,
there exist $V_1,...,V_k$ that satisfy $\E(\rho) = \sum_{i=1}^{k} V_i
\rho V_i^\dagger$ and $\sum_{i=1}^{k}V_i^\dagger V_i =I$.
For each CP map $\F$,
there exist $W_1,...,W_l$ that satisfy $\F(\rho) = \sum_{j=1}^{l} W_j
\rho W_j^\dagger$.

\subsection*{Quantum Measurement}
To obtain classical information from a quantum system in a certain state,
we have to
\emph{measure} some physical value of the system in the state. A quantum
measurement may change the state of the target system.
A physical value that can be measured is called an \emph{observable}.
An observable of a system $\ket{\psi} \in \H$
is denoted by an Hermitian operator on $\H$, namely,
an operator $A$ satisfying $A = A^\dagger$.
An Hermitian operator $A$ has an eigenvalue decomposition
$A = \sum_{i \in I} \lambda_i \ket{i}\bra{i}$, where $\lambda_i$ is an
eigenvalue and $\ket{i}$ is the eigenvector corresponding to $\lambda_i$.
$A$ has the unique spectral decomposition $A = \sum_{j \in J}
\lambda_j P_j$, where $\lambda_j = \lambda_{j'}$ implies $j = j'$
for all $j, j' \in J$. $P_j$ is called the projector to the 
eigenspace of $\lambda_j$.

When we measure an observable $A = \sum_i \lambda_j P_j$ of
a system in a pure state $\ket{\psi}$, we obtain
the result $\lambda_j$ with probability $\bra{\psi}P_j
\ket{\psi}$, and the post-measurement state is 
$\frac{P_j \ket{\psi}}{\sqrt{\bra{\psi}P_j\ket{\psi}}}$.
For example, if we measure an observable $Z = 1\ket{0}\bra{0}
+ (-1)\ket{1}\bra{1}$ of
a state $\alpha \ket{0} + \beta \ket{1}$, we obtain
the result $1$ and the post-measurement
state $\ket{0}$
with probability $|\alpha|^2$, and 
the result $-1$ and the post-measurement
state $\ket{1}$
with probability $|\beta|^2$.
We can calculate the probability of
obtaining each measurement result from a mixed state.
Let the objective mixed state is $\rho = 
\sum_i p_i \ket{\psi_i}\bra{\psi_i}$.
When we measure an observable $A = \sum_i \lambda_i P_i$,
the probability that we obtain $\lambda_j$ is
\[
 \sum_i p_i \bra{\psi_i} P_j \ket{\psi_i} 
 = \tr{}{P_j \rho P_j}
 = \tr{}{P_j \rho},
\]
and the post-measurement state is
\[
 \sum_i \frac{p_i \bra{\psi_i}P_j\ket{\psi_i}}{\tr{}{P_j \rho}} \frac{P_j\ket{\psi_i}\bra{\psi_i}P_j}
 {\bra{\psi_i}P_j\ket{\psi_i}}
 =
 \frac{P_j \rho P_j}{\tr{}{P_j \rho}}.
\]

Some abbreviations about measurements are often used. We say
``measure $\ket{\psi}$ in the $\{\ket{0},\ket{1}\}$ basis'' for
``measure an observable $0\ket{0}\bra{0}+1\ket{1}\bra{1}$ of
$\ket{\psi}$''. Similarly, 
we may say ``measure $\ket{\psi}$ in the $\{\ket{+},\ket{-}\}$ basis'' for
``measure an observable $0\ket{+}\bra{+}+1\ket{-}\bra{-}$ of
$\ket{\psi}$''.

For a density operator $\rho$ and an observable $\sum_j \lambda_j P_j$,
$\frac{P_j \rho P_j}{\tr{}{P_j \rho}}$ is the density operator
denoting the conditional probability distribution given the result $\lambda_j$ 
of the measurement. When the pre-measurement state is $\rho$,
the probability distribution obtained after the measurement of
an observable $\sum_j \lambda_j P_j$ is
\[
 \sum_j \tr{}{P_j \rho} \frac{P_j \rho P_j}{\tr{}{P_j \rho}} =
 \sum_j P_j \rho P_j.
\]
The map $\E_{\mathit{projmeas}}(\rho) =  \sum_j P_j \rho P_j$ is a
TPCP map. For all $j$, the map $\E^j_{\mathit{projmeas}}(\rho) =
P_j \rho P_j$ is a CP map.

\subsection*{Partial Trace}
Let $\rho \in \D(\H_1 \otimes \H_2)$.
$\rho$ can be written as $\sum_i C_i \otimes D_i$, where
$C_i$ is a linear operator on $\H_1$
and $D_i$ is on $\H_2$ for all $i$. 
The \emph{partial trace} of $\rho$ by $\H_1$, denoted
by $\mathrm{tr}_{\H_1}(\rho)$, is defined as
$\sum_i \tr{}{C_i}D_i$. When one
measures an observable $I \otimes (\sum_j \lambda_j P_j)$ on
$\H_1 \otimes \H_2$, he obtains the result
$\lambda_j$ and the post-measurement state is
$\frac{(I \otimes P_j)\rho(I \otimes P_j)}
{\tr{}{(I \otimes P_j)\rho}}$
with probability $\tr{}{(I \otimes P_j)\rho}$ for
a pre-measurement state $\rho$.
Let us write $\rho_2$ for $\tr{\H_1}{\rho}$.
We have
\[
 \tr{}{(I \otimes P_j)\rho} = \tr{}{P_j\rho_2}
\]
and
\[
 \tr{\H_1}{\frac{(I \otimes P_j)\rho(I \otimes P_j)}
 {\tr{}{(I \otimes P_j)\rho}}}
=
 \frac{P_j \rho_2 P_j}
{\tr{}{P_j\rho_2}}.
\]
Hence, the above measurement can be considered as the measurement of
the observable $\sum_j{\lambda_j}P_j$ for the 
pre-measerment state $\rho_2$.
% $A$ of the state $\mathrm{tr}_{\H_1}(\rho)$
% and that of $A \otimes I \in \H_1 \otimes \H_1$ gives him or her
% identical measurement results and post-measurement states
% with identical probability.
Assume that Bob can measure an observable 
only on the partial quantum system $\H_2$.
% but the total system corresponds to
%$\H_1 \otimes \H_2$.
For the above reason,
we may say that $\mathrm{tr}_{\H_1}(\rho)$
is the quantum state that he has access or
his {\it view}, in the following chapters.

\section{Quantum Error Correcting Code}
The quantum key distribution (QKD) protocols that are
our target of formal verification in this thesis include an error
correction step after quantum communication.
The error correction step is based on Calderbank-Shor-Steane (CSS)
\cite{CalderbankShor1996} quantum error correcting code (QECC), and
it is described in the stabilizer formalism. In this 
section, we introduce 
necessary definitions and properties of the stabilizer formalism and
CSS QECC.

We rely on the textbook by Nielsen and Chuang 
\cite[Chapter 10]{NielsenChuang-Kimura2004}.
\subsection{Stabilizer Formalism}
\subsubsection{Stabilizers}
A quantum state $\ket{\psi}$ is \emph{stabilized} by a unitary operator
$U$ if $U\ket{\psi} = \ket{\psi}$. Let the Pauli group $G_n$ on
$2^n$-dimensional space be defined as
\[
 G_n := \{\pm g, \pm ig \,|\, g = A_1 \otimes A_2 \cdots \otimes A_n,
 A_j \in \{I, X, Y, Z\} \mbox{ for all } j\}.
\]
For $g, g' \in G_n$, $g$ and $g'$ is said to be \emph{commutative}
if $gg' = g'g$ and \emph{anticommutative} if $gg' = -g'g$. For all $g, g'
\in G_n$, $g$ and $g'$ are either commutative or
anticommutative.

Let $S$ be a subgroup of $G_n$ and a set $V_S$ be defined
as
\[
 V_S := \{\ket{\psi} \,|\, \mbox{ For all } g \in S,\, g \mbox{ stabilizes
} \ket{\psi} \}.
\]
$S$ is said to be commutative if $g$ and $g'$ are
commutative for all $g, g' \in S$.
$V_S$ is a vector space and $S$ is called the stabilizer of $V_S$.
$V_S$ is said to be non-trivial if $V_S \neq \{0\}$.
The condition that $V_S$ is non-trivial is characterized as follows.

\begin{prop}
 $V_S$ is non-trivial if and only if
 $S$ is commutative and $-I \notin S$ holds.
\end{prop}

Let $\{g_1, g_2, ...,g_l\} \subseteq S$.
If for all $a \in S$, $a$ can be written as a product of
the elements in $\{g_1, g_2, ...,g_l\}$, $\{g_1,g_2,...,g_l\}$ is
said to be a generator of $S$, and we write $S = \langle
g_1, g_2,...,g_l \rangle$. 
A generator $\{g_1, g_2, ...,g_l\}$ is said to be
\emph{independent} if for all $i$, $g_i$ cannot be written as a
product of the elements in $\{g_1, g_2, ...,g_l\} \backslash \{g_i\}$.
The following proposition gives the dimension of the space
of quantum codewords in the later discussion.

\begin{prop}
 Let $S = \langle g_1, g_2,...,g_{n - k} \rangle$. $V_S$ is 
 a $2^k$-dimensional vector space if
 $\{g_1, g_2, ...,g_{n - k}\}$ is independent and commutative,
 and $-I \notin S$ holds.
\end{prop}

We have the following way to check independence and commutativity
of generators $g_1, g_2, ..., g_l$ using bit vectors and matrices.
For $g \in G_n$, let $r(g)$ be the bit vector with the length $2n$ defined
as
\begin{align*}
 &r(g) := 
  \begin{bmatrix}
   b_1 & b_2 & ... & b_n & b_{n+1} & ... & b_{2n}
  \end{bmatrix}
 \mbox{ where}\\
 &g = cA_1 \otimes A_2 \otimes \cdots A_n \mbox{ and for all }j 
 \mbox{ and } c \in \{\pm 1, \pm i\},\\
 &\mbox{ if } A_j = I \mbox{ then } b_j = 0 \mbox{ and } b_{n+j} = 0\\
 &\mbox{ if } A_j = X \mbox{ then } b_j = 1 \mbox{ and } b_{n+j} = 0\\
 &\mbox{ if } A_j = Z \mbox{ then } b_j = 0 \mbox{ and } b_{n+j} = 1\\
 &\mbox{ otherwise } \mbox{ then } b_j = 1 \mbox{ and } b_{n+j} = 1
\end{align*}
Next, let $\Lambda$ be
\[
 \begin{bmatrix}
    0 & I\\
    I & 0
  \end{bmatrix}.
\]
We then have that
$r(g) \Lambda r(g') = 0$ if and only if $g$ and $g'$ are commutative.
We can describe a generator $\{g_1,...,g_l\}$ as a matrix
\[
 \begin{bmatrix}
  r(g_1)\\
  r(g_2)\\
  ... \\
  r(g_l)
 \end{bmatrix}
\]
though it does not keep the information of
scalar multiplication $\pm 1$, $\pm i$.

In the following discussions, we assume 
$\{g_1, g_2,...,g_l\}$ are independent and commutative, and
$-I \notin S$ holds for each
stabilizer $S = \langle g_1, g_2,...,g_l \rangle$. 

\subsubsection{Unitary Operations in Stabilizer Formalism}
Let $V_S$ be stabilized by a subgroup $S = 
\langle g_1, g_2,...,g_l \rangle$. Let $\ket{\psi}$ be an 
arbitrary element in $V_S$. For all unitary operator $U$ and $g \in S$,
we have 
\[
 U\ket{\psi} = Ug\ket{\psi} = UgU^\dagger U\ket{\psi}.
\]
Therefore, $UV_S := \{U\ket{\psi}\,|\,\ket{\psi} \in V_S\}$ is
stabilized by $USU^\dagger := \{UgU^\dagger \,|\, g \in S \}$.

\subsubsection{Measurement in Computational Bases in Stabilizer 
Formalism}
Let us consider the measurement of an observable 
\[
g \in
\{A_1 \otimes A_2 \otimes \cdots \otimes
A_n, A_i \in \{I, X, Y, Z\} \mbox{ for all } i\} \subseteq G_n.
\]
Assume the system is in a state $\ket{\psi}$ that is stabilized
by $S = \langle g_1,g_2,...,g_l \rangle$. There are two possibilities.
\begin{enumerate}
 \item $g$ is commutative with all $g_1, g_2,...,g_l$.
 \item $g$ is anticommutative with some of $g_1, g_2,...,g_l$.
       In this case, we can assume $g$ is anticommutative
       with $g_1$ and commutative with
       $g_2,...,g_l$ without loss of generality.
       When $g$ is anticommutative with $g_i$, we can relabel
       $g_i$ to $g_1$ and $g_1$ to $g_i$. We then have that
       $g$ is commutative with $g_1 g_j$ if $g_j$ is not commutative
       with $g$. We can replace $g_j$ with $g_1 g_j$.
\end{enumerate}
In fact, the result of the measurement is as follows in each case.
\begin{enumerate}
 \item Either $g$ or $-g$ is in $S$. If $g \in S$ holds, then
       the measurement result is $1$ with probability $1$.
       If $-g \in S$ holds, then
       the measurement result is $-1$ with probability $1$.
       In both cases, the measurement does not change the state
       $\ket{\psi}$.
 \item Neither $g$ nor $-g$ is in $S$.
       We have the measurement result $1$ or $-1$ with 
       probability $\frac{1}{2}$. The state after the measurement
       is stabilized by $\langle g, g_2, ..., g_l \rangle$ if the
       result is $1$ or by $\langle -g, g_2, ..., g_l \rangle$ if
       $-1$.
\end{enumerate}

\subsection{Stabilizer Codes}
A vector space that is stabilized by 
$S = \langle g_1, g_2,..., g_{n-k} \rangle$ is called
\emph{$[n,k]$-stabilizer code} and written $C(S)$. The elements in $C(S)$
are called codewords.
We define the \emph{logical} computational basis states
as follows. Let $\bar Z_1, \bar Z_2, ..., \bar Z_k \in G_n$ 
make the set $\{g_1, g_2,...,g_{n-k}, \bar Z_1, \bar Z_2,
..., \bar Z_k \}$ independent and commutative.
The state that is stabilized by 
\[
\langle
g_1, g_2,...,g_{n-k}, (-1)^{x_1}\bar Z_1, (-1)^{x_2}\bar Z_2,
..., (-1)^{x_k}\bar Z_k \rangle
\]
 is defined to be a
logical computational basis state $\ket{x_1 x_2 \cdots x_k}_L$.
For $j \in [1..k]$, $\bar Z_j$ is the logical Pauli operator $Z$
that acts on the $j$-th logical qubit.
Let $\bar X_j \in G_n$ be an operator that satisfies
$\bar X_j \bar Z_j \bar X_j^\dagger = - \bar Z_j$ and
$\bar X_i \bar Z_j \bar X_i^\dagger = \bar Z_i$ for all $i$ with
$i \neq j$. $\bar X_j$ is a logical $X$ operator that
acts on the $j$-th logical qubit.

In fact, it is sufficient to
consider bit flip and phase flip errors
to consider correction of general errors.
Let us take an arbitrary element $E \in G_n$ acting on
$C(S)$, where $S = \langle g_1, g_2,..., g_{n-k} \rangle$.
There are the following 3 cases.
\begin{enumerate}
 \item $E$ is anticommutative with some $g_i$ for $i \in [1..n-k]$.
       By the error $E$, the stabilizer becomes 
       $\langle g_1, g_2,...,-g_i,...,g_{n-k} \rangle$.
       When the observables \\
       $g_1,...,g_i,...,g_{n-k}$ are measured,
       the results are $1,...,-1,...,1$. Therefore, we
       can identify the position $i$ by the measurement.
 \item $E \in S$. The error $E$ does not change the objective state.
 \item $E$ is commutative with $g_i$ for all $i \in [1..n-k]$ and 
       $E \notin S$. Such errors maps a codeword in $C(S)$ to
       a codeword in $C(S)$.
       In fact, some of the errors cannot be corrected.
\end{enumerate}

Let the \emph{centralizer} $Z(S)$ of $S$ be defined as
\[
Z(S) := \{E \,|\, Eg=gE \mbox{ for all } g \in S \}.
\]
We have the following theorem.

\begin{thm}
\label{pre:correctable}
 Let $C(S)$ be a stabilizer code and $\{E_j\}_{j \in J}$ be a set of
 operators in $G_n$. If $E_j^\dagger E_k \notin Z(S) - S$ for all 
$j, k \in J$, then $\{E_j\}_{j \in J}$ is the set of errors
that can be corrected.
\end{thm}
We describe the way to correct errors. 
Let $S = \langle g_1, g_2,..., g_{n-k} \rangle$ and $\{E_j\}_{j \in J}$
be a set of errors satisfying the condition of Theorem
\ref{pre:correctable}. Assume that an arbitrary error $E_j$ has
performed to an arbitrary codeword in $C(S)$. The error correction
goes as follows.
\begin{itemize}
 \item We measure the observables $g_1,g_2,...,g_{n-k}$, and
       let measurement results, namely the \emph{syndrome},
       be $\beta_1,\beta_2,...,\beta_{n-k} \in
       \{1,-1\}$. $E_j g_l E_j^\dagger = \beta_l g_l$ holds for all $l
       \in [1..n-k]$.
 \item If $E_j$ is the only error that has the syndrome
       $\beta_1,\beta_2,...,\beta_{n-k}$, then it is sufficient
       to correct the error to apply $E_j$ to the objective system,
       because $E_j\beta_l g_l E_j^\dagger = \beta_l^2 g_l = g_l$ holds.
 \item If there is an error $E_{j'}$ in $\{E_j\}_{j \in J}$
       that has the same syndrome as $E_j$, then it is sufficient
       to correct the error to apply $E_{j'}^\dagger$ to the objective
       system.
       The reason is as follows. Let $P$ be a projector to $C(S)$.
       Since $E_j$ and $E_{j'}$ have the same error syndrome,
       $E_jPE_j^\dagger = E_{j'}PE_{j'}^\dagger$ holds.
       This implies $E_{j'}^\dagger E_jPE_j^\dagger E_{j'} = P$.
       Because of the assumption that $\{E_j\}_{j \in J}$ satisfies
       the condition of Theorem \ref{pre:correctable},
       $E_{j'}^\dagger E_{j} \in S$ holds. Namely, $E_{j'}^\dagger E_j$
       stabilizes the objective state.
\end{itemize}
\subsubsection{Distance of Quantum Codes}
Similarly to classical codes, the notion of distance of
quantum codes is defined. The weight of $E \in G_n$ is defined
as the number of the factors of $E$ that are not equal to $I$.
The \emph{distance} of a stabilizer code $C(S)$
is defined as the minimum weight of $Z(S) - S$.
When $C(S)$ is an $[n,k]$-stabilizer code with the distance $d$,
$C(S)$ is said to be a $[n,k,d]$-stabilizer code.
By Theorem $\ref{pre:correctable}$, a stabilizer code with the distance
$2t+1$ can correct arbitrary errors in $t$ qubits.

\subsection{CSS Quantum Error Correcting Code}
\subsubsection*{Stabilizer Form}
CSS quantum error correcting code (QECC) \cite{CalderbankShor1996} is
described in
the stabilizer formalism. It employs a classical $[n,k_1]$ code $C_1$ and
a $[n,k_2]$ code $C_2$ satisfying $C_2 \subseteq C_1$.
It is also assumed that both $C_1$ and $C_2^\perp$ correct $t$ 
errors. Let $\CSS(C_1,C_2)$ be a $[n, k_1 - k_2]$ stabilizer code that
is stabilized
by the set whose generator is described by the following matrix
\[
 \begin{bmatrix}
    H(C_2^\perp) & 0\\
    0 & H(C_1)
  \end{bmatrix},
\]
where $H(C_2^\perp)$ and $H(C_1)$ are parity check matrices of
$C_2^\perp$ and $C_1$, whose types are $(n-k_2) \times n$ and $k_1
\times n$.
The generators $g_1,g_2,...,g_{n-(k_1 - k_2)}$ are commutative and 
independent since $C_2 \subseteq C_1$.
The distance of $\CSS(C_1,C_2)$ is at least $2t + 1$.

\subsubsection*{Construction of CSS code}
Let $w \in \{0,1\}^k$ and assume $\ket{w}$ be the state to be coded.
Let $x = G_1 w \in C_1$, where $G_1$ is
a generator matrix of $C_1$. The codeword $\ket{x +
C_2}$ for $w$ is defined as
\[
 \ket{x + C_2} := \frac{1}{\sqrt{|C_2|}}\sum_{y \in C_2} \ket{x+y}.
\]
A parametrized $\CSS_{u,v}(C_1,C_2)$ code defined as follows
is equivalent to $\CSS(C_1,C_2)$ for $u \in C_2$ and 
$v \in \{0,1\}^n - C_1$.
\[
 \ket{x + C_2} := \frac{1}{\sqrt{|C_2|}}\sum_{y \in C_2}(-1)^{u \cdot y}
 \ket{x+y+v}.
\]
A parametrized $\CSS_{u,v}(C_1,C_2)$ is considered in the discussion of
equivalence of BB84 and the EDP-based protocol.

\subsection{Entanglement Distillation based on an Error Correcting Code}
Let $\ket{\beta_{00}} := \frac{\ket{00}+\ket{11}}{\sqrt{2}}$. 
Two qubits in the state $\ket{\beta_{00}}$ are called an EPR pair.
Let us consider the following scenario. 
First, Alice prepares $\ket{\beta_{00}}^{\otimes n} \simeq
\sum_{i \in \{0,1\}^n}\ket{i}\ket{i} \in \H_A
\otimes \H_B$.
%First, Alice prepares $\ket{\beta_{00}}^{\otimes n}$,
%which is identified with $\sum_{i \in \{0,1\}^n}\ket{i}\ket{i} \in \H_A
%\otimes \H_B$.
Second, Alice sends the qubit string whose state is
in $\H_B$ to Bob through a noisy quantum channel. 
Alice and Bob want to share the halves of 
a smaller number $k (\le n)$ of EPR pairs from the given state that
may be influenced by noise. In fact, this is possible by
quantum error correction if the number of errors is small enough to
be corrected.

Let $\{ g_1,...,g_{n-k} \}$ be commutative, independent, and do not
produce $-I$.
Then, $C(S)$ with $S := \langle g_1 \otimes I,...,g_{n-k} \otimes I, I
\otimes g_1$,...,$I \otimes g_{n-k} \rangle$ is a $[2n,2k]$-stabilizer
code. When the observables $g_1 \otimes I,...,g_{n-k} \otimes I, I
\otimes g_1$,...,$I \otimes g_{n-k}$ of the state
$\sum_{i \in \{0,1\}^n}\ket{i}\ket{i}$ are measured, 
the resulting state is stabilized by 
$\langle (-1)^{b_1} g_1 \otimes I,..., (-1)^{b_{n-k}}g_{n-k} \otimes I, I
(-1)^{b'_1}\otimes g_1,...,(-1)^{b'_{n-k}}I \otimes g_{n-k} \rangle$,
where $b_1,...,b_{n-k}$ and $b'_1,...,b'_{n-k}$ are measurement results
obtained by Alice and Bob.
If there is no error,
$b_i = b'_i$ holds for all $i$ because of the entanglement. In fact,
the resulting state can be regarded as
$\ket{\beta_{00}}^{\otimes k}$. If there are some errors,
$b_i \neq b'_i$ possibly holds for some positions $i$.
In such a case, if $b_i = 0$ and $b'_i = 1$ for example, then 
the state after the measurement is stabilized by
$\langle ..., g_i \otimes I,...,-I \otimes g_i,... \rangle$.
Alice can inform $b_i$ to Bob so that
they can modify the state to be stabilized by
$\langle ..., g_i \otimes I,...,I \otimes g_i,... \rangle$.
Similarly, by Alice's informing her measurement result to Bob,
they can correct the difference.
Let $C(\langle g_1,...,g_{n-k}\rangle)$ 
corrects $t$ errors. In fact, $\ket{\beta_{00}}^{\otimes k}$ can be
obtained even if the second halves contain at most $t$ errors.

\section{Quantum Key Distribution Protocols}
The word BB84 does not identify one unique
protocol, because there are
several possible methods for
\emph{error correction} and \emph{privacy amplification} after
the quantum communication.
In this paper, since
we formalize Shor and Preskill's proof,
the implementation of BB84 follows their paper \cite{ShorPreskill2000}.
It employs two classical linear codes $C_1, C_2$ that satisfy
$C_2 \subseteq C_1$.
 In this paper, the protocol is slightly modified for simplicity:
 Alice only generates $2n$ qubits. 
This modification causes Bob to store qubits in his side,
 but does not affect the security at all.

\subsection{BB84 (slightly modified)}
{\bf Assumptions}
\begin{itemize}
 \item The length of codeword $n \in \N$ and the error threshold
       $h \in [0..n]$
       are defined and known to Alice, Bob, and Eve.
 \item Classical linear codes $C_1$ and $C_2$ with length $n$ are
       defined and known to Alice, Bob, and Eve. 
       $C_1$ and $C_2$ satisfies 
       $\{0^n\} \subseteq C_2 \subseteq C_1 \subseteq \{0,1\}^n$.
 \item They use quantum and public classical channels.
       Eve can interpolate qubits passing
       through the quantum channel, and listen data passing through the
       public classical channel.
\end{itemize}
{\bf Protocol}\\
We denote the following protocol as $\mathsf{BB84}^{n,h}_{C_1,C_2}$
\begin{enumerate}
 \item Alice generates two random $2n$-bit strings
       $d_1,...,d_{2n}$ and $b_1,...,b_{2n}$.
 \item Alice prepares a $2n$-qubit string $q_1,...,q_{2n}$
       according to the randomness:
       for each $q_i\,(1 \leq i \leq 2n)$, 
       Alice prepares the state $\ket{0}$ if $d_i=0, b_i=0$,
       $\ket{1}$ if $d_i=1, b_i=0$, $\ket{+}$ if $d_i=0,
       b_i=1$, $\ket{-}$ if $d_i=1, b_i=1$.
 \item Alice sends $q_1, ..., q_{2n}$ to Bob through the quantum channel.
 \item Bob receives them and announces Alice
       that fact.
 \item Alice announces $b_1, ..., b_{2n}$ using the classical channel.
 \item For each $i$, Bob measures $q_i$ in $\{\ket{0},\ket{1}\}$
       basis if $b_i=0$; in $\{\ket{+},\ket{-}\}$ basis
       if $b_i=1$. Let the results, which
       are either $0$ or $1$, of the measurement be
       $c_1,...,c_{2n}$. (If no error occurs,
       $d_i = c_i$ for all $i$.)
       Alice randomly chooses $n$ bits from them as
       check bits. Let the indices of the 
       check bits be $k_1,...,k_n$. Alice tells Bob
       $k_1,...,k_n \, (k_1 < ... < k_n)$.
 \item Bob tells Alice $c_{k_1},...,c_{k_n}$
       using the classical channel. Alice counts the
       number of $j$'s with $d_{k_j} \neq c_{k_j}$.
       If the number is greater than the threshold $h$, 
       they abort the protocol. 
 \item Let $x$ be the bitstring with the length $n$ obtained by
       eliminating
       $d_{k_1},...,d_{k_n}$ from $d_1,...d_{2n}$.
       Alice chooses a codeword $u$\,$\in C_1$ at random,
       and announces $u + x$.
 \item (Error correction) 
       Bob lets $y$ be the bitstring with the length $n$
       obtained by eliminating
       $c_{k_1},...,c_{k_n}$ from $c_1,...c_{2n}$, and
       $\tilde w$ be $u + x + y$.
       (Ideally, the condition $x + y = 0$ is expected to hold.)
       Bob performs error correction of $\tilde w$ to obtain $w$.
       If he succeeds to
       correct errors, $w = u$ holds.
 \item (Privacy amplification)
       Alice lets her secret key $k_A$ be $u + C_2$ and
       Bob lets his secret key $k_B$ be $w + C_2$, where
       $u + C_2 := u + \sum_{y \in C_2} y$
\end{enumerate}

BB84 is transformed into the following EDP-based protocol,
which is a modification of the Lo and Chau's protocol
\cite{LoChau1999}.

\subsection{The EDP-based Protocol}
\label{pre:EDPbased}
{\bf Assumptions}
\begin{itemize}
 \item The length of codeword $n \in \N$ and the error threshold 
       $h \in [0..n]$ are
       defined and known to Alice, Bob, and Eve.
 \item Classical linear codes $C_1$ and $C_2$ with length $n$ are
       defined and known to Alice, Bob, and Eve. 
       $C_1$ and $C_2$ satisfies 
       $\{0^n\} \subseteq C_2 \subseteq C_1 \subseteq \{0,1\}^n$.
       Alice and Bob use the CSS code constructed from 
       $C_1$ and $C_2$.
 \item They use quantum, public classical, and a private classical
       channels. Eve can interpolate qubits passing
       through the quantum channel, and listen data passing 
       through the public classical channel.
\end{itemize}
{\bf Protocol}\\
We denote the following protocol as $\mathsf{EDP}^{n,h}_{C_1,C_2}$
\begin{enumerate}
\item Alice prepares $2n$ EPR pairs
      $(\frac{\ket{00}+\ket{11}}{\sqrt{2}})^{\otimes 2n}$ and 
      a random bitstring $b_{1},...,b_{2n}$. 
\item For each $i$, Alice performs Hadamard 
      transformation on the second half of 
      $i$-th pair of $(\frac{\ket{00}+\ket{11}}{\sqrt{2}})^{\otimes 2n}$
      if $b_i = 1$.
      She then sends the second halves of the pairs to Bob.
\item Bob receives the halves and announces Alice that fact.
\item Alice announces $b_1,...,b_{2n}$
      through the public classical channel. For each $i$, Bob
      performs Hadamard transformations to $i$-th half if
      $b_i = 1$. 
\item Alice randomly chooses $n$ pairs from the pairs
      for error check. Let $k_1,...,k_n$ be the positions.
      Alice tells Bob $k_1,...,k_n$.
\item For each $j \in [1..n]$, 
      Alice and Bob measure their halves of $k_j$-th pair in
      $\{\ket{0},\ket{1}\}$ basis,
      and share the measurement results. (If no error occurs, they
      have the same values as the results.)
      If the number of errors is greater than the threshold $h$, 
      they abort the protocol.
\item (Entanglement Distillation)
      Let $H(C_1)$ and $H(C_2^\perp)$ be the parity check matrices of 
      $C_1$ and $C^\perp_2$.
      Alice and Bob measures the observables
      which are the generators described by the matrix
      \[
      \begin{bmatrix}
       H(C_2^\perp) & 0\\
       0 & H(C_1)
      \end{bmatrix}.
      \]
      Alice informs the her measurement results to Bob, and
      Bob corrects errors using them. 
      The measurement results corresponding to $H(C_2^\perp)$ and 
      $H(C_1)$ are sent through the private and public channel respectively.
      If the error correction succeeds,
      they share logical $\ket{\beta_{00}}^{\otimes (k_1 - k_2)}$.
\item Alice and Bob measure their qubits in $\{|0\rangle, |1\rangle\}$
      basis to obtain shared secret keys $k_A$ and $k_B$.
\end{enumerate}

\subsection{Security of Quantum Key Distribution}
We describe here the security criteria introduced in Nielsen and Chuaung's
book \cite[Chapter 12]{NielsenChuang-Kimura2004}.
First, we introduce the notions of \emph{negligible} and
\emph{overwhelming} functions.

\begin{defi}
 A function $f:\mathbb{N}\rightarrow [0, 1]$ is negligible if
 for all polynomial $p(\cdot)$, there exists a natural number
 $N$ such that for all $n \ge N$, $f(n) \le \frac{1}{p(n)}$ holds. 
 A function $f$ is non-negligible if $f$ is not negligible.
\end{defi}

\begin{defi}
 A function $f:\mathbb{N}\rightarrow [0, 1]$ is overwhelming
 if $1 - f$ is negligible, where $(1 - f)(n) = 1 - f(n)$.
\end{defi}
The security criteria is defined as follows.
\begin{defi}
 Let $k_A$, $k_B$, and $k_E$ are random variables
 of Alice's, Bob's, and Eve's keys under the probability 
 distribution after the execution of a QKD protocol.
 The protocol is secure with respect to
 security parameters $s > 0$ and $l > 0$ if
 Alice and Bob have aborted the protocol or
 $\Pr(k_A = k_B)$ is overwhelming with respect to $s$
 and $I(k_A;k_E)$ is negligible with respect to $l$.
\end{defi}

In the definition above, confidentiality of the secret key is 
stated as ``$I(k_A;k_E)$ is negligible'' and correctness of
the keys
is stated as ``$\Pr(k_A = k_B)$ is overwhelming''.
However, Eve can block the protocols by jamming the quantum channel
\cite{Mayers2001}.
If she intercepts all qubits sent from Alice and performs some operations
to change their states and resends them to Bob, then
the errors in check bits will be large and the protocol will be aborted.

\section{Shor and Preskill's Security Proof}
The flow of Shor and Preskill's security proof \cite{ShorPreskill2000}
is as follows. First, BB84 is shown to be equivalent for Eve to the
EDP-based protocol. Concretely, equivalence means that 
the information obtained by Eve who adopts
an arbitrary strategy is equal in the both protocols.
We call this step {\it the transformation step}.
Next, the security of the EDP-based protocol is proven. This implies
the security of BB84.
We call this step {\it the analysis step}.
We explain the discussions of the two steps briefly.

\subsection{Transformation step}
The transformation starts at $\mathsf{EDP}^{n,h}_{C_1,C_2}$.
The first observation is that it does not matter even if
Alice measures her check bits before she sends the other halves
of EPR pairs to Bob. It is the same as her choosing $\ket{0}$ or
$\ket{1}$ at random. Moreover, it does not matter,
even if she first measures the observables
for entanglement distillation for her code bits.
In fact, this is equivalent to sending $k_1 - k_2$ halves
of EPR pairs encoded by the $\CSS_{u,v}(C_1,C_2)$ code for
two random parameters $u,v \in \{0,1\}^n$. 
$u$ and $v$ are determined by the measurement results of the observables 
corresponding to $H(C_2^\perp)$ and $H(C_1)$.
Eventually, instead of measuring Alice's halves, she
can encode a random $k_1 - k_2$ bit string using $\CSS_{u,v}(C_1,C_2)$
with randomly chosen $u$ and $v$.
The following $\mathsf{CSS}^{n,h}_{C_1,C_2}$ QKD protocol is then
obtained, which is an intermediate one in the transformation.

\subsection*{CSS Codes Protocol}
{\bf Assumptions}\\
 The same assumptions as the EDP-based protocol are used.\\
{\bf Protocol}
\begin{enumerate}
\item Alice prepares $k_1 - k_2$ code bits, $u$, and $v$ at random.
      Alice then encodes the code bits using $\CSS_{u,v}(C_1,C_2)$ code.
      Alice next prepares $n$ random check bits
      and a random bitstring $b_{1},...,b_{2n}$. 
      The string of code bits is Alice's secret key.
\item Alice randomly chooses $n$ out of $2n$ positions,
      put check bits in the positions, and put code bits in the
      remaining positions. Let $k_1,...,k_n$ be the check positions.
\item For each $i$, Alice performs Hadamard 
      transformation to the qubits in the positions with
      $b_i = 1$. She then sends the qubits to Bob.
\item Bob receives the halves and announces Alice that fact.
\item Alice announces $b_1,...,b_{2n}$
      through the public classical channel. For each $i$, Bob
      performs Hadamard transformations to $i$-th half if
      $b_i = 1$.
\item Alice tells Bob the positions of check bits $k_1,...,k_n$.
\item For each $j \in [1..n]$, 
      Bobs measure qubits in the position of $k_j$ in
      $\{\ket{0},\ket{1}\}$ basis,
      and share the measurement results.
      If the number of errors is greater than the threshold $h$, 
      they abort the protocol.
\item Alice tells Bob $u$ through the public classical channel and
      $v$ through the secret classical channel.
\item Bob decodes his qubits using $u$ and $v$, and obtains 
      his secret key.
\end{enumerate}
Next, $\mathsf{CSS}^{n,h}_{C_1,C_2}$ is transformed into 
$\mathsf{BB84}^{n,h}_{C_1,C_2}$. When the coded secret key in $C_1$
prepared by Alice is $k'_A \in C_1$, the CSS codeword is
\[
 \frac{1}{\sqrt{|C_2|}}\sum_{y \in C_2}(-1)^{u \cdot y}
 \ket{k'_A+y+v}.
\]
Since Bob only wants to have $k'_A$, the value of $u$ is not necessary.
Indeed, he can measure the state in $\{\ket{0}, \ket{1}\}$ basis to have
the bitstring $k'_A+y_0+v$ for some $y_0 \in C_2$. He subtracts $v$ from
it to have $k'_A+y_0$. As the secret key is the coset $k'_A + C_2$,
the value of $y_0$ does not matter. We then assume
Alice does not send $u$ to Bob. In Bob's view, the state of
the given qubit is the mixed state
\[
 \sum_u (\sum_{y}(-1)^{u \cdot y}
 \ket{k'_A+y+v})
 (\sum_{y}(-1)^{u \cdot y}
 \ket{k'_A+y+v})^\dagger = \sum_{y}\ket{k'_A + y + v}
 \bra{k'_A + y + v}
\]
The state of the right-hand side can be prepared taking $y \in C_2$ at 
random. Let us focus on Bob's view before obtaining $v$. Recall that
the value of $k_A'$ is also taken uniformly.
The state is the mixed state
\[
 \sum_{k'_A \in C_1} \sum_{v \in \{0,1\}^n - C_1} \sum_{y \in C_2} 
  \ket{k'_A + y + v}\bra{k'_A + y + v} = 
 \sum_{v \in \{0,1\}^n - C_1} \sum_{k'' \in C_1}\ket{k''+ v}\bra{k''+ v}.
\]
We observe that $k'' + v$ is uniform random in $\{0,1\}^n$.
Therefore, the related part of the protocol can be modified as follows.
\begin{itemize}
 \item Alice chooses $k'' \in C_1$ and $v \in \{0,1\}^n - C_1$
       at random, and sends $\ket{k''+v}$ to Bob, performing Hadamard
       transformation randomly.
 \item After Hadamard transformation, 
       Bob measures it and obtains classical bitstring $k'' + v + 
       \epsilon$, where $\epsilon$ is the error.
 \item Alice tells $v$ to Bob. Bob obtains $k'' + v + e + v = 
       k'' + e$. As $k''\in C_1$, Bob performs error correction.
       If he succeeds, he obtains $k''$. The shared key is the coset of 
       $k''$ in $C_2$.
\end{itemize}
We eventually obtain $\mathsf{BB84}^{n,h}_{C_1,C_2}$ by the transformation.

\subsection{Analysis step}
A key point is that
Alice and Bob can accurately judge
from the error rate obtained at the step 6
whether the error correction will succeed.
Since check bits are randomly chosen, the numbers of
errors contained in the code bits and check bits are close ($\sharp$).
The numbers of bit and phase flip errors are estimated from the check
bits with $b_i = 0$ and $b_i = 1$ in the step 2.
Shor and Preskill uses a lemma given by
Lo and Chau \cite{LoChau1999}. If Alice and Bob share a 
state having fidelity $F = 1 - 2^{-s}$
with ${\ket{\beta_{00}}}^{\otimes {k_1 - k_2}}$, $I(k_A;k_E)
 \le 2^{-s+\log_2(2(k_1 - k_2)+s+1/\log_e 2) } +
 2^{O(-2s)}$ holds. The fidelity is actually estimated from the
following fact.
\begin{align*}
 &F :=  \bra{\beta_{00}}^{\otimes m} \rho' \ket{\beta_{00}}^{\otimes m}
 \ge \tr{}{\Pi \rho} \mbox{ holds, where}\\
 &\rho \mbox{ and } \rho' \mbox{ are the states of the pairs before 
 and after the error correction, and}\\
 &\Pi \mbox{ is the projector to the space in which 
 errors in code bits are correctable.}
\end{align*}
By ($\sharp$), $\tr{}{\Pi \rho}$ is overwhelming if they have decided
not to abort the protocol.

Formally, the statement of security of BB84 is as follows.
\begin{thm}[Shor-Preskill \cite{ShorPreskill2000}]
 Let $[n,k_1]$-code $C_1$ and $[n,k_2]$-code $C_2$ satisfies
 $C_2 \subseteq C_1$, and $C_1$ and $C_2^\perp$ correct $t$ errors.
 $\mathsf{BB84}^{n, h}_{C_1, C_2}$
 is secure with respect to $n$. Concretely, 
 \begin{align*}
 &\Pr(k_A = k_B) \ge 1 - e^{-\frac{1}{4}\epsilon^{2}n / (\delta - 
 \delta^2)}, \mbox{ where }
 \delta = \frac{t}{n},  \epsilon = \delta - \frac{h}{n}, \mbox{ and }\\
 &I(k_A;k_E) \le 2^{-s+\log_2(2(k_1 - k_2)+s+1/\log_e 2) } +
 2^{O(-2s)}\mbox{ hold}, \mbox{ where } s \mbox{ satisfies}\\
 & s \ge \frac{1}{4}\epsilon^{2}n / (\delta - 
 \delta^2)
 \end{align*}
\end{thm}
\chapter{Automated Verification of Bisimilarity of qCCS configurations}
\label{symqccs}
\section{qCCS}
We introduce the qCCS formal framework presented by Deng and Feng
\cite{DengFeng2012}.
Three data types $\mathit{Bool}, \mathit{Real}$, and $\mathit{Qbt}$ are
used for booleans, real numbers, and qubits, respectively.
Let $\mathit{cVar}$ be a countably infinite set for
classical variables, and $\sfqv$ be a finite
set\footnote{The set of quantum variables is
countably infinite in the original qCCS and each element
represents a qubit, not a qubit string.} $\sfqv$
for quantum variables. $\mathit{cVar}$ and $\sfqv$ are
ranged over by $x, y, z,...$ and $q, r,...$.
For each $q \in \sfqv$, its qubit-length $|q|$ is defined.
A finite sequence of quantum variables
is written $\tilde q$. When $\tilde q = q_1,q_2,...,q_n$, 
$|\tilde q|$ represents $|q_1| + |q_2| + \cdots + |q_n|$.
A sequence $\tilde q = q_1,q_2,...,q_n$ may be regarded as a set
$\{q_1,q_2,...,q_n\}$ implicitly when there is no fear of confusion.
Let $\mathit{Exp}$ be a set of real expressions,
and $\mathit{BExp}$ be a set of boolean expressions.
$\mathit{Exp}$ is ranged over by $e, e',...$.
$\mathit{BExp}$, ranged over by $b, b',...$, is composed of constants
$\mathtt{true}, \mathtt{false}$, atomic expressions $e\,\mathrm
{rel}\,e'$, and logical connectives $\neg, \wedge, \vee$, and
$\rightarrow$, where $\mathrm{rel} \in \{>, <, \ge, \le, =\}$.

Let $\mathit{cChan}$ be a set of
classical channels, and $\mathit{qChan}$ be
a set of quantum channels.
$\mathit{cChan}$ is ranged over by $c,d,...$, and
$\mathit{qChan}$ is ranged over by $\mathsf{c}, \mathsf{d},...$.

For a Hilbert space $\H$, $\dim{(\H)}$ denotes the dimension of $\H$.
For a linear operator $A:\H \rightarrow \H$, $\dim{(A)}$ denotes
$\dim{(\H)}$. For a TPCP map $\E:\D(\H_A) \rightarrow \D(\H_B)$,
$\dom{\E}$ and $\cod{\E}$ denote its domain $\D(\H_A)$ and codomain
$\D(\H_B)$.
For $e \in \mathit{Exp}$ and $b \in \mathit{BExp}$, $\braw{e}$ and
$\braw{b}$ denote their evaluations.

Let $\mathit{Op}$, ranged
over by $\mathit{op}, \mathit{op}_1,...$, 
be a set of identifiers of TPCP maps.
For each $\mathit{op} \in \mathit{Op}$, 
a corresponding TPCP map $\E^{\mathit{op}}$ satisfying
$\dom{\E^{\mathit{op}}} = \cod{\E^{\mathit{op}}}$ is defined.

\subsection{Syntax}
While the original syntax of qCCS allows
recursive definitions of processes, we restricted them for simplicity.
The 
sub-language is still expressive to describe protocols including 
our target QKD protocols. We also eliminated the constructors of 
choice $+$, tau $\tau.P$ and relabeling $P[f]$
because we do not use them.
\begin{defi}
The syntax of qCCS process is given as follows.
\begin{align*}
 \mathit{Proc} \ni P,Q &::= ~\nil ~|~ c?x.P ~|~ c!e.P ~|~
\rcvq{c}{q}.P ~|~\sndq{c}{q}.P \\
 ~|~ \myif{b&}{P}
 ~|~ \mathit{op}[\tilde q].P ~|~ M[\tilde q;x].P~|~P||Q~|~P\backslash L
\end{align*}
where $M$ is an Hermitian operator and 
$L$ is a set of channels.

The set of quantum free variables in a process $P$, denoted
by $\qv{P}$, is inductively defined as follows. 
\begin{align*}
 &\qv{\nil} = \emptyset
 &&\qv{c!e.P} = \qv{P}\\
 &\qv{c?x.P} = \qv{P}
 &&\qv{\sndq{c}{q}.P} = \{q\} \cup \qv{P}\\
 &\qv{\rcvq{c}{q}.P} = \qv{P} - \{q\}
 &&\qv{\myif{b}{P}} = \qv{P}\\
 &\qv{\mathit{op}[\tilde q].P} = \tilde q \cup \qv{P}
 &&\qv{M[\tilde q;x].P} = \tilde q \cup \qv{P}\\
 &\qv{P||Q} = \qv{P} \cup \qv{Q}
 &&\qv{P \backslash L} = \qv{P}
\end{align*}
The constructors $c?x, M[\tilde q;x]$, and $\rcvq{c}{q}$ bind
a classical variable $x$ and a quantum variable $q$.
Bound quantum variables in $P$ is denoted $\mathrm{qbv}(P)$.

For a process to be legal, the following conditions are required.
\begin{enumerate}
\item $\sndq{c}{q}.P \in \Proc$ only if $q \notin \qv{P}$,
\item $P||Q \in \Proc$ only if $\qv{P} \cap \qv{Q} = \emptyset$.
\end{enumerate}
\end{defi}
We explain intuitive meanings of the constructors. The process $\nil$
does nothing. The process $c?x.P$ receives a value of the type 
$\mathit{Real}$ through the channel $c$, binds it to the variable $x$,
and executes $P$. The process
$c!e.P$ sends a value that is obtained evaluating the
expression $e$ through the channel $c$, and executes $P$. The process
$\rcvq{c}{q}.P$
receives a qubit through the channel $\mathsf{c}$, and executes $P$.
The process $\sndq{c}{q}.P$
sends a qubit indicated by the quantum variable $q$ through the channel
$\mathsf{c}$, and executes $P$. The requirement {\it 1} says that
a qubit string, which is a physical object, becomes inaccessible after one
sends it.
The process $\myif{b}{P}$ executes $P$ iff the evaluation of the 
condition $b$ is $\mathit{true}$. The process
$\mathit{op}[\tilde q].P$ performs the corresponding 
TPCP map $\E^{\mathit{op}}$ to
the Hilbert space indicated by $\tilde q$, and executes $P$.
The process $M[\tilde q;x].P$ measures an observable $M$ of
the quantum state indicated by $\tilde q$, stores the result of the
measurement into a classical variable $x$, and executes $P$. 
The process $P||Q$ executes the process $P$ and $Q$ in parallel.
The requirement {\it 2} means that $P$ and $Q$ do not share quantum
systems.
The process $P \backslash L$ executes the process $P$ with private 
channels in $L$. 

For a classical variable $x$ and a value $v$ of
the type $\mathit{Real}$, $P\subst{v}{x}$ is the process obtained
replacing $x$ with $v$.
For quantum variables $q$ and $r$,
$P\subst{r}{q}$ is the process obtained
replacing $q$ with $r$.

\begin{ex} Examples of the processes are as follows,
\label{symqccs:processex}
 \begin{align*}
  &\sndq{c}{r}. M_1[q;x].\nil\\
  &\mathtt{measure}[r]. \sndq{c}{r}. \nil\\
  &M_1[q;x].M_2[r,s;y].\myif{x + y \le
	4}{(c!(x+y).\sndq{c}{r}.\nil||
	c?z.d!z.\rcvq{d}{t}.\nil)}\backslash\{c\}
 \end{align*}
where $x, y \in \mathit{cVar}$, $q,r,s,t \in \sfqv$, $c, d \in
 \mathit{cChan}$, $\mathsf{c}, \mathsf{d} \in
 \mathit{qChan}$. $M_1 = \ket{1}\bra{1}$ and
 $M_2 = \ket{001}\bra{001} + 2(\ket{010}\bra{010}
 + \ket{011}\bra{011})
 + 6\ket{110}\bra{110}$
 with $|q|=|r|=|t|=1$ and $|s|=2$.
 $\mathtt{measure} \in \mathit{Op}$ corresponds a 
 TPCP map
 $\E^{\mathtt{measure}}(\rho) = \ket{0}\bra{0}\rho\ket{0}\bra{0} +
\ket{1}\bra{1}\rho\ket{1}\bra{1}$.
\end{ex}

\subsection{Semantics}
For each $q \in \sfqv$, there assumed to be a corresponding $2^{|q|}$
dimensional
Hilbert space $\mathcal{H}_q$.
For $\tilde q = q_1,q_2,...,q_l$, let $\H_{\tilde q}$ be $\H_{q_1}
\otimes \H_{q_2} \otimes \cdots \otimes \H_{q_l}$.
Let $\mathcal{H}_S=\bigotimes_{q \in {\it S}} \mathcal{H}_q$ for $S
\subseteq \sfqv$ and let $\H = \H_{\sfqv}$
\footnote{As assumed in Chapter \ref{prel}, we identify $H_1 \otimes
H_2$ with $H_2 \otimes H_1$ for Hilbert spaces $H_1$ and $H_2$.
Therefore, the order of $\H_{q}$ with respect to $\otimes$ for $q \in S$
is not significant here.}.
Let $\mathcal{D}(\mathcal{H})$, ranged over by $\rho, \sigma,...$, be the
set of all density operators on $\H$.
For a process to be legal with respect to 
the semantics, the following conditions are additionally required.
\begin{enumerate}
\setcounter{enumi}{2}
 \item $\mathit{op}[\tilde q].P \in \Proc$
       only if $\dom{\E^{\mathit{op}}} =
       \D(\H_{\tilde q})$,
 \item $M[\tilde q;x].P \in \Proc$ only if $\dom{M} = \H_{\tilde q}$,
\end{enumerate}
For $\E^{\mathit{op}}$ with $\dom{\E^{\mathit{op}}} = \H_{\tilde q}$,
let $\E^{\mathit{op}}_{\tilde q}:\D(\H) \rightarrow \D(\H)$ be
$I_{\D(\H_S)} \otimes \E^{\mathit{op}} \otimes I_{\D(\H_T)}$
for $S \cup T =
\sfqv - \tilde q$, where $I_{\D(\H_S)}$ and $I_{\D(\H_T)}$ are identity
operators on
$\D(\H_{S})$ and $\D(\H_{T})$.
Similarly, for an Hermitian operator $M:\H_{\tilde q} \rightarrow
\H_{\tilde q}$ with spectrum decomposition $M = \sum_{i} \lambda_i E^i$,
$E^i_{\tilde q}:\H \rightarrow \H$ is defined as
$I_{\H_S} \otimes E^i_{\tilde q} \otimes I_{\H_T}$.

Let $\Con = \Proc \times \D(\H)$. An
element of $\Con$ is called a configuration.
A configuration consisting of $P \in \Proc$ and $\rho \in \D(\H)$
is written $\con{P}{\rho}$\footnote{In \cite{FengDuanJiYing2007,
Ying2009, FengDuanYing2011, DengFeng2012,
FengDengYing2012}, a configuration is written
$\langle P, \rho \rangle$ using angle brackets. In this thesis,
we write $\con{P}{\rho}$ since we frequently write density operators using
bra-ket notation.}.

\begin{ex} 
Examples of the configurations are as follows,
  \begin{align*}
  &\con{\sndq{c}{r}. M_1[q;x].\nil}{\mathit{EPR}_{q,r}
   \otimes \ket{01}\bra{01}_s \otimes \ket{-}\bra{-}_t}\\
  &\con{\mathtt{measure}[r]. \sndq{c}{r}. \nil}
   {\mathit{EPR}_{q,r} \otimes \ket{00}\bra{00}_s
   \otimes \ket{+}\bra{+}_t}\\
  &\con{M_1[q;x].M_2[r,s;y].\myif{x + y \le
	4}{(c!(x+y).\sndq{c}{r}.\nil||
	c?z.d!z.\rcvq{d}{t}.\nil)}\backslash\{c\}}
   {\\&\ket{+}\bra{+}_q \otimes \ket{+}\bra{+}_r \otimes \ket{10}\bra{10}_s
   \otimes \ket{0}\bra{0}_t}
 \end{align*}
where the sets $\mathit{cVar}$,
$\mathit{qVar}$, $\mathit{Op}$,
and the Hermitian operators $M_1$ and $M_2$
are defined in Example \ref{symqccs:processex}.
\end{ex}

qCCS has a nondeterministic and 
finite-support probabilistic transition system.
The set of all finite-support 
probability distribution on $\mathit{Con}$ is denoted $D(\mathit{Con})$,
which is ranged over by $\mu, \nu,...$.
Namely,
\begin{align*}
D(\Con)=\{\mu &\,|\, \sum_{\con{P}{\rho} \in \Con}
\mu(\con{P}{\rho}) = 1, \mbox{ and for only finitely many }
\con{P}{\rho},\\
&\mbox{we have } \mu(\con{P}{\rho}) > 0 \}. 
\end{align*}
For $\mu \in D(\Con)$, we write
$\mu = \boxplus_{i \in I} p_i \bullet \con{P_i}{\rho_i}$ if
$\mu(\con{P_i}{\rho_i}) = p_i$ and $\sum_{i \in I} p_i = 1$ hold.
For a point distribution, we may simply write $\con{P}{\rho}$ instead of
$1 \bullet \con{P}{\rho}$.
We also write $\mu = \sum_{i \in I} p_i \mu_i$ if $\mu(\con{P}{\rho}) =
\sum_{i \in
I} p_i \mu_i(\con{P}{\rho})$ for all $\con{P}{\rho} \in \Con$ and $\mu_i
\in D(\Con)$.

Let the set of actions $\mathit{Act}_{\tau}$, ranged over by
$\alpha,...$, be $\{c?v, c!v, \sndq{c}{q}, \rcvq{c}{q} \,|\,
c \in \mathit{cChan}, \mathsf{c} \in \mathit{qChan}, 
v \mbox{ is of the type } \mathit{Real}, q \in \sfqv\} \cup {\tau}$.
Channel name $\mathrm{cn}(\alpha)$ in $\alpha$
is defined as $\mathrm{cn}(c?v) =
\mathrm{cn}(c!v) = \{c\}$, $\mathrm{cn}(\sndq{c}{q}) =
\mathrm{cn}(\rcvq{c}{q}) = \{\mathsf{c}\}$, and
$\mathrm{cn}(\tau) = \emptyset$.
Quantum bound variable $\mathrm{qbv}(\alpha)$ in $\alpha$ is defined as
$\mathrm{qbv}(c!v) = \mathrm{qbv}(c?v) =
\mathrm{qbv}(\sndq{c}{q}) = 
\mathrm{qbv}(\tau) = \emptyset$, and
$\mathrm{qbv}(\rcvq{c}{q}) = \{q\}$.

\begin{defi}
The relation of transitions $\rightarrow \subseteq \Con \times
\mathit{Act}_\tau \times D(\Con)$
is defined by the rules in Figure \ref{fig:qccs-ltr}.
We regard $\xrightarrow{\alpha}$ as the subset of
$\Con \times D(\Con)$ for fixed $\alpha$.
The relation $\xrightarrow{\hat \alpha} \subseteq \Con \times D(\Con)$
is defined as follows.
\[
 \xrightarrow{\hat \alpha} := \begin{cases}
			       \xrightarrow{\tau} \cup \{(\con{P}{\rho},
			       1\bullet
			       \con{P}{\rho})\}~~~&(\alpha
			       \mbox{ is } \tau)\\
			       \xrightarrow{\alpha}~~~&(\mbox{otherwise})
			      \end{cases} 
\]
\end{defi}
\begin{figure}
\begin{align*}
&\frac{ v \mbox{ is of the type } \mathit{Real}
}
{
~\con{c?x.P}{\rho} \xrightarrow{c?v}
 \con{P\brac{v/x}}{\rho}~
}(\mbox{C-Inp})~~~~~
\frac{
 \braw{e} = v
}
{
~\con{c!e.P}{\rho} \xrightarrow{c!v}
 \con{P}{\rho}~
}(\mbox{C-Outp})\\
\\
&~\frac{\con{P_1}{\rho} \xrightarrow{c!v} \con{P_1'}{\rho}~~
\con{P_2}{\rho} \xrightarrow{c?v} \con{P_2'}{\rho}~
}
{
\con{P_1||P_2}{\rho} \xrightarrow{\tau} \con{P_1'||P_2'}{\rho}
}(\mbox{C-Com})
\\
\\
&\frac{
~r \notin \qv{P}\backslash{\{q\}}
}
{
~\con{\rcvq{c}{q}.P}{\rho} \xrightarrow{\rcvq{c}{r}}
 \con{P\brac{r/q}}{\rho}~
}(\mbox{Q-Inp})~~~~~
\frac{
}
{
~\con{\sndq{c}{q}.P}{\rho} \xrightarrow{\sndq{c}{q}}
 \con{P}{\rho}~
}(\mbox{Q-Outp})
\\
\\
&\frac{
}
{
~\con{\mathit{op}[\tilde q].P}{\rho} \xrightarrow{\tau}
 \con{P}{\E^{\mathit{op}}_{\tilde q}(\rho)}~
}(\mbox{Oper})~
\frac{
\con{P_1}{\rho} \xrightarrow{\rcvq{c}{r}} \con{P'_1}{\rho}\,
\con{P_2}{\rho} \xrightarrow{\sndq{c}{r}}
 \con{P'_2}{\rho}
}
{
\con{P_1||P_2}{\rho} \xrightarrow{\tau}
 \con{P'_1||P'_2}{\rho}
}(\mbox{Q-Com})
\\
\\
&\frac{
\con{P}{\rho} \xrightarrow{\alpha} \mu,
\braw{b} = \mathit{true}
}
{
~\con{\myif{b}{P}}{\rho} \xrightarrow{\alpha} \mu
}(\mbox{Cho})
~~~~~
\frac{\con{P}{\rho} \xrightarrow{\alpha}
 \boxplus_i p_i \bullet \con{P_i}{\rho_i}
~~\mathrm{cn}(\alpha) \cap
 L = \emptyset}
{\con{P \backslash L}{\rho} \xrightarrow{\alpha} 
 \boxplus_i p_i \bullet \con{P_i \backslash L}{\rho_i} }(\mbox{Res})
\\
\\
&\frac{
\con{P}{\rho} \xrightarrow{\alpha}
\boxplus_i p_i \bullet\con{P'_i}{\rho_i}~\mathrm{qbv}(\alpha) \cap
 \mathrm{qv}(Q) = \emptyset
}
{
~\con{P||Q}{\rho} \xrightarrow{\alpha}
\boxplus_i p_i \bullet\con{P'_i||Q}{\rho_i}
}(\mbox{IntL})\\
\\
&\frac{
\con{P}{\rho} \xrightarrow{\alpha}
\boxplus_i p_i \bullet\con{P'_i}{\rho_i}~\mathrm{qbv}(\alpha) \cap
 \mathrm{qv}(Q) = \emptyset
}
{
~\con{Q||P}{\rho} \xrightarrow{\alpha}
\boxplus_i p_i \bullet\con{Q||P'_i}{\rho_i}
}(\mbox{IntR})\\
\\
&\frac{
}
{
~\con{M[\tilde r;x].P}{\rho} \xrightarrow{\tau}
 \sum_i p_i \bullet \con{P\brac{\lambda_i/x}}
{E^i_{\tilde r} \rho E^i_{\tilde r}/p_i}~
}(\mbox{Meas})\\
&\mbox{where }M\mbox{ has the spectrum decomposition}\\
&M = \sum_i \lambda_i E^i,~{\rm and}
~p_i = \mathrm{tr}(E^i_{\tilde r}\rho)
\end{align*}
\caption{Labelled Transition Rule}
\label{fig:qccs-ltr}
\end{figure}

\begin{figure}[htbp]
 \begin{center}
  \includegraphics[width=110mm]{transex1.png}
 \end{center}
 \caption{Examples of the Transitions (1)}
 \label{fig:transex1}
\end{figure}
\begin{figure}[htbp]
 \begin{center}
  \includegraphics[width=130mm]{transex2.png}
 \end{center}
 \caption{Examples of the Transitions (2)}
 \label{fig:transex2}
\end{figure}

\begin{ex} Examples of the transitions are described in Figure
 \ref{fig:transex1} and \ref{fig:transex2}.
\end{ex}

\subsection{Lifting Relations}
To define weak
bisimilarity, the relations of transitions $\xrightarrow{\alpha},
\xrightarrow{\hat \alpha}
\subseteq \Con \times D(\Con)$ are lifted to subsets of $D(\Con)
\times
D(\Con)$. We introduce the definitions by
Deng and Feng \cite{DengFeng2012} here with some of the useful
properties.
Some definitions are rephrased in equivalent forms.

\begin{defi}
 For $\R \subseteq \Con \times D(\Con)$, its lifted
 relation $\R^\dagger \subseteq D(\Con) \times
 D(\Con)$ is defined as the smallest relation that satisfies
 \begin{itemize}
  \item $\con{P}{\rho} \R \mu$ implies
	$1 \bullet \con{P}{\rho} \R^\dagger \mu$,
	and
  \item (Linearity) $\mu_i \R^\dagger \nu_i$
	for any $i \in I$ implies
	$\sum_{i \in I}p_i \mu_i \R^\dagger
	\sum_{i \in I}p_i \nu_i$ for any $p_i \in [0,1]$ with 
	$\sum_{i \in I} p_i = 1$, where $I$ is a finite
	index set.
 \end{itemize}
\end{defi}

\begin{prop}
 $\mu \R^\dagger \nu$ if and only if there is a 
 finite set $I$ such that
 \begin{itemize}
  \item $\mu = \sum_{i \in I} p_i \con{P_i}{\rho_i}$,
  \item $\nu = \sum_{i \in I} p_i \nu_i$,
  \item $\con{P_i}{\rho_i} \R \nu_i$ for all
	$i \in I$.
 \end{itemize}
\end{prop}

$\R^\dagger$ may be simply written
 $\R$. We have that
$\con{P}{\rho} \xrightarrow{\alpha} \mu$ implies 
$\con{P}{\rho} (\xrightarrow{\alpha})^\dagger \mu$. 
The converse is not true. Indeed, $\con{c!1.\nil||c!1.\nil}{\rho}
 (\xrightarrow{c!1})^\dagger \frac{1}{2}\con{\nil||c!1.\nil}{\rho} \boxplus
\frac{1}{2}\con{c!1.\nil||\nil}{\rho}$ holds but the statement does not
 hold that is obtained replacing $(\xrightarrow{c!1})^\dagger$ to 
$\xrightarrow{c!1}$.

The internal action is then defined. It represents actions that are
not observed by the outsider and is important to define weak bisimilarity.
\begin{defi}
The internal action $\Rightarrow \subseteq
D(\Con) \times D(\Con)$ is defined as $((\xrightarrow{\hat
 \tau})^\dagger)^{\ast}$.
\end{defi}

\begin{prop}
 The relation 
 $\Rightarrow \xrightarrow{\hat \tau} \Rightarrow$ is equal to
 $\Rightarrow$.
 The relation $\Rightarrow \xrightarrow{\hat \alpha} \Rightarrow$
 is linear for all $\alpha$.
\end{prop}


Relations on $\Con$ is also lifted to those on $D(\Con)$.

\begin{defi}
For $\R \subseteq \Con \times \Con$, $\R^\dagger \subseteq D(\Con) \times
D(\Con)$ is defined as
\begin{align*}
 \{(\mu, \nu) \,|\, & \exists I:\mbox{finite index set}.\,\mu = \sum_{i \in
 I}p_i \con{P_i}{\rho_i}, \nu = \sum_{i \in I} p_i \con{Q_i}{\sigma_i},\\
 &\forall i \in I. \, \con{P_i}{\rho_i} \R \con{Q_i}{\sigma_i}\}.
\end{align*}
\end{defi}

\begin{prop}
  For $\R \subseteq \Con \times \Con$, $\R^\dagger \subseteq D(\Con) \times
D(\Con)$ is linear.
\end{prop}

\subsection{Bisimulation}
The strong and weak open bisimulation relation of qCCS configurations
defined by
Deng and Feng \cite{DengFeng2012} is introduced.
Let $\H_{\overline{\qv{P}}}$ be $\bigotimes_{q \in \mathit{\sfqv}-
\qv{P}}\H_q$.

\begin{defi}
A relation $\R \subseteq \Con \times \Con$ is a strong simulation
 if
 $\con{P}{\rho} \R \con{Q}{\sigma}$
implies $\qv{P}=\qv{Q}$, $\tr{\qv{P}}{\rho}=\tr{\qv{Q}}{\sigma}$ and
for all TPCP map $\E$ that acts on $\H_{\overline{\qv{P}}}$,
\begin{itemize}
\item whenever $\con{P}{\E(\rho)} \xrightarrow{\alpha} \mu$,
      there exists $\nu$ such that
      $\con{Q}{\E(\sigma)}
      \xrightarrow{\alpha} \nu$ and $\mu \R^\dagger
      \nu$.
\end{itemize}
$\R$ is a strong bisimulation if $\R$ and $\R^{-1}$ are strong simulations. 
The relation $\strg$ is defined as the largest strong bisimulation.
 If $\con{P}{\rho} \strg
 \con{Q}{\sigma}$, we say they are strongly bisimilar.
\end{defi}

\begin{defi}
A relation $\R \subseteq \Con \times \Con$ is a weak  
simulation if $\con{P}{\rho} \R \con{Q}{\sigma}$
implies $\qv{P}=\qv{Q}$, $\tr{\qv{P}}{\rho}=\tr{\qv{Q}}{\sigma}$ and
for all TPCP map $\E$ that acts on $\H_{\overline{\qv{P}}}$.
\begin{itemize}
\item whenever $\con{P}{\E(\rho)} \xrightarrow{\alpha} \mu$,
      there exists $\nu$ such that
      $\con{Q}{\E(\sigma)}\Rightarrow
      \xrightarrow{\hat \alpha} \Rightarrow \nu$ and $\mu \R^\dagger
      \nu$
\end{itemize}
$\R$ is a weak bisimulation if $\R$ and $\R^{-1}$ are weak simulations.
The relation $\approx$ is defined as
the largest weak bisimulation. If $\con{P}{\rho} \approx
 \con{Q}{\sigma}$, we may simply say they are bisimilar instead of
weakly bisimilar.
\end{defi}

For the bisimulation relations of qCCS configurations, ownership of
quantum variables, which represent physical objects, is significant.
The first condition $\qv{P}=\qv{Q}$ implies $\sfqv - \qv{P} = 
\sfqv - \qv{Q}$, which means the equality of quantum variables
that the outsider possesses. The second condition
$\tr{\qv{P}}{\rho}=\tr{\qv{Q}}{\sigma}$ means the equality of quantum
states that the outsider can access. In the next condition, 
an arbitrary TPCP map $\E$ that acts on $\sfqv - \qv{P}$ is taken.
This allows the outsider to perform an arbitrary operation to quantum
systems that she can access.

\begin{rem}
 There is another way to define probabilistic bisimulation based on
 equivalence classes \cite{Larsen1991, Goubault2007, Davidson-etal2012}.
 When we define by this way, an equivalent notion is in fact defined.
 Concretely, if $\con{P}{\rho} \R \con{Q}{\sigma}$ holds for some 
 strong bisimulation relation $\R$, then 
 \begin{align*}
  & \qv{P} = \qv{Q}, \tr{\qv{P}}{\rho} = \tr{\qv{Q}}{\sigma}, \mbox{ and }\\
  &\forall \E_{\tilde r}:\D(\H_{\sfqv - \qv{P}})\rightarrow \D(\H_{\sfqv
  - \qv{P}}). ~ \forall S \in \Con/\R.\\
  (&\exists \mu.\,(\con{P}{\E_{\tilde r}(\rho)} \xrightarrow{\alpha} \mu \mbox{ and }
  \sum_{S \R X_i} \mu(X_i)=p)\\
  \Leftrightarrow&
 \exists \nu.\,(\con{Q}{\E_{\tilde r}(\sigma)} \xrightarrow{\alpha} \nu \mbox{ and }
 \sum_{S \R Y_i} \nu(Y_i)=p))
 \end{align*}
hold, and conversely.
\end{rem}

Although we consider a little different formal framework,
$\approx$ has the following properties.
Proposition \ref{symqccs:equivalence} and Theorem 
\ref{symqccs:congruence} are proven similarly to the
original \cite{DengFeng2012}.
Theorem \ref{symqccs:coinduction} is proven similarly to
the previous version \cite{FengDuanYing2011}.

\begin{prop}
\label{symqccs:equivalence}
 $\approx$ is an equivalence relation.
\end{prop}
\begin{thm}
\label{symqccs:coinduction}
$\con{P}{\rho} \approx \con{Q}{\sigma}$
if and only if
$\qv{P}=\qv{Q}$,\\
$\tr{\qv{P}}{\rho}=\tr{\qv{Q}}{\sigma}$ and
for all TPCP map $\E$ that acts on $\H_{\overline{\qv{P}}}$,
\begin{enumerate}
 \item whenever $\con{P}{\E(\rho)} \xrightarrow{\alpha} \mu$,
       there exists $\nu$ such that
       $\con{Q}{\E(\sigma)}\Rightarrow
       \xrightarrow{\hat \alpha} \Rightarrow \nu$ and
       $\mu \approx^\dagger \nu$,
 \item whenever $\con{Q}{\E(\sigma)} \xrightarrow{\alpha} \nu$,
       there exists $\mu$ such that
       $\con{P}{\E(\rho)}\Rightarrow
       \xrightarrow{\hat \alpha} \Rightarrow \mu$ and $\mu
       \approx^\dagger \nu$.
\end{enumerate}
\end{thm}

Especially, the next theorem is useful to examine equivalence of
protocols under the existence of other protocols.
\begin{thm}
\label{symqccs:congruence}
If $\con{P}{\rho} \approx \con{Q}{\sigma}$,
\begin{itemize}
 \item $\con{P \backslash L}{\rho} \approx \con{Q \backslash L}{\sigma}$,
       and
 \item $\con{P||R}{\rho} \approx \con{Q||R}{\sigma}$
\end{itemize}
hold for all set of channels $L$ and process $R$ with
$\qv{P} \cap \qv{Q} = \emptyset$.
\end{thm}

We also use the following properties of {\it strong} bisimulation to
prove the soundness of our verifier. 
The properties are proven
similarly to those of weak bisimulation \cite{DengFeng2012}.
\begin{prop}
\label{symqccs:strgequiv} 
$\strg$ is an equivalence relation.
\end{prop}

\begin{prop}
\label{symqccs:strgopclose}
 If $\con{P}{\rho} \strg \con{Q}{\sigma}$, then
$\con{P}{\E(\rho)} \strg \con{Q}{\E(\sigma)}$ for all
TPCP map acting on $\H_{\overline{\qv{P}}}$.
\end{prop}
\begin{thm}
\label{symqccs:strgcong}
 If $\con{P}{\rho} \strg \con{Q}{\sigma}$, then
\begin{itemize}
 \item $\con{P \backslash L}{\rho} \strg \con{Q \backslash L}{\sigma}$,
       and
 \item $\con{P||R}{\rho} \strg \con{Q||R}{\sigma}$
\end{itemize}
hold for all set of channels $L$ and process $R$ with
$\qv{P} \cap \qv{Q} = \emptyset$.
\end{thm}

We call the properties of $\approx$ and $\strg$ {\it congruence}
that are stated by Theorem \ref{symqccs:congruence} and Theorem
\ref{symqccs:strgcong}, although the relations are {\it not} closed
under application of {\it all} constructors.

Finally, we introduce some example and counter-example of bisimulation.
In the following examples, let $\EPR$ be
$(\frac{\ket{00}+\ket{11}}{\sqrt{2}})
(\frac{\ket{00}+\ket{11}}{\sqrt{2}})^\dagger$.
\begin{ex}
 The following two configurations are bisimilar
for an arbitrary process $P(q^A)$ 
satisfying $q^A \in \qv{P(q^A)}$ and quantum state $\rho^E \in
\D(\H_{\overline{\{q^A, q^B\}}})$.

 \begin{enumerate}
  \item $X \defequiv \con{
	\sndq{c}{q^B}. {\tt measure}[q^A].P(q^A)
	}{
	\EPR_{q^A, q^B} \otimes \rho^E
	}$
  \item $Y \defequiv \con{
        {\tt measure}[q^A]. \sndq{c}{q^B}.P(q^A)
	}{
	\EPR_{q^A, q^B} \otimes \rho^E
	}$
 \end{enumerate}
\end{ex}
 A proof of the bisimilarity using Theorem \ref{symqccs:coinduction} is
 as follows.
 For $X = \con{P}{\rho} \in \Con$, let $\E(X)$ be $\con{P}{\E(\rho)}$
 for a TPCP map $\E$.
 For $X$ and $Y$, the conditions of quantum variables and
 partial traces are easily checked. Without loss of generality, 
we can take $I \otimes \E_1$ as an arbitrary
TPCP map acting on $\H_{\over{\{q_A, q_B\}}}$.
Let $\rho'$ be $\E_1(\rho^E)$.
For the transition
 \[
  (I \otimes \E_1)(X)
 \xrightarrow{\sndq{c}{q^B}} 
 \con{
 {\tt measure}[q^A].P(q^A)
 }{
 \EPR_{q^A, q^B} \otimes \rho'
 },
 \]
 we have
 \[
 (I \otimes \E_1)(Y)
 \Rightarrow \xrightarrow{\sndq{c}{q^B}} 
 \con{P(q^A)}{(\frac{1}{2}\ket{00}\bra{00} +
 \frac{1}{2}\ket{11}\bra{11})_{q^A, q^B} \otimes \rho'}.
 \]
 For the transition
 \[
 (I \otimes \E_1)(Y) \xrightarrow{\tau}
 \con{\sndq{c}{q^B}.P(q^A)}
{(\frac{1}{2}\ket{00}\bra{00}+\frac{1}{2}\ket{11}\bra{11})_{q^A, q^B}
 \otimes \rho'},
 \]
 we have $(I \otimes \E_1) (X) \Rightarrow 
(I \otimes \E_1) (X)$. To prove $X \approx Y$, it is sufficient to
show 
\begin{align*}
 \con{{\tt measure}[q^A].P(q^A)}
 {\EPR_{q^A, q^B} \otimes \rho'}
 &\approx
 \con{P(q^A)}{(\frac{1}{2}\ket{00}\bra{00} +
 \frac{1}{2}\ket{11}\bra{11})_{q^A, q^B} \otimes \rho'}~(\sharp)\\
 \mbox{ and }
 (I \otimes \E_1) (X)
 &\approx
 \con{\sndq{c}{q^B}.P(q^A)}{\EPR_{q^A, q^B} \otimes \rho'}~(\flat).
\end{align*}
 For $(\sharp)$, the condition of partial
 trace holds
 because
 \[
  \tr{q^A}{\EPR_{q^A,q^B}} = 
 \tr{q^A}{(\frac{1}{2}\ket{00}\bra{00} +
 \frac{1}{2}\ket{11}\bra{11})_{q^A,q^B}}
 \]
 holds.
 Let $\E_2$ be an arbitrary TPCP map acting on $\H_{\over{\{q^A\}}}$.
 For the transition
\begin{align*}
 &\con{{\tt measure}[q^A].P(q^A)}
 {\E_2(\EPR_{q^A, q^B} \otimes \rho')}\\
 \xrightarrow{\tau}
 &\con{P(q^A)}
 {\E_2(\frac{1}{2}\ket{00}\bra{00} +
 \frac{1}{2}\ket{11}\bra{11})_{q^A, q^B} \otimes \rho')} \defequiv Z,
\end{align*}
 we have
\begin{align*}
 &\con{P(q^A)}
 {\E_2(\frac{1}{2}\ket{00}\bra{00} +
 \frac{1}{2}\ket{11}\bra{11})_{q^A, q^B} \otimes \rho')}
 \Rightarrow Z
\end{align*}
 and $Z \approx^\dagger Z$.
 Next, for an arbitrary TPCP map $\E_3$ acting on $\H_{\over{\{q^A\}}}$
 and transition,
\[
 \con{P(q^A)}
 {\E_3(\frac{1}{2}\ket{00}\bra{00} +
 \frac{1}{2}\ket{11}\bra{11})_{q^A, q^B} \otimes \rho^E)}
 \xrightarrow{\alpha} \mu,\]
 we have the transition
 \[
\con{{\tt measure}[q^A].P(q^A)}
 {\E_3(\EPR_{q^A, q^B} \otimes \rho^E)} \Rightarrow
 (\xrightarrow{\alpha})^\dagger
 \mu
\]
 and $\mu \approx^\dagger \mu$ holds.
 The case when $P(q^A)$ does not perform any transition is
 easily checked.

The condition $(\flat)$ can
 be similarly checked.

\begin{ex}
 The following two configurations are not bisimilar in general.
 \begin{enumerate}
  \item $X \defequiv \con{
	\sndq{c}{q^B}. \ket{1}\bra{1}[q^A;x].P(q^A)
	}{
	\EPR_{q^A, q^B} \otimes \rho^E
	}$
  \item $Y \defequiv \con{
        \ket{1}\bra{1}[q^A;x]. \sndq{c}{q^B}.P(q^A)
	}{
	\EPR_{q^A, q^B} \otimes \rho^E
	}$
 \end{enumerate}
 We prove one of the necessary conditions of bisimulation cannot be
satisfied. Let $P(q^A)$ do not perform any transition.
 For the transition 
\[
  X \xrightarrow{\sndq{c}{q}} \con{
 \ket{1}\bra{1}[q^A;x].P(q^A)
 }{
 \EPR_{q^A, q^B} \otimes \rho^E
 },
\]
 the only possible action by $Y$ that perform $\Rightarrow
 \xrightarrow{\sndq{c}{q}} \Rightarrow$ is
\begin{align*}
 Y \Rightarrow \xrightarrow{\sndq{c}{q}} 
 &\frac{1}{2} \bullet \con{P(q^A)\subst{0}{x}}{
	\ket{00}\bra{00}_{q^A, q^B} \otimes \rho^E
	} \\
 &\boxplus
 \frac{1}{2} \bullet \con{P(q^A)\subst{1}{x}}{
	\ket{11}\bra{11}_{q^A, q^B} \otimes \rho^E
	}
 \defequiv \nu.
\end{align*}
 Therefore, it is necessary for bisimilarity to 
\[
 \con{
 \ket{1}\bra{1}[q^A;x].P(q^A)
 }{
 \EPR_{q^A, q^B} \otimes \rho^E
 }
 \approx \nu.
\]
However, this does not hold because
 $\tr{q^A}{\EPR_{q^A, q^B}} \neq \tr{q^A}{\ket{ii}\bra{ii}_{q^A, q^B}}$
 for $i \in \{0,1\}$.
\end{ex}

\section{Simplification of qCCS's Syntax}
\subsection{Motivation}
\subsubsection{On Formalization of Measurement}
qCCS's syntax has the constructors of TPCP map
application $\mathit{op}[\tilde q].P$
and quantum measurement $M[\tilde q, x].P$.
Since a quantum measurement can also be formalized as a TPCP map,
we have two ways to formalize a measurement.
For example, quantum measurement of the quantum
state $\ket{+}\bra{+}$
is formalized in the following two ways,
where the TPCP map $\E^{\mathtt{measure}}(\rho)$
that corresponds to $\mathtt{measure}[q]$ is $\ket{0}\bra{0}\rho
\ket{0}\bra{0}+\ket{1}\bra{1}\rho
\ket{1}\bra{1}$, $\rho^E \in \H_{\sfqv -\{q\}}$ is an arbitrary quantum
states, and $P(q)$ is an arbitrary process with $q \in \qv{P(q)}$.
 \begin{align*}
  &1.~~\con{\ket{1}\bra{1}[q;x].P(q)}{\ket{+}\bra{+}_q \otimes \rho^E}
  \xrightarrow{\tau}
  \frac{1}{2} \bullet \con{P(q)}
  {\ket{0}\bra{0}_q \otimes \rho^E} \boxplus\\
  &~~~~~~~~~~~~~~~~~~~~~~~~~~~~~~~~~~~~~~~~~~~~~~~~~~
  \frac{1}{2} \bullet \con{P(q)}
  {\ket{1}\bra{1}_q \otimes \rho^E}\\
 &2.~~\con{\mathtt{measure}[q].P(q)}{\ket{+}\bra{+}_q \otimes \rho^E}
  \xrightarrow{\tau}
  \con{P(q)}
  {1/2(\ket{0}\bra{0}+\ket{1}\bra{1})_q}
\end{align*}
Although the two processes apparently formalize the same deed,
they are not bisimilar.

Indeed, the way to formalize a quantum measurement is important in the
formal verification of Shor and Preskill's security proof using qCCS.
In the transformation step, the EDP-based protocol
is converted to the next protocol based on the
fact that nobody outside cannot distinguish the following 
two processes:
\begin{itemize}
 \item[A.] Alice measures a half of an EPR pair
       and then sends the other half to the outside.
 \item[B.] Alice sends a half of an EPR pair to the outside
       and then measures the other half.
\end{itemize}
First, when the measurement is formalized using the
constructor $M[\tilde q, x].P$, the following two configurations are
obtained formalizing the above two, where
$\EPR =
(\frac{\ket{00}+\ket{11}}{\sqrt{2}})
(\frac{\ket{00}+\ket{11}}{\sqrt{2}})^\dagger$,
$\rho^E \in \H_{\sfqv - \{q^A,q^B\}}$
is an arbitrary quantum states, and $Q(q^A)$ is
the successive process. They are not bisimilar.
\begin{itemize}
 \item[A-1.] $\con{
       \sndq{c}{q^B}. \ket{1}\bra{1}[q^A;x].Q(q^A)
       }{\EPR_{q^A, q^B} \otimes \rho^E}$
 \item[B-1.] $\con{
        \ket{1}\bra{1}[q^A;x]. \sndq{c}{q^B}.Q(q^A)
       }{\EPR_{q^A, q^B} \otimes \rho^E}$
\end{itemize}
Second, when the measurement is formalized as 
a TPCP map, the following two configurations are obtained
formalizing the example. They are bisimilar.
\begin{itemize}
 \item[A-2.] $\con{
       \sndq{c}{q^B}. {\tt measure}[q^A].Q(q^A)
       }{
       \EPR_{q^A, q^B} \otimes \rho^E
       }$,
 \item[B-2.] $\con{
        {\tt measure}[q^A]. \sndq{c}{q^B}.Q(q^A)
       }{
       \EPR_{q^A, q^B} \otimes \rho^E
       }$, where
\end{itemize}
\[
\E^{\tt measure}_{q^A}(\rho) = 
\ket{0}\bra{0}_{q^A}\rho\ket{0}\bra{0}_{q^A} + \ket{1}\bra{1}_{q^A}
\rho \ket{1}\bra{1}_{q^A}.
\]

\subsubsection*{Criteria to Select the Way to Formalize}
\label{symqccs:criteria}
By the definition of weak bisimulation relation,
whether probabilistic branches
evoked by $M[\tilde q;x]$ \emph{exist or not} is
significant in transition trees of qCCS configurations.
Therefore, the two different formalization of a
quantum measurement are considered to be 
different from the view of the outsider.
In general, it is unnatural that the outsider recognize
the existence of probabilistic branches {\it without viewing
configuration's different behaviour} that depends on 
the result of the branch.
Hence, we think that if probabilistic branches are evoked, then
the insider must perform a different labelled transition.
We accordingly propose a criteria to select one way from the two to
formalize a quantum measurement.
\begin{itemize}
 \item If transitions with different labels occur according to
       the result of the measurement, the measurement should be formalized
       using the constructor $M[\tilde q;x].P$;
 \item otherwise, it should be formalized
       as a TPCP map, namely, using the constructor $\op{op}{\tilde
       q}.P$.
\end{itemize}
By our criteria, we should formalize the measurement of $\ket{+}\bra{+}$
in the first example as
\begin{align*}
  &1.~~\con{\ket{1}\bra{1}[q;x].P(q)}{\ket{+}\bra{+}_q \otimes \rho^E}
  \xrightarrow{\tau}
  \frac{1}{2} \bullet \con{P(q)}
  {\ket{0}\bra{0}_q \otimes \rho^E} \boxplus\\
  &~~~~~~~~~~~~~~~~~~~~~~~~~~~~~~~~~~~~~~~~~~~~~~~~~~
  \frac{1}{2} \bullet \con{P(q)}
  {\ket{1}\bra{1}_q \otimes \rho^E}
\end{align*}
if $P(q)$ performs different labeled transitions according to the
result. A typical case is when $P \equiv \myif{x =
1}{\sndq{c}{q}.P'}$ holds for some $\mathsf{c}$ and $P'$.
Otherwise, we should formalize it as
\[
 2.~~\con{\mathtt{measure}[q].P(q)}{\ket{+}\bra{+}_q \otimes \rho^E}
  \xrightarrow{\tau}
  \con{P(q)}
  {1/2(\ket{0}\bra{0}+\ket{1}\bra{1})_q}.
\]
Next, let us consider the processes A and B in the second example.
In fact, it is natural that we assume the successive process $Q(q^A)$
does not perform different labeled transitions according to the result of
the measurement of $q^A$. By the definition of the QKD protocols we
consider, which channels Alice and Bob use does not depend on
the result of the measurement of $q^A$.
Hence, we should formalize them as follows by our
criteria.
\begin{enumerate}
 \item[A-2.] $\con{
       \sndq{c}{q^B}. {\tt measure}[q^A].Q(q^A)
       }{
       \EPR_{q^A, q^B} \otimes \rho^E
       }$
 \item[B-2.] $\con{
        {\tt measure}[q^A]. \sndq{c}{q^B}.Q(q^A)
       }{
       \EPR_{q^A, q^B} \otimes \rho^E
       }$.
\end{enumerate}

We simplified the syntax so that
it reflects these criteria.
We eliminated the constructions $M[\tilde q;x].P$ and 
$\myif{b}{P}$.
Instead, we introduced a new syntax $\measure{q}{P}$, where the
observable $\ket{1}\bra{1}$ on the space corresponding to the 
\emph{qubit}\footnote{Note that the meta variable $b$ stands for a boolean
condition in the original syntax but
the meta variable $b$ stands for a quantum variable with length $1$ in
our simplified syntax.}
 $b$ (i.e. $|b|=1$ must be satisfied.) is measured, and if the
result is 1, then it behaves like $P$, else it terminates.
In the new syntax, the qubit $b$ represents the condition for the branch,
which
is supposed to be computed beforehand by some TPCP map. % The final
% condition we impose is that $P$ must contain at least one $\sndq{c}{q}$
% or
% $\rcvq{c}{q} \in \mathcal{P}$ with non-restricted channel $\mathsf{c}$
% to make a visible labeled transition occur.
Besides, we eliminated classical communications for simplicity.
Since classical data can be represented by quantum data,
the elimination of the use of classical data
does not weaken crucially the expressiveness of the language.
Indeed, a distribution where we have the value $0$ with probability $p$
and $1$ with probability $1-p$ is represented by the diagonal 
density operator $p\ket{0}\bra{0} + (1-p)\ket{1}\bra{1}$.

\subsubsection{On Ownership of Quantum System}
By the definition of bisimulation relation, 
if $\con{P}{\rho} \approx \con{Q}{\sigma}$, then
$\tr{\qv{P}}{\rho}=\tr{\qv{Q}}{\sigma}$. 
Intuitively, $\qv{P}$ is considered as the set of quantum variables of
the process
$P$'s own, and $\sfqv - \qv{P}$ is the outsider's. 
$\tr{\qv{P}}{\rho} \in \D(\H_{\sfqv - \qv{P}})$ is considered as the 
quantum states that the outsider can access. For the bisimulation
relation, ownership of quantum variables is significant. In
the transitions of qCCS processes, the ownership changes by
the communication between the process and the outsider by $\sndq{c}{q}$
and
$\rcvq{c}{q}$. However, there are cases where ownership changes
without communication between a process and it's outsider.
\[
 \con{\mathtt{hadamard}[q].\nil}{\ket{0}\bra{0}_{q}
 \otimes \rho^E} \xrightarrow{\tau}
 \con{\nil}{\ket{0}\bra{0}_{q}
 \otimes \rho^E} 
\]
In the above configurations, $\qv{\mathtt{hadamard}[q].\nil} = \{q\}$ and
$\qv{\nil} = \emptyset$. The process loses in the 
transition the ownership of the variable
$q$ without sending it to the outside. We added the restriction
that $\op{op}{\tilde q}.P$ is defined only if $\tilde q \subseteq
\qv{P}$, and changed $\nil$ to a new constructor $\discard{\tilde q}$
that terminates keeping the quantum variables $\tilde q$ inside.
In the original qCCS, process's termination keeping 
quantum variables $\tilde q$ inside is realized for example by
$\myif{\mathtt{false}}{\op{\mathit{I}}{\tilde
q}.\nil}$, where $\mathit{I}$ is the identity operator.

\subsection{Simplified Syntax}
\label{symqccs:modifiedsyntax}
\begin{defi}
The simplified qCCS syntax is given as follows.
\begin{align*}
\mathcal{P} \ni P,Q ::=\, &\discard{\tilde q} ~|~ \sndq{c}{q}.P ~|~
 \rcvq{c}{q}.P
 ~|~\op{\mathit{op}}{\tilde q}.P\\
&~|~ P||Q ~|~ \measure{b}{P} ~|~ P
 \backslash L
\end{align*}
$\qv{\cdot}$ for the simplified syntax is defined as
$\qv{\discard{\tilde q}} = \tilde q$ and \\
$\qv{\measure{b}{P}} = \qv{P}$.
For a process to be legal, the following conditions are required.
 \begin{enumerate}
 \item $\sndq{c}{q}.P \in \mathcal{P}$ iff $q \notin \qv{P}$.
 \item $\rcvq{c}{q}.P \in \mathcal{P}$ iff $q \in \qv{P}$.
 \item $P||Q \in \mathcal{P}$ iff $\qv{P} \cap \qv{Q} = \emptyset$.
 \item $\op{\mathit{op}}{\tilde q}.P \in \mathcal{P}$ iff $\tilde q \subseteq \qv{P}$.
 \item $\measure{b}{P} \in \mathcal{P}$ iff $b \in \qv{P}$.%  and $P$ contains at least one $\sndq{c}{q}$ or
       % $\rcvq{c}{q} \in \mathcal{P}$ with non-restricted channel $\mathsf{c}$.
 \end{enumerate}
\end{defi}

\section{Simplification of Operational Semantics}
We simplified the operational semantics for convenience of implementation.
Instead of considering a probability distribution on configurations,
we consider a probability-weighted quantum states represented by probability-weighted
density operators. For example, instead of considering
\begin{align*}
 \con{\measure{b}{P}}{\rho} \, \xrightarrow{\tau} \,
 & p \bullet \con{P}{\frac{\ket{1}\bra{1}_b\rho\ket{1}\bra{1}_b}{p}} \boxplus \\
 & (1-p) \bullet
 \con{\discard{\qv{P}}}{\frac{\ket{0}\bra{0}_b\rho\ket{0}\bra{0}_b}{1-p}}, \\
\mbox{where } p = \tr{}{\ket{1}\bra{1}_b\rho},
\end{align*}
we consider
\begin{align*}
 \con{\measure{b}{P}}{\rho} \, \xrightarrow{\tau} \,
 & \con{P}{\ket{1}\bra{1}_b\rho\ket{1}\bra{1}_b}\\
 \con{\measure{b}{P}}{\rho} \, \xrightarrow{\tau} \,
 & \con{\discard{\qv{P}}}{\ket{0}\bra{0}_b\rho\ket{0}\bra{0}_b}.
\end{align*}
For this purpose, we define the set of probability-weighted quantum 
states
$\Delta(\H)$, ranged over by $\rho, \sigma,...$, as
$\{p\rho\,|\, p \in [0,1], \rho \in \D(\H)\}$. Any element $\rho \in
\Delta(\H)$ can be converted to an ordinary density operator
 $\frac{\rho}{\tr{}{\rho}} \in \D(\H)$.
If there is no fear of confusion, we may simply say quantum states
instead of
probability-weighted quantum states. We again call a pair of a process and
a probability-weighted quantum state a configuration.
The set of configurations $\C$ is defined as $\mathcal{P} \times
\Delta(\H)$. For a configuration $\con{P}{\rho} \in \C$, $\tr{}{\rho}$
can be interpreted as the probability of reaching it from another
configuration. By this simplification, 
probability was excluded from the transition system. The simplified
transition system is only nondeterministic, not probabilistic.

\begin{defi}
\label{symqccs:simpletrans}
Let $\A_{\tau} := \{\tau\} \cup
 \{\sndq{c}{q},\rcvq{c}{q} \,|\, \mathsf{c} \in \mathit{qChan}, q \in 
\sfqv \}$ be the set of actions.
The transition $\rightarrow \subseteq \C \times \A_{\tau} \times \C$ is 
defined by the rules in Figure \ref{fig:simplified-semantics}.
The transition $\xrightarrow{\hat \alpha}$ is defined as follows.
\[
 \xrightarrow{\hat \alpha} := \begin{cases}
			       \xrightarrow{\tau} \cup \{(\con{P}{\rho},
			       \con{P}{\rho})\}~~~&(\alpha
			       \mbox{ is } \tau)\\
			       \xrightarrow{\alpha}~~~&(\mbox{otherwise})
			      \end{cases} 
\]
\begin{figure}[htbp]
\begin{minipage}{0.5\hsize}
\begin{align*}
%send
&\frac{}
{\con{\sndq{c}{q}.P}{\rho} \xrightarrow{\sndq{c}{q}}
\con{P}{\rho}}
\,\mbox{(In)}\\
\\
%receive
&\frac{r \in \sfqv - \qv{P}}
{\con{\rcvq{c}{q}.P}{\rho} \xrightarrow{\rcvq{c}{r}}
\con{P\subst{r}{q}}{\rho}}
\,\mbox{(Out)}\\
\\
%operator
&\frac{}
{\con{\op{op}{\tilde q}.P}{\rho} \xrightarrow{\tau} 
\con{P}{\E^\mathit{op}_{\tilde q}(\rho)}}
\,\mbox{(Op)}
\end{align*}
\end{minipage}
\begin{minipage}{0.5\hsize}
 \begin{align*}
%restriction
&\frac{
\con{P}{\rho} \xrightarrow{\alpha} 
\con{P'}{\rho'}~~~
\mathrm{cn}(\alpha) \cap L = \emptyset
}
{
\con{P \backslash L}{\rho} \xrightarrow{\alpha} 
\con{P' \backslash L}{\rho'}
}\,\mbox{(Res)}\\
\\
%interleavingR
&\frac{
\con{Q}{\rho} \xrightarrow{\alpha} 
\con{Q'}{\rho'}~~~
}
{
\con{P||Q}{\rho} \xrightarrow{\alpha} 
\con{P||Q'}{\rho'}~~~
}\,\mbox{(Right)}\\
\\
%interleavingL
&\frac{
\con{P}{\rho} \xrightarrow{\alpha} 
\con{P'}{\rho'}
}
{
\con{P||Q}{\rho} \xrightarrow{\alpha} 
\con{P'||Q}{\rho'}~~~
}\,\mbox{(Left)}
 \end{align*}
\end{minipage}

\begin{align*}
  %communication
 &\frac{
 \con{P}{\rho} \xrightarrow{\sndq{c}{q}} 
 \con{P'}{\rho}~~~
 \con{Q}{\rho} \xrightarrow{\rcvq{c}{q}} 
 \con{Q'}{\rho}
 }
 {
 \con{P||Q}{\rho} \xrightarrow{\tau} 
 \con{P'||Q'}{\rho}
 }\,\mbox{(Comm)}\\
\\
%measure
&~\frac{ \ket{1}\bra{1}^{b} \rho \ket{1}\bra{1}^{b} \neq O
}
{
\con{\measure{b}{P}}{\rho} \xrightarrow{\tau}
\con{P}{\ket{1}\bra{1}^{b} \rho \ket{1}\bra{1}^{b}}
}\,\mbox{(Meas1)}\\
\\
&\frac{ \ket{0}\bra{0}^{b} \rho \ket{0}\bra{0}^{b} \neq O
}
{
\con{\measure{b}{P}}{\rho} \xrightarrow{\tau}
\con{\discard{\qv{P}}}{\ket{0}\bra{0}^{b} \rho \ket{0}\bra{0}^{b}}
}\,\mbox{(Meas0)}
 \end{align*}
\caption{Simplified Semantics}
\label{fig:simplified-semantics}
\end{figure}
\end{defi}
We call a new formal framework {\it nondeterministic qCCS}
whose set of configuration is $\C$ and transition rules are defined in
Definition \ref{symqccs:simpletrans}. Although our verifier, 
called \emph{Verifier1}, is
implemented based on nondeterministic qCCS, it verifies the
relation $\approx$ defined by Deng et al \cite{DengFeng2012}.
We call the property of Verifier1 \emph{soundness}, which is
further discussed in Section \ref{symqccs:sndness}.

\section{Automated Verification of Bisimilarity}
Verifier1 handles the configurations $\con{P}{\rho}$ that consist
of a process $P \in \mathcal{P}$ and a symbolic representation
$\rho \in \mathcal{S}$ of a probability-weighted quantum state $\braw{\rho} \in
\Delta(\H)$. Verifier1 obeys the simplified transition rules
defined in Definition \ref{symqccs:simpletrans} except for (Meas$i$)
for $i = 0,1$: it performs the transition even if 
$ \braw{\opsym{\mathtt{proj}i}{b}{\rho}} = 
\ket{i}\bra{i}_b \braw{\rho}\ket{i}\bra{i}_b
= O$ for $\rho \in \mathcal{S}$.

\subsection{Symbolic Representation of Quantum States}
Since cryptographic protocols are defined with security parameters,
the dimensions of quantum states, which are data in protocols,
are unfixed. In our verifier, quantum states are represented as
symbols. First, finite sets
$S_{\mathit{nat}},
S_{\mathit{stat}},
S_{\mathit{op}}$ of symbols respectively
representing
natural numbers, quantum states, and TPCP maps are assumed. 
\begin{itemize}
 \item $S_{\mathit{nat}}$ is a set of symbols representing
       natural numbers. A symbol $\mathtt{1}$ is
       an element of $S_{\mathit{nat}}$.
 \item $S_{\mathit{stat}}$ is a set of symbols representing quantum
       states.
 \item $S_{\mathit{op}}$ is a set of symbols representing TPCP maps.
 \item A function $\mathrm{len}:\sfqv \rightarrow
       S_{\mathit{nat}}$ carries each quantum variable to its 
       qubit-length. $b \in \sfqv$ is called
       a qubit variable if $\mathrm{len}(b) = \mathtt{1}$.
 \item A function $\mathrm{arg}:S_{\mathit{stat}} \cup
       S_{\mathit{op}} \rightarrow \bigcup_{n \in
       \mathbb{N_{+}}}(S_{\mathit{nat}})^n$ carries
       each symbol of quantum states or TPCP map to
       qubit-lengths of its arguments. For example,
       an EPR pair $(\frac{\ket{00}+\ket{11}}{\sqrt{2}})
       (\frac{\ket{00}+\ket{11}}{\sqrt{2}})_{q,r}^\dagger$
       is represented as
       $\texttt{EPR[}q,r\texttt{]}$, where $\mathtt{EPR} \in
       S_{\mathit{stat}}$,
       $\mathrm{len}(q)=\mathrm{len}(r)=\mathtt{1}$, and
       $\mathrm{arg}(\mathtt{EPR})=(\mathtt{1},\mathtt{1})$.
\end{itemize}

After the sets $S_{\mathit{nat}}$,
$S_{\mathit{stat}}$, $S_{\mathit{op}}$, $\mathrm{len}(\cdot)$, and
$\mathrm{arg}(\cdot)$ are defined,
the syntax of symbolic representations of quantum states are
defined.

\begin{defi}
\label{symqccs:symbrep}
The syntax of symbolic representations of quantum states 
are given as  follows,
\begin{align*}
\mathcal{S} \ni \rho, \sigma ::=\,\, &X\mathtt{[}\tilde q\mathtt{]} ~|~
\mathit{op}\mathtt{[}\tilde q\mathtt{](}\rho\mathtt{)} ~|~ \rho \,
  \mathtt{*} \, \sigma
  ~|~\mathtt{proj0}\mathtt{[}b\mathtt{](}\rho\mathtt{)}
~|~\mathtt{proj1}\mathtt{[}b\mathtt{](}\rho\mathtt{)}
~|~\mathtt{Tr}\mathtt{[}\tilde q\mathtt{](}\rho\mathtt{)}
\end{align*}
where $b$ is a qubit variable, $X \in S_{\mathit{stat}}$, and 
$\mathit{op} \in S_{\mathit{op}}$. The set of quantum variables 
$\qv{\rho}$ in symbolic representations $\rho$ are defined as follows.
\begin{align*}
&\qv{X\mathtt{[}\tilde q\mathtt{]}} = \tilde q,
&&\qv{\mathit{op}\mathtt{[}\tilde
 q\mathtt{](}\rho\mathtt{)}}=\qv{\rho},\\
&\qv{\rho \,\mathtt{*}\, \sigma} = 
 \qv{\rho} \cup \qv{\sigma},
&&\qv{\mathtt{proj}i\mathtt{[}b\mathtt{](}\rho\mathtt{)}}
= \qv{\rho},\\
&\qv{\mathtt{Tr}\mathtt{[}\tilde
 q\mathtt{](}\rho\mathtt{)}}=\qv{\rho} - \tilde q.
\end{align*}
For a symbolic representation to be legal, the following conditions are
required.
\begin{enumerate}
 \item $X\mathtt{[}q_1,q_2,...,q_n\mathtt{]} \in \mathcal{S}$ iff 
       $\mathrm{arg}(X)=(\mathrm{len}(q_1),\mathrm{len}(q_2),...,\mathrm{len}(q_
       n))$.
 \item $\mathit{op}\mathtt{[}q_1,q_2,...,q_n\mathtt{](}\rho\mathtt{)}
       \in \mathcal{S}$ iff
       $\mathrm{arg}(op)=(\mathrm{len}(q_1),\mathrm{len}(q_2),...,
       \mathrm{len}(q_n))$ and \\
       $\{q_1,q_2,...,q_n\} \subseteq \qv{\rho}$.
 \item $\rho \, \mathtt{*} \, \sigma \in \mathcal{S}$ iff $\qv{\rho} \cap
       \qv{\sigma} = \emptyset$.
 \item For $i \in \{\mathtt{0}, \mathtt{1}\}$,
       $\mathtt{proj}i\mathtt{[}b\mathtt{](}\rho\mathtt{)} \in \mathcal{S}$ 
       iff $b \in \qv{\rho}$.
 \item $\mathtt{Tr}\mathtt{[}\tilde q \mathtt{](}\rho\mathtt{)} \in
       \mathcal{S}$ iff $\tilde q \subseteq \qv{\rho}$.
\end{enumerate}
If $\rho, \sigma \in \mathcal{S}$ are syntactically equal, we write
 $\rho \equiv \sigma$.
\end{defi}
Intuitive meanings are as follows.
$X\texttt{[}\tilde q\texttt{]}$ means that $\tilde q$'s quantum state
is $X$.
$\mathit{op}\texttt{[}\tilde q\texttt{](}\rho\texttt{)}$
is a quantum state obtained after application of a TPCP map
$\mathit{op}$ that
acts on $\tilde q$, 
to $\rho$.  $\rho \,  \mathtt{*} \, \sigma$ is the tensor product of
states $\rho$ and $\sigma$. $\mathtt{proj}i\texttt{[}b\texttt{](}\rho\texttt{)}$
means the quantum state $\rho$ obtained after application of the projector
$\ket{i}\bra{i}_b$. $\mathtt{Tr}\texttt{[}\tilde
q\texttt{](}\rho\texttt{)}$ means the partial trace of $\rho$ by
$\tilde q$.

Next, we define the formal interpretation of the symbolic
representations.
To define it, interpretations of the elements of 
$S_{\mathit{nat}}$, $S_{\mathit{stat}}$, and $S_{\mathit{op}}$ must be
defined beforehand. The interpretations depend on a set of
security parameters. Let $\Lambda$ be the product of
the ranges of security parameters.
The types of the interpretations $\braw{\cdot}$, which 
depend on $\Lambda$, are as
follows. \footnote{For $f : \Lambda \rightarrow \mathbb{N}_+$, 
$2^{f} : \Lambda \rightarrow \mathbb{N}_+$ represents
$2^{f}(\lambda) = 2^{f(\lambda)}$ for all $\lambda \in \Lambda$.}
\begin{itemize}
 \item For $n \in S_{\mathit{nat}}$, $\braw{n} :
       \Lambda \rightarrow \mathbb{N}_+$. For $q \in \sfqv$ with 
       $\mathrm{len}(q) = n$, $\H_q$ is $2^{\braw{n}}$-dimensional.
 \item For $X \in S_{\mathit{stat}}$ with
       $\mathrm{arg}(X) = (n_1,n_2,...,n_m)$, $\braw{X}$
       is an element of a Hilbert space with dimension
       $2^{\braw{n_1}+
       \braw{n_2} + \cdots\braw{n_m}}$.
 \item For $\mathit{op} \in S_{\mathit{op}}$ with
       $\mathrm{arg}(\mathit{op}) = (n_1,n_2,...,n_m)$,
       $\braw{\mathit{op}}$
       is a TPCP map on Hilbert space with dimension
       $2^{\braw{n_1} +
       \braw{n_2} + \cdots\braw{n_m}}$.
\end{itemize}
The interpretation of the symbolic representations is then 
defined as follows.
\begin{itemize}
 \item $\braw{X\texttt{[}\tilde q\texttt{]}} =
       \braw{X} \in \Delta(\H_{\tilde q})$
 \item $\braw{\mathit{op}\texttt{[}\tilde
       q\texttt{](}\rho\texttt{)}} =
       \braw{\mathit{op}}_{\tilde q}(\braw{\rho})
       \in \Delta(\H_{\qv{\rho}})$
 \item $\braw{\rho \,\mathtt{*} \, \sigma} = 
       \braw{\rho} \otimes
       \braw{\sigma} \in \Delta(\H_{\qv{\rho}} \otimes
       \H_{\qv{\sigma}})$
 \item $\braw{\mathtt{proj0}\texttt{[}b\texttt{](}\rho\texttt{)}}
       =
       \ket{0}\bra{0}_b \braw{\rho} \ket{0}\bra{0}_b \in
       \Delta(\H_{\qv{\rho}})$
 \item $\braw{\mathtt{proj}1\texttt{[}b\texttt{](}\rho\texttt{)}} 
       =
       \ket{1}\bra{1}_b\braw{\rho}\ket{1}\bra{1}_b \in
       \Delta(\H_{\qv{\rho}})$
 \item $\braw{\mathtt{Tr}\texttt{[}\tilde q\texttt{](}\rho\texttt{)}}
       =
       \tr{\tilde q}{\braw{\rho}} \in
       \Delta(\H_{\qv{\rho} - \tilde q})$
\end{itemize}
\begin{ex}
 Let $\Lambda = \mathbb{N}_+$, $\sfqv = \{\mathtt{q}, \mathtt{r}\}$,
 $\mathrm{len}(\mathtt{q})=
 \mathrm{len}(\mathtt{r})=\mathtt{n}$ and let
 $\braw{\mathtt{n}}(k) = k$, $\braw{\mathtt{EPR}} =
 ((\frac{\ket{00}
 + \ket{11}}{\sqrt{2}})
 (\frac{\ket{00}
 + \ket{11}}{\sqrt{2}})^\dagger)^{\otimes k} \defequiv \mathit{EPR}$,
 $\mathrm{arg}(\mathtt{EPR})=(\mathtt{n},\mathtt{n})$, \\
 $\braw{\mathtt{measure}} (\rho) = \ket{0}\bra{0}^{\otimes k}
 \rho \ket{0}\bra{0}^{\otimes k} + \ket{1}\bra{1}^{\otimes k} \rho
 \ket{1}\bra{1}^{\otimes k}$, and
 $\mathrm{arg}(\mathtt{measure})=(\mathtt{n})$.
 The interpretation of the symbolic representation
 $\mathtt{measure[q](EPR[q,r])}$ is calculated as follows.
 \begin{align*}
  \braw{\mathtt{measure[q](EPR[q,r])}}
  &= \braw{\mathtt{measure}}_{\mathtt{q}} (\mathit{EPR}_{\mathtt{q},
  \mathtt{r}})\\
  &= \braw{\mathtt{measure}} \otimes I_{\H_\mathtt{r}}
   (\mathit{EPR}_{\mathtt{q}, \mathtt{r}})\\
  &= \sum_{j \in \{0,1\}^k} \frac{1}{2^k}\ket{jj}\bra{jj}_{\mathtt{q},
  \mathtt{r}}
 \end{align*}
\end{ex}

\subsection{Equality Test of Partial Traces}
\subsubsection{Calculation of Partial Traces}
\label{symqccs:traceoutalgo}
To verify bisimilarity, the equality of partial traces
must be checked.
In fact, partial traces can be to some extent calculated
quite simply focusing on the structure of the expression of the quantum
states. 
For example,
suppose there are 2 qubits named $q$ and $r$, and the 
the outsider has only $r$. When the quantum state of the total system
is $\E_q(\ket{0}\bra{0}_q\otimes\ket{1}\bra{1}_r)$,
the quantum state that the outsider can access is
$\tr{q}{\E_q(\ket{0}\bra{0}_q\otimes\ket{1}\bra{1}_r)}$.
We have $\tr{q}{\E_q(\ket{0}\bra{0}_q\otimes\ket{1}\bra{1}_r)} =
\ket{1}\bra{1}_r$ for an arbitrary operator $\E_q$ that acts on $q$,
simply eliminating the state $\ket{0}\bra{0}_q$ and
the operator $\E_q$.
This is intuitively interpreted that 
the outsider cannot observe what happens to quantum system that
he or she cannot access.
For the symbolic representations, they can be simplified
focusing on occurrence of quantum variables.
Formulating such calculation, we obtain the following rewriting
rules, where interpretation of the right-hand side of $=$ is equal to
that of the left-hand side regardless of $S_{\mathit{nat}},
S_{\mathit{stat}}, S_{\mathit{op}}$, their interpretations, 
and definitions of $\mathrm{len}(\cdot)$ and $\mathrm{arg}(\cdot)$.
\begin{align}
& \ptr{\tilde q}{\rho}=\ptr{\tilde r}{\ptr{\tilde s}{\rho}}
 \mbox{ if } \tilde q = \tilde r \cup \tilde s\\
& \ptr{\tilde q}{\opsym{op}{\tilde
 r}{\rho}} = \ptr{\tilde q}{\rho} \mbox{ if }
 \tilde r \subseteq \tilde q \\
&\ptr{\tilde q}{\opsym{op}{\tilde r}{\rho}}=
 \opsym{op}{\tilde r}{\ptr{\tilde q}{\rho}}~\mathrm{if}~\tilde q \cap
 \tilde r = \emptyset\\
& \ptr{\tilde q}{\opsym{\mathtt{proj}i}{b}{\rho}}=
\opsym{\mathtt{proj}i}{b}{\ptr{\tilde q}{\rho}}
 \mbox{ if } b \notin \tilde q\\
& \ptr{\tilde q}{\opsym{\mathtt{proj}i}{b}{\rho}}=
\ptr{b}{\opsym{\mathtt{proj}i}{b}{\ptr{\tilde q -
 \{b\}}{\rho}}} \mbox{ if } b \in \tilde q\\
&\ptr{\tilde q}{\rho * \rho_{\tilde q} * \sigma} = \rho *
 \sigma, \mbox{ where } \qv{\rho_{\tilde q}} = \tilde q
\end{align}

\subsubsection{Algorithm to Trace Out}
Verifier1 uses the rewriting rules above.
The procedure goes as follows. Let
$\ptr{\tilde q}{\E_1\texttt{[}\tilde q_1\texttt{]}
\texttt{(}...\texttt{(}\E_n\texttt{[}\tilde 
q_n\texttt{](}\rho_1 \texttt{*} ... \texttt{*} \rho_m \texttt{))}...
\texttt{)}}$
be the objective quantum state, where $\E_i$ is either TPCP map symbol or
$\mathtt{proj}j \, (j \in \{\mathtt{0}, \mathtt{1}\})$ for $0 \le i \le n$.
\begin{enumerate}
 \item A set $S_0$ is initialized to be $\tilde q$.
 \item For each $i\,(1 \leq i \leq n)$, $\E_i$'s are
       successively processed.
       \begin{itemize}
	\item If $\E_i$ is a TPCP
	      map symbol and $\tilde q_i \subseteq S_{i - 1}$ holds,
	      then 
	      $\E_i\texttt{[}\tilde q_i\texttt{]}$ is
	      eliminated by rule (3.2), and $S_i$ is defined to be
	      $S_{i - 1}$.
	\item If $\E_i$ is a TPCP map symbol and $\tilde q_i \subseteq
	      S_{i - 1}$ does not hold, then 
	      $S_i$ is defined to be $S_{i - 1}\backslash \tilde
	      q_i$, which is application of rules (3.1) and (3.3).
	\item If $\E_i\texttt{[}\tilde q\texttt{]}$ is
	      $\mathtt{proj}i\texttt{[}b\texttt{]}$
	      and $b \in S_{i - 1}$ holds, then 
	      $S_i$ is defined to be $S_{i - 1}$,
	      which is application of rules (3.4).
	\item If $\E_i\texttt{[}\tilde q\texttt{]}$ is
	      $\mathtt{proj}i\texttt{[}b\texttt{]}$
	      and $b \notin S_{i - 1}$ holds, then 
	      $S_i$ is defined to be $S_{i - 1}\backslash \{b\}$,
	      which is application of rules (3.5).
       \end{itemize}
 \item A set $T$, recording which quantum variables related to
       the state has been deleted by rule (3.6), is initialized
       to $\emptyset$.
       For each $j\,(1 \leq j \leq m)$, if $\qv{\rho_j} \subseteq S_n$,
       then $\rho_j$ is eliminated and $T$ is updated to 
       $T \cup \qv{\rho_j}$.
 \item $\texttt{Tr[}\tilde q \texttt{]}$ is rewritten to
       $\texttt{Tr[}\tilde q - T \texttt{]}$.
\end{enumerate}
\begin{ex}
 A symbolic representation 
\[
\mathtt{Tr[q,r,b](neg[b](proj0[b](hadamard[r](cnot[q,b](EPR[q,s]*R[r]*R[b])))))}
\]
is simplified by the trace out procedure as follows.
\begin{align*}
&\mathtt{Tr[q,r,b](neg[b](proj0[b](hadamard[r](cnot[q,b](EPR[q,s]*R[r]*R[b])))))}\\
=&\mathtt{Tr[b](proj0[b](Tr[q,r](hadamard[r](cnot[q,b](EPR[q,s]*R[r]*R[b])))))}\\
=&\mathtt{Tr[b](proj0[b](Tr[q,r](cnot[q,b](EPR[q,s]*R[r]*R[b]))))}\\
=&\mathtt{Tr[b](proj0[b](Tr[q](cnot[q,b](Tr[r](EPR[q,s]*R[r]*R[b])))))}\\
=&\mathtt{Tr[b](proj0[b](Tr[q](cnot[q,b](EPR[q,s]*R[b]))))}\\
=&\mathtt{Tr[b,q](proj0[b](cnot[q,b](EPR[q,s]*R[b])))}
\end{align*}
\end{ex}

\subsubsection{User-defined Equations}
Verifier1 also takes user-defined equations to verify equality
of quantum states that are symbolically represented. The equations
are of the form $\rho = \sigma$, where $\rho, \sigma \in \mathcal{S}$.
An equation $\rho = \sigma$ is said to be valid if $\braw{\rho} =
\braw{\sigma}$.

There is a restriction on user-defined equations $\rho = \sigma$:
$\rho$ and $\sigma$ must contain the same number of
$\mathtt{proj}i[b]$ for $i = \mathtt{0}, \mathtt{1}$ and for all
$b \in \sfqv$. This makes the proof
of the soundness (Theorem \ref{symqccs:correspondence}) easier.

\subsubsection{Application of User-defined Equations}
If an objective quantum state has a part that
matches to the left-hand side of a user-defined equation,
the part is rewritten to the right-hand side.
To apply a user-defined equation, Verifier1 automatically
solves commutativity of TPCP maps or partial traces
for disjoint sets of quantum variables.
For example, if the objective quantum
state is $\texttt{Tr[q](hadamard[s]}\texttt{(EPR[q,r]*X[s]))}$ and a user
defines an equation $\texttt{Tr[q](EPR[q,r])=Tr[q](PROB[q,r])}~\mathrm{(E1)}$, 
the application procedure goes as follows.
\begin{align*}
&\texttt{Tr[q](hadamard[s](EPR[q,r]*X[s]))}\\
=~&\texttt{hadamards[s](Tr[q](EPR[q,r]*X[s]))} &&\mbox{(by (3.2))}\\
=~&\texttt{hadamard[s](Tr[q](PROB[q,r]*X[s]))} &&\mbox{(by E1)}
\end{align*}
Since trace-out may have become applicable by application of
user-defined rules, trace-out procedure is applied again.
In each opportunity to test the equality of quantum states, 
each user-defined equation is applied only once.
This guarantees whatever rules a user defines, the equality test
terminates.

\subsubsection{Equality Test after the Rewriting}
After the rewriting by
user-defined equations and trace out,
equality of the two symbolic representations are
checked up to exchange of the order of TPCP map application and
tensor product. For example, symbolic expressions
\begin{align*}
&\texttt{Tr[q](hadamard[s](bitflip[r](EPR[q,r]*X[s])))} \mbox{ and}\\
&\texttt{Tr[q](bitflip[r](hadamard[s](X[s]*EPR[q,r])))}
\end{align*}
must be judged to be equal.
Verifier1 automatically judges the equality by
syntactically checking disjointness of TPCP maps'
arguments and by sorting environment symbols by name, which are
concatenated by ``$\texttt{*}$''.

\subsection{Algorithm to Check Bisimilarity}
\label{symqccs:algorithmforbisim}
The recursive procedure to verify bisimilarity is as follows.
It returns either $\mathit{true}$ or $\mathit{false}$.
\begin{enumerate}
\item The procedure takes as input two configurations $\con{P_0}{\rho_0}$,
      $\con{Q_0}{\sigma_0}$ and user-defined equations $\mathit{eqs}$
      on quantum states.
\item If $P_0$ and $Q_0$ can perform any $\tau$-transitions of
      TPCP map applications, they are all performed at this point.
      Let $\con{P}{\rho}$ and $\con{Q}{\sigma}$ be the configurations
      thus obtained.
\item Whether $\qv{P} = \qv{Q}$ is checked. If it does not hold, the
      procedure returns $\mathit{false}$.
\item Whether $\ptrtt{\qv{P}}{\rho} = \ptrtt{\qv{Q}}{\sigma}$ is 
      checked
      using $\mathit{eqs}$.
      The procedure to check equality of quantum states are described
      in the previous subsection.
      If it does not hold, the procedure returns $\mathit{false}$.
\item A new TPCP map symbol $\E\texttt{[}\qv{\rho} - \qv{P}\texttt{]}$
      that stands for an
      arbitrary operation is generated.
\item \begin{enumerate}
       \item For each $\con{P'}{\rho'}$ such that
	     \begin{itemize}
	      \item $\con{P}{\opsym{\E}{\qv{\rho} - \qv{P}}{\rho}}
		    \xrightarrow{\alpha}
		    \con{P'}{\rho'}$ holds, and
	      \item  neither 
		     \begin{itemize}
		      \item $\rho' \equiv 
			    \opsym{\texttt{proj}0}{b}{
			    \opsym{\E}{\qv{\rho} -\qv{P}}{\rho}}$ nor
		      \item $\rho' \equiv 
			    \opsym{\texttt{proj}1}{b}{
			    \opsym{\E}{\qv{\rho} -\qv{P}}{\rho}}$
		     \end{itemize}
		     holds for any $b \in \qv{P}$,
	     \end{itemize}
	     the procedure checks whether there exists
	     $\con{Q'}{\sigma'}$
	     such that
	     \[
	      \con{Q}{\opsym{\E}{\qv{\sigma} -
	     \qv{Q}}{\sigma}} \weak{\hat \alpha}
	     \con{Q'}{\sigma'}
	     \]
	     holds and 
	     the procedure returns $\mathit{true}$ with the input
	     $\con{P'}{\rho'}$,
	     $\con{Q'}{\sigma'}$, and $\mathit{eqs}$. 
       \item For each pair $(\con{P'}{\rho'}, \con{P''}{\rho''})$
	     such that
	     \begin{itemize}
	      \item $\con{P}{\E\texttt{[}\qv{\rho} -
		    \qv{P}\texttt{](}\rho\texttt{)}} \xrightarrow{\tau}
		    \con{P'}{\rho'}$,
	      \item $\rho' \equiv 
		    \opsym{\texttt{proj0}}{b}{
                    \opsym{\E}{\qv{\rho} -\qv{P}}{\rho}}$,
	      \item $\con{P}{\E\texttt{[}\qv{\rho} -
		    \qv{P}\texttt{](}\rho\texttt{)}} \xrightarrow{\tau}
		    \con{P''}{\rho''}$, and
	      \item $\rho'' \equiv 
		    \opsym{\texttt{proj1}}{b}{
                    \opsym{\E}{\qv{\rho} -\qv{P}}{\rho}}$
	     \end{itemize}
	     hold for some $b \in \qv{P}$,
	     the procedure checks whether there exists a pair
	     $(\con{Q'}{\sigma'}, \con{Q''}{\sigma''})$
	     such that
	     \begin{itemize}
	      \item $\con{Q}{\E\texttt{[}\qv{\sigma} -
		    \qv{Q}\texttt{](}\sigma\texttt{)}}
		    \xrightarrow{\tau\ast}
		    \con{\hat Q}{\hat \sigma}$,
	      \item $\con{\hat Q}{\hat \sigma}
		    \xrightarrow{\tau}
		    \con{\hat Q'}
                        {\opsym{\texttt{proj0}}{b}{\hat \sigma}}
		    \xrightarrow{\tau\ast}
		    \con{Q'}{\sigma'}$, and
	      \item $\con{\hat Q}{\hat \sigma}
		    \xrightarrow{\tau}
		    \con{\hat Q''}
                        {\opsym{\texttt{proj1}}{b}{\hat \sigma}}
		    \xrightarrow{\tau\ast}
		    \con{Q''}{\sigma''}$
	     \end{itemize}
	     hold for some $\hat Q$, $\hat Q'$, and $\hat Q''$, and
	     \begin{itemize}
	      \item Verifier1
		    returns $\mathit{true}$ with $(\con{P'}{\rho'},
		    \con{Q'}{\sigma'})$ and $\mathit{eqs}$, and
	      \item Verifier1
		    returns $\mathit{true}$ with $(\con{P''}{\rho''},
		    \con{Q''}{\sigma''})$ and $\mathit{eqs}$.
	     \end{itemize}
      \end{enumerate}
      If there exists, it goes
      to the next step 7. Otherwise, it returns $\mathit{false}$.
\item For each $\con{Q'}{\sigma'}$ such that
      $\con{Q}{\E\texttt{[}\qv{\sigma} -
      \qv{Q}\texttt{](}\sigma\texttt{)}} \xrightarrow{\alpha}
      \con{Q'}{\sigma'}$,
      the procedure checks the symmetric condition of the step 6.
      If there exists, it returns $\mathit{true}$. Otherwise,
      it returns $\mathit{false}$.
\end{enumerate}
The procedure always terminates.
This is because the transition of the processes is finite and
equality check in the step 4 always terminates.

The step 2 prominently decreases the spaces to search.
This is based on the fact that 
$\con{\mathit{op}_1[\tilde q].P||\mathit{op}_2[\tilde
r].Q}{\rho}$ and $ 
\con{P||Q}{\F_{op_2}^{\tilde r}(\E_{op_1}^{\tilde q}(\rho))}$ are
bisimilar, and \\
 $\F_{op_2}^{\tilde r}(\E_{op_1}^{\tilde q}(\rho)) =
\E_{op_1}^{\tilde q}(\F_{op_2}^{\tilde r}(\rho)) $ holds because $\tilde
q \cap \tilde r = \emptyset$ and $\qv{P}\cap\qv{Q}=\emptyset$ hold.

In Section \ref{symqccs:sndness}, when we prove soundness of
Verifier1, we apply the fact that the numbers of $\mathtt{proj}i$'s
are equal in $\rho$ and $\sigma$ if Verifier1 returns $\mathit{true}$
with $\con{P}{\rho}$ and $\con{Q}{\sigma}$. The reason is as follows.
There is the restriction that the both sides of an equation contains
the same number of $\mathtt{proj}i$'s.
This implies that rewriting by an arbitrary user-defined equation
does not change the number of $\mathtt{proj}i$'s in a symbolic
representation. Besides, the trace out procedure does not
eliminate $\mathtt{proj}i$'s. Therefore, the numbers of 
$\mathtt{proj}i$'s must be equal to have passed the equality test
in the step 4. 

We make here a remark about the step 6 (b).
Suppose the following transitions are performed.
\begin{itemize}
 \item $ \con{P}{\E\texttt{[}\qv{\rho} -
		    \qv{P}\texttt{](}\rho\texttt{)}}
       \xrightarrow{\tau}
       \con{P'}{\opsym{\mathtt{proj0}}{b}{\E\texttt{[}\qv{\rho} -
       \qv{P}\texttt{](}\rho\texttt{)}}}$
 \item $ \con{P}{\E\texttt{[}\qv{\rho} -
		    \qv{P}\texttt{](}\rho\texttt{)}}
       \xrightarrow{\tau}
       \con{P''}{\opsym{\mathtt{proj1}}{b}{\E\texttt{[}\qv{\rho} -
       \qv{P}\texttt{](}\rho\texttt{)}}}$
\end{itemize}
To return $\mathit{true}$ with $\con{P}{\rho},
\con{Q}{\sigma}$, and $\mathit{eqs}$, 
it requires the existence of $\con{Q'}{\sigma'}$ and
$\con{Q''}{\sigma''}$ satisfying the conditions mentioned in 6
(b), \emph{even if} $\braw{\opsym{\mathtt{proj0}}{b}{\E\texttt{[}\qv{\rho} -
       \qv{P}\texttt{](}\rho\texttt{)}}} = O$ \emph{holds}, which
means the probability of this transition is $0$.
As for this case, in fact, only the 
existence of $\con{Q''}{\sigma''}$ is necessary 
in the proof of the soundness but that of
$\con{Q'}{\sigma'}$ is not. Therefore, the condition
that Verifier1 returns $\mathit{true}$ with $\con{P}{\rho}$ and
$\con{Q}{\sigma}$ is stronger than the condition that
the two qCCS configurations corresponding to them are bisimilar.

\subsubsection{Memoization}
We also employ a memoization technique. Let $\con{P}{\rho}$ and
$\con{Q}{\sigma}$
have the transitions $\con{P}{\rho} \xrightarrow{\alpha} \con{P'}{\rho'}$ and 
$\con{Q}{\sigma} \xrightarrow{\alpha} \con{Q'}{\sigma'}$, and assume $\con{P'}{\rho'} \approx \con{Q'}{\sigma'}$.
When checking whether $\con{P}{\rho} \approx \con{Q}{\sigma}$ holds, Verifier1 
first checks $\con{Q}{\sigma}$ simulates $\con{P}{\rho}$'s transition. For $\con{P}{\rho}
\xrightarrow{\alpha} \con{P'}{\rho'}$, Verifier1 finds $\con{Q}{\sigma} \xrightarrow{\alpha}
\con{Q'}{\sigma'}$, and then recursively checks $\con{P'}{\rho'} \approx \con{Q'}{\sigma'}$.
Verifier1 then checks $\con{P}{\rho}$ simulates $\con{Q}{\sigma}$'s transition. For the transition $\con{Q}{\sigma} \xrightarrow{\alpha} \con{Q'}{\sigma'}$, it finds
the transition $\con{P}{\rho} \xrightarrow{\alpha} \con{P'}{\rho'}$, and next
checks $\con{Q'}{\sigma'} \approx \con{P'}{\rho'}$. Since $\approx$ is a 
symmetric relation, the last condition has been
already obtained when checking $\con{P'}{\rho'} \approx
\con{Q'}{\sigma'}$.
Verifier1 reuses the result.

\section{Soundness of Verifier1}
\label{symqccs:sndness}
Verifier1 is designed to be sound in the following sense.
If Verifier1 returns $\mathit{true}$ for two configurations
$\con{P}{\rho},
\con{Q}{\sigma} \in \mathcal{P} \times \mathcal{S}$,
and some valid user-defined equations, then
the corresponding two configurations that are elements of $\Con$
are bisimilar in the original qCCS. Our goal is to prove Theorem
 \ref{symqccs:correspondence} that states the correspondence formally.
In this section, we prepare lemmas to prove it.

Suppose $\con{P}{\rho} \xrightarrow{\alpha} \con{P'}{\rho'}$ holds. 
We say the transition $\xrightarrow{\alpha}$ is caused by {\it rule's
name}, where {\it rule's name} is either $\mathrm{(In)},
\mathrm{(Out)}, \mathrm{(Op)}, \mathrm{(Meas1)}$, or
$\mathrm{(Meas0)}$, if the derivation tree begins with the application
of the rule and
$\mathrm{(Comm)}$ rule is not used. If $\mathrm{(Comm)}$ rule is used,
we say the transition is caused by $\mathrm{(Comm)}$.
We first prepare the notation to focus on a part of a process that causes
a transition.

\begin{defi} The evaluation contexts are defined as follows.
 \[
  C[\_] ::= \_ ~~|~~  C[\_]\lVert P ~~|~~ P\lVert C[\_] ~~|~~ C[\_]
 \backslash L \]
\end{defi}
\begin{lem}
\label{symqccs:redex}
 If $\mathcal{P} \times \mathcal{S}
 \ni \con{P}{\rho} \xrightarrow{\sndq{c}{q}} \con{P'}{\rho'}$,
 then
 $P = C[\sndq{c}{q}.P_0]$ and $P' = C[P_0]$
 hold for some process $P_0$ and evaluation context $C[\_]$ that does 
 not restrict $\mathsf{c}$.
\end{lem}

\begin{proof}
 We prove it by induction of the number $n$ of application
 of the transition rules. \\
 (Case 1) Assume $n = 1$. The only rule to derive
 $\xrightarrow{\sndq{c}{q}}$ with one time
 application is $\mathrm{(Out)}$. Therefore, $P = \sndq{c}{q}.P_0$
 holds.\\
 (Case 2) Assume $n > 1$. The last rule applied is either 
 $\mathrm{(Res)}, \mathrm{(Right)}$, or $\mathrm{(Left)}$.
 We prove the case of $\mathrm{(Res)}$ as other cases are similar.
 The last derivation is
\[
 \frac{~\con{P_1}{\rho} \xrightarrow{\sndq{c}{q}} \con{P'_1}{\rho}~~~
  \mathsf{c} \notin L~} 
 {\con{P_1 \backslash L}{\rho} \xrightarrow{\sndq{c}{q}}
  \con{P'_1 \backslash L}{\rho}},
\]
where $P = P_1 \backslash L$ and $P' = P'_1 \backslash L$ 
hold, for some $L$.
By I.H., $P_1 = C[\sndq{c}{q}.P_2]$ and $P'_1 = C[P_2]$ for 
some $C[\_]$ and $P_2$.
We take an evaluation context $C'[\_] = C[\_] \backslash L$.
As $\mathsf{c} \notin L$, $C'[\_]$ does not restrict $\mathsf{c}$.
We then have $P = C'[\sndq{c}{q}.P_2]$ and $P' = C'[P_2]$.
\end{proof}

Lemma \ref{symqccs:redex} is for transitions caused by (Out).
Transitions caused by (In), (Op), (Meas$i$), and (Comm) have
a similar property since the derivations start from those and
proceed by applying (Left), (Right), (Res) rules. The original
qCCS also has a similar property.

Next, we define the correspondence of processes in $\mathcal{P}$ and
those in $\Proc$, where $\mathcal{P}$ is the set of 
the processes of nondeterministic qCCS
and $\Proc$ is the set of the processes of original qCCS.

\begin{defi}
 The function $\mathrm{cnv}: \mathcal{P} \rightarrow \Proc$
 is inductively defined as follows.
 \begin{align*}
  &\cnv{\mathtt{discard[}\tilde q\mathtt{]}} =
  \myif{\mathtt{false}}{I[\tilde q].\nil}\\
  &\cnv{\sndq{c}{q}.P} = \sndq{c}{q}.\cnv{P} \\
  &\cnv{\rcvq{c}{q}.P} = \rcvq{c}{q}.\cnv{P}\\
  &\cnv{\mathit{op}\mathtt{[}\tilde q\mathtt{]}.P} =
  \mathit{op}\mathtt{[}\tilde q\mathtt{]}.\cnv{P} \\
  &\cnv{\measure{b}{P}} = \ket{1}\bra{1}[b;x].\myif{x = 1}{\cnv{P}}\\
  &\cnv{P||Q} = \cnv{P}||\cnv{Q}\\
  &\cnv{P \backslash L} = \cnv{P}\backslash L
 \end{align*}
For an evaluation context $C[\_]$, $\mathrm{cnv}(C)[\_]$ is the
context of original qCCS process obtained applying $\mathrm{cnv}$ to
all processes in $C[\_]$. 
\end{defi}
By the definition, we have the following proposition.
\begin{prop}
\label{symqccs:cnvinj}
 $\qv{P}=\qv{\mathrm{cnv(P)}}$ holds.
 If $\cnv{P} = \cnv{Q}$, then $P = Q$. 
\end{prop}
We then prove lemmas that state correspondence of original and
Verifier1's frameworks.
\begin{lem}
 $\mathcal{P} \times \mathcal{S} \ni
 \con{P}{\rho} \xrightarrow{\alpha} \con{P'}{\rho'}$ and
 $\tr{}{\braw{\rho}} =
 \tr{}{\braw{\rho'}}$ hold, then 
 \[
   \con{\cnv{P}}{\frac{\braw{\rho}}{\tr{}{\braw{\rho}}}} 
 \xrightarrow{\alpha} \mu \mbox{ and }
\mu (\strg)^\dagger 1\bullet
 \con{\cnv{P'}}{\frac{\braw{\rho'}}{\tr{}{\braw{\rho'}}}}
 \]
 hold for some $\mu \in D(\Con)$.
\end{lem}
\begin{proof}
%%%%
 ($\alpha$ is $\sndq{c}{q}$)
 By lemma \ref{symqccs:redex}, $P = C[\sndq{c}{q}.P_0]$ and 
 $P' = C[P_0]$ holds for some evaluation
 context $C[\_]$ and process $P_0$.
 Since 
 \[
 \cnv{P} = \cnv{C[\sndq{c}{q}.P_0]} =
 \cnv{C}[\sndq{c}{q}.\cnv{P_0}] \mbox{ and }
 \cnv{P'} = \cnv{C[P_0]}
 \]
 hold,
 $\con{\cnv{P}}
      {\frac{\braw{\rho}}{\tr{}{\braw{\rho}}}}
 \xrightarrow{\sndq{c}{q}}
 \con{\cnv{P'}}{\frac{\braw{\rho}}{\tr{}{\braw{\rho}}}}$ holds. 
 The conclusion of the lemma holds because identity is a 
strong bisimulation.
 \\
%%%%
 ($\alpha$ is $\rcvq{c}{q}$ or $\tau$ caused by
 $\mathrm{Comm}$) Similar to the above case.\\
%%%%
 ($\alpha$ is $\tau$ caused by $\mathrm{Op}$) Similar to the above
 cases
 except that the quantum state changes. The correctness of 
 the statement is checked observing that $\rho' =
 \op{\mathit{op}}{\tilde r}{\mathtt{(}\rho\mathtt{)}}$ for some
 $\mathit{op}[\tilde q]$and
 $\E^{op}_{\tilde r}(\frac{\braw{\rho}}{\tr{}{\braw{\rho}}}) =
 \frac{\braw{\rho'}}{\tr{}{\braw{\rho'}}}$ holds because $\E^{op}$ is
 trace-preserving.\\
%%%%
 ($\alpha$ is $\tau$ caused by $\mathrm{Meas1}$)
 Similarly to lemma \ref{symqccs:redex}, $P = C[\measure{b}{P_0}]$ and 
 $P' = C[P_0]$ holds for some evaluation
 context $C[\_]$ and process $P_0$.
 By $\tr{}{\braw{\rho}} = \tr{}{\braw{\rho'}} = 
 \tr{}{\braw{\mathtt{proj1}[b](\rho)}} = 
 \tr{}{\ket{1}\bra{1}_b\braw{\rho}}$,
\begin{align*}
 &\con{\cnv{C}[\ket{1}\bra{1}[b;x].\myif{x = 1}{\cnv{P_0}}]}
 {\frac{\braw{\rho}}{\tr{}{\braw{\rho}}}} \xrightarrow{\tau}\\
 &1 \bullet \con{\cnv{C}[\myif{1=1}{\cnv{P_0}}]}
 {\frac{\ket{1}\bra{1}_b\braw{\rho}}
  {\tr{}{\ket{1}\bra{1}_b\braw{\rho}}}}
\end{align*}
 holds.
 Since
 \[
\con{\myif{1=1}{\cnv{P_0}}}
{\frac{\ket{1}\bra{1}_b\braw{\rho}\ket{1}\bra{1}_b}
      {\tr{}{\ket{1}\bra{1}_b\braw{\rho}}}}
 (\strg)^\dagger
\con{\cnv{P_0}}{\frac{\ket{1}\bra{1}_b\braw{\rho}\ket{1}\bra{1}_b}
{\tr{}{\ket{1}\bra{1}_b\braw{\rho}}}}
\]
 and $\ket{1}\bra{1}_b\braw{\rho}\ket{1}\bra{1}_b = \braw{\rho'}$
 hold, 
\[
 1 \bullet \con{\cnv{C}[\myif{1 = 1}{\cnv{P_0}}]}
 {\frac{\ket{1}\bra{1}_b\braw{\rho}\ket{1}\bra{1}_b}
       {\tr{}{\ket{1}\bra{1}_b\braw{\rho}}}} (\strg)^\dagger
1 \bullet \con{\cnv{P'}}{\frac{\braw{\rho'}}{\tr{}{\braw{\rho'}}}}
.
\]
holds by the congruence of $\strg$ (Proposition \ref{symqccs:strgcong}).
\\
%%%%
 ($\alpha$ is $\tau$ caused by $\mathrm{Meas0}$) Similar to the
 case of $\mathrm{Meas1}$.
\end{proof}

\begin{lem}
\label{symqccs:weak}
  $ \mathcal{P} \times \mathcal{S} \ni
\con{P}{\rho} \weak{\hat \alpha} \con{P'}{\rho'}$ and
 $\tr{}{\braw{\rho}} =
 \tr{}{\braw{\rho'}}$ holds, then
 $\con{\cnv{P}}{\frac{\braw{\rho}}{\tr{}{\braw{\rho}}}} 
 \Rightarrow (\xrightarrow{\hat \alpha})^\dagger \Rightarrow 
 \mu$ and 
 $\mu (\strg)^\dagger
 \con{\cnv{P'}}{\frac{\braw{\rho'}}{\tr{}{\braw{\rho'}}}}$
 holds
 for some $\mu \in D(\Con)$.
\end{lem}
\begin{proof}
\label{symqccs:prfofweak}
By assumption, we have
\begin{itemize}
 \item $\con{P}{\rho} \xrightarrow{\tau} \con{P_1}{\rho_1}
       \xrightarrow{\tau} \cdots \xrightarrow{\tau} \con{P_k}{\rho_k}
       \xrightarrow{\hat \alpha} 
       \con{\hat P} {\hat \rho}
       \xrightarrow{\tau} \con{P'_1}{\rho'_1}
       \xrightarrow{\tau} \cdots \xrightarrow{\tau} \con{P'_m}{\rho'_m},$
 \item $\tr{}{\braw{\rho}} = \tr{}{\braw{\rho_1}} = 
       \cdots = \tr{}{\braw{\rho'}}$, and
 \item $\con{P'_m}{\rho'_m} = \con{P'}{\rho'}$
\end{itemize}
for some $k$, $m$,
$P_1$,...,$P_k$,$\hat P$,$P'_1$,...,$P'_m$
$\rho_1$,...,$\rho_k$,$\hat \rho$,$\rho'_1$,...,$\rho'_m$.
By $\con{P}{\rho} \xrightarrow{\tau} \con{P_1}{\rho_1}$ and
$\tr{}{\braw{\rho}} = \tr{}{\braw{\rho_1}}$
and the previous lemma, 
\[
 \con{\cnv{P}}{\frac{\braw{\rho}}{\tr{}{\braw{\rho}}}} 
 \xrightarrow{\tau} \mu_1 (\strg)^\dagger 
 1 \bullet \con{\cnv{P_1}}{\frac{\braw{\rho_1}}{\tr{}{\braw{\rho_1}}}} 
\]
holds for some $\mu_1 \in D(\Con)$. Next, we prove for all 
$i \, (1 \le i \le k-1)$ that 
$\con{P_i}{\rho_i}
\xrightarrow{\tau} \con{P_{i+1}}{\rho_{i+1}}$ and
$\tr{}{\braw{\rho_i}} = \tr{}{\braw{\rho_{i+1}}}$
and
\[
 \con{\cnv{P}}{\frac{\braw{\rho}}{\tr{}{\braw{\rho}}}} 
 \Rightarrow \mu_i (\strg)^\dagger 
 1 \bullet \con{\cnv{P_i}}{\frac{\braw{\rho_i}}{\tr{}{\braw{\rho_i}}}} 
 \mbox{ for some } \mu_i
\]
imply
\[
 \con{\cnv{P}}{\frac{\braw{\rho}}{\tr{}{\braw{\rho}}}} 
 \Rightarrow
 \mu_{i+1} (\strg)^\dagger 
 1 \bullet
 \con{\cnv{P_{i+1}}}{\frac{\braw{\rho_{i+1}}}{\tr{}{\braw{\rho_{i+1}}}}} 
 \mbox{ for some } \mu_{i+1}
 .
\]
By $\con{P_i}{\rho_i}
\xrightarrow{\tau} \con{P_{i+1}}{\rho_{i+1}}$ and
$\tr{}{\braw{\rho_i}} = \tr{}{\braw{\rho_{i+1}}}$ and the previous
lemma, we have 
\[
 \con{\cnv{P_i}}{\frac{\braw{\rho_i}}{\tr{}{\braw{\rho_i}}}} 
 \Rightarrow
 \mu'_{i+1} (\strg)^\dagger 
 1 \bullet
 \con{\cnv{P_{i+1}}}{\frac{\braw{\rho_{i+1}}}{\tr{}{\braw{\rho_{i+1}}}}} 
 \mbox{ for some } \mu'_{i+1}.
\]
By $\mu_i (\strg)^\dagger 1 \bullet
 \con{\cnv{P_i}}{\frac{\braw{\rho_i}}{\tr{}{\braw{\rho_i}}}}$ and
 $\con{\cnv{P_i}}{\frac{\braw{\rho_i}}{\tr{}{\braw{\rho_i}}}}
 \Rightarrow
 \mu'_{i+1}$, $\mu_i \Rightarrow \mu_{i+1}$ and\\
 $\mu_{i+1} (\strg)^\dagger \mu'_{i+1}$ holds
 for some $\mu_{i+1}$. We then have
 \[
 \con{\cnv{P}}{\frac{\braw{\rho}}{\tr{}{\braw{\rho}}}} 
 \Rightarrow
 \mu_i
 \Rightarrow
 \mu_{i+1} (\strg)^\dagger \mu'_{i+1} 
 (\strg)^\dagger
 1 \bullet 
\con{\cnv{P_{i+1}}}
    {\frac{\braw{\rho_{i+1}}}{\tr{}{\braw{\rho_{i+1}}}}},
 \]
 namely,
 \[
 \con{\cnv{P}}{\frac{\braw{\rho}}{\tr{}{\braw{\rho}}}} 
 \Rightarrow
 \mu_{i+1} 
 (\strg)^\dagger
 1 \bullet
 \con{\cnv{P_{i+1}}}
     {\frac{\braw{\rho_{i+1}}}{\tr{}{\braw{\rho_{i+1}}}}}
 \mbox{ for some } \mu_{i+1}.
 \]
 Applying this argument repeatedly, we have
 \[
 \con{\cnv{P}}{\frac{\braw{\rho}}{\tr{}{\braw{\rho}}}} 
 \Rightarrow
 \mu_{k} 
 (\strg)^\dagger
 1 \bullet
 \con{\cnv{P_{k}}}
     {\frac{\braw{\rho_{k}}}{\tr{}{\braw{\rho_{k}}}}}
 \mbox{ for some } \mu_{k}.
 \]
By the similar argument, we have
 \[
 \con{\cnv{P}}{\frac{\braw{\rho}}{\tr{}{\braw{\rho}}}} 
 \Rightarrow
 (\xrightarrow{\hat \alpha})^\dagger
 \hat \mu 
 (\strg)^\dagger
 1 \bullet
 \con{\cnv{\hat P}}
     {\frac{\braw{\hat \rho}}{\tr{}{\braw{\hat \rho}}}}
 \mbox{ for some } \hat \mu.
 \]
Furthermore, we have
 \[
 \con{\cnv{P}}{\frac{\braw{\rho}}{\tr{}{\braw{\rho}}}} 
 \Rightarrow
 (\xrightarrow{\hat \alpha})^\dagger
 \Rightarrow
 \mu
 (\strg)^\dagger
 1 \bullet
 \con{\cnv{P'}}
     {\frac{\braw{\rho'}}{\tr{}{\braw{\rho'}}}}
 \mbox{ for some } \mu.
 \]
\end{proof}

\begin{lem}
 If 
 \begin{itemize}
  \item $ \mathcal{P} \times \mathcal{S} \ni
	\con{P}{\rho} = \con{C[\measure{b}{P_0}]}{\rho}
	\xrightarrow{\tau} 
	\con{C[P_0]}{\mathtt{proj1}[b](\rho)} \defeq \con{P'}{\rho'}$ and
  \item $\con{P}{\rho} 
	\xrightarrow{\tau} 
	\con{C[\mathtt{discard}(\qv{P_0})]}{\mathtt{proj0}[b](\rho)} \defeq
	\con{P''}{\rho''}$
 \end{itemize}
 hold, then 
\begin{itemize}
 \item  $\con{\cnv{P}}{\frac{\braw{\rho}}{\tr{}{\braw{\rho}}}} 
	\xrightarrow{\tau} \mu$ and
 \item  $\mu (\strg)^\dagger
	\frac{\tr{}{\braw{\rho'}}}{\tr{}{\braw{\rho}}} \bullet
	\con{\cnv{P'}}{\frac{\braw{\rho'}}{\tr{}{\braw{\rho'}}}} +
	\frac{\tr{}{\braw{\rho''}}}{\tr{}{\braw{\rho}}} \bullet
	\con{\cnv{P''}}{\frac{\braw{\rho''}}{\tr{}{\braw{\rho''}}}}$
\end{itemize}
 hold for some $\mu \in D(\Con)$.
\end{lem}
\begin{proof}
 We have
 \begin{align*}
 &\con{\cnv{P}}{\frac{\braw{\rho}}{\tr{}{\braw{\rho}}}}=
 \con{\cnv{C}[\ket{1}\bra{1}[b;x].\myif{x =
 1}{\cnv{P_0}}]}{\frac{\braw{\rho}}{\tr{}{\braw{\rho}}}}\\
\xrightarrow{\tau}
 &\frac{\tr{}{\ket{0}\bra{0}_b\braw{\rho}}}{\tr{}{\braw{\rho}}} \bullet
 \con{\cnv{C}[\myif{0 =
 1}{\cnv{P_0}}]}{\frac{\ket{0}\bra{0}_b\braw{\rho}\ket{0}\bra{0}_b}
                    {\tr{}{\ket{0}\bra{0}_b\braw{\rho}}}}\\
 &+\frac{\tr{}{\ket{1}\bra{1}_b\braw{\rho}}}{\tr{}{\braw{\rho}}} \bullet
 \con{\cnv{C}[\myif{1 = 1}{\cnv{P_0}}]}{\frac{\ket{1}\bra{1}_b\braw{\rho}
  \ket{1}\bra{1}_b}{
  \tr{}{\ket{1}\bra{1}_b\braw{\rho}}}} \defeq \mu.
\end{align*}
Besides, we have
\begin{align*}
&\con{\cnv{C}[\myif{0 =1}{\cnv{P_0}}]}
 {\frac{\ket{0}\bra{0}_b\braw{\rho}\ket{0}\bra{0}_b}
                    {\tr{}{\ket{0}\bra{0}_b\braw{\rho}}}}
\\
 \strg &\con{\cnv{C}[\myif{0 = 1}{I[\qv{P_0}].\nil}]}
              {\frac{\ket{0}\bra{0}_b\braw{\rho}\ket{0}\bra{0}_b}
                    {\tr{}{\ket{0}\bra{0}_b\braw{\rho}}}} =
        \con{\cnv{P''}}{\frac{\braw{\rho''}}{\tr{}{\braw{\rho''}}}},
\\
&\mbox{and}\\
&\con{\cnv{C}[\myif{1 = 1}{\cnv{P_0}}]}
     {\frac{\ket{1}\bra{1}_b\braw{\rho}\ket{1}\bra{1}_b}
                    {\tr{}{\ket{1}\bra{1}_b\braw{\rho}}}}
\\
 \strg &\con{\cnv{C}[\cnv{P_0}]}
              {\frac{\ket{1}\bra{1}_b\braw{\rho}\ket{1}\bra{1}_b}
                    {\tr{}{\ket{1}\bra{1}_b\braw{\rho}}}} = 
        \con{\cnv{P'}}{\frac{\braw{\rho'}}{\tr{}{\braw{\rho'}}}}.
\end{align*}
We have the conclusion of the lemma by the linearity of
$(\strg)^\dagger$.
\end{proof}

\begin{lem}
\label{symqccs:prfofweakmeas}
If
\begin{itemize}
 \item  $\mathcal{P} \times \mathcal{S} \ni
	\con{P}{\rho} \xrightarrow{\tau\ast}
	\con{C[\measure{b}{P'}]}{\rho'} 
	\xrightarrow{\tau}
	\con{C[P']}{\mathtt{proj1}[b](\rho')} 
	\xrightarrow{\tau\ast} \con{P_1}{\rho_1}$,
 \item  $\con{C[\measure{b}{P'}]}{\rho'} \xrightarrow{\tau}
	\con{C[\mathtt{discard}(\qv{P'})]}{\mathtt{proj0}[b](\rho')}
	\xrightarrow{\tau\ast}\con{P_0}{\rho_0}$, and
 \item  $\tr{}{\braw{\rho}} = \tr{}{\braw{\rho'}} = 
	\tr{}{\braw{\rho_0}} + \tr{}{\braw{\rho_1}}$
\end{itemize}
hold, then
\begin{itemize}
 \item $\con{\cnv{P}}{\frac{\braw{\rho}}{\tr{}{\braw{\rho}}}}
       \Rightarrow \mu$ and
 \item $\mu (\strg)^\dagger
       \frac{\tr{}{\braw{\rho_0}}}{\tr{}{\braw{\rho}}} \bullet 
       \con{\cnv{P_0}}{\frac{\braw{\rho_0}}{\tr{}{\braw{\rho_0}}}}
       +\frac{\tr{}{\braw{\rho_1}}}{\tr{}{\braw{\rho}}} \bullet
       \con{\cnv{P_1}}{\frac{\braw{\rho_1}}{\tr{}{\braw{\rho_1}}}}$
\end{itemize}
hold for some $\mu \in D(\Con)$.
\end{lem}
\begin{proof}
 By the same argument as the proof of Lemma \ref{symqccs:weak}, we have
\begin{align*}
\con{\cnv{P}}{\frac{\braw{\rho}}{\tr{}{\braw{\rho}}}}
 \Rightarrow 
\mu_0 (\strg)^\dagger
1 \bullet \con{\cnv{C[\measure{b}{P'}]}}{\frac{\braw{\rho'}}{\tr{}{\braw{\rho'}}}}
\end{align*}
for some $\mu_0$.
By the previous lemma, we have
\begin{align*}
 &1 \bullet \con{\cnv{C[\measure{b}{P'}]}}
      {\frac{\braw{\rho'}}{\tr{}{\braw{\rho'}}}} \Rightarrow \mu_1,
\mbox{ and }\\
\mu_1 (\strg)^\dagger
 &\frac{\tr{}{\braw{\mathtt{proj0}[b](\rho')}}}{\tr{}{\braw{\rho}}}
 \bullet 
\con{\cnv{C[\mathtt{discard}(\qv{P'})]}}
{\frac{\braw{\mathtt{proj0}[b](\rho')}}
      {\tr{}{\braw{\mathtt{proj0}[b](\rho')}}}}\\
&+\frac{\tr{}{\braw{\mathtt{proj1}[b](\rho')}}}{\tr{}{\braw{\rho}}}
 \bullet
\con{\cnv{C[P']}}{\frac{\braw{\mathtt{proj1}[b](\rho')}}{\tr{}{\braw{\mathtt{proj1}[b](\rho')}}}}.
\end{align*}
for some $\mu_1$.
By
\begin{align*}
 &\tr{}{\braw{\rho'}} = \tr{}{\braw{\mathtt{proj0}[b](\rho')}} +
 \tr{}{\braw{\mathtt{proj1}[b](\rho')}},\\
 &\tr{}{\braw{\mathtt{proj0}[b](\rho')}} \ge \tr{}{\braw{\rho_0}},
 \mbox{ and}\\
 &\tr{}{\braw{\mathtt{proj1}[b](\rho')}} \ge \tr{}{\braw{\rho_1}},
\end{align*}
we have
$\tr{}{\braw{\mathtt{proj0}[b](\rho')}} = \tr{}{\braw{\rho_0}}$
and
$\tr{}{\braw{\mathtt{proj1}[b](\rho')}} = \tr{}{\braw{\rho_1}}$.
Let $\frac{\tr{}{\braw{\rho_0}}}{\tr{}{\braw{\rho}}} \defeq
p_0$ and 
$\frac{\tr{}{\braw{\rho_1}}}{\tr{}{\braw{\rho}}} \defeq
p_1$.
Now, we apply the same argument as the proof of 
Lemma \ref{symqccs:weak}
to each configuration. We have
\begin{align*}
 X \defeq &\con{\cnv{C[\mathtt{discard}(\qv{P'})]}}
{\frac{\braw{\mathtt{proj0}[b](\rho')}}
      {\tr{}{\braw{\mathtt{proj0}[b](\rho')}}}}
\Rightarrow \mu_2 (\strg)^\dagger 
\con{\cnv{P_0}}{\frac{\braw{\rho_0}}{\tr{}{\braw{\rho_0}}}}\\
\mbox{and}&\\
Y \defeq &\con{\cnv{C[P']}}
 {\frac{\braw{\mathtt{proj1}[b](\rho')}}
 {\tr{}{\braw{\mathtt{proj1}[b](\rho')}}}} 
\Rightarrow \mu_3 (\strg)^\dagger 
\con{\cnv{P_1}}{\frac{\braw{\rho_1}}{\tr{}{\braw{\rho_1}}}}
\end{align*}
for some $\mu_2$ and $\mu_3$. We then have
\begin{align*}
\mu_1 (\strg)^\dagger &
 p_0 \bullet X + p_1 \bullet Y
 \Rightarrow
 p_0 \mu_2 +  p_1 \mu_3 ~~ (\sharp)~~ \mbox{ and }\\
 p_0 \mu_2 +  p_1 \mu_3
 (\strg)^\dagger &
 p_0 \bullet \con{\cnv{P_0}}{\frac{\braw{\rho_0}}{\tr{}{\braw{\rho_0}}}}
+
 p_1 \bullet \con{\cnv{P_1}}{\frac{\braw{\rho_1}}{\tr{}{\braw{\rho_1}}}}.
\end{align*}
By $(\sharp)$, we have $\mu_1 \Rightarrow \mu$ and $\mu (\strg)^\dagger
p_0 \mu_2 + p_1 \mu_3$ for some $\mu$.
Therefore,
 \begin{align*}
  &\con{\cnv{P}}{\frac{\braw{\rho}}{\tr{}{\braw{\rho}}}}
  \Rightarrow \mu_0 \Rightarrow \mu_1 \Rightarrow \mu \mbox{ and}\\
  &\mu  (\strg)^\dagger 
  p_0 \bullet \con{\cnv{P_0}}{\frac{\braw{\rho_0}}
  {\tr{}{\braw{\rho_0}}}}
  +
  p_1 \bullet
  \con{\cnv{P_1}}{\frac{\braw{\rho_1}}{\tr{}{\braw{\rho_1}}}}
 \end{align*}
hold.
\end{proof}

\begin{lem}
 \label{symqccs:csimscon_point}
 If
 $\Proc \times \D(\H) \ni
 \con{\cnv{P}}{\frac{\braw{\rho}}{\tr{}{\braw{\rho}}}}
 \xrightarrow{\alpha}
 \mu$ holds and $\mu$ is a point distribution, then
 $\mathcal{P} \times \mathcal{S} \ni 
 \con{P}{\rho} \xrightarrow{\alpha} \con{P'}{\rho'}$ and
 $\mu (\strg)^\dagger
 1 \bullet \con{\cnv{P'}}{\frac{\braw{\rho'}}{\tr{}{\braw{\rho'}}}}$
 for some $\con{P'}{\rho'}$.
\end{lem}
\begin{proof}
%%
 ($\alpha$ is $\sndq{c}{q}$) There exist a qCCS's
evaluation context $D[\_]$
that does not restrict $\mathsf{c}$ and process $\tilde P \in
\Proc$ such that
$\cnv{P} = D[\sndq{c}{q}.\tilde P]$. There exist
an evaluation context $C[\_]$ of simplified processes not 
restricting $\mathsf{c}$ and
$P_0 \in \mathcal{P}$ such that 
$D[\_] = \cnv{C}[\_]$ and $\tilde P = \cnv{P_0}$.
Therefore, $\cnv{P} = \cnv{C[\sndq{c}{q}.P_0]}$ holds.
By Proposition \ref{symqccs:cnvinj}, $P = C[\sndq{c}{q}.P_0]$ holds.
We then have $\con{P}{\rho} \xrightarrow{\sndq{c}{q}} 
\con{C[P_0]}{\rho}$.
We also have
\[
 \con{\cnv{P}}{\frac{\braw{\rho}}{\tr{}{\braw{\rho}}}}
 \xrightarrow{\sndq{c}{q}}
 \con{D[\tilde P]}{\frac{\braw{\rho}}{\tr{}{\braw{\rho}}}}
 =
 \con{\cnv{C[P_0]}}{\frac{\braw{\rho}}{\tr{}{\braw{\rho}}}}
\]
%%
 ($\alpha$ is $\rcvq{c}{q}$) This case is similar to the above case.\\
%%
 ($\alpha$ is $\tau$ caused by application of a TPCP map or
communication) These cases are
also similar to the case of $\sndq{c}{q}$.\\
%%
($\alpha$ is $\tau$ caused by measurement) We assume 
the result of the measurement is $1$ with probability $1$.
The argument of the other case is similar. We omit the
similar argument as that in the case of $\sndq{c}{q}$.
We have
\begin{itemize}
 \item $P = C[\measure{b}{P_0}]$, 
 \item $\con{\cnv{P}}{\frac{\braw{\rho}}{\tr{}{\braw{\rho}}}}
       \xrightarrow{\tau}
       \con{D[\myif{1 = 1}{\tilde P}]}
       {\frac{\ket{1}\bra{1}_b\braw{\rho}\ket{1}\bra{1}_b}
       {\tr{}{\ket{1}\bra{1}_b\braw{\rho}}}}$
 \item $\con{D[\myif{1 = 1}{\tilde P}]}
       {\frac{\ket{1}\bra{1}_b\braw{\rho}\ket{1}\bra{1}_b}
       {\tr{}{\ket{1}\bra{1}_b\braw{\rho}}}}
       (\strg)^\dagger
       \con{D[\tilde P]}
       {\frac{\ket{1}\bra{1}_b\braw{\rho}\ket{1}\bra{1}_b}
       {\tr{}{\ket{1}\bra{1}_b\braw{\rho}}}}$, \mbox{ and}
 \item $D[\tilde P] = \cnv{C[P_0]}$
\end{itemize}
for some $C[\_], b$, $P_0$, $D[\_]$, and $\tilde P$.
\end{proof}
\begin{lem}
 \label{symqccs:csimscon_notpoint}
 If
 $\con{\cnv{P}}{\frac{\braw{\rho}}{\tr{}{\braw{\rho}}}}
 \xrightarrow{\alpha}
 \mu$ holds and $\mu$ is not a point distribution, then
\begin{enumerate}
 \item $\alpha$ is $\tau$,
 \item $\con{P}{\rho} = \con{C[\measure{b}{P'}]}{\rho}$
       for some evaluation context $C[\_]$, qubit variable $b$, process
       $P'$,
 \item $\con{P}{\rho} 
       \xrightarrow{\tau}
       \con{C[\discard{\qv{P'}}]}{\rho_1}
       \defeq \con{P_1}{\rho_1}$,
 \item $\con{P}{\rho} 
       \xrightarrow{\tau}
       \con{C[P']}{\rho_2} \defeq \con{P_2}{\rho_2}$, and
 \item $\mu (\strg)^\dagger
       \frac{\tr{}{\braw{\rho_1}}}{\tr{}{\braw{\rho}}}
       \bullet 
       \con{\cnv{P_1}}
           {\frac{\braw{\rho_1}}{\tr{}{\braw{\rho_1}}}} + 
       \frac{\tr{}{\braw{\rho_2}}}{\tr{}{\braw{\rho}}}
       \bullet 
       \con{\cnv{P_2}}
           {\frac{\braw{\rho_2}}{\tr{}{\braw{\rho_2}}}}$
\end{enumerate}
\end{lem}
\begin{proof}
 Since $\mu$ is not a point distribution, 
 the transition is caused by measurement (1).
 Therefore, we have 
 $P = C[\measure{b}{P'}]$ for some evaluation context
 $C[\_]$, qubit variable $b$, and process $P'$ (2).
 We have (3) and (4) immediately.
 We also have
\begin{align*}
 \mu =
 &\frac{\tr{}{\braw{\rho_1}}}{\tr{}{\braw{\rho}}}
 \bullet \con{\cnv{C}[\myif{0 = 1}{\cnv{P'}}]}
              {\frac{\braw{\rho_1}}{\tr{}{\braw{\rho_1}}}}\\
 + 
 &\frac{\tr{}{\braw{\rho_1}}}{\tr{}{\braw{\rho}}}
 \bullet \con{\cnv{C}[\myif{1 = 1}{\cnv{P'}}]}
             {\frac{\braw{\rho_2}}{\tr{}{\braw{\rho_2}}}}\\
(\strg)^\dagger &
\frac{\tr{}{\braw{\rho_1}}}{\tr{}{\braw{\rho}}}
 \bullet \con{\cnv{P_1}}
              {\frac{\braw{\rho_1}}{\tr{}{\braw{\rho_1}}}}\\
 + 
 &\frac{\tr{}{\braw{\rho_2}}}{\tr{}{\braw{\rho}}}
 \bullet \con{\cnv{P_2}}
             {\frac{\braw{\rho_2}}{\tr{}{\braw{\rho_2}}}}.
\end{align*}
\end{proof}

\subsection{The Correspondence}
The following theorem states the soundness of Verifier1.
\begin{thm}
\label{symqccs:correspondence}
 If Verifier1 returns $\mathit{true}$ with the input $\con{P}{\rho},
 \con{Q}{\sigma} \in \mathcal{P} \times \mathcal{S}$ satisfying
 $\tr{}{\braw{\rho}}=\tr{}{\braw{\sigma}}=1$,
 and a set of valid equations $\mathit{eqs}$, then
 $\con{\cnv{P}}{\braw{\rho}}
 \approx \con{\cnv{Q}}{\braw{\sigma}}$ holds.
\end{thm}
\begin{proof}
We assume 
that Verifier1 uses a simplified algorithm without the step 2 in
which TPCP maps are performed without any transition.
The theorem is still proven to hold with the step 2 extending the
proof.

 Assume all equations in $\mathit{eqs}$ are valid. We define
\[
 \R_{\mathit{eqs}} :=
 \Set{(X, Y) |
 \begin{array}{l}
  \text{$X \strg \con{\cnv{P}}{\frac{\braw{\rho}}{\tr{}{\braw{\rho}}}},
   Y \strg \con{\cnv{Q}}{\frac{\braw{\sigma}}{\tr{}{\braw{\sigma}}}}$,
   and}\\
  \text{Verifier1 returns $\mathit{true}$ with 
 $\con{P}{\rho}$ and $\con{Q}{\sigma}$ using $\mathit{eqs}$.}
 \end{array}
 }.
\]
We then have $\con{\cnv{P}
 }{\braw{\rho}} \R_{\mathit{eqs}} \con{\cnv{Q}}{\braw{\sigma}}$ if
Verifier1 returns $\mathit{true}$ with the input $\con{P}{\rho},
 \con{Q}{\sigma} \in \C$, $\tr{}{\braw{\rho}}=\tr{}{\braw{\sigma}}=1$,
 and $\mathit{eqs}$. It is sufficient to show that $\R_{\mathit{eqs}}$
 is a weak bisimulation relation. Let
 $(X, Y)$ be
 an arbitrary element in $\R_{\mathit{eqs}}$. The condition of
 quantum variable is satisfied by the definition of $\cnv{}$.
 The condition of partial trace is checked as follows.
 \begin{align*}
  \tr{\qv{\cnv{P}}}{\frac{\braw{\rho}}{\tr{}{\braw{\rho}}}} 
  &= 
  \frac{1}{\tr{}{\braw{\rho}}}\braw{\mathtt{Tr}[\qv{P}](\rho)}
  && (\mbox{by the definition of } \braw{})\\
  &= \frac{1}{\tr{}{\braw{\sigma}}}\braw{\mathtt{Tr}[\qv{Q}](\sigma)}
  && (\mbox{by the validity of } \mathit{eqs})\\
  &= \tr{\qv{\cnv{Q}}}{\frac{\braw{\sigma}}{\tr{}{\braw{\sigma}}}} 
  && (\mbox{by the definition of } \braw{})
 \end{align*}
Next, we check the condition of simulation. Let $\E_{\tilde r}$ be
an arbitrary TPCP map acting on $\tilde r \subseteq \sfqv - \qv{P}$.
Assume $X \xrightarrow{\alpha} \mu$.
By strong bisimulation,
$\con{\cnv{P}}{\E_{\tilde r}(\frac{\braw{\rho}}{\tr{}{\braw{\rho}}})}
\xrightarrow{\alpha} \mu'$ and $\mu (\strg)^\dagger \mu'$ hold.
\\
{\bf (Case 1)} Assume $\mu'$ is a point distribution.
 By lemma \ref{symqccs:csimscon_point},
 $\con{P}{\op{\bar \E}{\tilde r}(\rho)} \xrightarrow{\alpha}
 \con{P'}{\rho'}$ interpreting $\bar \E$ as $\E_{\tilde r}$, 
 and $\mu' (\strg)^\dagger
 \con{\cnv{P'}}{\frac{\braw{\rho'}}{\tr{}{\braw{\rho'}}}}$
 hold for some $\con{P'}{\rho'}$. 
Since Verifier1 returns $\mathit{true}$, 
 there exists $\con{Q'}{\sigma'}$ such that 
$\con{Q}{\op{\bar \E}{\tilde
 r}(\sigma)} \weak{\hat \alpha} \con{Q'}{\sigma'}$ holds and
 Verifier1
 returns $\mathit{true}$ with $\con{P'}{\rho'}$. This implies 
 $\tr{}{\braw{\sigma'}} = \tr{}{\braw{\rho'}} = 
\tr{}{\braw{\op{\bar \E}{\tilde r}(\rho)}} = \tr{}{\braw{\bar \E[\tilde
 r](\sigma)}}$. Now, we can apply Lemma \ref{symqccs:weak}.
 We have 
\[
 \con{\cnv{Q}}{\E_{\tilde r}
               (\frac{\braw{\sigma}}{\tr{}{\braw{\sigma}}})} 
 \Rightarrow
 \xrightarrow{\hat \alpha} 
 \Rightarrow 
 \nu'
 (\strg)^\dagger
 1 \bullet
 \con{\cnv{Q'}}{\frac{\braw{\sigma'}}{\tr{}{\braw{\sigma'}}}}
\]
 for some $\nu$.
 Next, by strong bisimulation,
 $Y 
  \Rightarrow 
  \xrightarrow{\hat \alpha} 
  \Rightarrow 
  \nu
 $ and
 $\nu' (\strg)^\dagger \nu$.
 We then have
\begin{align*}
&\mu 
 (\strg)^\dagger 
 \con{\cnv{P'}}{\frac{\braw{\rho'}}{\tr{}{\braw{\rho'}}}}
 \mbox{  and  }
\nu 
 (\strg)^\dagger 
 \con{\cnv{Q'}}{\frac{\braw{\sigma'}}{\tr{}{\braw{\sigma'}}}}.
\end{align*}
By the definition of $(\cdot)^\dagger$, $\mu$ can be
written as $\sum_i p_i X'_i$ and
$X'_i \strg  \con{\cnv{P'}}{\frac{\braw{\rho'}}{\tr{}{\braw{\rho'}}}}$
for all $i$.
Similarly, $\nu$ can be written as $\sum_j q_j Y'_j$ and
$Y'_j \strg  \con{\cnv{Q'}}
                 {\frac{\braw{\sigma'}}{\tr{}{\braw{\sigma'}}}}$
for all $j$.
Since Verifier1 returns $\mathit{true}$ with $\con{P'}{\rho'}$ and 
$\con{Q'}{\sigma'}$, $X_i \R_{\mathit{eqs}} Y_j$ for all $i, j$.
 holds. Therefore, $\sum_{i,j} p_i q_j X_i \R_{\mathit{eqs}}^\dagger
 \sum_{i,j} p_i q_j Y_j$ holds. This is equivalent to
 $\mu \R_{\mathit{eqs}}^\dagger \nu$.
 \\
{\bf (Case 2)}
 Assume $\mu'$ is not a point distribution. 
 By lemma \ref{symqccs:csimscon_notpoint},
 $\con{P}{\op{\bar \E}{\tilde r}(\rho)} 
  \xrightarrow{\tau}
  \con{P_1}{\rho_1}$ and
 $\con{P}{\op{\bar \E}{\tilde r}(\rho)}
  \xrightarrow{\tau}
  \con{P_2}{\rho_2}$ interpreting $\bar \E$ as
 $\E_{\tilde r}$, and
\[
 \mu' (\strg)^\dagger 
 \frac{\tr{}{\braw{\rho_1}}}{\tr{}{\braw{\rho}}}
 \bullet 
 \con{\cnv{P_1}}
 {\frac{\braw{\rho_1}}{\tr{}{\braw{\rho_1}}}} + 
 \frac{\tr{}{\braw{\rho_2}}}{\tr{}{\braw{\rho}}}
 \bullet 
 \con{\cnv{P_2}}
 {\frac{\braw{\rho_2}}{\tr{}{\braw{\rho_2}}}}
\]
hold.
 Since Verifier1 returns $\mathit{true}$, there exists 
 configurations $\con{Q_1}{\sigma_1}$ and $\con{Q_2}{\sigma_2}$ 
 such that
 \begin{itemize}
  \item $\con{Q}{\op{\bar \E}{\tilde r}(\sigma)} \xrightarrow{\tau\ast} 
	\con{D[\measure{b}{Q'}]}{\sigma'}$,
  \item $\con{D[\measure{b}{Q'}]}{\sigma'} \xrightarrow{\tau}
	\con{D[Q']}{\mathtt{proj1}[b](\sigma')} \xrightarrow{\tau\ast}
	\con{Q_1}{\sigma_1}$,
  \item $\con{D[\measure{b}{Q'}]}{\sigma'} \xrightarrow{\tau}
	\con{D[\mathtt{discard}(\qv{Q'})]}{\mathtt{proj0}[b](\sigma')}\\
	\xrightarrow{\tau\ast} \con{Q_2}{\sigma_2}$,
 \end{itemize}
 hold for some $D, b, Q'$ and $\sigma'$, and
\begin{itemize}
 \item Verifier1 returns $\mathit{true}$ with $\con{P_1}{\rho_1}$,
       $\con{Q_1}{\sigma_1}$, and $\mathit{eqs}$.
 \item Verifier1 returns $\mathit{true}$ with $\con{P_2}{\rho_2}$,
       $\con{Q_2}{\sigma_2}$, and $\mathit{eqs}$.
\end{itemize}
 Moreover, $\tr{}{\braw{\sigma}} = \tr{}{\braw{\sigma'}}$ holds; 
 Otherwise, $\tr{}{\braw{\sigma}} > \tr{}{\braw{\sigma'}}$ holds.
 Since Verifier1 returns $\mathit{true}$ with two pairs
 $\con{P}{\rho}$ and $\con{Q}{\sigma})$,
 the numbers of
 $\mathtt{proj}i$'s occurring in $\rho$ and $\sigma$ are equal
 (Section \ref{symqccs:algorithmforbisim}).
 Let the number be $N$.
 In $\rho_1$, there are $N + 1$ $\mathtt{proj}i$'s.
 By the transition, there are more than $N + 2$ $\mathtt{proj}i$'s or
 $N + 2$ $\mathtt{proj}i$'s in $\sigma_1$. This contradicts that 
 Verifier1 returned $\mathit{true}$ with 
 $\con{P_1}{\rho_1}$ and $\con{Q_1}{\sigma_1}$, and thus
 the numbers of $\mathtt{proj}i$'s in $\rho_1$ and $\sigma_1$ are
 equal.

 Next, by the validity of $\mathit{eqs}$,
 $\tr{}{\braw{\rho_1}} = \tr{}{\braw{\sigma_1}}$ and
 $\tr{}{\braw{\rho_2}} = \tr{}{\braw{\sigma_2}}$ hold.
 Thus we have $\tr{}{\braw{\sigma_1}} + \tr{}{\braw{\sigma_2}} = 
 \tr{}{\braw{\rho_1}} + \tr{}{\braw{\rho_2}} = 
 \tr{}{\braw{\rho}} = \tr{}{\braw{\sigma}} = \tr{}{\braw{\sigma'}}$.
 Now, we can apply the Lemma \ref{symqccs:prfofweakmeas} to have
\begin{align*}
&\con{\cnv{Q}}{\frac{\braw{\sigma}}{\tr{}{\braw{\sigma}}}}
 \Rightarrow
 \nu'\\
&\nu'
 (\strg)^\dagger
 \frac{\tr{}{\braw{\rho_1}}}{\tr{}{\braw{\rho}}} \bullet
 \con{\cnv{Q_1}}{\frac{\braw{\sigma_1}}{\tr{}{\braw{\sigma_1}}}}
 +\frac{\tr{}{\braw{\rho_2}}}{\tr{}{\braw{\rho}}} \bullet
 \con{\cnv{Q_2}}{\frac{\braw{\sigma_2}}{\tr{}{\braw{\sigma_2}}}}
\end{align*}
for some $\nu'$.
By strong bisimulation, 
$Y \Rightarrow \nu$ and $\nu' (\strg)^\dagger \nu$ hold for some
$\nu$.  We then have
\begin{align*}
 &\mu 
 (\strg)^\dagger 
 \frac{\tr{}{\braw{\rho_1}}}{\tr{}{\braw{\rho}}}
 \bullet 
 \con{\cnv{P_1}}
 {\frac{\braw{\rho_1}}{\tr{}{\braw{\rho_1}}}} + 
 \frac{\tr{}{\braw{\rho_1}}}{\tr{}{\braw{\rho}}}
 \bullet 
 \con{\cnv{P_2}}
 {\frac{\braw{\rho_2}}{\tr{}{\braw{\rho_2}}}} \defeq U\\
 &\nu
 (\strg)^\dagger
 \frac{\tr{}{\braw{\rho_1}}}{\tr{}{\braw{\rho}}} \bullet
 \con{\cnv{Q_1}}{\frac{\braw{\sigma_1}}{\tr{}{\braw{\sigma_1}}}}
 +\frac{\tr{}{\braw{\rho_2}}}{\tr{}{\braw{\rho}}} \bullet
 \con{\cnv{Q_2}}{\frac{\braw{\sigma_2}}{\tr{}{\braw{\sigma_2}}}}
 \defeq V
\end{align*}
Let $A, B, C$ and $D$ be
$\con{\cnv{P_1}}{\frac{\braw{\rho_1}}{\tr{}{\braw{\rho_1}}}}$,
$\con{\cnv{P_2}}{\frac{\braw{\rho_2}}{\tr{}{\braw{\rho_2}}}}$,
$\con{\cnv{Q_1}}{\frac{\braw{\sigma_1}}{\tr{}{\braw{\sigma_1}}}}$,
and
$\con{\cnv{Q_2}}{\frac{\braw{\sigma_2}}{\tr{}{\braw{\sigma_2}}}}$,
respectively.
By the definition of $(\cdot)^\dagger$, $\mu$ is written as\\
$\sum_{i \in I}p_i X_i + \sum_{j \in J}q_j X_j$ with 
$I \cap J = \emptyset$, and
$U$ is written as $\sum_{i \in I}p_i A + \sum_{j \in J}q_j B$, and
$X_i \strg A$ for all $i \in I$, and
$X_j \strg B$ for all $j \in J$. 
$\sum_{i \in I} p_i = \frac{\tr{}{\braw{\rho_1}}}{\tr{}{\braw{\rho}}}$
 and
$\sum_{j \in J} q_j =
\frac{\tr{}{\braw{\rho_2}}}{\tr{}{\braw{\rho}}}$ hold.
Similarly,
$\nu$ is written as\\
$\sum_{k \in K}p'_k Y_k + \sum_{l \in L}q'_l Y_l$ with 
$K \cap L = \emptyset$, and
$V$ is written as $\sum_{k \in K}p'_k C + \sum_{l \in L}q'_l D$, and
$Y_k \strg C$ for all $k \in K$, and
$Y_l \strg D$ for all $l \in L$. 
$\sum_{k \in K} p'_k = \frac{\tr{}{\braw{\rho_1}}}{\tr{}{\braw{\rho}}}$
 and
$\sum_{l \in L} q'_l =
\frac{\tr{}{\braw{\rho_2}}}{\tr{}{\braw{\rho}}}$ hold.
Since Verifier1 returns $\mathit{true}$ with \\
$(\con{P_m}{\rho_m}, \con{Q_m}{\sigma_m})\,(m = 1,2)$,
$X_i \R_{\mathit{eqs}} Y_k$ for all $i \in I, k \in K$, and
$X_j \R_{\mathit{eqs}} Y_l$ for all $j \in J, l \in L$.
We then have the conclusion $\mu (\R_{\mathit{eqs}})^\dagger \nu$
observing 
\begin{align*}
 &\mu = \frac{\tr{}{\braw{\rho}}}{\tr{}{\braw{\rho_1}}}
         \sum_{i,k}p_i p'_k X_i +
        \frac{\tr{}{\braw{\rho}}}{\tr{}{\braw{\rho_2}}}
         \sum_{j,l}q_j q'_l X_j \mbox{ and }\\
 &\nu = \frac{\tr{}{\braw{\rho}}}{\tr{}{\braw{\rho_1}}}
         \sum_{i,k}p_i p'_k Y_k +
        \frac{\tr{}{\braw{\rho}}}{\tr{}{\braw{\rho_2}}}
         \sum_{j,l}q_j q'_l Y_l.
\end{align*}
By definition of $\R$, $\R$ is symmetric and thus $\R^{-1}$ also
satisfies the conditions.
\end{proof}

\section{Discussion}
\subsection{On Completeness}
Let us consider an ``ideal'' verifier that can test
equality of partial traces perfectly.
It takes two configurations
$\con{P}{\rho}$ and $\con{Q}{\sigma}$, but does not take a
set of user-defined equations. In the step 4 of the algorithm,
it goes to the next step 
if and only if $\braw{\ptrtt{\qv{P}}{\rho}} =
\braw{\ptrtt{\qv{Q}}{\sigma}}$.
Let us then consider the following statement.
For $\con{P}{\rho},\con{Q}{\sigma} \in \mathcal{P} \times \mathcal{S}$,
if $\tr{}{\braw{\rho}}=\tr{}{\braw{\sigma}}=1$ and
$\con{\cnv{P}}{\braw{\rho}} \approx
\con{\cnv{Q}}{\braw{\sigma}}$ hold, then
the ideal verifier returns $\mathit{true}$ with
the input $\con{P}{\rho},\con{Q}{\sigma}$.
We call this
\emph{the completeness of nondeterministic qCCS with respect to qCCS}.
In fact, this statement is not true.
A counter example is as follows.
\begin{align*}
 \con{P}{\braw{\rho}} &\defeq \con{\measure{b}{\discard{b,q}}}
 {\ket{\psi}\bra{\psi}_b \otimes \ket{0}\bra{0}_q \otimes \rho^E} \mbox{ and}
\\
 \con{Q}{\braw{\sigma}} &\defeq \con{\discard{b,q}}
 {\ket{\psi}\bra{\psi}_b \otimes \ket{0}\bra{0}_q \otimes \rho^E},
 \mbox{ where}\\
 \ket{\psi} &= \sqrt{\frac{1}{3}}\ket{0} + \sqrt{\frac{2}{3}}\ket{1}.
\end{align*}
The two transitions of $\con{P}{\braw{\rho}}$ are
\begin{align*}
 \con{P}{\braw{\rho}} &\xrightarrow{\tau} \con{\discard{b,q}}
 {\frac{1}{3}\ket{0}\bra{0}_b \otimes \ket{0}\bra{0}_q \otimes \rho^E}
 \mbox{ and}\\
 \con{P}{\braw{\rho}} &\xrightarrow{\tau} \con{\discard{b,q}}
 {\frac{2}{3}\ket{1}\bra{1}_b \otimes \ket{0}\bra{0}_q \otimes \rho^E}
\end{align*}
but partial traces of them ($\frac{1}{3}\rho^E$ and 
$\frac{2}{3}\rho^E$) are not equal to that of $\con{Q}{\braw{\sigma}}$ (
namely $\rho^E$). In our simplified formal framework,
two configurations $\con{P_0}{\braw{\rho_0}}$ and $\con{Q_0}{\braw{\sigma_0}}$
with $\tr{}{\rho_0} \neq \tr{}{\sigma_0}$ are always separated even if
$\con{P_0}{\frac{\braw{\rho_0}}{\tr{}{\braw{\rho_0}}}}$ and
$\con{Q_0}{\frac{\braw{\sigma_0}}{\tr{}{\braw{\sigma_0}}}}$ are identified.
To identify such configurations, we must withdraw the simplification of 
operational semantics.

However, there seem to be a number of cases where we can assume that
processes behave differently according to the result of quantum
measurements.
In general, we can formalize a QKD protocol of the form 
\[
 ... \mathtt{abort\_flag[...,}b\mathtt{,...].}\measure{b}{P'} ...,
\]
where an operator $\mathtt{abort\_flag}$ calculates the number of errors
in check bits and sets a bit $b$ representing whether to abort the
protocol. Alice and Bob continue to communicate
only if the protocol has not aborted, which the outsider can
recognize. 

Besides, with our criteria discussed in Section \ref{symqccs:criteria},
we expect that there is a transition of $R$
with a label $\sndq{c}{q}$ or $\rcvq{c}{q}$, when a process
$\measure{b}{R}$ is considered.
Only for processes satisfying the condition, it is still possible that
the completeness holds. It needs to be discussed precisely.

% However, it seems not to be usual to consider a process like $P$
% which behaves equivalently whichever result of a measurement 
% it obtains. 
% With our criteria, which we discussed in Section \ref{symqccs:criteria},
% we expect that there is a transition of $R$
% with a label $\sndq{c}{q}$ or $\rcvq{c}{q}$, when a process
% $\measure{b}{R}$ is considered.
% Only for processes satisfying the condition, it is still possible that
% the statement be true. It needs to be discussed precisely.

% Before simplifying the operational semantics, 
% quantum states are always normalized (namely, trace is $1$) and thus
% $\con{P}{\braw{\rho}} \approx \con{Q}{\braw{\sigma}}$ holds.
% The key point is that
% \begin{align*}
%  \con{P}{\braw{\rho}} \xrightarrow{\tau}& \frac{1}{3}\con{\discard{b,q}}
%  {\ket{0}\bra{0}_b \otimes \ket{0}\bra{0}_q \otimes \rho^E}\\
%  &\boxplus \frac{2}{3}\con{\discard{b,q}}
%  {\ket{1}\bra{1}_b \otimes \ket{0}\bra{0}_q \otimes \rho^E} \defeq \mu,\\
%  \con{\discard{b,q}}{&\ket{0}\bra{0}_b \otimes \ket{0}\bra{0}_q \otimes \rho^E}
%  \approx \con{Q}{\braw{\sigma}}, \mbox{ and}\\
%  \con{\discard{b,q}}{&\ket{1}\bra{1}_b \otimes \ket{0}\bra{0}_q \otimes \rho^E}
%  \approx \con{Q}{\braw{\sigma}}
% \end{align*}
% hold and implies $\mu \approx^\dagger 1 \bullet \con{Q}{\braw{\sigma}}$.

%\usepackage{graphics}
\chapter{Approximate Bisimulation for Quantum Processes}
\label{prob_bisim}
Two notions of approximate bisimulation
are defined in our simplified formal framework named 
nondeterministic qCCS, which is 
defined in the previous chapter, namely, the 
relations are on $\C = \mathcal{P} \times \Delta(\H)$
with simplified operational semantics.
The relation $\sim_{\zeta, \eta}$ is defined in Section \ref{par} and
the relation $\sim$ is in Section \ref{neg}.
Before the definitions, we introduce some preliminaries.

\section{Preliminaries}
\subsection{Negligible Functions}
\begin{defi}
 A function $f:\mathbb{N}\rightarrow [0, 1]$ is \emph{negligible} if and
 only if for all polynomial $p(\cdot)$, there exists a natural number
 $N$ such that $f(n) \le \frac{1}{p(n)}$ holds
 for all $n \ge N$. $f$ is
 \emph{non-negligible} if $f$ is not negligible.
\end{defi}
\begin{rem}
 A function $\frac{f}{g}(n) \defeq \frac{f(n)}{g(n)}$ can be
 non-negligible even if $f$ is negligible and $g$ is non-negligible.
 An example is $f(n) = \frac{1}{2^n}$ and
\begin{displaymath}
 g(n) = \left\{
\begin{array}{l}
\frac{1}{2^n}~~~(n \mbox{ is even})\\
\frac{1}{n^2}~~~(\mbox{otherwise}).
\end{array}
\right.
\end{displaymath}
 In this thesis, we say $g$ is \emph{greater than negligible} if the
 function
 $\frac{f}{g}$ is negligible for all
 negligible function $f$.
\end{rem}

\begin{prop}
\label{par:propneg}
 If $f$ and $g$ are negligible, then $f + g$ and $cf$ is negligible for
 all $c \ge 0$.
\end{prop}

\subsection{Trace Distance of Probability-Weighted Quantum States}
Trace distance is a metric on a set of
linear operators.
To compare quantum states, trace distance on $\D(\H)$
is usually considered \cite[Chapter
9]{NielsenChuang-Kimura2004}.
Since we consider probability-weighted quantum states, 
we consider trace distance on $\Delta(\H)$, and discuss an
interpretation of it with respect to our transition system.

Let $\rho, \sigma \in \Delta(\H)$. $\rho - \sigma$ is an Hermitian
operator. Let $\sqrt{A}$ be $\sum_i \sqrt{\lambda_i}P_i$ for
an Hermitian operator $A$ with the spectrum decomposition
$\sum_i \lambda_i P_i$. Let $|A|$ be $\sqrt{A^\dagger A}$.

\begin{defi}
 Trace distance $d: \Delta(\H) \times \Delta(\H) \rightarrow [0,1]$ is
 defined as 
\[
 d(\rho, \sigma) = \frac{1}{2}\mathrm{tr}|\rho - \sigma|.
\]
\end{defi}
If $\tr{}{\rho} = \tr{}{\sigma} = 1$, trace distance can be thought as
a generalization of Kolmogorov distance.
Indeed, if $\rho$ and $\sigma$ are diagonal with respect to
orthonormal basis
$\{\ket{i}\}_{i}$, then $\rho = \sum_i p_i \ket{i}\bra{i}$,
$\rho = \sum_i q_i \ket{i}\bra{i}$, and $d(\rho,\sigma) = 
\frac{1}{2}\sum_i |p_i - q_i|$ hold for some unique $p_i$'s and $q_i$'s
satisfying $\sum_i p_i = \sum_i q_i = 1$.
We introduce some useful properties of trace distance as follows.

\begin{prop}
\label{par:trdismax}
 If $\tr{}{\rho} = \tr{}{\sigma} = 1$ holds, then 
\[
 d(\rho, \sigma) = 
 \max\{\tr{}{P\rho} - \tr{}{P\sigma}\,|\, P \mbox{ is a projector.}\}
\]
holds.
\end{prop}

\begin{prop}
\label{par:tpcpnonincrease}
If $\tr{}{\rho} = \tr{}{\sigma} = 1$ holds, then
 $d(\E(\rho), \E(\sigma))
\le d(\rho, \sigma)$ holds for all TPCP map $\E$.
\end{prop}
Proposition \ref{par:tpcpnonincrease}
can be extended to trace non-increasing cases.
In the proof, we use the fact that $\mathrm{tr}|\cdot|$ is a
norm on a set of linear operators.

\begin{prop}
\label{par:positivenonincrease}
 $d(\E(\rho), \E(\sigma)) \le d(\rho, \sigma)$ for all
 trace non-increasing positive map $\E$.
\end{prop}

\begin{proof}
 Let $\sum_i \lambda_i \ket{i}\bra{i}$ be an arbitrary eigenvalue
 decomposition of $\rho - \sigma$. 
 We obtain the proposition by the following calculation.
 \begin{align*}
  \mathrm{tr}|\E(\rho - \sigma)|
  &= \mathrm{tr}|\sum_i \lambda_i \E(\ket{i}\bra{i})| \\
  &\le \sum_i |\lambda_i| \cdot \mathrm{tr}|\E(\ket{i}\bra{i})| 
  &&\mbox{(by the axiom of trace norm)}\\
  &\le \sum_i |\lambda_i| \cdot \mathrm{tr}(\ket{i}\bra{i})
  &&\mbox{(} \E \mbox{ is trace non-increasing and positive.)}\\
  &= \sum_i |\lambda_i| = \mathrm{tr}|\rho - \sigma|
 \end{align*}
\end{proof}

For a configuration $\con{P}{\rho} \in \C$, $\tr{}{\rho}$ is 
interpreted
as the probability to reach $\con{P}{\rho}$, and the quantum state  
is $\frac{\rho}{\tr{}{\rho}}$.
The next proposition 
gives a way to interpret that $d(\rho, \sigma)$ is 
small for two configurations $\con{P}{\rho}, \con{Q}{\sigma} \in \C$.
\begin{prop}
\label{par:trdisproperty}
 Let $\rho, \sigma : \mathbb{N} \rightarrow \Delta(\H)$ be functions
 and regard $d(\rho, \sigma):\mathbb{N} \rightarrow [0,1]$ 
 as a function. $d(\rho, \sigma)$
 is negligible iff $|\tr{}{\rho} - \tr{}{\sigma}|$ and
 $\tr{}{\rho}\cdot d(\frac{\rho}{\tr{}{\rho}},
 \frac{\sigma}{\tr{}{\sigma}})$ are negligible.
\end{prop}

\begin{proof}
 $\mathrm{(\Rightarrow)}$ As $\tr{}{\cdot}$ is a TPCP map, $d(\rho,
 \sigma) \ge d(\tr{}{\rho}, \tr{}{\sigma}) = \frac{1}{2}|\tr{}{\rho} - 
 \tr{}{\sigma}|$. $|\tr{}{\rho} -  \tr{}{\sigma}|$ is negligible by
 Proposition \ref{par:propneg}.
 $\tr{}{\rho}\cdot d(\frac{\rho}{\tr{}{\rho}}, \frac{\sigma}{\tr{}{\sigma}})$
 is shown to be negligible by the following calculation.
 \begin{align*}
  \tr{}{\rho}\cdot d(\frac{\rho}{\tr{}{\rho}}, \frac{\sigma}{\tr{}{\sigma}})
  &\le \tr{}{\rho}\cdot (d(\frac{\rho}{\tr{}{\rho}},
  \frac{\sigma}{\tr{}{\rho}}) + d(\frac{\sigma}{\tr{}{\rho}},
  \frac{\sigma}{\tr{}{\sigma}}))\\
  &= d(\rho, \sigma) + |\tr{}{\rho} -
  \tr{}{\sigma}| \cdot \mathrm{tr}|\frac{\sigma}{\tr{}{\sigma}}|\\
  &= d(\rho, \sigma) + |\tr{}{\rho} - \tr{}{\sigma}|
 \end{align*}
$\mathrm{(\Leftarrow)}$ By triangle inequality, we have
$d(\rho, \frac{\tr{}{\rho}}{\tr{}{\sigma}}\sigma) + 
 d(\frac{\tr{}{\rho}}{\tr{}{\sigma}}\sigma, \sigma)
 \ge d(\rho, \sigma)$. This is equivalent to
$\tr{}{\rho} \cdot d(\frac{\rho}{\tr{}{\rho}},
 \frac{\sigma}{\tr{}{\sigma}}) + 
 d(\frac{\tr{}{\rho}}{\tr{}{\sigma}}\sigma, \sigma)
 \ge d(\rho, \sigma)$. The left-hand side is shown to be negligible
 by the calculation $d(\frac{\tr{}{\rho}}{\tr{}{\sigma}}\sigma, \sigma)
 = 
|\tr{}{\rho} -
  \tr{}{\sigma}| \cdot \mathrm{tr}|\frac{\sigma}{\tr{}{\sigma}}|$.
\end{proof}

For $\con{P}{\rho}, \con{Q}{\sigma}$, assume
$d(\rho, \sigma)$ is negligible. It is equivalent to that
$|\tr{}{\rho} - \tr{}{\sigma}|$ and $\tr{}{\rho} \cdot
d(\frac{\rho}{\tr{}{\rho}}, \frac{\sigma}{\tr{}{\sigma}})$ are
negligible. By Proposition \ref{par:trdismax}, 
we have that $\tr{}{\rho} \cdot \tr{}{P\frac{\rho}{\tr{}{\rho}}} - 
\tr{}{\rho} \cdot \tr{}{P\frac{\sigma}{\tr{}{\sigma}}}$ is negligible
for all projector $P$. As $|\tr{}{\rho}$ - $\tr{}{\sigma}|$ is
negligible, we have that
$\tr{}{\rho} \cdot \tr{}{P\frac{\rho}{\tr{}{\rho}}} - 
\tr{}{\sigma} \cdot \tr{}{P\frac{\sigma}{\tr{}{\sigma}}}$ is negligible.
$\tr{}{\rho} \cdot \tr{}{P\frac{\rho}{\tr{}{\rho}}}$ is the joint
probability that a process reaches $\con{P}{\rho}$ and observes
the measurement result corresponding to the projector $P$.
Therefore, that $d(\tr{}{\rho}, \tr{}{\sigma})$ is negligible implies
that the difference of the joint probabilities is negligible for an
arbitrary measurement.

\section{Approximate Bisimulation up to Parameters}
\label{par}
We define the first approximate bisimulation relation.
\begin{defi}
\label{par:defofzetaetarel}
 Let $0 \le \zeta, \eta \le 1$. A symmetric relation $\R \subseteq \C \times \C$
 is called an $(\zeta,\eta)$-bisimulation if for all
 $\con{P}{\rho},\con{Q}{\sigma}$,$\con{P}{\rho} \R \con{Q}{\sigma}$
 implies 
\begin{enumerate}
 \item $\qv{P}=\qv{Q} \defeq \tilde q$,
 \item $d(\tr{\tilde q}{\rho}, \tr{\tilde q}{\sigma}) \le \zeta$, and
 \item for all CP map $\E_{\tilde r}$ acting on 
       $\tilde r \subseteq \sfqv - \tilde q$,
       if $\con{P}{\E_{\tilde r}(\rho)} \xrightarrow{\alpha}
       \con{P'}{\rho'}$ and $\tr{}{\rho'} \ge \eta$ hold,
       then $\con{Q}{\E_{\tilde r}(\sigma)} \xrightarrow{\tau*}
       \xrightarrow{\hat \alpha}
       \xrightarrow{\tau*} \con{Q'}{\sigma'}$ and
       $\con{P'}{\rho'}\mathcal{R} \con{Q'}{\sigma'}$ hold for some
       $\con{Q'}{\sigma'}$
\end{enumerate}
We call the conditions 1 and 2 the static conditions, and
the condition 3 the simulation condition.
\end{defi}
Precisely in the condition 3, it is possible that $\E_{\tilde r}(\rho) = O
\notin \Delta(\H)$ holds for some $\E_{\tilde r}$. We exclude such
CP maps in our discussions.

\begin{defi}
\label{par:defofzetaeta}
 We define 
 \[
 \sim_{\zeta, \eta} := \{(\C, \D) \in \Con \times \Con~|~
 \C\R\D \mbox{ holds for some } (\zeta,\eta)\mbox{-bisimulation }
 \R\}.
 \]
 We say $\con{P}{\rho}$ and $\con{Q}{\sigma}$
 are $(\zeta, \eta)$-bisimilar if
 $\con{P}{\rho} \sim_{\zeta, \eta} \con{Q}{\sigma}$.
\end{defi}
The relation $\sim_{\zeta, \eta}$ has the similar properties
as the bisimulation relations defined in qCCS or other process 
calculi have \cite{Milner1999}.

\begin{lem}
\label{par:alsobisimulation}
$\sim_{\zeta,\eta}$ is a ($\zeta,\eta$)-bisimulation.
\end{lem}
\begin{proof}
By definition of $\sim_{\zeta, \eta}$, it is symmetric.
By $\con{P}{\rho}\sim_{\zeta, \eta}\con{Q}{\sigma}$,
there exists a $(\zeta, \eta)$-bisimulation
$\R$ satisfying $\con{P}{\rho}\R\con{Q}{\sigma}$.
The static conditions are easily checked.
Next, we have that for all CP map $\E_{\tilde r}$ that acts on 
$\tilde r \subseteq \sfqv - \qv{P}$, 
 if $\con{P}{\E_{\tilde r}(\rho)} \xrightarrow{\alpha} \con{P'}{\rho'}$
 and $\tr{}{\rho'} \ge \eta$ hold,
 then there exists
 $\con{Q'}{\sigma'}$ satisfying
 $\con{Q}{\E_{\tilde r}(\sigma)} \weak{\hat \alpha} \con{Q'}{\sigma'}$
 and 
 $\con{P'}{\rho'}\R\con{Q'}{\sigma'}$.
 This implies
 $\con{P'}{\rho'}\sim_{\zeta, \eta}\con{Q'}{\sigma'}$.
\end{proof}

\begin{lem}
\label{prob_bisim:parcoinduction}
 $\con{P}{\rho} \sim_{\zeta,\eta} \con{Q}{\sigma}$ if and
 only if 
\begin{enumerate}
 \item $\qv{P}=\qv{Q} \defeq \tilde q$,
 \item $d(\tr{\tilde q}{\rho}, \tr{\tilde q}{\sigma})  \le \zeta$,
 \item and for all CP map $\E_{\tilde r}$ acting on 
       $\tilde r \subseteq \sfqv - \tilde q$,
       \begin{itemize}
	\item
	     if $\con{P}{\E_{\tilde r}(\rho)} \xrightarrow{\alpha}
	     \con{P'}{\rho'}$ and $\tr{}{\rho'} \ge \eta$ hold,
	     then
	     $\con{Q}{\E_{\tilde r}(\sigma)} 
	     \xrightarrow{\tau*}
	     \xrightarrow{\hat \alpha}
	     \xrightarrow{\tau*} \con{Q'}{\sigma'}$ and
	     $\con{P'}{\rho'} \sim_{\zeta, \eta}
	     \con{Q'}{\sigma'}$ hold
	     for some  $\con{Q'}{\sigma'}$
	\item
	     if $\con{Q}{\E_{\tilde r}(\sigma)} \xrightarrow{\alpha}
	     \con{Q'}{\sigma'}$ and $\tr{}{\sigma'} \ge \eta$ hold,
	     then
	     $\con{P}{\E_{\tilde r}(\rho)} \xrightarrow{\tau*}
	     \xrightarrow{\hat \alpha}
	     \xrightarrow{\tau*} \con{P'}{\rho'}$ and
	     $\con{P'}{\rho'} \sim_{\zeta, \eta} \con{Q'}{\sigma'}$
	     hold for some
	     $\con{P'}{\rho'}$
       \end{itemize}
\end{enumerate}
\end{lem}
\begin{proof}
 ($\Rightarrow$) proven as the previous lemma.\\
 ($\Leftarrow$) We define $\hat \R\,:=\,\sim_{\zeta, \eta}\cup 
\{(\con{P}{\rho}, \con{Q}{\sigma} ) \} \cup \{(\con{Q}{\sigma},
 \con{P}{\rho}) \}$. $\hat \R$ is symmetric by the definition.
 It is sufficient to show $\hat
 \R$ is a ($\zeta, \eta$)-bisimulation relation. Let
 $(\con{P_0}{\rho_0},
 \con{Q_0}{\sigma_0})$
 be an arbitrary element of $\hat \R$.
 \begin{enumerate}
  \item  Suppose $(\con{P_0}{\rho_0}, \con{Q_0}{\sigma_0}) \in
	 \sim_{\zeta, \eta}$. Since $\sim_{\zeta, \eta}$
	 is a $(\zeta, \eta)$-bisimulation, the static
	 conditions are satisfied. Next, let
	 $\E_{\tilde r}$ be an arbitrary CP map acting on 	
	 $\tilde r \subseteq \sfqv - \qv{P_0}$, and assume
	 $\con{P_0}{\E_{\tilde r}(\rho_0)} 
	 \xrightarrow{\alpha}
	 \con{P'}{\rho'}$ and $\tr{}{\rho'} \ge \eta$ hold.
	 By the previous lemma,
	 we have
	 $\con{Q_0}{\E_{\tilde r}(\sigma_0)}
	 \xrightarrow{\tau*}
	 \xrightarrow{\hat \alpha}
	 \xrightarrow{\tau*} \con{Q'}{\sigma'}$ and
	 $\con{P'}{\rho'}\sim_{\zeta, \eta}
	 \con{Q'}{\sigma'}$
	 for some
	 $\con{Q'}{\sigma'}$.
	 This implies $\con{Q_0}{\E_{\tilde r}(\sigma_0)}
	 \xrightarrow{\tau*}
	 \xrightarrow{\hat \alpha}
	 \xrightarrow{\tau*} \con{Q'}{\sigma'}$ and
	 $\con{P'}{\rho'}\hat \R
	 \con{Q'}{\sigma'}$
	 for some $\con{Q'}{\sigma'}$ since 
	 $\sim_{\zeta, \eta} \subseteq \hat \R$.
  \item Suppose $(\con{P_0}{\rho_0},
	\con{Q_0}{\sigma_0}) = (\con{P}{\rho}, \con{Q}{\sigma})$.
	The static conditions are easily checked.
	The simulation condition holds by the assumption and
	$\sim_{\zeta, \eta} \subseteq \hat \R$.
  \item Suppose $(\con{P_0}{\rho_0},
	\con{Q_0}{\sigma_0}) = (\con{Q}{\sigma}, \con{P}{\rho})$.
	The proof is similar to the previous case.
 \end{enumerate}
\end{proof}

\begin{prop}
\label{par:sym-by-par}
 $\con{P||Q}{\rho} \sim_{0, 0} \con{Q||P}{\sigma}$.
\end{prop}
\begin{proof}
 This is proved by the definition of the transition rules, where
 (Left) and (Right) rules are symmetric.
\end{proof}

\begin{lem}
 If $\con{P}{\rho} \sim_{\zeta, \eta} \con{Q}{\sigma}$ and 
$\con{P}{\rho} \xrightarrow{\tau\ast} \con{P'}{\rho'}$ and
 $\tr{}{\rho'} \ge \eta$, then
$\con{Q}{\sigma} \xrightarrow{\tau\ast}
 \con{Q'}{\sigma'}$ and $\con{P'}{\rho'} \sim_{\zeta, \eta} \con{Q'}{\sigma'}$ hold
 for some $\con{Q'}{\sigma'}$.
\end{lem}
\begin{proof}
 Assume $\con{P}{\rho} \xrightarrow{\tau}^n \con{P'}{\rho'}$
 and let $\con{P_i}{\rho_i}$ be $i$-th configuration with $0 \le i \le
 n$. The case when $n = 0$ is trivial. Let $n > 0$.
 Since $\rho = \rho_0 \ge \rho_1 \ge \cdots \ge \rho_n =
 \rho'$ and $\tr{}{\rho'} \ge \eta$ hold, 
 $\tr{}{\rho_i} \ge \eta$ holds for all $i$. Therefore,
 $\con{Q}{\sigma} \xrightarrow{\tau\ast}^n
 \con{Q'}{\sigma'}$ and $\con{P'}{\rho'} \sim_{\zeta, \eta} \con{Q'}{\sigma'}$ hold
 for some $\con{Q'}{\sigma'}$.
\end{proof}

\begin{lem}
\label{par:weaksimulated}
 If $\con{P}{\rho} \sim_{\zeta, \eta} \con{Q}{\sigma}$
 and $\con{P}{\rho} \weak{\hat \alpha} \con{P'}{\rho'}$ and
 $\tr{}{\rho'} \ge \eta$ hold, then 
 $\con{Q}{\sigma} \weak{\hat \alpha} \con{Q'}{\sigma'}$ and
 $\con{P'}{\rho'} \sim_{\zeta, \eta} \con{Q'}{\sigma'}$ hold for some
 $\con{Q'}{\sigma'}$.
\end{lem}
\begin{proof}
 It is proven similarly to the previous lemma.
\end{proof}

The relation $\sim_{\zeta, \eta}$ is closed under application
of an arbitrary CP map by the outsider,
namely, one acts on $\sfqv - \tilde q$.
This is one of the 
similar properties as the original qCCS's largest 
bisimulation relation $\approx$ has \cite{DengFeng2012}.
\begin{lem}
\label{par:cpclosed}
 If $\con{P}{\rho} \sim_{\zeta, \eta} \con{Q}{\sigma}$,
 then $\con{P}{\E_{\tilde r}(\rho)} \sim_{\zeta, \eta}
 \con{Q}{\E_{\tilde r}(\sigma)}$ holds for all 
 CP map $\E$ acting on $\tilde r \subseteq \sfqv - \qv{P}$.
\end{lem}
\begin{proof}
 We use $(\Leftarrow)$ implication of Lemma
 \ref{prob_bisim:parcoinduction}.
 By Lemma \ref{par:positivenonincrease},
 $d(\E_{\tilde r}(\rho), \E_{\tilde
 r}(\sigma)) \leq d(\rho, \sigma)$ holds for all
 CP map $\E_{\tilde r}$ and thus the static condition on partial trace holds.
 Since $\E_{\hat r}$ ranges over arbitrary CP map in the definition,
 the simulation condition holds.
\end{proof}

The relation $\sim_{\zeta, \eta}$ is reflexive and symmetric but
not transitive because of the condition of trace distance. Instead,
it has the following properties.
\begin{prop}
\label{par:transitivitylike}
 \begin{enumerate}
  \item If $\con{P}{\rho} \sim_{\zeta, \eta} \con{Q}{\sigma}$,
	$\zeta \le \zeta'$, and $\eta \le \eta'$ hold, 
	then $\con{P}{\rho} \sim_{\zeta', \eta'} \con{Q}{\sigma}$ holds.
  \item If $\con{P}{\rho} \sim_{\zeta, \eta} \con{Q}{\sigma}$ and
	$\con{Q}{\sigma} \sim_{\zeta', \eta'} \con{R}{\theta}$ hold,
	then
	\[
	 \con{P}{\rho} \sim_{\zeta + \zeta',\,\mathrm{max}\{\eta,
	\eta'\} + 2(\zeta + \zeta')} \con{R}{\theta}
	\] 
	holds.
 \end{enumerate}
\end{prop}
\begin{proof}
 \begin{enumerate}
  \item It is sufficient to prove that $\sim_{\zeta, \eta}$ is 
	a $(\zeta', \eta')$-bisimulation. 
	The relation $\sim_{\zeta, \eta}$ is symmetric by Lemma \ref{par:alsobisimulation}.
	The static conditions is checked observing
	$d(\tr{\tilde q}{\rho}, \tr{\tilde q}{\sigma}) \le
	\zeta \le \zeta'$. For the simulation condition,
	assume 
	$\con{P}{\E_{\tilde r}(\rho)} \xrightarrow{\alpha}
	\con{P'}{\rho'}$ and $\tr{}{\rho'} \ge \eta'$
	for a CP map $\E_{\tilde r}$. Since $\tr{}{\rho'} \ge \eta' \ge
	\eta$, there exists a configuration $\con{Q'}{\sigma'}$
	satisfying $\con{Q}{\E_{\tilde r}(\sigma)} \weak{\hat \alpha} 
	\con{Q'}{\sigma'}$ and $\con{P'}{\rho'} \sim_{\zeta, \eta}
	\con{Q'}{\sigma'}$.
  \item We define a relation $\R \subseteq \C \times \C$ as follows.
	\begin{align*}
	 \R' :=& \{(\con{P}{\rho}, \con{R}{\theta})\,|\,
	            \con{P}{\rho} \sim_{\zeta, \eta} \circ
	 \sim_{\zeta', \eta'} \con{R}{\theta}\}\\
	 \R := & \R' \cup \R'^{-1}
	\end{align*}
	The assumption implies $\con{P}{\rho} \R \con{R}{\theta}$.
	It is sufficient to prove that $\R$ is a $(\max\{\eta,
	\eta'\}+2(\zeta + \zeta'))$-bisimulation. By definition,
	$\R$ is symmetric. \\
	{\bf (Case 1)} Let an arbitrary element
	$(\con{P}{\rho},\con{R}{\theta}) \in \R'$.
	There exists $\con{Q}{\sigma}$ satisfying
	$\con{P}{\rho} \sim_{\zeta, \eta} \con{Q}{\sigma}$ and
	$\con{Q}{\sigma} \sim_{\zeta', \eta'} \con{R}{\theta}$.
	The static condition is checked observing
	$d(\tr{\tilde q}{\rho}, \tr{\tilde q}{\theta}) \le
	d(\tr{\tilde q}{\rho}, \tr{\tilde q}{\sigma}) +
	d(\tr{\tilde q}{\sigma}, \tr{\tilde q}{\theta}) \le
	\zeta + \zeta'$. For the simulation condition, assume
	$\con{P}{\E_{\tilde r}(\rho)} \xrightarrow{\alpha}
	\con{P'}{\rho'}$ and $\tr{}{\rho'} \ge \max\{\eta,
	\eta'\}+2(\zeta + \zeta')$ for a CP map $\E_{\tilde r}$.
	Since $\tr{}{\rho'} \ge \eta$ and
 	$\con{P}{\rho} \sim_{\zeta, \eta} \con{Q}{\sigma}$
	hold, there exists
	$\con{Q'}{\sigma'}$ satisfying
	$\con{Q}{\E_{\tilde r}(\sigma)} \weak{\hat \alpha}
	\con{Q'}{\sigma'}$ and
	$\con{P'}{\rho'} \sim_{\zeta, \eta} \con{Q'}{\sigma'}$.
	Applying Lemma \ref{par:cpclosed} to
	$\con{Q}{\sigma} \sim_{\zeta', \eta'} \con{R}{\theta}$, we
	have
	$\con{Q}{\E_{\tilde r}(\sigma)} \sim_{\zeta', \eta'}
	\con{R}{\E_{\tilde r}(\theta)}$. We also have
	$\tr{}{\sigma'} \ge \tr{}{\rho'} - 2\zeta \ge \eta'$.
	By Lemma \ref{par:weaksimulated}, there exists
	$\con{R'}{\theta'}$
	satisfying $\con{R}{\E_{\tilde r}(\theta)} \weak{\hat \alpha}
	\con{R'}{\theta'}$ and $\con{Q'}{\sigma'} \sim_{\zeta', \eta'}
	\con{R'}{\theta'}$. We also have $\con{P'}{\rho'} \R' \con{R'}{\theta'}$.
	\\
	{\bf (Case 2)} Let an arbitrary element
	$(\con{R}{\theta}, \con{P}{\rho}) \in \R'^{-1}$.
	The proof is similar to the previous case.
 \end{enumerate}
\end{proof}

Examples of the relation are as follows.
\begin{ex}
\label{par:ex1}
  \begin{align*}
   (1)~~&\con{\measure{b}{\sndq{c}{b}.\discard{}}}
   {\ket{+}\bra{+}_b \otimes \ket{+}\bra{+}_q}\\
   \sim_{\frac{1}{2}, \frac{1}{2}}
   &\con{\measure{b}{\sndq{c}{b}.\discard{}}}
   {(\frac{1}{2}\ket{00}\bra{00} +
   \frac{1}{2}\ket{11}\bra{11})_{b,q}} \mbox{ holds.}\\
   (2)~~&\con{\measure{b}{\sndq{c}{q}.\discard{}}}
   {\ket{\psi}\bra{\psi}_b \otimes \ket{0}\bra{0}_q}\\
   \sim_{0, \frac{1}{4}}
   &\con{\measure{b}{\sndq{c}{q}.\discard{}}}
    {\ket{\psi}\bra{\psi}_b \otimes \ket{1}\bra{1}_q} \mbox{ holds,
   where}\\
   & \ket{\psi} = \frac{\sqrt{3}\ket{0} + \ket{1}}{2}
  \end{align*}
\end{ex}
We prove that the relation $\sim_{\zeta, \eta}$ is
closed under application of evaluation contexts.
We first show that $\sim_{\zeta, \eta}$ is
closed under restriction.
\begin{lem}
\label{par:resclosed}
 If $\con{P}{\rho} \sim_{\zeta, \eta} \con{Q}{\sigma}$ holds, then
 $\con{P \backslash L}{\rho} \sim_{\zeta, \eta}
 \con{Q \backslash L}{\sigma}$ holds.
\end{lem}
\begin{proof}
\label{par:resclosed}
 Let $\R := \{(\con{P \backslash L}{\rho}, \con{Q \backslash
 L}{\sigma}) \,|\, \con{P}{\rho} \sim_{\zeta, \eta} \con{Q}{\sigma}\}$.
 It is
 sufficient to show that $\R$ is an approximate bisimulation relation.
 $\R$ is symmetric by the definition.
 Let $(\con{P \backslash L}{\rho}, \con{Q \backslash
 L}{\sigma})$ be an arbitrary element of $\R$. The
 static conditions are easily checked. Assume $\con{P \backslash
 L}{\E_{\tilde r}(\rho)} \xrightarrow{\alpha} \con{P' \backslash
 L}{\rho'}$ and $\tr{}{\rho'} \ge \eta$.
 This implies $\con{P}{\E_{\tilde r}(\rho)} \xrightarrow{\alpha}
 \con{P'}{\rho'}$ and $\mathrm{cn}(\alpha) \cap L = \emptyset$.
 We have
 $\con{P}{\E_{\tilde r}(\rho)} \sim \con{Q}{\E_{\tilde r}(\sigma)}$
 from $\con{P}{\rho} \sim_{\zeta, \eta} \con{Q}{\sigma}$. We then have 
 $\con{Q}{\E_{\tilde r}(\sigma)} \weak{\hat \alpha} \con{Q'}{\sigma'}$ 
 and $\con{P'}{\rho'} \sim_{\zeta, \eta} \con{Q'}{\sigma'}$ for some
 $\con{Q'}{\sigma'}$. As $\mathrm{cn}(\hat \alpha) \cap L = \emptyset$
 and $\mathrm{cn}(\tau) = \emptyset$, we have 
 $\con{Q \backslash L}{\E_{\tilde r}(\sigma)} \weak{\hat \alpha}
 \con{Q' \backslash L}{\sigma'}$ and 
 $\con{P' \backslash L}{\rho'} \R \con{Q' \backslash L}{\sigma'}$. 
\end{proof}

The relation $\sim_{\zeta, \eta}$ is
closed under parallel composition of the processes.

\begin{thm}
\label{par:pallaclosed}
 If $\con{P}{\rho} \sim_{\zeta, \eta} \con{Q}{\sigma}$ holds, 
 $\con{P||R}{\rho} \sim_{\zeta, \eta} \con{Q||R}{\sigma}$ holds for
 all process $R$.
\end{thm}
\begin{proof}
\label{par:prfof-pallaclosed}
 We define 
\[
 \hat \R := \{(\con{P||R}{\rho}, \con{Q||R}{\sigma})\,|\,
 \con{P}{\rho} \sim_{\zeta, \eta} \con{Q}{\sigma}, R \in \mathcal{P}
 \}. 
\]
 As $\sim_{\zeta, \eta}$ is symmetric, $\hat \R$ is symmetric.
 It is sufficient to show $\hat \R$ is a 
$(\zeta, \eta)$-bisimulation.
 Let $(\con{P||R}{\rho}, \con{Q||R}{\sigma})$ be an arbitrary element
 in $\hat \R$.
 The static conditions are checked as follows. By the definition
 of $\qv{\cdot}$ and the condition $\qv{P} = \qv{Q}$
 obtained from $\con{P}{\rho} \sim_{\zeta, \eta} \con{Q}{\sigma}$,
 $\tilde q \defeq 
 \qv{P||R}=\qv{P}\cup\qv{R}=\qv{Q}\cup\qv{R}=\qv{Q||R}$ holds.
 Next, 
 \begin{align*}
 d(\tr{\tilde q}{\rho},
 \tr{\tilde q}{\sigma})
 = \, &
 d(\tr{\qv{R}}{\tr{\qv{P}}{\rho}}, \tr{\qv{R}}{\tr{\qv{Q}}{\sigma}})\\
 \le \, & d(\tr{\qv{P}}{\rho}, \tr{\qv{Q}}{\sigma}) \le \zeta.
 \end{align*}
 holds.
 We then show that the simulation condition is satisfied.
 Let $\E_{\tilde r}$ be an arbitrary TPCP map acting on $\tilde r
 \subseteq \sfqv - \tilde q$.
 A transition of a parallely-composed process is either the 
 3 cases by the transition rules.\\
 {\bf (Case 1)}The transition is performed only by
       $P$.
       Assume $\con{P}{\E_{\tilde r}(\rho)} \xrightarrow{\alpha}
       \con{P'}{\rho'}$ and $\tr{}{\rho'} \ge \eta$ hold.
       By $\con{P}{\rho} \sim_{\zeta, \eta} \con{Q}{\sigma}$,
       there exists $\con{Q'}{\sigma'}$ satisfying
       $\con{Q}{\E_{\tilde r}(\sigma)} 
       \weak{\hat \alpha} \con{Q'}{\sigma'}$ and
       $\con{P'}{\rho'}\sim_{\zeta, \eta}\con{Q'}{\sigma'}$.
       Therefore, $\con{Q||R}{\sigma}
       \weak{\hat \alpha} \con{Q'||R}{\sigma'}$ holds by (Left) rule
       and
       $\con{P'||R}{\rho'} \hat \R \con{Q'||R}{\sigma'}$ holds
       by the definition of $\hat \R$.\\
 {\bf (Case 2)} The transition is performed only by
       $\con{R}{\rho}$.
       Assume $\con{R}{\E_{\tilde r}(\rho)} \xrightarrow{\alpha}
       \con{R'}{\rho'}$ and $\tr{}{\rho'} \ge \eta$ hold.
       Because $R$ has a redex that
       causes the transition $\xrightarrow{\alpha}$,
       $\con{Q||R}{\E_{\tilde r}(\sigma)}
       \xrightarrow{\alpha} \con{Q||R'}{\sigma'}$ holds for some
       $\sigma'$. It is sufficient to show
       $\con{P}{\rho'} \sim_{\zeta, \eta} \con{Q}{\sigma'}$ by the
       definition of $\hat \R$. 
       $\rho' = \F_{\tilde s} \circ \E_{\tilde r}(\rho)$ and
       $\sigma' = \F_{\tilde s} \circ \E_{\tilde r}(\sigma)$ hold
       for some CP map $\F_{\tilde s}$ acting on $\tilde s \subseteq
       \qv{R}$. As $\con{P}{\rho} \sim_{\zeta, \eta}
       \con{Q}{\sigma}$ and $\tilde s, \tilde r \subseteq \sfqv - 
       \qv{P}$ hold, $\con{P}{\rho'} \sim_{\zeta, \eta}
       \con{Q}{\sigma'}$ holds by Lemma \ref{par:cpclosed}.
       This implies $\con{P||R'}{\rho'} \hat \R
       \con{Q||R'}{\sigma'}$.\\
 {\bf (Case 3)} The transition is performed by 
       communication of $P$ and $R$. As the communication rule is 
       applied, the $P||R$ can be written as
       $\con{C_1[\sndq{c}{q}.P']||C_2[\rcvq{c}{r}.R']}
       {\E_{\tilde r}(\rho)}$
       for some evaluation contexts $C_1[\_], C_2[\_]$, processes
       $P', R'$, and
       non-restricted channel $\mathsf{c}$.
       The transition to consider is
       \[
       \con{C_1[\sndq{c}{q}.P']||C_2[\rcvq{c}{r}.R']}
       {\E_{\tilde r}(\rho)}
       \xrightarrow{\tau}
       \con{C_1[P']||C_2[R']}{\E_{\tilde r}(\rho)}.
       \]
       Assume $\tr{}{\rho} \ge \eta$.
       By $\con{C_1[\sndq{c}{q}.P']}{\rho}
       \sim_{\zeta, \eta} 
       \con{Q}{\sigma}$
       and
       $\con{C_1[\sndq{c}{q}.P']}{\E_{\tilde r}(\rho)}
       \xrightarrow{\sndq{c}{q}}
       \con{C_1[P']}{\E_{\tilde r}(\rho)}$,
       $\con{Q}{\E_{\tilde r}(\sigma)} \weak{\sndq{c}{q}}
       \con{Q'}{\sigma'}$ and
       $\con{C_1[P']}{\E_{\tilde r}(\rho)} \sim_{\zeta, \eta}
       \con{Q'}{\sigma'}$ hold for some $\con{Q'}{\sigma'}$.
       This implies $\mathsf{c}$ is not restricted in $Q$
       and thus receive redex
       $\rcvq{c}{r}$ in $C_2[\rcvq{c}{r}.R']$
       can be react.
       The reaction does not influence the quantum state $\sigma'$. 
       Therefore,
       \[
         \con{Q||R}{\E_{\tilde r}(\sigma)} \xrightarrow{\tau \ast}
         \con{Q'||C_2[R']}{\sigma'} \mbox{ and }
         \con{C_1[P']||C_2[R']}{\E_{\tilde r}(\rho)} 
         \hat \R
         \con{Q'||C_2[R']}{\sigma'}
       \]
       hold by the definition of $\hat \R$.       
\end{proof}
By the two previous lemmas and Proposition \ref{par:sym-by-par},
we have the following corollary.
\begin{col}
\label{par:congruence}
 If $\con{P}{\rho} \sim_{\zeta, \eta} \con{Q}{\sigma}$ holds, 
 $\con{C[P]}{\rho} \sim_{\zeta, \eta} \con{C[Q]}{\sigma}$ holds for
 all non-aborting evaluation context $C[\_]$.
\end{col}

\section{Approximate Bisimulation up to Negligible Difference}
\label{neg}
In this section we assume that quantum states depend on
security parameters, namely, 
we assume that $\H$ is a function of natural numbers.
For example, an $n$ bit randomness is represented as
the operator 
$(\frac{1}{2}\ket{0}\bra{0} + \frac{1}{2}\ket{1}\bra{1})^{\otimes n}$
whose state space $\H$ is $2^n$-dimensional.
As for transitions of configurations, we define
that $\con{P}{\rho} \xrightarrow{\alpha} \con{P'}{\rho'}$ holds if and
only if 
$\con{P}{\rho(n)} \xrightarrow{\alpha} \con{P'}{\rho'(n)}$ holds 
for all $n$. Trace distance
$d(\rho, \sigma)$ is also regarded as a
function of natural numbers.
For example, $d(\ket{0}\bra{0}^{\otimes n}, \ket{+}\bra{+}^{\otimes n})
= 1 - \frac{1}{2^n}$ holds.
if $d(\rho, \sigma)$ and $d(\sigma, \theta)$ are negligible,
then $d(\rho, \theta)$ is negligible. 
As a result, we could define the
approximate bisimulation relation $\sim$, which is
transitive and thus is an equivalence relation.
The definition of an approximate bisimulation relation 
is as follows.
\begin{defi}
\label{neg:defofapproxbisim}
 A symmetric relation $\R \subseteq \C \times \C$ is called an
 approximate bisimulation
 if for all $\con{P}{\rho} \R \con{Q}{\sigma}$,
 \begin{enumerate}
  \item $\qv{P} = \qv{Q} \defeq \tilde q$,
  \item $d(\tr{\tilde q}{\rho}, \tr{\tilde q}{\sigma})$ 
	is negligible, and
  \item for all CP map $\E_{\tilde r}$ acting
	on $\tilde r \subseteq \sfqv - \tilde q$,
	if $\con{P}{\E_{\tilde r}(\rho)}
	\xrightarrow{\alpha}
	\con{P'}{\rho'}$ holds and $\tr{}{\rho'}$ is
	non-negligible, then
	$\con{Q}{\E_{\tilde r}(\sigma)} \weak{\hat \alpha}
	\con{Q'}{\sigma'}$ and
	$\con{P'}{\rho'} \R \con{Q'}{\sigma'}$ holds for
	some $\con{Q'}{\sigma'}$. 
 \end{enumerate}
\end{defi}
We call the above conditions {\it 1}, {\it 2} the static conditions and
{\it 3} the simulation condition.

\begin{defi}
\label{neg:defofapprox}
 We define 
 \[
 \sim = \{(\C, \D) \in 
 \Con \times \Con~|~
 \C\R\D \mbox{ holds for some approximate bisimulation } \R \}.
 \]
 We say $\C$ and $\D$ are approximately bisimilar if
 $\C \sim \D$. 
\end{defi}

There is another possible definition of the relation.
Let us replace the requirement ``$\tr{}{\rho'}$ is non-negligible'' in the condition
3 with ``$\frac{\tr{}{\rho'}}{\tr{}{\E_{\tilde r}(\rho)}}$ is non-negligible'', and
let $\simeq$ be the relation defined similarly to
Definition \ref{neg:defofapprox}.
Since $\frac{\tr{}{\rho'}}{\tr{}{\E_{\tilde r}(\rho)}} \ge
\tr{}{\rho'}$ holds, $\simeq \subseteq \sim$
holds. In fact, the relation $\simeq$
has properties that are similar to those discussed in the following
propositions, lemmas, and theorem.
We adopt Definition \ref{neg:defofapproxbisim} for the following reason.
It is natural that we assume a configuration $\con{P_0}{\rho_0}$ satisfies
$\tr{}{\rho_0} = 1$ when it formalizes a protocol.
Suppose we have $\con{P_0}{\rho_0} \xrightarrow{\alpha_0} \cdots
\xrightarrow{\alpha_k} \con{P}{\rho} \xrightarrow{\alpha}
\con{P'}{\rho'}$, that $\tr{}{\rho}$ is non-negligible and
that $\tr{}{\rho'}$ is negligible. The probability to reach
$\con{P}{\rho}$ is non-negligible and to reach $\con{P'}{\rho'}$ is
negligible.
Therefore, we want to care $\con{P}{\rho}$ but ignore $\con{P'}{\rho'}$.
However, a case is possible where a configuration $\con{Q_0}{\sigma_0}$
must simulate the transition $\con{P}{\rho} \xrightarrow{\alpha}
\con{P'}{\rho'}$ to satisfy $\con{P_0}{\rho_0} \simeq
\con{Q_0}{\sigma_0}$. This is because $\frac{\tr{}{\rho'}}{\tr{}{\rho}}$
can be non-negligible even if $\tr{}{\rho'}$ is negligible and
$\tr{}{\rho}$ is non-negligible by the definition of negligible
functions. We thus cannot ignore $\con{P'}{\rho'}$.

The following lemmas are similarly proven as the previous section.
\begin{lem}
\label{neg:coinduction}
$\con{P}{\rho} \sim \con{Q}{\sigma}$ holds,
 iff
 \begin{enumerate}
  \item $\qv{P} = \qv{Q} \defeq \tilde q$,
  \item $d(\tr{\tilde q}{\rho}, \tr{\tilde q}{\sigma})$
	is negligible, and
  \item for all CP map $\E_{\tilde r}$ acting
	on $\tilde r \subseteq \sfqv - \tilde q$,
	\begin{itemize}
	 \item if $\con{P}{\E_{\tilde r}(\rho)} \xrightarrow{\alpha}
	       \con{P'}{\rho'}$ holds and $\tr{}{\rho'}$ is
	       non-negligible, then
	       there exists $\con{Q'}{\sigma'}$ satisfying
	       $\con{Q}{\E_{\tilde r}(\sigma)} \weak{\hat \alpha}
	       \con{Q'}{\sigma'}$ and
	       $\con{P'}{\rho'} \sim \con{Q'}{\sigma'}$,
	       and
	 \item if $\con{Q}{\E_{\tilde r}(\sigma)} \xrightarrow{\alpha}
	       \con{Q'}{\sigma'}$ holds and $\tr{}{\sigma'}$ is
	       non-negligible, then
	       there exists $\con{P'}{\rho'}$ satisfying
	       $\con{P}{\E_{\tilde r}(\rho)} \weak{\hat \alpha}
	       \con{P'}{\rho'}$ and 
	       $\con{P'}{\rho'} \sim \con{Q'}{\sigma'}$.
	\end{itemize}
 \end{enumerate}
\end{lem}
\begin{lem}\label{neg:soclosed}
 If $\con{P}{\rho} \sim \con{Q}{\sigma}$, then 
 $\con{P}{\E_{\tilde r}(\rho)} \sim \con{Q}{\E_{\tilde r}(\sigma)}$
 for all CP map $\E_{\tilde r}$ that acts on $\tilde r$. 
\end{lem}
\begin{prop}
\label{neg:sym-by-par}
 $\con{P||Q}{\rho} \sim \con{Q||P}{\sigma}$.
\end{prop}

We then prepare lemmas to prove transitivity of the relation
$\sim$.
\begin{lem}
\label{neg:weaksimulated}
 If $\con{P}{\rho} \sim \con{Q}{\sigma}$ and 
$\con{P}{\rho} \xrightarrow{\tau\ast} \con{P'}{\rho'}$ and
 $\tr{}{\rho'}$ is non-negligible, then
$\con{Q}{\sigma} \xrightarrow{\tau\ast}
 \con{Q'}{\sigma'}$ and $\con{P'}{\rho'} \sim \con{Q'}{\sigma'}$ hold
 for some $\con{Q'}{\sigma'}$.
\end{lem}
\begin{proof}
 Assume $\con{P}{\rho} \xrightarrow{\tau}^n \con{P'}{\rho'}$
 and let $\con{P_i}{\rho_i}$ be $i$-th configuration with $0 \le i \le
 n$. The case when $n = 0$ is trivial. Let $n > 0$.
 Since $\rho = \rho_0 \ge \rho_1 \cdot\cdot\cdot \ge \rho_n =
 \rho'$ holds and $\tr{}{\rho'}$ is non-negligible, 
 $\tr{}{\rho_i}$ is non-negligible for all $i$. Therefore,
 $\con{Q}{\sigma} \xrightarrow{\tau\ast}^n
 \con{Q'}{\sigma'}$ and $\con{P'}{\rho'} \sim \con{Q'}{\sigma'}$ hold
 for some $\con{Q'}{\sigma'}$.
\end{proof}

\begin{lem}
\label{neg:simulateweak}
If $\con{P}{\rho} \sim \con{Q}{\sigma}$ and 
$\con{P}{\rho} \weak{\hat \alpha} \con{P'}{\rho'}$ and
 $\tr{}{\rho'}$ is non-negligible, then
$\con{Q}{\sigma} \weak{\hat \alpha}
 \con{Q'}{\sigma'}$ and $\con{P'}{\rho'} \sim \con{Q'}{\sigma'}$
 for some $\con{Q'}{\sigma'}$.
\end{lem}

\begin{proof}
 It is similarly proven as the previous lemma.
\end{proof}

\begin{lem}
\label{neg:equivalence}
 $\sim$ is an equivalence relation.
\end{lem}
\begin{proof}
 {\bf (Reflexivity)} 
 Let $\mathit{Id}_\C$ be the identity relation on $\C$.
 For all $\con{P}{\rho} \in \C$, 
 $\con{P}{\rho}\mathit{Id}_\C \con{P}{\rho}$ holds.
 It is sufficient to show
 $\mathit{Id}_\C$ is an approximate bisimulation. 
 Assume $(\con{P}{\rho},
 \con{P}{\rho})$ is an arbitrary element in $\mathit{Id}_\C$ and
 $\tr{}{\rho}$ is non-negligible. The static conditions are 
 easily checked.
 Let $\E_{\tilde r}$ be an arbitrary CP map and assume
 $\con{P}{\E_{\tilde r}(\rho)} \xrightarrow{\alpha}
 \con{P'}{\rho'}$ and $\tr{}{\rho'}$ is negligible. As
 $\con{P'}{\rho'}\mathit{Id}_\C \con{P'}{\rho'}$ holds,
 $\mathit{Id}_\C$
 is an approximate bisimulation. \\
 {\bf (Symmetry)} 
 ($\Rightarrow$) implication of
 Lemma \ref{neg:coinduction} is the condition that $\sim$ is 
 an approximate bisimulation.
 An approximate bisimulation relation
 is defined to be symmetric. 
 \\
 {\bf (Transitivity)} It is sufficient to show $\sim \circ \sim$
 is an approximate bisimulation relation. Let $(\con{P}{\rho},
 \con{R}{\theta})$ be an arbitrary element of $\sim \circ
 \sim$. There exists $\con{Q}{\sigma}$ satisfying
 $\con{P}{\rho} \sim \con{Q}{\sigma}$ and $\con{Q}{\sigma}
 \sim {\con{R}{\theta}}$. 
 The static conditions are easily checked using
 triangle inequality of trace distance $d(\cdot, \cdot)$.
 Let $\E_{\tilde r}$ be an arbitrary
 CP map acting on $\tilde r \subseteq \sfqv - \qv{P}$ and
 assume $\con{P}{\E_{\tilde r}(\rho)} \xrightarrow{\alpha}
 \con{P'}{\rho'}$ and $\tr{}{\rho'}$ is non-negligible.
 By $\con{P}{\rho} \sim \con{Q}{\sigma}$, there exists
 $\con{Q'}{\sigma'}$ satisfying $\con{Q}{\E_{\tilde r}(\sigma)} 
 \weak{\hat \alpha} \con{Q'}{\sigma'}$ and $\con{P'}{\rho'} \sim
 \con{Q'}{\sigma'}$. By its static conditions, we have
 $d(\tr{\qv{P'}}{\rho'}, \tr{\qv{Q'}}{\sigma'})$ is negligible.
 This implies $|\tr{}{\rho'} - \tr{}{\sigma'}|$ is 
 negligible and thus we have
 that $\tr{}{\sigma'}$ is non-negligible.
 We have $\con{Q}{\E_{\tilde r}(\sigma)}
 \sim \con{R}{\E_{\tilde r}(\theta)}$ applying lemma
 \ref{neg:soclosed} to $\con{Q}{\sigma}
 \sim {\con{R}{\theta}}$. Next by lemma
 \ref{neg:simulateweak}, we have $\con{R}{\E_{\tilde r}(\theta)}
 \weak{\hat \alpha} \con{R'}{\theta'}$ and $\con{Q'}{\sigma'}
 \sim \con{R'}{\theta'}$ for some $\con{R'}{\theta'}$. 
 Therefore, $\con{P'}{\rho'}\sim \circ \sim \con{R'}{\theta'}$.
\end{proof}
We prove congruence of the relation $\sim$. 
We first show that $\sim$ is closed by restriction.
\begin{lem}
\label{neg:resclosed}
 If $\con{P}{\rho} \sim \con{Q}{\sigma}$ holds, then
 $\con{P \backslash L}{\rho} \sim \con{Q \backslash L}{\sigma}$ holds.
\end{lem}
\begin{proof}
It is similarly proven as Lemma \ref{par:resclosed}.
\end{proof}

The next theorem states that the relation $\sim$ is closed
under parallel composition of processes.
With this theorem and Lemma \ref{neg:resclosed}, we immediately 
have that $\sim$ is closed by application of an arbitrary evaluation
context.
The structure of the proof is same as \ref{par:pallaclosed}. 
\begin{thm}
\label{neg:parclosed}
  If $\con{P}{\rho} \sim \con{Q}{\sigma}$, then
 $\con{P||R}{\rho} \sim \con{Q||R}{\sigma}$ for all
 process $R$.
\end{thm}
\begin{proof}
 We define 
\[
 \R := \{(\con{P||R}{\rho}, \con{Q||R}{\sigma})\,|\,
 \con{P}{\rho} \sim \con{Q}{\sigma}, R \in \mathcal{P}
 \}. 
\]
 It is sufficient to show $\R$ is an approximate bisimulation.
 $\R$ is symmetric by the definition.
 Let $(\con{P||R}{\rho}, \con{Q||R}{\sigma})$ be an arbitrary element in
 $\R$. The static conditions and the simulation condition 
 are checked similarly to Theorem \ref{par:pallaclosed}.
\end{proof}
Similarly to the discussion in the previous section,
we have the following corollary.
\begin{col}
\label{neg:congruence}
 If $\con{P}{\rho} \sim \con{Q}{\sigma}$ holds, then
 $\con{C[P]}{\rho} \sim \con{C[Q]}{\sigma}$ holds for
 all evaluation context $C[\_]$.
\end{col}
We have proved that the relation $\sim$ is equivalence and closed
by application of an arbitrary evaluation context.
We thus say it is congruent. The congruence 
suggests sanity of the definition as well as feasibility in practice.
For example, it allows us to infer equivalence of
multiple sessions of protocols.
\begin{col}
\label{neg:multisession}
 If $\con{P_1}{\rho_1 \otimes \rho^E_1} \sim \con{Q_1}{\sigma_1
 \otimes \rho^E_1}$,
$\con{P_2}{\rho_2 \otimes \rho^E_2} \sim \con{Q_2}{\sigma_2 \otimes
 \rho^E_2}$ and $\qv{P_1}\cap \qv{P_2}=\qv{P_1}\cap \qv{Q_2}=\qv{Q_1}\cap
 \qv{P_2}=\qv{Q_1}\cap \qv{Q_2}=\emptyset$
 hold for all $\rho^E_1, \rho^E_2$, then
$\con{P_1 || P_2}{\rho_1 \otimes \rho_2 \otimes \rho^E} \sim \con{Q_1 ||
 Q_2}{\sigma_1 \otimes \sigma_2 \otimes \rho^E}$ holds for all $\rho^E$.
\end{col}
\begin{proof}
We have $\con{P_1}{\rho_1 \otimes \rho_2 \otimes \rho^E}
\sim \con{Q_1}{\sigma_1 \otimes \rho_2 \otimes \rho^E}$ by substituting
$\rho_2 \otimes \rho^E$ for $\rho^E_1$ in the assumption.
By congruence, we have $\con{P_1||P_2}{\rho_1 \otimes \rho_2 \otimes \rho^E} \sim
 \con{Q_1||P_2}{\sigma_1 \otimes \rho_2 \otimes \rho^E}$.
Similarly, we have $\con{Q_1||P_2}{\sigma_1 \otimes \rho_2 \otimes \rho^E} \sim
 \con{Q_1||Q_2}{\sigma_1 \otimes \sigma_2 \otimes \rho^E}$.
By transitivity of $\sim$, we obtain the conclusion.
\end{proof}
Let configurations $\con{P_i}{\rho \otimes \rho^E_i}$ and
$\con{Q_i}{\sigma \otimes \rho^E_i}$ formalize
an actual and an ideal protocols for $i = 1,2$.
By the above corollary, we have
$\con{P_1 || P_2}{\rho_1 \otimes \rho_2 \otimes \rho^E} \sim \con{Q_1 ||
 Q_2}{\sigma_1 \otimes \sigma_2 \otimes \rho^E}$.
This means that $\con{P_1 || P_2}{\rho_1 \otimes \rho_2 \otimes \rho^E}$
is approximately secure,
provided that the ideal protocol is secure even if they run in
parallel. The latter condition depends on protocols but possibly be
satisfied. In fact, EDP-ideal protocol that we consider in the next
chapter satisfies the condition, because Alice and Bob generate
a shared key using pre-shared EPR pairs.
Another example of application is discussed
in the subsection Outputting Secret Keys in Section
\ref{fml:policiesandtechs}.

% \section{Comparison of the Two Relations}
% The relation $\sim$ is 
% defined only when the notion of negligibility makes sense
% while the relation $\sim_{\zeta,\eta}$ is defined only when
% the inequality $\ge$ makes sense. It is not a trivial task
% compare the two relations. It is possible to assume that 
% the parameters $\zeta$ and $\eta$ are functions of security parameters, but
% to define an inequality $\ge$ on such functions appropriately is not a
% trivial task. 

% Let us assume again that quantum states depend on security parameters,
% and for $f,g:\mathbb{N}\rightarrow [0,1]$, define $f \ge g$ iff $f(n)
% \ge g(n)$ for all $n$. If
% $\con{P}{\rho}\sim_{\zeta, \eta}\con{Q}{\sigma}$ holds for some 
% negligible functions $\zeta, \eta$ with respect to $n$,
% then $\con{P}{\rho} \sim \con{Q}{\sigma}$ holds. We do not know, however,
% whether the converse is true. When we assume $\con{P}{\rho} \sim
% \con{Q}{\sigma}$, by the condition 3 in Lemma \ref{neg:coinduction}, we
% have
% \[
%  \forall \E_{\tilde r}:\mbox{CP map}.\, \exists
%  \eta:\mbox{non-negligible}. \cdots
% \]
% but this does not immediately imply
% \[
%  \exists \eta:\mbox{non-negligible}.\, 
% \forall \E_{\tilde r}:\mbox{CP map}. \cdots.
% \]

Although we use only the relation $\sim$ for
the verification in this thesis, the relation 
$\sim_{\zeta, \eta}$ will be useful when we
evaluate the gap of two configurations quantitatively.
By $\con{P}{\rho}\sim \con{Q}{\sigma}$, the value of the trace distance
is simply
understood to be negligible, but it cannot be evaluated more explicitly.
% Besides, let us focus again on parallel composition.
% For the relation $\sim$, we
% have already discussed Corollary \ref{neg:multisession} but
% the number of parallel composition cannot depend on a security
% parameter.
Using the relation $\sim_{\zeta, \eta}$,
the gap can be evaluated concretely. For example,
if $\con{P_i}{\rho_i \otimes \rho^E_i} \sim_{\zeta,\eta}
\con{Q_i}{\sigma_i \otimes \rho^E_i}$ holds
for all $\rho^E_i \in \D(\H_{\sfqv - \qv{P_i}})$
for all $i \in [1..k]$ and $\qv{P_1} \cap \cdots \cap \qv{P_k} =
\emptyset$ holds, then we have
\[
 \con{P_1||\cdots||P_k}{\rho} \sim_{\zeta', \eta'}
 \con{Q_1||\cdots||Q_k}{\sigma},
\]
where $\zeta' = k\zeta$ and $\eta' = (k-1)(k+2)\zeta + \eta$.

\section{Guarantees of Approximate Bisimulation}
\subsection{Application to Verification of QKD protocols' Security}
We explain about feasibility of the relation $\sim$ for verification
of security of QKD protocols. In the next chapter, we will verify
$\con{P}{\rho} \sim \con{Q}{\sigma}$, where
$\con{P}{\rho}$ and $\con{Q}{\sigma}$ are configurations formalizing the EDP-based protocol
(Section \ref{pre:EDPbased}) and EDP-ideal (Section \ref{fml:formalverif2}).
% a sort of cheating protocol named
% EDP-ideal. In EDP-ideal protocol, 
% Alice and Bob initially share EPR pairs, and
% they execute the same protocol as 
% the EDP-based protocol until the decision of continue or aborting
% by the result of the error checking. Only when they
% decide to continue, they create the secret keys using pre-shared EPR
% pairs instead of pairs obtained after the entanglement distillation
% protocol.
% Therefore, if the secret keys is created, Eve has no information
% of the keys.
Assume $\con{P}{\rho} \sim \con{Q}{\sigma}$ and
\[
 \con{P}{\E_{\tilde r}(\rho)} \xrightarrow{\alpha} 
 \con{P_1}{\E^1_{\tilde r_1}(\rho_1)} \xrightarrow{\alpha_1} \cdots 
 \xrightarrow{\sndq{ska}{k_A}} \con{P'}{\rho'}
\]
and $\tr{}{\rho'}$ is non-negligible, where
the last transition $\xrightarrow{\sndq{ska}{k_A}}$
represents that Alice's key $k_A$ is created\footnote{In Chapter 5, we
actually formalize the protocols as configurations
that do such transitions. This point is discussed in Section \ref{fml:outputtingsks}}.
Then, there exists the following
transition
\[
 \con{Q}{\E_{\tilde r}(\sigma)} \weak{\alpha} 
 \con{Q_1}{\E^1_{\tilde r_1}(\sigma_1)} \weak{\alpha_1} \cdots 
 \weak{\sndq{ska}{k_A}} \con{Q'}{\sigma'}
\]
such that $d(\tr{\qv{P'}}{\rho'}, \tr{\qv{Q'}}{\sigma'})$ is negligible.
By Proposition \ref{par:trdisproperty}, we have that
\begin{align*}
 &|\tr{}{\rho'} - \tr{}{\sigma'}| \mbox{ and }
 |\tr{}{\rho'}\cdot\tr{}{\pi\frac{\tr{\qv{P'}}{\rho'}}{\tr{}{\rho'}}} -
 \tr{}{\sigma'}\cdot\tr{}{\pi\frac{\tr{\qv{Q'}}{\sigma'}}{\tr{}{\sigma'}}} | 
\end{align*}
are negligible for all projector $\pi$. Especially, let $\pi$ be
the projector to the subspace where $i$-th bits of Alice's key and
Eve's key are equal. We can rephrase the above expression as follows.
\begin{align*}
 &|\Pr(A) - \Pr(B)| \mbox{ and }
 |\Pr(A)\Pr(k_{A,i} = k_{E,i}|A) - \Pr(B)\Pr(k'_{A,i} = k'_{E,i}|B)|
\end{align*}
are negligible, where 
\begin{itemize}
 \item $k_{A,i}$ and $k_{E,i}$ are random variables of $i$-th bits of
       Alice's and Eve's keys in the EDP-based protocol, 
 \item $k'_{A,i}$ and $k'_{E,i}$ are those in EDP-ideal, and
 \item $A$ and $B$ are the events that
       $\con{P}{\rho}$ reaches $\con{P'}{\rho'}$ and
       $\con{Q}{\sigma}$ reaches $\con{Q'}{\sigma'}$.
\end{itemize}
Moreover, we have
\begin{align*}
    &|\Pr(A)\Pr(k_{A,i} = k_{E,i}|A) - \Pr(B)\Pr(k'_{A,i} = k'_{E,i}|B)|\\
\le &\Pr(A)|\Pr(k_{A,i} = k_{E,i}|A) - \Pr(k'_{A,i} = k'_{E,i}|B)|\\
 &+ \Pr(k'_{A,i} = k'_{E,i}|B)|\Pr(A) - \Pr(B)|\\
= & \Pr(A)|\Pr(k_{A,i} = k_{E,i}|A) - \frac{1}{2}| + \frac{1}{2}|\Pr(A) - \Pr(B)|.
\end{align*}
The equation $\Pr(k'_{A,i} = k'_{E,i}|B) = \frac{1}{2}$ holds by the
definition
of EDP-ideal. If $\Pr(A)$ is greater than negligible, we have
that $|\Pr(k_{A,i} = k_{E,i}|A) - \frac{1}{2}|$ is negligible.
It seems possible to derive that the mutual information of Alice's and
Eve's keys is negligible. Similarly, we have that Alice's and Bob's
keys are identical with overwhelming probability.

\subsection{On Guarantees in General Cases}
As long as we verify security of an actual protocol $\con{P}{\rho}$ by
proving $\con{P}{\rho} \sim \con{Q}{\sigma}$ for an ideal protocol
$\con{Q}{\sigma}$, we can discuss similarly to the previous
subsection.
Concretely, if $\con{P}{\rho} \sim \con{Q}{\sigma}$, then
for all $\con{P'}{\rho'}$ such that
\[
 \con{P}{\E_{\tilde r}(\rho)} \xrightarrow{\alpha} 
 \con{P_1}{\E^1_{\tilde r_1}(\rho_1)} \xrightarrow{\alpha_1} \cdots 
 \xrightarrow{\alpha_m} \con{P'}{\rho'}
\]
and $\tr{}{\rho'}$ is non-negligible,
there exists $\con{Q'}{\sigma'}$ such that
\[
 \con{Q}{\E_{\tilde r}(\sigma)} \weak{\alpha} 
 \con{Q_1}{\E^1_{\tilde r_1}(\sigma_1)} \weak{\alpha_1} \cdots 
 \weak{\alpha_m} \con{Q'}{\sigma'}
\]
and $\con{P'}{\rho'} \sim \con{Q'}{\sigma'}$. 
The last condition implies $d(\tr{\qv{P'}}{\rho'},
\tr{\qv{Q'}}{\sigma'})$ is negligible.
By proposition \ref{par:trdisproperty}, we have
$$|\tr{}{\rho'} - \tr{}{\sigma'}|
\mbox{ and } \tr{}{\rho'}d(\frac{\tr{\qv{P'}}{\rho'}}{\tr{}{\rho'}},
\frac{\tr{\qv{Q'}}{\sigma'}}{\tr{}{\sigma'}})$$
are negligible. 
 We thus have the following conclusions.
\begin{enumerate}
 \item The probabilitiy to reach $\con{P'}{\rho'}$ from
       $\con{P}{\rho}$ is negligibily colse
       to that to reach $\con{Q'}{\sigma'}$ from $\con{Q}{\sigma}$.
 \item The greater $\tr{}{\rho'}$ we have, the less 
       $d(\frac{\tr{\qv{P'}}{\rho'}}{\tr{}{\rho'}},
       \frac{\tr{\qv{Q'}}{\sigma'}}{\tr{}{\sigma'}})$ we have.
       Especially, if $\tr{}{\rho'}$ is greater than negligible,
       then $d(\frac{\tr{\qv{P'}}{\rho'}}{\tr{}{\rho'}},
       \frac{\tr{\qv{Q'}}{\sigma'}}{\tr{}{\sigma'}})$ is negligible.
\end{enumerate}
When the protocol $\con{P}{\rho}$ is for generation of certain data,
the data will be sent to the outside by the
final transition $\xrightarrow{\alpha_m}$ of the form
$\xrightarrow{\sndq{c}{q}}$,
where $\H_q$ is the state space of the data.
By the condition 2 above, we have that 
whenever the probability to reach $\con{P'}{\rho'}$
from the start point $\con{P}{\rho}$ is greater than negligible,
the data have been almost correctly generated
at $\con{P'}{\rho'}$.

Another important way to examine end guarantees of the approximate
bisimulation is to define \emph{testing equivalence}
\cite{DeNicola1984, Deng2009}
or \emph{observational equivalence} 
\cite{Blanchet2008cryptoverif, Yasuda2014} that includes the
approximate bisimulation. Observationally equivalent processes
become ready to use the same channel with the same probability
under the application of an arbitrary evaluation context.
It is an open problem to define such equivalence that
includes the approximate bisimulation.
Before the approximation,
to define testing or observational equivalence that includes
bisimulation in qCCS is still an open problem.
\emph{Reduction barbed congruence} $\approx_r$ is defined
\cite{DengFeng2012} but it
seems more distinguishing than what is meant by observational
equivalence, because it
requires \emph{reduction closedness}:
if $\con{P}{\rho} \approx_r \con{Q}{\sigma}$ and $\con{P}{\rho}
\Rightarrow \mu$ hold, then there exists $\nu$ satisfying
$\con{Q}{\sigma} \Rightarrow \nu$ and $\mu \approx_r \nu$.
The relation $\approx_r$ in fact coincides with the bisimulation
relation $\approx$.
Yasuda defined two kinds of observational equivalence
$\approx_{\mathit{oe}}$ and $\approx_{\mathit{oe}}^{\mathit{st}}$
on qCCS configurations \cite{Yasuda2014} but neither of them includes
$\approx$.

\section{Automated Verification of Approximate Bisimulation}
\subsection{Extension of the Symbolic Representation and  the Algorithm}
\label{neg:extensionofalgorithm}
We extended Verifier1, which is described in 
Chapter \ref{symqccs}, to verify the approximate bisimilarity.
Let us call the extended verifier \emph{Verifier2}.
We applied Verifier2 to the second part of
Shor-Preskill's security proof. This is described in the next chapter.

Since the outsider's quantum operations are assumed 
to be CP (Definition \ref{neg:defofapproxbisim}),
it was necessary to extend the syntax of symbolic representations of
probability-weighted quantum states. The set $\mathcal{S}$ of the symbolic
representations was defined in Definition \ref{symqccs:symbrep}.
In the definition, we assumed $\mathit{op}$ is an element of 
$S_{\mathit{op}}$, which is a set of TPCP map symbols.
The expressions 
$\mathtt{proj0}\mathtt{[}b\mathtt{](}\rho\mathtt{)}$ and
$\mathtt{proj1}\mathtt{[}b\mathtt{](}\rho\mathtt{)}$ 
represent applications of special CP maps
$\ket{0}\bra{0}_b\braw{\rho}\ket{0}\bra{0}_b$ and
$\ket{1}\bra{1}_b\braw{\rho}\ket{1}\bra{1}_b$
but application of general CP maps
was not allowed in the syntax.

For the extension, we first add a definition
\begin{itemize}
 \item $S_{\mathit{cp}}$ is a set of symbols representing CP maps.
       Symbols $\mathtt{proj0}$ and $\mathtt{proj1}$ are elements of
       $S_{\mathit{cp}}$.
\end{itemize}
We then define the set $\mathcal{S}'$ of extended
symbolic representations. In the definition, 
we do not refer $S_\mathit{op}$ because
TPCP maps are also CP maps.
\begin{defi}
\label{neg:exsymbrep}
\begin{align*}
\mathcal{S}' \ni \rho, \sigma ::=\,\, &X\mathtt{[}\tilde q\mathtt{]} ~|~
\mathit{cp}\mathtt{[}\tilde q\mathtt{](}\rho\mathtt{)} ~|~ \rho \,
  \mathtt{*} \, \sigma
~|~\mathtt{Tr}\mathtt{[}\tilde q\mathtt{](}\rho\mathtt{)}
\end{align*}
where $\mathit{cp} \in S_{\mathit{cp}}$.
\end{defi}

The interpretation $\braw{\cdot}:\mathcal{S}' \rightarrow \Delta(\H)$
is naturally extended.
Verifier2 obeys the same transition rules as the
previous one, and
takes as input two elements in
$\mathcal{P} \times \mathcal{S'}$,
a user-defined set of equations $\mathit{eqs}$
on symbolic quantum states,
and additionally a user-defined set of triples
$\mathit{inds} \subseteq \mathcal{S'} \times \mathcal{S'} \times
S_\mathit{nat}$, which we call {\it indistinguishability expressions}.
An indistinguishability expression $(\rho, \sigma, n)$ intuitively means
the trace distance of $\rho$ and $\sigma$ is negligible with respect to
$n$.

We modified the steps 1, 4, 5, 6, and 7 in the algorithm described in
Section \ref{symqccs:algorithmforbisim}.
The new algorithm of the recursive procedure is as follows.
\begin{enumerate}
 \item The procedure takes as input two configurations
       $\con{P_0}{\rho_0}$,
       $\con{Q_0}{\sigma_0}$ and user-defined equations $\mathit{eqs}$
       and indistinguishability expressions $\mathit{inds}$
       on quantum states.
 \item If $P_0$ and $Q_0$ can perform any $\tau$-transitions of
       TPCP map applications, they are all performed at this point.
       Let $\con{P}{\rho}$ and $\con{Q}{\sigma}$ be obtained
       configurations.
 \item Whether $\qv{P} = \qv{Q}$ is checked. If it does not hold, the
       procedure returns $\mathit{false}$.
 \item Whether $\ptrtt{\qv{P}}{\rho} = \ptrtt{\qv{Q}}{\sigma}$ is
       checked using $\eqs$ and $\inds$.
       If it does not hold, the procedure returns $\mathit{false}$.
 \item A new {\it CP} map symbol $\E\texttt{[}\qv{\rho} - \qv{P}\texttt{]}$
       that stands for an
       arbitrary operation is generated. 
 \item For each $\con{P'}{\rho'}$ such that
       $\con{P}{\E\texttt{[}\qv{\rho} -
       \qv{P}\texttt{](}\rho\texttt{)}} \xrightarrow{\alpha}
       \con{P'}{\rho'}$,
       the procedure checks whether there exists $\con{Q'}{\sigma'}$ such
       that \\
       $\con{Q}{\E\texttt{[}\qv{\sigma} -
       \qv{Q}\texttt{](}\sigma\texttt{)}} \weak{\hat \alpha}
       \con{Q'}{\sigma'}$ and 
       the procedure returns $\mathit{true}$ with input
       $\con{P'}{\rho'}$,
       $\con{Q'}{\sigma'}$, and $\mathit{eqs}$. If there exists, it goes
       to the
       next step 7. Otherwise, it returns $\mathit{false}$.
 \item For each $\con{Q'}{\sigma'}$ such that
       $\con{Q}{\E\texttt{[}\qv{\sigma} -
       \qv{Q}\texttt{](}\sigma\texttt{)}} \xrightarrow{\alpha}
       \con{Q'}{\sigma'}$,
       the procedure checks whether there exists $\con{P'}{\rho'}$ such
       that\\
       $\con{P}{\E\texttt{[}\qv{\rho} -
       \qv{P}\texttt{](}\rho\texttt{)}} \weak{\hat \alpha}
       \con{P'}{\rho'}$ and 
       the procedure returns $\mathit{true}$ with input
       $\con{P'}{\rho'}$ and
       $\con{Q'}{\sigma'}$, and $\mathit{eqs}$. If there exists, it
       returns
       $\mathit{true}$. Otherwise, it returns $\mathit{false}$.
\end{enumerate}
The way to use $\inds$ to test $\ptrtt{\qv{P}}{\rho} =
\ptrtt{\qv{Q}}{\sigma}$
is similar to that of $\eqs$, that is,
for $(\rho, \sigma, n) \in \inds$,
a part in an objective quantum state
that matches to $\rho$ is rewritten to $\sigma$.

\subsection{Correctness of the Extended Verifier}
A user-defined set $\inds$ is said to be valid if for all
element $(\rho, \sigma, n) \in \mathit{inds}$, $d(\braw{\rho},
\braw{\sigma})$ is a negligible function of $\braw{n}$.
Let $\mathit{eqs}$ and $\inds$ be valid.
Let a relation $\R_{\eqs, \inds} \subseteq \C \times \C$ be defined
as follows.
\begin{align*}
 \R_{\eqs, \inds} :=
 \{&(\con{P}{\braw{\rho}},
 \con{Q}{\braw{\sigma}}) \,|\,
 \mbox{ Verifier2 returns } \mathit{true}
 \mbox{ with }\\
 & \con{P}{\rho}, \con{Q}{\sigma} \mbox{ using } \mathit{eqs}
 \mbox{ and } \mathit{inds}.
\}
\end{align*}
The relation $\R_{\eqs, \inds}$ is an approximate bisimulation
relation.

The argument about the correctness of Verifier2
is basically similar to that about the original one.
We focused on the two different points. The first is
that whether the trace distance of objective quantum
states is negligible or not is tested instead of the equality of
the states. The second point is about the simulation condition.

\subsubsection{On the Static Condition of Partial Trace}
It is necessary to check the partial rewriting of quantum states
done by Verifier2 is correct. It rewrites a
symbolic representation of the form $\rho_l \, \mathtt{*}\,
 \rho \,\mathtt{*}\, \rho_r$ to
the symbolic representation $\rho_l \, \mathtt{*}\, \sigma
 \,\mathtt{*}\, \rho_r$ given
$(\rho, \sigma, n) \in \inds$. 
The correctness is guaranteed from the fact that 
$d(X, Y) = d(X_l \otimes X \otimes X_r,
X_l \otimes Y \otimes X_r)$ holds for all $X, Y, X_l, X_r \in
\Delta(\H)$. If $(\rho, \sigma, n)$ is valid, namely 
$d(\braw{\rho}, \braw{\sigma})$ is negligible with respect to $n$, 
then $d(\braw{\rho_l \, \mathtt{*}\,
 \rho \,\mathtt{*}\, \rho_r},
\braw{\rho_l \, \mathtt{*}\,
 \sigma \,\mathtt{*}\, \rho_r})$ is negligible
 with respect to $n$ because
$d(\braw{\rho_l \, \mathtt{*}\,
 \rho \,\mathtt{*}\, \rho_r},
\braw{\rho_l \, \mathtt{*}\,
 \sigma \,\mathtt{*}\, \rho_r}) =
d(\braw{\rho_l} \otimes
 \braw{\rho} \otimes \braw{\rho_r},
\braw{\rho_l} \otimes
 \braw{\sigma} \otimes \braw{\rho_r}
)=
d(\braw{\rho},\braw{\sigma})
$ holds.

\subsubsection{On the Simulation Condition}
The simulation condition of approximate bisimulation is only required
to transitions with non-negligible probability, stating
\begin{itemize}
 \item If $\con{P}{\E_{\tilde r}(\rho)}
       \xrightarrow{\alpha}
       \con{P'}{\rho'}$ holds and $\tr{}{\rho'}$ is
       non-negligible, then\\
       $\con{Q}{\E_{\tilde r}(\sigma)} \weak{\hat \alpha}
       \con{Q'}{\sigma'}$ and 
       $\con{P'}{\rho'} \R \con{Q'}{\sigma'}$ hold
       for some $\con{Q'}{\sigma'}$.
\end{itemize}
However, Verifier2 does not check whether the probability of a
transition is non-negligible or not. It returns $\mathit{false}$
when a transition cannot be simulated even if the probability is
negligible. As for simulation,
the condition that Verifier2 returns $\mathit{true}$
is strictly stronger than the simulation condition of the approximate
bisimulation.

\section{Discussion}
\subsection{Relation between the Verifiers}
Before the extension, the steps 6 and 7 required the 
correspondence of qubit variable $b$ when there is a transition
caused by (Meas0) or (Meas1) rules.
%, in other words, caused by a part of a process of the form
%$\measure{b}{P}$.
The proof of Verifier1's soundness was made easier by this condition.
Verifier2 does not check such correspondence, which straightforwardly
checks the
conditions stated in the definition of the relation $\sim$.
On this point, Verifier1 checks more strict condition.
On the other hand, for the step 5, Verifier2
generates a CP map symbol representing the outsider's operation, which
is not necessarily TPCP, while Verifier1 generates a TPCP map symbol.
On this point, Verifier2 checks more strict condition.

In fact, whether an outsider's operation is TPCP or CP
does not matter in verification in most cases.
Only the rewriting rule (3.2) of
partial traces in Section \ref{symqccs:traceoutalgo} cannot
be applied if $\mathit{op}$ is a CP map symbol. Except for
it, quantum operators are treated equivalently in the both verifiers
whether they are TPCP or CP.
Moreover, even if outsider's operators are TPCP,
the rule (3.2) cannot be applied to them
in most cases, where we assume the outsider has her own
``local memory''.
Assume she does not send a quantum variable $q^E$
to the insider (the process).
It is in the domain of her operation $\E$ in general.
Hence, an arbitrary symbol generated in the step 5 is of the form 
$\E\texttt{[}\cdots, q^E, \cdots\texttt{]}$.
However, she does not send $q^E$ to the process.
Therefore, in any expressions of the 
form 
\[
 \ptrtt{\tilde q}{\cdots\E\texttt{[}\cdots, q^E, \cdots\texttt{]}
 \cdots},
\]
which appear in the test of equality of partial traces, 
$q^E \notin \tilde q$ holds and thus the rule (3.2) cannot be applied.

Let us now consider a new verifier, which we call Verifier3,
that generates {\it CP} maps in the 
step 5 but executes the same algorithm as Verifier1.
We have that Verifier3 verifies more strict condition
than Verifier2, in other words, if Verifier3 returns 
$\mathit{true}$ with configurations $\con{P}{\rho}$, $\con{Q}{\sigma}$,
and user-defined equations $\mathit{eqs}$, Verifier2
returns $\mathit{true}$ with the same input.
 % negligible
\chapter{Formal Verification of Quantum Cryptographic Protocols Using
the Verifiers}
\label{formaliz}
\section{Overview}
We implemented a software tool to verify weak bisimilarity 
(being in the relation $\approx$) of
configurations of the original qCCS \cite{DengFeng2012}.
In Chapter \ref{symqccs}, we described
the design and soundness of the verifier, which
we call Verifier1.
In Chapter \ref{prob_bisim},
we defined the approximate bisimulation relation $\sim$ and
extended the verifier to verify approximate
bisimilarity (being in the relation $\sim$).
We call the extended one Verifier2.
We summarized the difference
of them in Table \ref{fml:differofverifers}.
\begin{table}[htb]
  \begin{tabular}{|l|c|c|} \hline
   ~ & Verifier1 & Verifier2 \\ \hline
   syntax of processes & $\mathcal{P}$ & $\mathcal{P}$ \\
   syntax of symb. rep. & $\mathcal{S}$ (Def. \ref{symqccs:symbrep})
       & $\mathcal{S'}$
	   (Def. \ref{neg:exsymbrep}) \\
   the outsider performs & TPCP maps & CP maps \\
   rewriting using & $\mathit{eqs}$ & $\mathit{eqs}$ and
	   $\mathit{inds}$ \\
   %transition rule & no limitation & (Comm) rule is limited \\
   the relation to verify & $\approx$ in the original
       \cite{DengFeng2012} & $\sim$ defined in Chapter \ref{prob_bisim}
	   \\ 
   applied to verify & $\mathtt{BB84} \approx \mathtt{EDPbased}$ & 
	   $\mathtt{BB84} \sim \mathtt{EDPbased} \sim
	   \mathtt{EDPideal}$ \\\hline
  \end{tabular}
\caption{Difference of the Verifiers}
\label{fml:differofverifers}
\end{table}
The package of the verifiers is available from the following URL.
\url{http://hagi.is.s.u-tokyo.ac.jp/~tk/qccsverifer.tar.gz}

In this chapter,
we describe the applications of the verifiers to
Shor and Preskill's security proof of BB84. The formal verification
consists of the following 2 steps.
\begin{enumerate}
 \item BB84 and the EDP-based protocol are formalized as configurations
       $\mathtt{BB84}$ and $\mathtt{EDPbased}$.
       Bisimilarity of the configurations is verified 
       by Verifier1.
 \item A new protocol EDP-ideal is defined.
       In the protocol, Alice and Bob 
       initially share EPR-pairs, whose
       number is the same as that of the secret key's bit length.
       Apart from that, they executes the same protocol
       as the EDP-based protocol before creating their secret keys.
       When the protocol is not aborted, they create their secret keys
       just measuring their halves of pre-shared EPR-pairs.
       Since the pre-shared EPR pairs will not be influenced
       by Eve, Alice and Bob can create a shared secret key
       without leaking any information.
       The protocol EDP-ideal is formalized as a configuration 
       $\mathtt{EDPideal}$. Approximate bisimilarity of 
       $\mathtt{EDPbased}$ and $\mathtt{EDPideal}$ is 
       verified by Verifier2.
\end{enumerate}
%We also have $\mathtt{BB84}$ and $\mathtt{EDPideal}$ are
%approximately bisimilar.

\section{Input and Output for the Verifiers}
\subsection{Scripts}
Input files, which we call scripts,
contain the following descriptions.
Although the algorithms are different,
scripts for the verifiers are almost the same: only Verifier2
takes indistinguishability expressions.

Before formalizing processes and
quantum states, symbols need to be declared.
\begin{itemize}
\item natural number symbols, which are 
      elements of $S_{\mathit{nat}}$, in the form \\
      {\tt nat $\mathit{n}$;}.
\item channel names, which are elements of $\mathit{qChan}$,
      in the form \\{\tt channel $\mathit{c}$ :
      $\mathit{n}$;},\\
      where $\mathit{n}$
      is a natural number symbol defined beforehand. Through channel $c$,
      quantum variables with length $\mathit{n}$ are communicated.
\item quantum variables, which are elements of $\sfqv$, 
      in the form \\
      {\tt qvar $\mathit{q}$ : $\mathit{n}$;},\\
      where $n$ is the qubit-length of $q$.
\item symbols of quantum states, which are 
      elements of $S_{\mathit{stat}}$, in the form\\
      $\texttt{dsym}$ $\mathit{X}$ $\texttt{:}$ $n_1 \texttt{,}
      ...\texttt{,}
      n_k\texttt{;}$.\\
      $X$ is a quantum state which $k$ quantum variables with
      qubit-length
      $n_1,...,n_k$ are in. ``dsym'' stands for 
      ``density operator symbol''.
\item symbols of TPCP maps (or CP maps), which are elements of
      $S_{\mathit{op}}$ (or $S_{\mathit{cp}}$), in the form\\
      {\tt operator $\mathit{op}$ : $n_1 \texttt{,} ...\texttt{,}
      n_k\texttt{;}$}.\\
      $\mathit{op}$ acts on 
      quantum variables with qubit-length $n_1,...,n_k$.
      For Verifier2, user-defined CP map symbols are assumed to 
      represent TPCP. Recall that for the process construction 
      $\op{\mathit{op}}{\tilde q}.P$, $\mathit{op}$ is assumed to
      be a TPCP map.
\end{itemize}
Processes, quantum states, configurations, and 
equations on quantum states are then defined.
\begin{itemize}
\item A process is defined in the form\\
      {\tt process} {\it process\_name}\\
      ~$P$\\
      {\tt end}.
\item A quantum state is defined in the form\\
      {\tt environment} {\it environment\_name}\\
      ~$\rho$\\
      {\tt end}.
\item A configuration is defined in the form\\
      {\tt configuration}\\
      ~{\tt proc} {\it process\_name}\\
      ~{\tt env}  {\it environment\_name}\\
      {\tt end}.
\item An equation is defined in the form\\
      {\tt equation} {\it equation\_name}\\
      ~$\rho\mathrel{\texttt{=}}\sigma$\\
      {\tt end},\\
      where $\rho$ and $\sigma$ are quantum states. For a quantum
      state, description $\texttt{\_\_}\texttt{[}\tilde q\texttt{]}$ is
      permitted, which
      matches arbitrary quantum state of $\tilde q$.
\end{itemize}
Only for Verifier2, indistinguishability expressions
on quantum states are defined.
\begin{itemize}
 \item An indistinguishability expression is defined in the form\\
       {\tt indistinguishable} {\it indexpression\_name} $n$\\
       ~$\rho\mathrel{\texttt{=}}\sigma$\\
       {\tt end},\\
       where $n$ is a natural number symbol, and
       $\rho$ and $\sigma$ are quantum state symbols.
       For a quantum
       state, description $\texttt{\_\_}\texttt{[}\tilde q\texttt{]}$ is
       permitted, which
       matches arbitrary quantum state of $\tilde q$.
       Moreover, as a CP map, description
       $\texttt{\_\_}\texttt{[}\tilde q\texttt{]}$ is permitted,
       which matches an arbitrary CP map acting on $\tilde q$.
\end{itemize}

\subsection{Outputs}
If no option is set, the verifiers find two configuration in a
script, verify their (approximate) bisimilarity using the defined
equations (and indistinguishability expressions),
and then output $\mathit{true}$ or $\mathit{false}$.
The verifiers have options with which they 
show information for debugging.
The information is about
the reason why the recursive procedure returns $\mathit{false}$.
Concretely, for configurations $\con{P}{\rho}$ and $\con{Q}{\sigma}$, 
the verifiers show
\begin{enumerate}
 \item $P$, $Q$, $\qv{P}$, and $\qv{Q}$
       if $\qv{P} \neq \qv{Q}$.
 \item $\tr{\qv{P}}{\rho}$ and $\tr{\qv{Q}}{\sigma}$ if
       $\tr{\qv{P}}{\rho} \neq \tr{\qv{Q}}{\sigma}$\\
       (or $d(\tr{\qv{P}}{\rho}, \tr{\qv{Q}}{\sigma})$ 
       is not verified to be negligible.)
 \item $\alpha$, $P$, and $Q$ if $Q \not\xrightarrow{\alpha}$
       and $P \xrightarrow{\alpha}$, or 
       $P \not\xrightarrow{\alpha}$
       and $Q \xrightarrow{\alpha}$.
\end{enumerate}
Especially, the information 2 can be used for finding equations that are
necessary for the verification.
For more details, readers can find user manual
of the verifier contained in the package.

We next introduce as examples formal verification of correctness of
the quantum teleportation protocol and the super dense coding
protocol using Verifier1. The protocols were formally verified in
\cite{FengDuanYing2011} using bisimulation. Our way of formalization
is slightly different from theirs, because we represent classical data
as quantum data.
\begin{figure}
\begin{minipage}{0.5\hsize}
\begin{verbatim}
nat 2;
nat m;
channel c : 2;
channel d : 1;
qvar q  : 1;
qvar q1 : 1;
qvar q2 : 1;
qvar x : 2;
qvar qE : m;
dsym EPR : 1,1;
dsym ZERO : 2;
dsym AFTER : 1,1,2;
dsym ANY : 1;
dsym EVE : m;
operator cnot : 1,1;
operator hadamard : 1;
operator measure : 1,1,2;
operator telproc : 2,1;
operator swap : 1,1;

process Tel_Proc
 ((cnot[q,q1].
   hadamard[q].
   measure[q,q1,x].
   c!x.discard(q,q1)
 ||
   c?y.telproc[y,q2].
   d!q2.discard(y)
 )/{c})
end

environment Tel_Env
 EPR[q1,q2] * ZERO[x]
 * ANY[q] * EVE[qE]
end
\end{verbatim}
\end{minipage}
\begin{minipage}{0.5\hsize}
\begin{verbatim}
configuration Tel
 proc Tel_Proc
 env  Tel_Env
end

process TelSpec_Proc
 swap[q,q2].d!q2.discard(q1,x,q)
end

environment TelSpec_Env
 EPR[q1,q2] * ZERO[x]
 * ANY[q] * EVE[qE]
end

configuration TelSpec
 proc TelSpec_Proc
 env  TelSpec_Env
end

equation E1
 telproc[x,q2](measure[q,q1,x](
 hadamard[q](cnot[q,q1](
 EPR[q1,q2] * ZERO[x] * ANY[q])
 )))
 = 
 ANY[q2] * AFTER[q,q1,x]
end

equation E2
 swap[q,q2](EPR[q1,q2] * ANY[q])
 =
 EPR[q1,q] * ANY[q2]
end
\end{verbatim}
\end{minipage}
\caption{Formalization of Quantum Teleportation}
\label{fml:codetel}
\end{figure}

\begin{ex}
\label{fml:teleportation}
 An example of formal verification of the quantum teleportation protocol is
 shown in Figure \ref{fml:codetel}.
 The protocol is formalized as a configuration {\tt
 Tel}. A configuration {\tt TelSpec} is a specification of the protocol, which
 merely swaps input's and output's quantum states.
 With equation {\tt E1} and {\tt E2}, {\tt Tel} and {\tt TelSpec} are
 automatically proven to be bisimilar.
\end{ex}
The interpretations of natural number symbols, TPCP maps and
quantum states in the script of Example \ref{fml:teleportation} are as follows.
\begin{itemize}
\item The natural number symbol {\tt 2} is interpreted to the
      natural number $2$.
      {\tt m} is interpreted to an arbitrary natural number $m$.
\item The quantum state symbols are interpreted as follows.
      \begin{itemize}
       \item $\braw{\mathtt{EPR}} = (\frac{\ket{00}+\ket{11}}{\sqrt{2}})
	     (\frac{\ket{00}+\ket{11}}{\sqrt{2}})^\dagger$
       \item $\braw{\mathtt{ZERO}} = \ket{00}\bra{00}$
       \item $\braw{\mathtt{AFTER}} = \frac{1}{4}(
	     \ket{0000}\bra{0000}+\ket{0101}\bra{0101}
	     +\ket{1010}\bra{1010}+\ket{1111}\bra{1111})$
       \item {\tt ANY} and {\tt EVE} are interpreted to arbitrary quantum
	     states with dimension $1$ and $m$.
      \end{itemize}
\item The TPCP map symbols 
      are interpreted as follows.
      \begin{itemize}
       \item $\braw{\texttt{cnot}}_{q,r}$ is CNOT operator
	     in which the control qubit is $q$ and the target qubit
	     is $r$.
       \item $\braw{\texttt{hadamard}}_s$ is Hadamard
	     transformation to $s$.
       \item $\braw{\texttt{swap}}_{t,u}$ is the operation
	     swapping the state of $t$ and $u$.
       \item $\braw{\texttt{measure}}(\cdot) = A(\cdot)A^\dagger$,
	     where $A = \ket{00}\bra{00}\otimes I \otimes
	     I + \ket{01}\bra{01} \otimes I \otimes X + 
	     \ket{10}\bra{10} \otimes X \otimes I + \ket{11}\bra{11}
	     \otimes X \otimes X$.     
       \item $\braw{\texttt{telproc}}(\cdot) = B(\cdot)B^\dagger$,
	     where $B = \ket{00}\bra{00}\otimes I +
	     \ket{01}\bra{01} \otimes X + 
	     \ket{10}\bra{10} \otimes Z + \ket{11}\bra{11} \otimes XZ$.
      \end{itemize}
\end{itemize}
Under the above definitions of interpretations, 
validity of equations {\tt E1} and {\tt E2} are checked by hand.

\begin{figure}
\begin{minipage}{0.5\hsize}
\begin{verbatim}
nat 2;
nat m;
channel c : 1;
channel d : 2;
qvar q  : 2;
qvar q1 : 1;
qvar q2 : 1;
qvar x  : 2;
qvar qE : m;
dsym EPR : 1,1;
dsym ZERO : 2;
dsym ANY2bit : 2;
dsym EVE : m;
operator cnot : 1,1;
operator hadamard : 1;
operator swap : 2,2;
operator measure : 1,1,2;
operator sdcproc : 2,1;

process Sdc_Proc
 ((sdcproc[q,q1].
   c!q1.discard(q)
 ||
   c?y.cnot[y,q2].
   hadamard[y].
   measure[y,q2,x].
   d!x.discard(y,q2)
 )/{c})
end

environment Sdc_Env
 EPR[q1,q2] * ZERO[x] *
 ANY2bit[q] * EVE[qE]
end
\end{verbatim}
\end{minipage}
\begin{minipage}{0.5\hsize}
\begin{verbatim}
configuration Sdc
 proc Sdc_Proc
 env  Sdc_Env
end

process SdcSpec_Proc
 swap[q,x].d!x.discard(q,q1,q2)
end

environment SdcSpec_Env
 EPR[q1,q2] * ZERO[x] *
 ANY2bit[q] * EVE[qE]
end

configuration SdcSpec
 proc SdcSpec_Proc
 env  SdcSpec_Env
end

equation E1
 Tr[q1,q2,q](
  measure[q1,q2,x](
  hadamard[q1](
  cnot[q1,q2](
  sdcproc[q,q1](
   EPR[q1,q2] * ZERO[x]
   * ANY2bit[q])))))
 = 
 ANY2bit[x]
end

equation E2
 swap[q,x](ZERO[x] * ANY2bit[q])
 =
 ANY2bit[x] * ZERO[q]
end
\end{verbatim}
\end{minipage}
\caption{Formalization of Super Dence Coding}
\label{fml:codesdc}
\end{figure}

\begin{ex}
\label{fml:densecoding}
 An example of formal verification of the super dense cording protocol is
 shown in Figure \ref{fml:codesdc}.
 The protocol is formalized as a configuration {\tt
 Sdc}. A configuration {\tt SdcSpec} is a specification of the protocol,
 which merely swaps input's and output's quantum states.
 With equation {\tt E1} and {\tt E2}, {\tt Sdc} and {\tt SdcSpec} are
 automatically proven to be bisimilar.
\end{ex}
The interpretations of natural number symbols, TPCP maps and
quantum states in the script of Example \ref{fml:densecoding} are as follows.
\begin{itemize}
\item The natural number symbol {\tt 2} are interpreted to the
      natural number $2$.
      {\tt m} is interpreted to an arbitrary natural number $m$.
\item The quantum state symbols are interpreted as follows.
      \begin{itemize}
       \item $\braw{\mathtt{EPR}} = (\frac{\ket{00}+\ket{11}}{\sqrt{2}})
	     (\frac{\ket{00}+\ket{11}}{\sqrt{2}})^\dagger$
       \item $\braw{\mathtt{ZERO}} = \ket{00}\bra{00}$
       \item The symbol {\tt ANY2bit} is interpreted to either
	     $\ket{00}\bra{00}$, $\ket{01}\bra{01}$,
	     $\ket{10}\bra{10}$, or $\ket{11}\bra{11}$.
       \item The symbol {\tt EVE} is interpreted to arbitrary quantum
	     states with dimension $m$.
      \end{itemize}
\item TPCP map symbols 
      are interpreted as follows.
      \begin{itemize}
       \item $\braw{\texttt{cnot}}_{q,r}$ is CNOT operator
	     in which the control qubit is $q$ and the target qubit
	     is $r$.
       \item $\braw{\texttt{hadamard}}_s$ is Hadamard
	     transformation to $s$.
       \item $\braw{\texttt{swap}}_{t,u}$ is the operation
	     swapping the state of $t$ and $u$.
       \item $\braw{\texttt{measure}}(\cdot) = A(\cdot)A^\dagger$,
	     where $A = \ket{00}\bra{00}\otimes I \otimes
	     I + \ket{01}\bra{01} \otimes I \otimes X + 
	     \ket{10}\bra{10} \otimes X \otimes I + \ket{11}\bra{11}
	     \otimes X \otimes X$.     
       \item $\braw{\texttt{sdcproc}}(\cdot) = B(\cdot)B^\dagger$,
	     where $B = \ket{00}\bra{00}\otimes I +
	     \ket{01}\bra{01} \otimes X + 
	     \ket{10}\bra{10} \otimes Z + \ket{11}\bra{11} \otimes XZ$.
      \end{itemize}
\end{itemize}
Under the above definitions of interpretations, 
validity of equations {\tt E1} and {\tt E2} are checked by hand. 

\section{Policies and Techniques of Formalization}
\label{fml:policiesandtechs}
\subsection*{Naming of Quantum Variables}
There are principals Alice, Bob, and Eve in QKD protocols that we
consider.
For readability, quantum variables that Alice, Bob, and Eve initially
have are appended with $\texttt{\_A}$, $\texttt{\_B}$ and
$\texttt{\_E}$ respectively in the scripts.
The exception is that $\texttt{EVE\_2[r\_B]}$ is 
initially the state of Eve's variable but she will be able to
send it to Bob through $\texttt{c2?r\_B}$ because $\texttt{c2}$ is
public. This means that arbitrary quantum state that
Eve has prepared can be sent to Bob through the public channel.

\subsection*{Formalization of Channels}
As in general QKD protocols, three kinds of channels are used:
public quantum channels, private classical channels, and public
no-interpolate
classical channels.
Whether values
themselves are quantum or classical does not matter here, since
classical values are expressed as quantum states.
A diagonal density operator can be regarded to represent a 
classical value. Let us say that a quantum variable $q$ is assigned
a classical value when $q$'s quantum state is represented as a 
diagonal operator.

Since the syntax has channel restriction $P \backslash L$,
formalization of the private channels is straightforward.
The public quantum channels and the public
no-interpolate classical channels are realized by copying
the data.
If a quantum variable $q$ that is assigned a classical value
is sent through a public no-interpolate channel $c$, this is 
formalized as
\[
...\texttt{copy[}q\texttt{,}Q\texttt{].}c\texttt{!}q\texttt{.}d\texttt{!}Q...\backslash
\{...,c,...\},
\]
where $Q$ is a new quantum variable, an operator $\texttt{copy}$ 
copies the value of $q$ to $Q$, and $d$
is a new non-restricted channel. Concretely, the operator
$\texttt{copy[}q,Q\texttt{]}$ initializes the state of $Q$ to
$\ket{0 \cdots 0}\bra{0 \cdots 0}$ and apply CNOT with each
qubit of $q$ as the control and each qubit of $Q$ as the target.
The variable $q$ will be securely 
sent through the restricted channel $c$ and Eve
obtains the same value accessing $Q$ through the public channel
$d$. 

\subsection*{Aborting}
Error checking and aborting is important in QKD protocols.
When Alice and Bob decide to abort an execution of a protocol, what they
do after the aborting is often not explicitly written
\cite{BennetBrassard1984, ShorPreskill2000}.
Although there are several possibilities, we merely write
processes that do nothing after the aborting.

\subsection*{Outputting Secret Keys}
\label{fml:outputtingsks}
In our formalization, the processes of QKD protocols 
send the completed secret keys
to the outside and terminate keeping quantum variables
that need not be sent to the outside.
The purpose is to verify that
the protocols produce the identical keys in BB84 and the EDP-based
protocol (or approximately identical keys in the EDP-based protocol and
EDP-ideal).

Congruence of the relation $\sim$
(Theorem \ref{neg:congruence}, in Chapter \ref{prob_bisim}) is 
useful in checking the behavior of the configurations of QKD under the presence
of additional processes.
Let us write 
the configurations of EDP-ideal and BB84 as
$\con{\mathit{EDPideal}}{\rho}$ and $\con{\mathit{BB84}}{\sigma}$.
By congruence, if
$\con{\mathit{EDPideal}}{\rho} \sim
\con{\mathit{BB84}}{\sigma}$ holds, which can be verified using
Verifier2, we have
\[
 \con{\mathit{EDPideal}||
P_{\mathrm{Alice}}||
P_{\mathrm{Bob}}
}{\rho} \sim
 \con{\mathit{BB84}||
P_{\mathrm{Alice}}||
P_{\mathrm{Bob}}
}{\sigma},
\]
where $P_{\mathrm{Alice}}$ and $P_{\mathrm{Bob}}$
are processes that run after given the secret keys from
$\mathit{EDPideal}$ or $\mathit{BB84}$.
Moreover, we have
\[
 \con{(\mathit{EDPideal}||
P_{\mathrm{Alice}}||
P_{\mathrm{Bob}})\backslash \{\mathit{cka}, \mathit{ckb}\}
}{\rho} \sim
 \con{(\mathit{BB84}||
P_{\mathrm{Alice}}||
P_{\mathrm{Bob}})\backslash \{\mathit{cka}, \mathit{ckb}\}
}{\sigma},\]
where $\mathit{cka}$ and $\mathit{ckb}$ are
secret channels to communicate Alice's and Bob's key.
This suggests that we immediately have that
$P_{\mathrm{Alice}}$ and $P_{\mathrm{Bob}}$ behave equivalently
with secret keys created by $\con{\mathit{EDPideal}}{\rho}$ and
$\con{\mathit{BB84}}{\sigma}$.

\section{Formal Verification of Equivalence of BB84 and \\
the EDP-based Protocol}
\label{fml:formalverif1}
\begin{figure}
\begin{minipage}{0.5\hsize}
\begin{verbatim}
process EDPbased
 ((hadamards[q2_A,r2_A,s_A].
  shuffle[q2_A,r2_A,t_A].
  c1!q2_A.c2!r2_A.c3?a_A.
  copyN[t_A,T_A].c4!t_A.d1!T_A.
  copy2n[s_A,S_A].c5!s_A.d2!S_A.
  measure[q1_A].
  c6?u_A.
  abort_alice[q1_A,u_A,b1_A].
  copy1[b1_A,b2_A].
  copy1[b1_A,B_A].
  c7!b1_A.d3!B_A.
  meas b2_A then
   css_projection[r1_A,x_A,z_A].
   copyn[x_A,X_A].
   css_decode[r1_A,x_A,z_A].
   measure[r1_A].
   c8!x_A.d4!X_A.
   c9!z_A.barrier!f_A.
   cka!r1_A.
   discard(q1_A,b2_A,a_A,
           u_A,v1_A,v_B)
  saem
 ||
  c1?q_B.c2?r_B.
  c3!a_B.d5!A_B.
  c4?t_B.unshuffle[q_B,r_B,t_B].
  c5?s_B.hadamards[q_B,r_B,s_B].
  measure[q_B].
  copyn[q_B,Q_B].c6!q_B.d6!Q_B.
  c7?b_B.
\end{verbatim}
\end{minipage}
\begin{minipage}{0.5\hsize}
\begin{verbatim}
   meas b_B then 
    c8?x_B.c9?z_B.
    css_syndrome[r_B,x_B,z_B,
                 sx_B,sz_B].
    css_correct[r_B,sx_B,sz_B].
    css_decode[r_B,x_B,z_B].
    measure[r_B].
    barrier?f_B.
    ckb!r_B.
    discard(b_B,s_B,t_B,x_B,
            z_B,sx_B,sz_B,f_B)
   saem)/{c3, c4, c5, c6, c7, c8, 
    c9, barrier})
end

environment EDPbased_ENV
 EPR[q1_A,q2_A] * EPR[r1_A,r2_A]
 * RND_2n[s_A] * RND_N[t_A] *
 Z_1[b1_A] * Z_1[b2_A] * Z_n[x_A]
 * Z_n[z_A] * Z_2n[S_A] * 
 Z_N[T_A] * 
 Z_1[B_A] * Z_n[X_A] * Z_1[f_A]
 * Z_1[a_B] * Z_1[A_B] * Z_n[Q_B]
 * Z_n[sx_B] * Z_n[sz_B]
 * EVE[q_E] * Z_n_n[v1_A,v_B]
 * EVE1[q_B] * EVE2[r_B]
end

configuration EDPbased
 proc EDPbased
 env EDPbased_ENV
end
\end{verbatim}
\end{minipage}
\caption{Formalization of the EDP-based Protocol}
\label{fml:EDPbased}
\end{figure}

\begin{figure}
\begin{minipage}{0.5\hsize}
%\scriptsize
\begin{verbatim}
process BB84
((hadamards[q2_A,r2_A,s_A].
 shuffle[q2_A,r2_A,t_A].
 c1!q2_A.c2!r2_A.c3?a_A.
 copyN[t_A,T_A].c4!t_A.d1!T_A.
 copy2n[s_A,S_A].c5!s_A.d2!S_A.
 c6?u_A.
 abort_alice[q1_A,u_A,b1_A].
 copy1[b1_A,b2_A].
 copy1[b1_A,B_A].
 c7!b1_A.d3!B_A.
 meas b2_A then 
  cnot[r1_A,x_A].
  copyn[x_A,X_A].
  cnot_and_swap[x_A,r1_A].
  key[r1_A].
  c8!x_A.d4!X_A.
  barrier!f_A.
  cka!r1_A.
  discard(q1_A,b2_A,a_A,
          z_A,u_A,v1_A,v_B)
 saem
||
 c1?q_B.c2?r_B.
 c3!a_B.d5!A_B.
 c4?t_B.unshuffle[q_B,r_B,t_B].
 c5?s_B.hadamards[q_B,r_B,s_B].
 measure[q_B].
 copyn[q_B,Q_B].c6!q_B.d6!Q_B.
 c7?b_B.meas b_B then
 c8?x_B.
\end{verbatim}
\end{minipage}
\begin{minipage}{0.5\hsize}
%\scriptsize
\begin{verbatim}
  measure[r_B].
  cnot[x_B,r_B].
  copyn[x_B,r_B].
  syndrome[r_B,sx_B].
  correct[r_B,sx_B].
  key[r_B].
  barrier?f_B.
  ckb!r_B.
  discard(b_B,s_B,t_B,
          x_B,sx_B,sz_B,f_B)
 saem)/{c3, c4, c5, c6,
        c7, c8, barrier})
end

environment BB84_ENV
 PROB[q1_A,q2_A] *
 PROB[r1_A,r2_A]
 * RND_2n[s_A] * RND_N[t_A]
 * Z_1[b1_A] * Z_1[b2_A]
 * RC1[x_A] * RC2[z_A] 
 * Z_2n[S_A]
 * Z_N[T_A] * Z_1[B_A] * Z_n[X_A]
 * Z_1[f_A] *  Z_1[a_B] 
 * Z_1[A_B]
 * Z_n[Q_B] * Z_n[sx_B]
 * Z_n[sz_B] * Z_n_n[v1_A,v_B]
 * EVE[q_E] * EVE1[q_B]
 * EVE2[r_B]
end

configuration BB84
 proc BB84
 env BB84_ENV
end
\end{verbatim}
\end{minipage}
\caption{Formalization of BB84 Protocol}
\label{fml:BB84}
\end{figure}

The scripts of formalization of 
the EDP-based protocol and BB84 is shown in
Figure \ref{fml:EDPbased} and \ref{fml:BB84}.

\subsection{Formalization of the EDP-based Protocol}
The EDP-based protocol employs CSS quantum error correcting code (QECC),
which
is constructed from two classical linear codes $C_1, C_2$.
CSS QECC can be parametrized with $u \in C_2$ and 
$v \in \{0,1\}^n - C_1$.
We write $\mathrm{CSS}_{u,v}(C_1,C_2)$ for CSS code parametrized $u$ and
$v$ that employ codes $C_1$ and $C_2$.

\subsection{Symbols and Operators in the EDP-based Protocol}
\subsubsection{Quantum State Symbols}
\begin{itemize}
 \item Alice first prepares EPR pairs. Let quantum variables $q$ and $r$
       be of the length {\tt n}, where {\tt n} interpreted as an
       arbitrary natural number $n$. 
       $\texttt{EPR[}q,r\texttt{]}$ is interpreted to EPR pairs
       $((\frac{\ket{00}+\ket{11}}{\sqrt{2}})
       (\frac{\ket{00}+\ket{11}}{\sqrt{2}})^\dagger)^{\otimes
       n}_{q,r}$.
 \item $\texttt{RND\_2n[}q\texttt{]}$
       $\texttt{RND\_N[}r\texttt{]}$
       are interpreted to
       $(\frac{1}{2}\ket{0}\bra{0}+
       \frac{1}{2}\ket{1}\bra{1})^{\otimes 2n}_{q}$ and
       $(\frac{1}{2}\ket{0}\bra{0}+
       \frac{1}{2}\ket{1}\bra{1})^{\otimes N}_{r}$, where $N$
       is represented by a natural number symbol $\mathtt{N}$.
       $\mathtt{N}$ is interpreted to $N = \ceil{\log_2(2n!)}$.
       This is the randomness
       to determine check bits.
 \item $\texttt{Z\_1[}q\texttt{]}$,
       $\texttt{Z\_n[}r\texttt{]}$,
       $\texttt{Z\_2n[}s\texttt{]}$, and
       $\texttt{Z\_n\_n[}t,u\texttt{]}$,
       are interpreted to\\
       $\ket{0}\bra{0}_{q}$,
       $\ket{0}\bra{0}^{\otimes n}_{r}$,
       $\ket{0}\bra{0}^{\otimes 2n}_{s}$, and
       $\ket{0}\bra{0}^{\otimes n}_{t} \otimes \ket{0}\bra{0}^{\otimes n}_{u}$,
       respectively.
 \item $\texttt{EVE}$, $\texttt{EVE1}$ and $\texttt{EVE2}$ are
       arbitrarily interpreted. They express quantum states that are
       prepared by the adversary. $\texttt{EVE}$ is one for a quantum
       variable with length $\texttt{m}$, where $\texttt{m}$ is
       interpreted as an arbitrary natural number $m$. $\texttt{EVE1}$
       and $\texttt{EVE2}$ are ones for quantum variables with
       length $\texttt{n}$.
\end{itemize}
\subsubsection*{TPCP Map Symbols}
\begin{itemize}
 \item $\texttt{hadamards[}q,r,s\texttt{]}$ randomly performs Hadamard
transformation
to qubit-string $q,r$ according to a bitstring $s$ which serves
as a seed of randomness.
 \item $\texttt{shuffle[}q,r,s\texttt{]}$ randomly permutates the bits
       of qubit-string $q,r$ according to the randomness $s$.
       In the formalization, $\mathtt{q\_A}$ and $\mathtt{r\_A}$ are
       supposed to be used as check bits and to generate secret keys. 
       By this procedure, they are uniformly shuffled. Later, they are
       reverted by $\texttt{unshuffle[}q,r,s\texttt{]}$ procedure.
 \item $\texttt{copy2n[}q,r\texttt{]}$ copies the value of $q$
       with length $\texttt{2n}$ to $r$, where $q$ is supposed to be
       assigned a classical value.
       $\texttt{copyN[}q,r\texttt{]}$ and $\texttt{copy1[}q,r\texttt{]}$
       are for quantum variables with length $\texttt{N}$ and
       $\texttt{1}$.
 \item $\texttt{measure[}q\texttt{]}$ is the projective measurement of
       $q$.
 \item $\texttt{abort\_alice[}q,r,s\texttt{]}$ compares two bitstrings 
       $q$ and $r$, and sets the value 0 to a bit $s$ if the difference between
       $q$ and $r$ is lower than the
       threshold $h$, else sets the value 1 to $s$.
       The threshold $h$ does not occur in the symbolic representation
       $\texttt{abort\_alice[}q,r,s\texttt{]}$, but it is defined
       when the interpretation $\braw{\texttt{abort\_alice}}$ is
       defined. The threshold $h$ can be defined appropriately so that
       the indistinguishability expressions are valid that are
       used to verify approximate bisimilarity of the EDP-based
       protocol and EDP-ideal.
 \item $\texttt{css\_projection[}q,r,s\texttt{]}$
       is the measurement of the observable of $q$'s state
       that is described by the parity check matrix determined from
       $C_1$ and $C_2$
       (Section \ref{pre:EDPbased}, Step 7).
       EPR pairs $q$ are converted to a random 
       codeword of $\mathrm{CSS}_{x,y}(C_1,C_2)$, where
       parameters $x,y$ are also uniformly distributed.
       The value of $x$ and $y$ are stored in $r$ and $s$.
 \item $\texttt{css\_decode[}q,r,s\texttt{]}$ decodes $q$ as 
       $\mathrm{CSS}_{x,y}(C_1,C_2)$
       codeword when the value of $r$ and $s$ are $x$ and $y$.
 \item $\texttt{unshuffle[}q,r,s\texttt{]}$ is the inverse of
       $\texttt{shuffle[}q,r,s\texttt{]}$.
 \item $\texttt{css\_syndrome[}q,r,s,u,v\texttt{]}$ calculates
       the error syndrome of $q$ as a codeword of
       $\mathrm{CSS}_{x,y}(C_1,C_2)$ when
       $r$ and $s$ have the value $x$ and $y$, and stores the syndrome in
       $u$ and $v$.
 \item $\texttt{css\_correct[}q,u,v\texttt{]}$ is error correction
       with the syndrome stored in $u, v$.
\end{itemize}

\subsection{Formalization of BB84}
BB84 employs classical codes $C_1$ and $C_2$ with $C_2 \subseteq C_1$,
which correspond to $\mathrm{CSS}_{x,y}(C_1,C_2)$
in the EDP-based protocol.

\subsection{Symbols and Operators in BB84}
\subsubsection{Quantum State Symbols}
\begin{itemize}
 \item Alice first prepares two same random bitstrings. This
       initial state is represented by $\texttt{PROB[}q,r\texttt{]}$ with
       $q$ for Alice and $r$ for Bob, which is interpreted as
       $(\frac{1}{2}\ket{00}\bra{00} + 
       \frac{1}{2}\ket{11}\bra{11})^{\otimes n}_{q,r}$.
 \item $\texttt{RC1[}q\texttt{]}$ is interpreted as  
       $\sum_{u\in C_1}\frac{1}{|C_1|}\ket{u}\bra{u}$.
 \item $\texttt{RC2[}q\texttt{]}$ is interpreted as 
       $\sum_{v\in C_2}\frac{1}{|C_2|}\ket{v}\bra{v}$.
\end{itemize}
\subsubsection{TPCP Map Symbols}
\begin{itemize}
 \item $\texttt{syndrome[}q,r\texttt{]}$ calculates the error syndrome of
       $q$ using as a codeword in $C_1$ and store the syndrome to
       $r$.
 \item $\texttt{correct[}q,r\texttt{]}$ corrects errors of $q$ with the
       syndrome $r$.
 \item $\texttt{key[}q\texttt{]}$ calculates with respect to $C_2$ the coset of the value
       that is an element of $C_1$ and stored in $q$.
\end{itemize}

\subsection{Equations for the Formal Verification}
We defined 6 equations in Verifier1.
They are described in Figure \ref{fml:eqsforbb84edp1}.
The equations
$\mathtt{E1}$, $\mathtt{E2}$, and $\mathtt{E3}$ are obtained formalizing
the inferences in Shor and Preskill's security proof.
The equations $\mathtt{E4}$, $\mathtt{E5}$, and $\mathtt{E6}$ are formalization
of basic properties of linear operators.
\begin{figure}
\begin{minipage}{0.5\hsize}
%\scriptsize
\begin{verbatim}
equation E1	
 measure[r1_A](
 css_decode[r1_A,x_A,z_A](
 copyn[x_A,X_A](
 css_projection[r1_A,x_A,z_A](
  EPR[r1_A,r2_A] *
  Z_n[x_A] * Z_n[z_A] *
  Z_n[X_A]))))
 = 
 key[r1_A](
 cnot_and_swap[x_A,r1_A](
 copyn[x_A,X_A](
 cnot[r1_A,x_A](
  PROB[r1_A,r2_A] *
  RC1[x_A] *
  RC2[z_A] * Z_n[X_A]))))
end

equation E2
 measure[r_B](
 css_decode[r_B,x_A,z_A](
 css_correct[r_B,sx_B,sz_B](
 css_syndrome[r_B,x_A,z_A,
              sx_B,sz_B](
 cnot_and_swap[x_A,r1_A](
 copyn[x_A,X_A](
 cnot[r1_A,x_A](
  PROB[r1_A,r2_A] *
  __[r_B] * RC1[x_A] *
  RC2[z_A] * Z_n[sx_B] *
  Z_n[sz_B] * Z_n[X_A])))))))
 = 
 key[r_B](
 correct[r_B,sx_B](
 syndrome[r_B,sx_B](
 copyn[x_A,r_B](
 cnot[r_B,x_A](
 measure[r_B](
 cnot_and_swap[x_A,r1_A](
 copyn[x_A,X_A](
 cnot[r1_A,x_A](
  PROB[r1_A,r2_A] * __[r_B]
  * RC1[x_A] * RC2[z_A]
  * Z_n[sx_B] * Z_n[sz_B] * 
  Z_n[X_A])))))))))
end
\end{verbatim}
\end{minipage}
\begin{minipage}{0.5\hsize}
\begin{verbatim}
equation E3
 measure[r2_A](
 css_decode[r2_A,x_A,z_A](
 css_correct[r2_A,sx_B,sz_B](
 css_syndrome[r2_A,x_A,z_A,
              sx_B,sz_B](
 cnot_and_swap[x_A,r1_A](
 copyn[x_A,X_A](
 cnot[r1_A,x_A](
  __[r1_A,r2_A] * 
  RC1[x_A] * RC2[z_A] * 
  Z_n[sx_B] * Z_n[sz_B] *
  Z_n[X_A])))))))
 = 
 key[r2_A](
 correct[r2_A,sx_B](
 syndrome[r2_A,sx_B](
 copyn[x_A,r2_A](
 cnot[r2_A,x_A](
 measure[r2_A](
 cnot_and_swap[x_A,r1_A](
 copyn[x_A,X_A](
 cnot[r1_A,x_A](
  __[r1_A,r2_A] * 
  RC1[x_A] * RC2[z_A] *
  Z_n[sx_B] * Z_n[sz_B] * 
  Z_n[X_A])))))))))
end

equation E4
 Tr[q1_A](EPR[q1_A,q2_A])
 =
 Tr[q1_A](PROB[q1_A,q2_A])
end

equation E5
 Tr[r1_A](EPR[r1_A,r2_A])
 =
 Tr[r1_A](PROB[r1_A,r2_A])
end

equation E6
 measure[q1_A](EPR[q1_A,q2_A])
 =
 PROB[q1_A,q2_A]
end
\end{verbatim}
\end{minipage}
\caption{Equations for BB84 and the EDP-based Protocol}
\label{fml:eqsforbb84edp1}
\end{figure}

\subsection{Experiment Result}
\subsubsection{Experiment 1}
We ran Verifier1 with the input of {\tt shor-preskill.scr}.
We used a laptop with Intel Core i5 CPU M 460 @ 2.53GHz and 1GB memory.
The transition tree of the EDP-based protocol has 621 nodes and 165
paths, and that of BB84 has 588 nodes and 165 paths.
The verifier checked the bisimilarity of the two protocols in 30.38
seconds. The recursive procedure was called 753 times. 
The number of configurations in the history was 653 and
history was hit 653 times. The number of application of each equation
is described as follows. The equations E1, E2, E3, E4, E5, and E6 are
applied 55, 24, 9, 73, 271, and 11 times, respectively.

\section{Formal Verification of Indistinguishability of the
EDP-based Protocol and EDP-ideal}
\label{fml:formalverif2}
The last protocol EDP-ideal is a sort of cheating protocol.
Alice and Bob are assumed to share EPR pairs initially in 
the protocol. Alice and Bob executes the same protocol as 
the EDP-based protocol until the decision of continue or aborting
by the result of the error checking. Only when Alice and Bob
decide to continue, 
they create the secret keys using pre-shared EPR pairs instead
of pairs obtained after the entanglement distillation protocol.

\subsection{Formalization of EDP-ideal}
The pre-shared EPR pairs are formalized as
$\mathtt{EPR[rx\_A,rx\_B]}$. The code of EDP-ideal is
almost the same as the EDP-based protocol. When
Alice and Bob decide to continue the protocol,
Alice creates her secret key from $\mathtt{rx\_A}$ and
renames to $\mathtt{r1\_A}$ operating
$\mathtt{create\_key[rx\_A,r1\_A]}$. Bob creates his key
similarly by $\mathtt{create\_key[rx\_B,r\_B]}$.

\begin{figure}
\begin{minipage}{0.5\hsize}
%\scriptsize
\begin{verbatim}
process EDP-IDEAL
 ((hadamards[q2_A,r2_A,s_A].
  shuffle[q2_A,r2_A,t_A].
  c1!q2_A.c2!r2_A.c3?a_A.
  copyN[t_A,T_A].c4!t_A.d1!T_A.
  copy2n[s_A,S_A].c5!s_A.d2!S_A.
  measure[q1_A].
  c6?u_A.
  abort_alice[q1_A,u_A,b1_A].
  copy1[b1_A,b2_A].
  copy1[b1_A,B_A].
  c7!b1_A.d3!B_A.
  meas b2_A then
   css_projection[r1_A,x_A,z_A].
   css_decode[r1_A,x_A,z_A].
   copyn[x_A,X_A].
   measure[r1_A].
   c8!x_A.d4!X_A.
   c9!z_A.
   create_key[rx_A,r1_A].
   barrier!f_A.
   cka!r1_A.
   discard(q1_A,b2_A,
           a_A,u_A,rx_A)
  saem
 ||
  c1?q_B.c2?r_B.
  c3!a_B.d5!A_B.
  c4?t_B.unshuffle[q_B,r_B,t_B].
  c5?s_B.hadamards[q_B,r_B,s_B].
  measure[q_B].
  copyn[q_B,Q_B].c6!q_B.d6!Q_B.
  c7?b_B.
\end{verbatim}
\end{minipage}
\begin{minipage}{0.5\hsize}
\begin{verbatim}
   meas b_B then 
    c8?x_B.c9?z_B.
    css_syndrome[r_B,x_B,
                 z_B,sx_B,sz_B].
    css_correct[r_B,sx_B,sz_B].
    css_decode[r_B,x_B,z_B].
    measure[r_B].
    create_key[rx_B,r_B].
    barrier?f_B.
    ckb!r_B.
    discard(b_B,s_B,t_B,x_B,z_B,
            sx_B,sz_B,f_B,rx_B)
   saem)/{c3, c4, c5, c6, 
          c7, c8, c9, barrier})
end

environment EDP-IDEAL_ENV
 EPR[q1_A,q2_A] * EPR[r1_A,r2_A]
 * RND_2n[s_A] * RND_N[t_A]
 * Z_1[b1_A] * Z_1[b2_A]
 * Z_n[x_A] 
 * Z_n[z_A] * Z_2n[S_A]
 * Z_N[T_A]
 * Z_1[B_A] * Z_n[X_A] * Z_1[f_A]
 * Z_1[a_B] * Z_1[A_B] * Z_n[Q_B]
 * Z_n[sx_B] * Z_n[sz_B]
 * EVE[q_E]
 * EVE1[q_B] * EVE2[r_B]
 * EPR[rx_A,rx_B]
end

configuration EDP-IDEAL
 proc EDP-IDEAL
 env  EDP-IDEAL_ENV
end
\end{verbatim}
\end{minipage}
\caption{Formalization of EDP-ideal}
\end{figure}
\subsection{Indistinguishability Expressions for the Verification}
We defined 24 indistinguishability expressions Verifier2.
One of the expressions $\mathtt{E1}$ is described in Figure 
\ref{fml:indexp}. The expression's meaning is as follows.
\begin{enumerate}
 \item The probability that they do not abort the protocol is negligibly
       close in the both protocols.
 \item If the both protocols are not aborted,
       Alice's secret key that is created from halves of qubit pairs
       whose states
       are obtained after the entanglement distillation in the EDP-based
       protocol is indistinguishable from Alice's key that is
       created from EPR pairs.
\end{enumerate}
The indistinguishability expressions are prepared for each Eve's choice: she
can choose to interfere or not to interfere communications through
the public quantum channels $\mathsf{c1}$ and $\mathsf{c2}$.
For example, if she interferes $\mathsf{c1}$ and does not interfere
$\mathsf{c2}$, a possible scheduling is as follows.
\begin{align*}
 \cdots \xrightarrow{\tau} &
 \con{(\sndq{c1}{\mathtt{q2\_A}}.\sndq{c2}{\mathtt{r2\_A}}.\cdots||
 \rcvq{c1}{\mathtt{q\_B}}.\rcvq{c2}{\mathtt{r\_B}}.\cdots)\backslash 
 \{\mathsf{c3},\cdots\}}{\rho}\\
 \xrightarrow{\sndq{c1}{\mathtt{q2\_A}}} 
 \xrightarrow{\rcvq{c1}{\mathtt{q\_B}}} &
 \con{(\sndq{c2}{\mathtt{r2\_A}}.\cdots||
 \rcvq{c2}{\mathtt{r\_B}}.\cdots)\backslash
 \{\mathsf{c3},\cdots\}}{\rho'}\\
 \xrightarrow{\tau} &
 \con{(\cdots||
 \cdots)\backslash
 \{\mathsf{c3},\cdots\}}{\rho''}
\end{align*}
$\mathtt{E1}$ is for the case where Eve chooses the scheduling
$\xrightarrow{\sndq{c1}{\mathtt{q2\_A}}}
\xrightarrow{\rcvq{c1}{\mathtt{q\_B}}}$ and\\
$\xrightarrow{\sndq{c2}{\mathtt{r2\_A}}}
\xrightarrow{\rcvq{c2}{\mathtt{r\_B}}}$. The order of sending and
receiving does not matter here.
 As for this point, we needed 4 types of indistinguishability expressions
for the cases where Eve interferes both $\mathsf{c1}, \mathsf{c2}$,
only $\mathsf{c1}$, only $\mathsf{c2}$, and does not interfere both.

The indistinguishability expressions are also prepared for certain
steps of the protocols: after completing the keys, Alice and Bob output
their keys but who sends the first is non-deterministic.
For instance, the quantum variable {\tt r\_B} is in the expression
{\tt Tr[b1\_A,b2\_A,q1\_A,q\_B,r\_B,rx\_A,rx\_B,s\_A,t\_A,x\_A,z\_A]} in
the first
line and this is for the step where Alice has already 
sent her key {\tt r1\_A} to the outside by {\tt cka!r1\_A}
but Bob has not yet his secret
key {\tt r\_B} by {\tt ckb!r\_B}. As for this point,
we needed 3 types of indistinguishability expressions
for the cases where the outsider has obtained only Alice's key, 
only Bob's key, and both.

We have explained the reason why we needed $3 \times 4 = 12$
indistinguishability expressions.
Finally, for each expression, we needed one equivalent expression
obtained replacing the order of the CP maps, because the pattern matching
algorithm for CP maps does not solve commutativity completely.
Hence, we prepared $12 \times 2 = 24$ expressions.
\begin{figure}
\begin{verbatim}
indistinguishable E1 n
 Tr[b1_A,b2_A,q1_A,q_B,r_B,rx_A,rx_B,s_A,t_A,x_A,z_A](
  create_key[rx_A,r1_A](proj1[b1_A](measure[r1_A](
  copyn[x_A,X_A](css_decode[r1_A,x_A,z_A](
  css_projection[r1_A,x_A,z_A](proj1[b2_A](
  copy1[b1_A,B_A](copy1[b1_A,b2_A](
  abort_alice[q1_A,q_B,b1_A](measure[q1_A](
  copyn[q_B,Q_B](measure[q_B](
  hadamards[q_B,r_B,s_A](copy2n[s_A,S_A](
  unshuffle[q_B,r_B,t_A](copyN[t_A, T_A](
  __[q2_A,r2_A,q_E,q_B,r_B](
  shuffle[q2_A,r2_A,t_A](hadamards[q2_A,r2_A,s_A](
    EPR[q1_A,q2_A] * EPR[r1_A,r2_A] * EPR[rx_A,rx_B] *
    RND_2n[s_A] * Z_2n[S_A] * RND_N[t_A] * Z_N[T_A] * 
    Z_1[b1_A] * Z_1[b2_A] * Z_1[B_A] * Z_n[Q_B] *
    Z_n[x_A] * Z_n[X_A] * Z_n[z_A] * 
    __[q_B] * __[r_B] * __[q_E]
 )))))))))))))))))))))
 =
 Tr[b1_A,b2_A,q1_A,q_B,r_B,s_A,t_A,x_A,z_A](
  proj1[b1_A](measure[r1_A](
  copyn[x_A,X_A](css_decode[r1_A,x_A,z_A](
  css_projection[r1_A,x_A,z_A](proj1[b2_A](
  copy1[b1_A,B_A](copy1[b1_A,b2_A](
  abort_alice[q1_A,q_B,b1_A](measure[q1_A](
  copyn[q_B,Q_B](measure[q_B](
  hadamards[q_B,r_B,s_A](copy2n[s_A,S_A](
  unshuffle[q_B,r_B,t_A](copyN[t_A, T_A](
  __[q2_A,r2_A,q_E,q_B,r_B](
  shuffle[q2_A,r2_A,t_A](hadamards[q2_A,r2_A,s_A](
    EPR[q1_A,q2_A] * EPR[r1_A,r2_A] * 
    RND_2n[s_A] * Z_2n[S_A] * RND_N[t_A] * Z_N[T_A] * 
    Z_1[b1_A] * Z_1[b2_A] * Z_1[B_A] * Z_n[Q_B] *
    Z_n[x_A] * Z_n[X_A] * Z_n[z_A] *
    __[q_B] * __[r_B] * __[q_E]
  ))))))))))))))))))))
end
\end{verbatim}
\label{fml:indexp}
\caption{Indistinguishable Expression E1}
\end{figure}
\subsection{Experiment Result}
\subsubsection{Experiment 2}
We performed the experiment in the same environment as the
previous part of the formal verification described
in Section \ref{fml:formalverif1}.
We ran Verifier2 with the input of {\tt edp-edpideal.scr}, where
the EDP based protocol and EDP-ideal are formalized.
As for the transition tree, the both protocols have 621 nodes and 165
paths. Verifier2 checked the bisimilarity of the two protocols in 112.50
seconds. The recursive procedure was called 907 times.
The number of configurations in the history was 763 and
history was hit 620 times. The number of application of each equation
is described as follows. The equations E1, E2, E3, E1-2, E2-2, and E3-2 are
applied 6, 12, 6, 12, 24, and 12 times, respectively.
The equations F1, F2, F3, F1-2, F2-2, and F3-2 are
applied 2, 4, 2, 4, 8, and 4 times, respectively.
The equations G1, G2, G3, G1-2, G2-2, and G3-2 are
applied 2, 4, 2, 4, 8, and 4 times, respectively.
The equations H1, H2, H3, H1-2, H2-2, and H3-2 are
applied 1, 2, 1, 2, 4, and 2 times, respectively.

\subsubsection{Experiment 3}
We ran Verifier2 with the input of {\tt bb84-edp.scr}, which is
identical to the script of Experiment 1.
It checked the bisimilarity of the two protocols in 39.50
seconds. The recursive procedure was called 1039 times. 
The transition tree of the EDP-based protocol has 621 nodes and 165
paths, and that of BB84 has 588 nodes and 165 paths, which is the same
result as the Experiment 1.
The number of configurations in the history was 796 and
history was hit 653 times. The number of application of each equation
is described as follows. The equations E1, E2, E3, E4, E5, and E6 are
applied 132, 24, 9, 73, 458, and 11 times, respectively.

\subsubsection{Discussion about the Results}
Although the number of the call of the recursive procedure is close,
it took more time, 112.50 seconds, in Experiment 2
compared to 30.38 seconds in Experiment 3.
There are following two reasons.
The first is that the algorithm of the Verifier2 
is more complex than that of Verifier1: 
wild card of CP map symbol $\texttt{\_\_[} \tilde q \texttt{]}$ 
is permitted in indistinguishability expressions. The algorithm
to match the left-hand side of the expressions is more complex for the
sake of the wild card matching.
The second is that both the number of indistinguishability expressions
and the sizes of them are larger. For each time of testing
indistinguishability, the procedure checks for each 
indistinguishability expression whether it matches to the
left-hand side of the objective quantum states.

Let us compare Experiment 1 and Experiment 3.
Although we used the same input file, the number of calling the 
recursive procedure is different. This is because the existence or
absence of the requirement that a branch caused by $\mathtt{meas}$
should be matched, which we mentioned in Section 
\ref{neg:extensionofalgorithm}. With
the requirement, to simulate a
transition caused by $\mathtt{meas}$, the number of transitions which
are candidate for success of the simulation is more limited:
Verifier1 only seeks a $\tau$ transition caused by $\mathtt{meas}$ by the
same qubit variable.

% \section{Related Work}
% \subsection{CryptoVerif}
% \label{fml:rel-cv}
% $\mathsf{CryptoVerif}$ \cite{Blanchet2008cryptoverif} 
% is a software tool to verify security of
% \emph{classical} protocols.
% As a proof technique, $\mathsf{CryptoVerif}$ applies 
% \emph{observational equivalence} of processes.
% Let $\Pr(P \rightsquigarrow a)$
% be the probability
% that the process $P$ transits to a process that is ready to send some
% data through the channel $a$.
% Two processes $P$ and $Q$ are observationally equivalent, written
% $P \cong Q$ here, if
% $$
% |\Pr(C[P] \rightsquigarrow a) - \Pr(C[Q] \rightsquigarrow a)|
% $$
% is negligible for all evaluation context
% $C[\_]$ that runs in polynomial time
% and channel $a$ that is
% not restricted. 
% The relation $\cong$ is \emph{congruent} by the definition,
% namely, if $P \cong Q$ holds, then $C[P] \cong C[Q]$ holds for all
% evaluation context $C[\_]$ running in polynomial time.
% When we consider $C[\_]$ as a polynomial time
% adversary that runs in parallel with the protocol $P$ or $Q$,
% the observational equivalence of them is intuitively interpreted
% as indistinguishability of the protocols from an adversary.

% In cryptographic proofs, security of a high-level protocol
% is reduced to security of the employed cryptographic primitives,
% which is assumed.
% Similarly, security of a cryptographic scheme is reduced to 
% assumed difficulty of computing certain functions.
% In $\mathsf{CryptoVerif}$, a user formalizes such assumptions as
% observational equivalence of processes.
% Given a process $P$ formalizing a target protocol and 
% user-defined observational equivalence, $\mathsf{CryptoVerif}$
% tries to rewrite $P$ to another process $Q$ that is obviously secure:
% if $P$ is of the form $C[X]$ and there is a user defined
% equivalence $X \cong Y$, then it is rewritten to $C[Y]$.
% On the other hand, our tools verify bisimilarity by
% tracing execution paths of configurations, not by
% rewriting processes.

% Fortunately, the bisimulation in qCCS is congruent, and
% it is thus possible that a verification tool is
% designed to verify bisimilarity by rewriting.
% With such a verifier, bisimilarity of 
% big-sized configurations is derived from that of some small-sized ones.
% Especially in proofs of security of QKD protocols,
% difficulty of computing certain functions is not assumed.
% Therefore, even if a verifier conducts the rewriting,
% we possibly need to prove bisimilarity of
% such small-sized configurations unlike
% verification using $\mathsf{CryptoVerif}$.

\chapter{Conclusions}
\section{Automated Verification of Bisimilarity of Configurations}
\subsection*{Impact of Automation}
Our automatic verification methods
broaden the range of application of qCCS.
In security proofs, equivalence of protocols is often discussed.
It can be described as bisimilarity but it is
difficult to check by hand when state transitions of processes
have many long branches.
 Besides, equality of outsider's views
between two protocols must be checked in each step.
Outsider's view is calculated from collective quantum state,
which is possibly denoted by a huge matrix.
One might prove bisimilarity with the insight of state transitions
without tracing them. However, it is not always possible and
possibly contradicts a purpose of formal methods, namely,
to make implicit inferences in proofs explicit.
 Our verifiers do exhausting parts of proofs of
behavioural equivalence: it checks correspondence
of all state transitions up to invisible ones and 
equality of outsider's views using equations and 
indistinguishability expressions.
Let us call equations and indistinguishability expressions {\it axioms}
here.
On the other hand, a user only has to
examine the correctness of formalization of protocols and validity of
axioms.
It could be difficult to find all appropriate axioms for a proof
immediately. The verifiers are also able to show quantum states,
outsider's views, and/or
processes when the recursive procedure returns $\mathit{false}$. With
the information,
a user can modify axioms to input.

\subsection*{Equations and Indistinguishability Expressions}
\label{concl:validityof}
Formal verification of validity of axioms is important but
not in the scope of this thesis. Most of the axioms in Chapter 
\ref{formaliz} are obtained formalizing properties of CSS-QECC
\cite{CalderbankShor1996} or Lo and Chau's theorem \cite{LoChau1999}
and their validity is not self-evident.
Nevertheless, the validity can be verified as just {\it equality or
negligible trace distance of density operators}.
One does not have to consider {\it communication} of principals
and {\it nondeterminism}
of execution models to verify the axioms, because
conditions about them are verified using the process calculus.

Application of a sequential quantum programming language such as 
QPL \cite{Selinger2004} is a possible way to verify axioms.
In QPL's semantics, the programs are interpreted to TPCP maps.
The bodies of symbolically-represented TPCP maps, such as 
$\mathtt{css\_projection[}q,r,s\mathtt{]}$ and
$\mathtt{css\_syndrome[}q,r,s\mathtt{]}$ described in Chapter
\ref{formaliz},
are possibly formalized as QPL programs. Validity of axioms could be
verified as equivalence or ``indistinguishability'' of the programs.

\section{Congruent Approximate Bisimulation Relation}
The notion of bisimulation proposed by Feng et al. \cite{DengFeng2012}
was applicable to verify equivalence of BB84 and the EDP-based protocol
\cite{Kubota2012}. The configurations that are bisimilar in the original
qCCS's definition behave equivalently from the outside.
However, it seemed not to applicable directly 
to verify the security proof of the latter. To do it, 
it seemed to be a possible way 
to consider an ideal protocol and prove that 
they behave equivalently or 
{\it almost} equivalently in the view of Eve 
but the EDP-based protocol is not
ideal, that is, it leaks negligible amount of information to Eve.
We presented two notions of approximate bisimulation, 
$\sim_{\zeta, \eta}$ and $\sim$, in Chapter \ref{prob_bisim}.
In our definition of $\sim$,
the configurations are approximately bisimilar as
long as they disclose quantum states to the outsider with negligible
trace distance and perform identical transitions up to
$\xrightarrow{\tau}$ if their probability is non-negligible.
As described in Chapter \ref{formaliz}, the notion of approximate
bisimulation $\sim$
was applicable to verify formally
the security proof of the EDP-based protocol.

We studied the properties of the approximate bisimulation relations.
Some of them, 
such as those stated in Proposition \ref{par:sym-by-par}, \ref{neg:sym-by-par}, Lemma
\ref{prob_bisim:parcoinduction}, \ref{neg:coinduction}, and \ref{neg:weaksimulated},
are analogy of those of conservative bisimulation relations \cite{Milner1999,
FengDuanYing2011, DengFeng2012}. 
As stated by Lemma \ref{par:transitivitylike}, the relation
$\sim_{\zeta, \eta}$ has a transitivity-like property. The relation $\sim$
is an equivalence relation, which is stated by Lemma
\ref{neg:equivalence}.
Furthermore, $\sim_{\zeta, \eta}$ and $\sim$ are closed under
application of an arbitrary
evaluation context as stated in Corollary \ref{par:congruence} and
\ref{neg:congruence}. The property suggest sanity of our definitions as
well as feasibility in practice. Concretely, they are useful when
we consider multiple sessions of a protocol or its behavior employed
as a primitive of another protocol.

\section{Formal Verification of Security Proofs of\\
Quantum Cryptographic Protocols}
We formally verified Shor and Preskill's security proof of BB84
\cite{ShorPreskill2000} using our verifiers.
The first verifier verified bisimilarity of $\mathtt{BB84}$ and
$\mathtt{EDPbased}$ and the second verifier
verified approximate bisimilarity of $\mathtt{BB84}$,
$\mathtt{EDPbased}$, and $\mathtt{EDPideal}$.
To the best of our knowledge, 
this is the first work where \emph{cryptographic security} of
a quantum cryptographic protocol 
is formally verified using a quantum process calculus 
a software tool.

Shor and Preskill's security proof is simple compared to
other proofs \cite{Mayers2001, KoashiPreskill2003}, where more general
execution models are assumed.
Actually,
Shor and Preskill's proof is clearly understood by researchers
in quantum cryptography \cite{Mayers2001, KoashiPreskill2003, Lo2003}
and it seems not to be an emergent task to verify it formally.
We consider our work as a step to
apply formal methods practically to general quantum protocols.

\section{Future Work}
Although we applied our verifiers to Shor and Preskill's security proof,
we did not to other proofs. For B92 and \cite{Bennett1992}
the six-state protocol \cite{Bruss1998}, which are QKD protocols,
the proofs \cite{TamakiKoashiImoto2003, Lo2001} that are similar to Shor
and Preskill's have been presented:
the security of the objective protocol is reduced to an EDP-based
protocol. The notions of bisimilarity and
approximate bisimilarity are possibly applied to them.

There are also several security proofs of BB84 with different
assumptions. To broaden the range of formal verification using
process calculi, we have to consider the way to verify proofs
with different patterns of arguments.

As mentioned in Subsection \ref{concl:validityof}, 
formal verification of validity of axioms is future work. 
Furthermore, to broaden the
range of automation,
the algorithms must be improved to test equality and
indistinguishability 
of symbolically-represented partial traces
using the axioms. In this thesis,
we adopt a quite simple algorithm for the tests:
each axiom is applied only once for each test
in the order of user's definition.
Although such simple
algorithm is applicable to verify relatively simple
verification 
objectives such as Shor and Preskill's security proof, improved
algorithms help us to verify more general ones.
It is possible to implement a \emph{completion} algorithm to the
verifiers and apply it to the rewriting system consisting of axioms.
An algorithm similar to Knuth-Bendix completion \cite{KnuthBendix1970}
is adopted by $\mathsf{CryptoVerif}$
\cite{Blanchet2008cryptoverif} to simplify terms that represent
computations on bitstrings using user-defined equations.
With a complete rewriting system, equality and indistinguishability
of the symbolic representations are decided.

If quantum cryptographic protocols are formalized as 
configurations and
proven bisimilar, we expect they are equivalent in purely quantum 
cryptographic sense. However, 
correspondence between bisimilarity and physical
equivalence is only intuitively understood.
Compared to the tools we introduced in Section \ref{intro:tools},
our verifiers seems to be relatively close to 
verifiers of quantum cryptographic proofs
in that they do not idealize cryptographic primitives.
To consider physical semantics of quantum process calculi
is also future work.

%-------------------
\nocite{*}
\bibliographystyle{plain}
%\bibliographystyle{unsrt}
\bibliography{main}

\end{document}