\chapter{Conclusions}
\section{Automated Verification of Bisimilarity of Configurations}
\subsection*{Impact of Automation}
Our automatic verification methods
broaden the range of application of qCCS.
In security proofs, equivalence of protocols is often discussed.
It can be described as bisimilarity but it is
difficult to check by hand when state transitions of processes
have many long branches.
 Besides, equality of outsider's views
between two protocols must be checked in each step.
Outsider's view is calculated from collective quantum state,
which is possibly denoted by a huge matrix.
One might prove bisimilarity with the insight of state transitions
without tracing them. However, it is not always possible and
possibly contradicts a purpose of formal methods, namely,
to make implicit inferences in proofs explicit.
 Our verifiers do exhausting parts of proofs of
behavioural equivalence: it checks correspondence
of all state transitions up to invisible ones and 
equality of outsider's views using equations and 
indistinguishability expressions.
Let us call equations and indistinguishability expressions {\it axioms}
here.
On the other hand, a user only has to
examine the correctness of formalization of protocols and validity of
axioms.
It could be difficult to find all appropriate axioms for a proof
immediately. The verifiers are also able to show quantum states,
outsider's views, and/or
processes when the recursive procedure returns $\mathit{false}$. With
the information,
a user can modify axioms to input.

\subsection*{Equations and Indistinguishability Expressions}
\label{concl:validityof}
Formal verification of validity of axioms is important but
not in the scope of this thesis. Most of the axioms in Chapter 
\ref{formaliz} are obtained formalizing properties of CSS-QECC
\cite{CalderbankShor1996} or Lo and Chau's theorem \cite{LoChau1999}
and their validity is not self-evident.
Nevertheless, the validity can be verified as just {\it equality or
negligible trace distance of density operators}.
One does not have to consider {\it communication} of principals
and {\it nondeterminism}
of execution models to verify the axioms, because
conditions about them are verified using the process calculus.

Application of a sequential quantum programming language such as 
QPL \cite{Selinger2004} is a possible way to verify axioms.
In QPL's semantics, the programs are interpreted to TPCP maps.
The bodies of symbolically-represented TPCP maps, such as 
$\mathtt{css\_projection[}q,r,s\mathtt{]}$ and
$\mathtt{css\_syndrome[}q,r,s\mathtt{]}$ described in Chapter
\ref{formaliz},
are possibly formalized as QPL programs. Validity of axioms could be
verified as equivalence or ``indistinguishability'' of the programs.

\section{Congruent Approximate Bisimulation Relation}
The notion of bisimulation proposed by Feng et al. \cite{DengFeng2012}
was applicable to verify equivalence of BB84 and the EDP-based protocol
\cite{Kubota2012}. The configurations that are bisimilar in the original
qCCS's definition behave equivalently from the outside.
However, it seemed not to applicable directly 
to verify the security proof of the latter. To do it, 
it seemed to be a possible way 
to consider an ideal protocol and prove that 
they behave equivalently or 
{\it almost} equivalently in the view of Eve 
but the EDP-based protocol is not
ideal, that is, it leaks negligible amount of information to Eve.
We presented two notions of approximate bisimulation, 
$\sim_{\zeta, \eta}$ and $\sim$, in Chapter \ref{prob_bisim}.
In our definition of $\sim$,
the configurations are approximately bisimilar as
long as they disclose quantum states to the outsider with negligible
trace distance and perform identical transitions up to
$\xrightarrow{\tau}$ if their probability is non-negligible.
As described in Chapter \ref{formaliz}, the notion of approximate
bisimulation $\sim$
was applicable to verify formally
the security proof of the EDP-based protocol.

We studied the properties of the approximate bisimulation relations.
Some of them, 
such as those stated in Proposition \ref{par:sym-by-par}, \ref{neg:sym-by-par}, Lemma
\ref{prob_bisim:parcoinduction}, \ref{neg:coinduction}, and \ref{neg:weaksimulated},
are analogy of those of conservative bisimulation relations \cite{Milner1999,
FengDuanYing2011, DengFeng2012}. 
As stated by Lemma \ref{par:transitivitylike}, the relation
$\sim_{\zeta, \eta}$ has a transitivity-like property. The relation $\sim$
is an equivalence relation, which is stated by Lemma
\ref{neg:equivalence}.
Furthermore, $\sim_{\zeta, \eta}$ and $\sim$ are closed under
application of an arbitrary
evaluation context as stated in Corollary \ref{par:congruence} and
\ref{neg:congruence}. The property suggest sanity of our definitions as
well as feasibility in practice. Concretely, they are useful when
we consider multiple sessions of a protocol or its behavior employed
as a primitive of another protocol.

\section{Formal Verification of Security Proofs of\\
Quantum Cryptographic Protocols}
We formally verified Shor and Preskill's security proof of BB84
\cite{ShorPreskill2000} using our verifiers.
The first verifier verified bisimilarity of $\mathtt{BB84}$ and
$\mathtt{EDPbased}$ and the second verifier
verified approximate bisimilarity of $\mathtt{BB84}$,
$\mathtt{EDPbased}$, and $\mathtt{EDPideal}$.
To the best of our knowledge, 
this is the first work where \emph{cryptographic security} of
a quantum cryptographic protocol 
is formally verified using a quantum process calculus 
a software tool.

Shor and Preskill's security proof is simple compared to
other proofs \cite{Mayers2001, KoashiPreskill2003}, where more general
execution models are assumed.
Actually,
Shor and Preskill's proof is clearly understood by researchers
in quantum cryptography \cite{Mayers2001, KoashiPreskill2003, Lo2003}
and it seems not to be an emergent task to verify it formally.
We consider our work as a step to
apply formal methods practically to general quantum protocols.

\section{Future Work}
Although we applied our verifiers to Shor and Preskill's security proof,
we did not to other proofs. For B92 and \cite{Bennett1992}
the six-state protocol \cite{Bruss1998}, which are QKD protocols,
the proofs \cite{TamakiKoashiImoto2003, Lo2001} that are similar to Shor
and Preskill's have been presented:
the security of the objective protocol is reduced to an EDP-based
protocol. The notions of bisimilarity and
approximate bisimilarity are possibly applied to them.

There are also several security proofs of BB84 with different
assumptions. To broaden the range of formal verification using
process calculi, we have to consider the way to verify proofs
with different patterns of arguments.

As mentioned in Subsection \ref{concl:validityof}, 
formal verification of validity of axioms is future work. 
Furthermore, to broaden the
range of automation,
the algorithms must be improved to test equality and
indistinguishability 
of symbolically-represented partial traces
using the axioms. In this thesis,
we adopt a quite simple algorithm for the tests:
each axiom is applied only once for each test
in the order of user's definition.
Although such simple
algorithm is applicable to verify relatively simple
verification 
objectives such as Shor and Preskill's security proof, improved
algorithms help us to verify more general ones.
It is possible to implement a \emph{completion} algorithm to the
verifiers and apply it to the rewriting system consisting of axioms.
An algorithm similar to Knuth-Bendix completion \cite{KnuthBendix1970}
is adopted by $\mathsf{CryptoVerif}$
\cite{Blanchet2008cryptoverif} to simplify terms that represent
computations on bitstrings using user-defined equations.
With a complete rewriting system, equality and indistinguishability
of the symbolic representations are decided.

If quantum cryptographic protocols are formalized as 
configurations and
proven bisimilar, we expect they are equivalent in purely quantum 
cryptographic sense. However, 
correspondence between bisimilarity and physical
equivalence is only intuitively understood.
Compared to the tools we introduced in Section \ref{intro:tools},
our verifiers seems to be relatively close to 
verifiers of quantum cryptographic proofs
in that they do not idealize cryptographic primitives.
To consider physical semantics of quantum process calculi
is also future work.
