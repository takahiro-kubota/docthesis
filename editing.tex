%#BIBTEX jbibtex main
\documentclass[doctor, 11pt, final]{iscs-thesis}

\usepackage{amsmath}
\usepackage{amssymb}
\usepackage{url}
\usepackage{proof}
\usepackage{multicol}
\usepackage[dvipdfmx]{graphicx}

%-------------------
\newcommand{\dom}[1]{\mathrm{dom}(#1)}
\newcommand{\cod}[1]{\mathrm{cod}(#1)}
\newcommand{\defequiv}[0]{\overset{\mathrm{def}}{\equiv}}
\newcommand{\defeq}[0]{\overset{\mathrm{def}}{=}}
\newcommand{\strg}[0]{\,\dot\approx\,}
\newcommand{\xgizarrow}[1]{\overset{#1}{\rightsquigarrow}}

\newcommand{\A}[0]{\mathcal{A}}
\newcommand{\C}[0]{\mathcal{C}}
\newcommand{\D}[0]{\mathcal{D}}
\newcommand{\R}[0]{\mathcal{R}}
\newcommand{\E}[0]{\mathcal{E}}
\newcommand{\F}[0]{\mathcal{F}}
\newcommand{\G}[0]{\mathcal{G}}
\renewcommand{\H}[0]{\mathcal{H}}
\newcommand{\I}[0]{\mathbb{I}}
\newcommand{\N}[0]{\mathbb{N}}

\newcommand{\CSS}[0]{\mathrm{CSS}}

\newcommand{\Con}[0]{\mathit{Con}}
\newcommand{\qChan}[0]{\mathit{qChan}}
\newcommand{\Proc}[0]{\mathit{Proc}}
\newcommand{\qVar}[0]{\mathit{qVar}}
\newcommand{\Env}[0]{\mathit{Env}}
\newcommand{\nil}[0]{\mathtt{nil}}
\newcommand{\inds}[0]{\mathit{inds}}
\newcommand{\eqs}[0]{\mathit{eqs}}
\newcommand{\pr}[1]{{\rm pr}(#1)}
\newcommand{\ptr}[2]{\mathtt{Tr}\texttt{[}#1\texttt{]}\texttt{(}#2\texttt{)}}
\newcommand{\ptrtt}[2]{\texttt{Tr[}#1\texttt{](}#2\texttt{)}}
\newcommand{\qv}[1]{\mathrm{qv}(#1)}
\newcommand{\cnv}[1]{\mathrm{cnv}(#1)}
\newcommand{\tr}[2]{{\rm tr}_{#1}(#2)}
\newcommand{\braw}[1]{[\![#1]\!]}
\newcommand{\brat}[1]{\langle #1 \rangle}
\newcommand{\vecb}[1]{{\bf #1}}
\newcommand{\con}[2]{\langle #1, #2 \rangle}
\newcommand{\sndq}[2]{\mathsf{#1}! #2}
\newcommand{\rcvq}[2]{\mathsf{#1}? #2}
\newcommand{\subst}[2]{\{#1 / #2\}}

\newcommand{\braket}[2]{\langle #1|#2\rangle}
\newcommand{\bra}[1]{\langle #1|}
\newcommand{\ket}[1]{|#1\rangle}
\newcommand{\ceil}[1]{\lceil #1 \rceil}

\newcommand{\discard}[1]{\mathtt{discard(} #1 \mathtt{)}}
\newcommand{\myif}[2]{\mathtt{if}~#1~\mathtt{then}~#2~\mathtt{fi}}
\newcommand{\measure}[2]{{\tt meas}~#1~{\tt then}~#2~{\tt saem}}

\newcommand{\op}[2]{#1[#2]}
\newcommand{\dsym}[2]{#1\texttt{[}#2\texttt{]}}
\newcommand{\opapp}[3]{#1_{#2}(#3)}
\newcommand{\opsym}[3]{#1\texttt{[}#2\texttt{]}\texttt{(}#3\texttt{)}}

\newcommand{\weak}[1]{\xrightarrow{\tau*} \xrightarrow{#1}
\xrightarrow{\tau*}}

\newcommand{\barb}[4]{#1 \Downarrow^{#2} (#3, #4)}

\newcommand{\brac}[1]{\{#1\}}
\newcommand{\EPR}[0]{\mathit{EPR}}
\newcommand{\sfqv}[0]{\mathit{qVar}}
\newcommand{\pair}[2]{(#1, #2)}
%-------------------
\newtheorem{defi}{Definition}
\newtheorem{prf}[defi]{Proof}
\newtheorem{prop}[defi]{Proposition}
\newtheorem{thm}[defi]{Theorem}
\newtheorem{lem}[defi]{Lemma}
\newtheorem{clm}[defi]{Claim}
\newtheorem{col}[defi]{Corollary}
\newtheorem{ex}[defi]{Example}

%-------------------

\etitle{Application of Formal Methods \\
to Quantum Cryptography}
\jtitle{����Ū��ˡ���̻ҰŹ�ؤα���}
\eauthor{Takahiro Kubota}
\jauthor{������ ����}
\esupervisor{Masami Hagiya}
\jsupervisor{��ë ����}
\supervisortitle{Professor} % Professor, etc.
\date{December 13, 2013}
%-------------------
\begin{document}
%\listoftables
%-------------------
We next introduce researches about formal verification of
classical security protocols.
\subsection{CryptoVerif}
$\mathsf{CryptoVerif}$ \cite{...} is a software tool to verify security of
\emph{classical} protocols, which is applied to a number of
protocols. 
There are two features of $\mathsf{CryptoVerif}$ that are 
related to our verifiers: it is designed on the basis of a probabilistic
process calculus, and it handles negligibility.
Of course, it cannot be applied to verify security of QKD
protocols, because a process, which may be an adversary,
described in $\mathsf{CryptoVerif}$'s framework 
is bounded in polynomial time with respect to security parameters, while
an advesary against a QKD protocol is not. 
Nevertheless, there are several points which are worth comparing to
ours.

As a proof technique, $\mathsf{CryptoVerif}$ applies 
\emph{observational equavalence} of processes.
Let $\Pr(P \rightsquigarrow a)$
be the probability\footnote{Semantics of a probabilistic process
calculus is possibily be both nondeterministic and probabilistic but
that of $\mathsf{CryptoVerif}$'s framework is only
probabilistic. When there are choices in a execution point, 
the same probability is assigned to each.}
that the process $P$ transits to a process that is ready to send some
data through the channel $a$.
Two processes $P$ and $Q$ are observationally equivalent, written
$P \approx Q$, if
$$
|\Pr(C[P] \rightsquigarrow a) - \Pr(C[Q] \rightsquigarrow a)|
$$
is negligible for all evaluation context\footnote{An evaluation context
is composed by a hole, channel restrictions, and
parallel compositions.} $C[\_]$ running in polynomial time
and channel $a$ that is
not restricted. 
The relation $\approx$ is congruent by the definition,
namely, if $P \approx Q$ holds, then $C[P] \approx C[Q]$ holds for all
evaluation context $C[\_]$ running in polynomial time.
When we consider $C[\_]$ as a polynomial time
adversary that runs in parallel with the protocol $P$ or $Q$,
the observational equivalence of them is intuitively considered
as indistinguishability of the protocols from the view point of the 
adversary.
% A strong point of $\mathsf{CryptoVerif}$ is that
% it does not idealize cryptographic primitives.

In general, security of cryptographic protocols or schemes
is reduced to security of the cryptographic primitives that a protocol
employs or to difficulty of computing certain functions.
For example, FDH signature scheme \cite{...}
is proven EUF-CMA secure assuming it employs
a one-way trapdoor permutation and a hash function
that can be replaced with the \emph{random oracle}.
In $\mathsf{CryptoVerif}$, a user formalizes such assumptions as
observational equivalence of processes. A possible flow to successfully
verify security of a protocol using $\mathsf{CryptoVerif}$ is as follows.
First, the user formalizes the protocol as a process $P$, which
may contain \emph{bad-event flags} in some conditional branches.
He then gives $\mathsf{CryptoVerif}$ some assumptions
$X_i \approx Y_i ~ (i=1,2,...)$. $\mathsf{CryptoVerif}$ finds
that $P \equiv C[X_i]$ for some evaluation
context $C[\_]$ and $i$, where $\equiv$ means syntactic equality.
Then, $P$ is rewritten to $P_1 \equiv C[Y_i]$. Thanks to the congruence,
$P \approx P_1$ holds. $\mathsf{CryptoVerif}$ continues to
apply the assumptions. $\mathsf{CryptoVerif}$ also rewrites the process
based on general rules preserving the relation $\approx$.
It removes assignments of variables.
Some of the assumptions possibly become applicable after the removing.
It also eliminates branches whose conditions are false.
In a case of success, all the branches contaning the bad-event flags are
eliminated. Let $Q$ be the process obtained after the rewriting and
containing no bad-event flag. Since 
the rewriting preserves $\approx$, $P \approx Q$ holds.
This means that the protocol 
$P$ is secure, because it is indistinguashable with $Q$ where
no bad-event happens.

Let us compare two frameworks, $\mathsf{CryptoVerif}$'s and ours.
First, equivalence of processes or configurations is characterized by
observational
equivalence in the former, while it is by weak bisimulation in the
latter. 
No notion of bisimulation processes in the former
framework has been proposed, although bisimulation is often easier to verify
than observational equivalence in which \emph{arbitary evaluation
contexts} are taken into account.

In $\mathsf{CryptoVerif}$'s framework, 
a case seems to be rare where we must directly prove observational
equivalence of processes $P$ and $Q$ taking an arbitarary 
evaluation context $C[\_]$. What one must prove for a classical
security protocol is that the security of it is reduced to the
assumed security of employed primitives or difficulity of computing
certain functions. Such assumptions are formalized as observational
equivalence but they are not what we must prove. The
observational equivalence that means the security of the target protocol
is derived from the assumed observational equivalence, where
arbitrary property of the adversary against the target protocol are
reduced to those of the adversary against the primitives or 
algorithm for the functions. As a proof strategy, syntactic
transformation of a code formalizing a protocol is
more efficient than tracing all the possible execution paths of it.

In qCCS's framework, no notion of observational equivalence that can be
considered as an analogy of $\mathsf{CryptoVerif}$'s has not been
proposed. The barbed congruence \cite{DengFeng2012} seems to be stronger
as it is equal to the bisimulation.
Fortunately, the bisimulation is congruence, which possibly allow us
syntactic reasoning: one may derive bisimilarity of small-size
configurations from small-size ones.
One main diffirence is that 
when we prove security of quantum protocols,
we do not assume security of primitives or difficulty of computing
certain functions. Therefore, we possibly need to prove bisimilarity of
small-size configurations unlike in verification using $\mathsf{CryptoVerif}$.

\nocite{*}
\bibliographystyle{plain}
%\bibliographystyle{unsrt}
\bibliography{main}

\end{document}
