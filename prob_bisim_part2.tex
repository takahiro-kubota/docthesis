\section{Approximate Bisimulation up to Negligible Difference}
\label{neg}
In this section we assume that quantum states depend on
security parameters, namely, 
we assume that $\H$ is a function of natural numbers.
For example, an $n$ bit randomness is represented as
the operator 
$(\frac{1}{2}\ket{0}\bra{0} + \frac{1}{2}\ket{1}\bra{1})^{\otimes n}$
whose state space $\H$ is $2^n$-dimensional.
As for transitions of configurations, we define
that $\con{P}{\rho} \xrightarrow{\alpha} \con{P'}{\rho'}$ holds if and
only if 
$\con{P}{\rho(n)} \xrightarrow{\alpha} \con{P'}{\rho'(n)}$ holds 
for all $n$. Trace distance
$d(\rho, \sigma)$ is also regarded as a
function of natural numbers.
For example, $d(\ket{0}\bra{0}^{\otimes n}, \ket{+}\bra{+}^{\otimes n})
= 1 - \frac{1}{2^n}$ holds.
if $d(\rho, \sigma)$ and $d(\sigma, \theta)$ are negligible,
then $d(\rho, \theta)$ is negligible. 
As a result, we could define the
approximate bisimulation relation $\sim$, which is
transitive and thus is an equivalence relation.
The definition of an approximate bisimulation relation 
is as follows.
\begin{defi}
\label{neg:defofapproxbisim}
 A symmetric relation $\R \subseteq \C \times \C$ is called an
 approximate bisimulation
 if for all $\con{P}{\rho} \R \con{Q}{\sigma}$,
 \begin{enumerate}
  \item $\qv{P} = \qv{Q} \defeq \tilde q$,
  \item $d(\tr{\tilde q}{\rho}, \tr{\tilde q}{\sigma})$ 
	is negligible, and
  \item for all CP map $\E_{\tilde r}$ acting
	on $\tilde r \subseteq \sfqv - \tilde q$,
	if $\con{P}{\E_{\tilde r}(\rho)}
	\xrightarrow{\alpha}
	\con{P'}{\rho'}$ holds and $\tr{}{\rho'}$ is
	non-negligible, then
	$\con{Q}{\E_{\tilde r}(\sigma)} \weak{\hat \alpha}
	\con{Q'}{\sigma'}$ and
	$\con{P'}{\rho'} \R \con{Q'}{\sigma'}$ holds for
	some $\con{Q'}{\sigma'}$. 
 \end{enumerate}
\end{defi}
We call the above conditions {\it 1}, {\it 2} the static conditions and
{\it 3} the simulation condition.

\begin{defi}
\label{neg:defofapprox}
 We define 
 \[
 \sim = \{(\C, \D) \in 
 \Con \times \Con~|~
 \C\R\D \mbox{ holds for some approximate bisimulation } \R \}.
 \]
 We say $\C$ and $\D$ are approximately bisimilar if
 $\C \sim \D$. 
\end{defi}

There is another possible definition of the relation.
Let us replace the requirement ``$\tr{}{\rho'}$ is non-negligible'' in the condition
3 with ``$\frac{\tr{}{\rho'}}{\tr{}{\E_{\tilde r}(\rho)}}$ is non-negligible'', and
let $\simeq$ be the relation defined similarly to
Definition \ref{neg:defofapprox}.
Since $\frac{\tr{}{\rho'}}{\tr{}{\E_{\tilde r}(\rho)}} \ge
\tr{}{\rho'}$ holds, $\simeq \subseteq \sim$
holds. In fact, the relation $\simeq$
has properties that are similar to those discussed in the following
propositions, lemmas, and theorem.
We adopt Definition \ref{neg:defofapproxbisim} for the following reason.
It is natural that we assume a configuration $\con{P_0}{\rho_0}$ satisfies
$\tr{}{\rho_0} = 1$ when it formalizes a protocol.
Suppose we have $\con{P_0}{\rho_0} \xrightarrow{\alpha_0} \cdots
\xrightarrow{\alpha_k} \con{P}{\rho} \xrightarrow{\alpha}
\con{P'}{\rho'}$, that $\tr{}{\rho}$ is non-negligible and
that $\tr{}{\rho'}$ is negligible. The probability to reach
$\con{P}{\rho}$ is non-negligible and to reach $\con{P'}{\rho'}$ is
negligible.
Therefore, we want to care $\con{P}{\rho}$ but ignore $\con{P'}{\rho'}$.
However, a case is possible where a configuration $\con{Q_0}{\sigma_0}$
must simulate the transition $\con{P}{\rho} \xrightarrow{\alpha}
\con{P'}{\rho'}$ to satisfy $\con{P_0}{\rho_0} \simeq
\con{Q_0}{\sigma_0}$. This is because $\frac{\tr{}{\rho'}}{\tr{}{\rho}}$
can be non-negligible even if $\tr{}{\rho'}$ is negligible and
$\tr{}{\rho}$ is non-negligible by the definition of negligible
functions. We thus cannot ignore $\con{P'}{\rho'}$.

The following lemmas are similarly proven as the previous section.
\begin{lem}
\label{neg:coinduction}
$\con{P}{\rho} \sim \con{Q}{\sigma}$ holds,
 iff
 \begin{enumerate}
  \item $\qv{P} = \qv{Q} \defeq \tilde q$,
  \item $d(\tr{\tilde q}{\rho}, \tr{\tilde q}{\sigma})$
	is negligible, and
  \item for all CP map $\E_{\tilde r}$ acting
	on $\tilde r \subseteq \sfqv - \tilde q$,
	\begin{itemize}
	 \item if $\con{P}{\E_{\tilde r}(\rho)} \xrightarrow{\alpha}
	       \con{P'}{\rho'}$ holds and $\tr{}{\rho'}$ is
	       non-negligible, then
	       there exists $\con{Q'}{\sigma'}$ satisfying
	       $\con{Q}{\E_{\tilde r}(\sigma)} \weak{\hat \alpha}
	       \con{Q'}{\sigma'}$ and
	       $\con{P'}{\rho'} \sim \con{Q'}{\sigma'}$,
	       and
	 \item if $\con{Q}{\E_{\tilde r}(\sigma)} \xrightarrow{\alpha}
	       \con{Q'}{\sigma'}$ holds and $\tr{}{\sigma'}$ is
	       non-negligible, then
	       there exists $\con{P'}{\rho'}$ satisfying
	       $\con{P}{\E_{\tilde r}(\rho)} \weak{\hat \alpha}
	       \con{P'}{\rho'}$ and 
	       $\con{P'}{\rho'} \sim \con{Q'}{\sigma'}$.
	\end{itemize}
 \end{enumerate}
\end{lem}
\begin{lem}\label{neg:soclosed}
 If $\con{P}{\rho} \sim \con{Q}{\sigma}$, then 
 $\con{P}{\E_{\tilde r}(\rho)} \sim \con{Q}{\E_{\tilde r}(\sigma)}$
 for all CP map $\E_{\tilde r}$ that acts on $\tilde r$. 
\end{lem}
\begin{prop}
\label{neg:sym-by-par}
 $\con{P||Q}{\rho} \sim \con{Q||P}{\sigma}$.
\end{prop}

We then prepare lemmas to prove transitivity of the relation
$\sim$.
\begin{lem}
\label{neg:weaksimulated}
 If $\con{P}{\rho} \sim \con{Q}{\sigma}$ and 
$\con{P}{\rho} \xrightarrow{\tau\ast} \con{P'}{\rho'}$ and
 $\tr{}{\rho'}$ is non-negligible, then
$\con{Q}{\sigma} \xrightarrow{\tau\ast}
 \con{Q'}{\sigma'}$ and $\con{P'}{\rho'} \sim \con{Q'}{\sigma'}$ hold
 for some $\con{Q'}{\sigma'}$.
\end{lem}
\begin{proof}
 Assume $\con{P}{\rho} \xrightarrow{\tau}^n \con{P'}{\rho'}$
 and let $\con{P_i}{\rho_i}$ be $i$-th configuration with $0 \le i \le
 n$. The case when $n = 0$ is trivial. Let $n > 0$.
 Since $\rho = \rho_0 \ge \rho_1 \cdot\cdot\cdot \ge \rho_n =
 \rho'$ holds and $\tr{}{\rho'}$ is non-negligible, 
 $\tr{}{\rho_i}$ is non-negligible for all $i$. Therefore,
 $\con{Q}{\sigma} \xrightarrow{\tau\ast}^n
 \con{Q'}{\sigma'}$ and $\con{P'}{\rho'} \sim \con{Q'}{\sigma'}$ hold
 for some $\con{Q'}{\sigma'}$.
\end{proof}

\begin{lem}
\label{neg:simulateweak}
If $\con{P}{\rho} \sim \con{Q}{\sigma}$ and 
$\con{P}{\rho} \weak{\hat \alpha} \con{P'}{\rho'}$ and
 $\tr{}{\rho'}$ is non-negligible, then
$\con{Q}{\sigma} \weak{\hat \alpha}
 \con{Q'}{\sigma'}$ and $\con{P'}{\rho'} \sim \con{Q'}{\sigma'}$
 for some $\con{Q'}{\sigma'}$.
\end{lem}

\begin{proof}
 It is similarly proven as the previous lemma.
\end{proof}

\begin{lem}
\label{neg:equivalence}
 $\sim$ is an equivalence relation.
\end{lem}
\begin{proof}
 {\bf (Reflexivity)} 
 Let $\mathit{Id}_\C$ be the identity relation on $\C$.
 For all $\con{P}{\rho} \in \C$, 
 $\con{P}{\rho}\mathit{Id}_\C \con{P}{\rho}$ holds.
 It is sufficient to show
 $\mathit{Id}_\C$ is an approximate bisimulation. 
 Assume $(\con{P}{\rho},
 \con{P}{\rho})$ is an arbitrary element in $\mathit{Id}_\C$ and
 $\tr{}{\rho}$ is non-negligible. The static conditions are 
 easily checked.
 Let $\E_{\tilde r}$ be an arbitrary CP map and assume
 $\con{P}{\E_{\tilde r}(\rho)} \xrightarrow{\alpha}
 \con{P'}{\rho'}$ and $\tr{}{\rho'}$ is negligible. As
 $\con{P'}{\rho'}\mathit{Id}_\C \con{P'}{\rho'}$ holds,
 $\mathit{Id}_\C$
 is an approximate bisimulation. \\
 {\bf (Symmetry)} 
 ($\Rightarrow$) implication of
 Lemma \ref{neg:coinduction} is the condition that $\sim$ is 
 an approximate bisimulation.
 An approximate bisimulation relation
 is defined to be symmetric. 
 \\
 {\bf (Transitivity)} It is sufficient to show $\sim \circ \sim$
 is an approximate bisimulation relation. Let $(\con{P}{\rho},
 \con{R}{\theta})$ be an arbitrary element of $\sim \circ
 \sim$. There exists $\con{Q}{\sigma}$ satisfying
 $\con{P}{\rho} \sim \con{Q}{\sigma}$ and $\con{Q}{\sigma}
 \sim {\con{R}{\theta}}$. 
 The static conditions are easily checked using
 triangle inequality of trace distance $d(\cdot, \cdot)$.
 Let $\E_{\tilde r}$ be an arbitrary
 CP map acting on $\tilde r \subseteq \sfqv - \qv{P}$ and
 assume $\con{P}{\E_{\tilde r}(\rho)} \xrightarrow{\alpha}
 \con{P'}{\rho'}$ and $\tr{}{\rho'}$ is non-negligible.
 By $\con{P}{\rho} \sim \con{Q}{\sigma}$, there exists
 $\con{Q'}{\sigma'}$ satisfying $\con{Q}{\E_{\tilde r}(\sigma)} 
 \weak{\hat \alpha} \con{Q'}{\sigma'}$ and $\con{P'}{\rho'} \sim
 \con{Q'}{\sigma'}$. By its static conditions, we have
 $d(\tr{\qv{P'}}{\rho'}, \tr{\qv{Q'}}{\sigma'})$ is negligible.
 This implies $|\tr{}{\rho'} - \tr{}{\sigma'}|$ is 
 negligible and thus we have
 that $\tr{}{\sigma'}$ is non-negligible.
 We have $\con{Q}{\E_{\tilde r}(\sigma)}
 \sim \con{R}{\E_{\tilde r}(\theta)}$ applying lemma
 \ref{neg:soclosed} to $\con{Q}{\sigma}
 \sim {\con{R}{\theta}}$. Next by lemma
 \ref{neg:simulateweak}, we have $\con{R}{\E_{\tilde r}(\theta)}
 \weak{\hat \alpha} \con{R'}{\theta'}$ and $\con{Q'}{\sigma'}
 \sim \con{R'}{\theta'}$ for some $\con{R'}{\theta'}$. 
 Therefore, $\con{P'}{\rho'}\sim \circ \sim \con{R'}{\theta'}$.
\end{proof}
We prove congruence of the relation $\sim$. 
We first show that $\sim$ is closed by restriction.
\begin{lem}
\label{neg:resclosed}
 If $\con{P}{\rho} \sim \con{Q}{\sigma}$ holds, then
 $\con{P \backslash L}{\rho} \sim \con{Q \backslash L}{\sigma}$ holds.
\end{lem}
\begin{proof}
It is similarly proven as Lemma \ref{par:resclosed}.
\end{proof}

The next theorem states that the relation $\sim$ is closed
under parallel composition of processes.
With this theorem and Lemma \ref{neg:resclosed}, we immediately 
have that $\sim$ is closed by application of an arbitrary evaluation
context.
The structure of the proof is same as \ref{par:pallaclosed}. 
\begin{thm}
\label{neg:parclosed}
  If $\con{P}{\rho} \sim \con{Q}{\sigma}$, then
 $\con{P||R}{\rho} \sim \con{Q||R}{\sigma}$ for all
 process $R$.
\end{thm}
\begin{proof}
 We define 
\[
 \R := \{(\con{P||R}{\rho}, \con{Q||R}{\sigma})\,|\,
 \con{P}{\rho} \sim \con{Q}{\sigma}, R \in \mathcal{P}
 \}. 
\]
 It is sufficient to show $\R$ is an approximate bisimulation.
 $\R$ is symmetric by the definition.
 Let $(\con{P||R}{\rho}, \con{Q||R}{\sigma})$ be an arbitrary element in
 $\R$. The static conditions and the simulation condition 
 are checked similarly to Theorem \ref{par:pallaclosed}.
\end{proof}
Similarly to the discussion in the previous section,
we have the following corollary.
\begin{col}
\label{neg:congruence}
 If $\con{P}{\rho} \sim \con{Q}{\sigma}$ holds, then
 $\con{C[P]}{\rho} \sim \con{C[Q]}{\sigma}$ holds for
 all evaluation context $C[\_]$.
\end{col}
We have proved that the relation $\sim$ is equivalence and closed
by application of an arbitrary evaluation context.
We thus say it is congruent. The congruence 
suggests sanity of the definition as well as feasibility in practice.
For example, it allows us to infer equivalence of
multiple sessions of protocols.
\begin{col}
\label{neg:multisession}
 If $\con{P_1}{\rho_1 \otimes \rho^E_1} \sim \con{Q_1}{\sigma_1
 \otimes \rho^E_1}$,
$\con{P_2}{\rho_2 \otimes \rho^E_2} \sim \con{Q_2}{\sigma_2 \otimes
 \rho^E_2}$ and $\qv{P_1}\cap \qv{P_2}=\qv{P_1}\cap \qv{Q_2}=\qv{Q_1}\cap
 \qv{P_2}=\qv{Q_1}\cap \qv{Q_2}=\emptyset$
 hold for all $\rho^E_1, \rho^E_2$, then
$\con{P_1 || P_2}{\rho_1 \otimes \rho_2 \otimes \rho^E} \sim \con{Q_1 ||
 Q_2}{\sigma_1 \otimes \sigma_2 \otimes \rho^E}$ holds for all $\rho^E$.
\end{col}
\begin{proof}
We have $\con{P_1}{\rho_1 \otimes \rho_2 \otimes \rho^E}
\sim \con{Q_1}{\sigma_1 \otimes \rho_2 \otimes \rho^E}$ by substituting
$\rho_2 \otimes \rho^E$ for $\rho^E_1$ in the assumption.
By congruence, we have $\con{P_1||P_2}{\rho_1 \otimes \rho_2 \otimes \rho^E} \sim
 \con{Q_1||P_2}{\sigma_1 \otimes \rho_2 \otimes \rho^E}$.
Similarly, we have $\con{Q_1||P_2}{\sigma_1 \otimes \rho_2 \otimes \rho^E} \sim
 \con{Q_1||Q_2}{\sigma_1 \otimes \sigma_2 \otimes \rho^E}$.
By transitivity of $\sim$, we obtain the conclusion.
\end{proof}
Let configurations $\con{P_i}{\rho \otimes \rho^E_i}$ and
$\con{Q_i}{\sigma \otimes \rho^E_i}$ formalize
an actual and an ideal protocols for $i = 1,2$.
By the above corollary, we have
$\con{P_1 || P_2}{\rho_1 \otimes \rho_2 \otimes \rho^E} \sim \con{Q_1 ||
 Q_2}{\sigma_1 \otimes \sigma_2 \otimes \rho^E}$.
This means that $\con{P_1 || P_2}{\rho_1 \otimes \rho_2 \otimes \rho^E}$
is approximately secure,
provided that the ideal protocol is secure even if they run in
parallel. The latter condition depends on protocols but possibly be
satisfied. In fact, EDP-ideal protocol that we consider in the next
chapter satisfies the condition, because Alice and Bob generate
a shared key using pre-shared EPR pairs.
Another example of application is discussed
in the subsection Outputting Secret Keys in Section
\ref{fml:policiesandtechs}.

% \section{Comparison of the Two Relations}
% The relation $\sim$ is 
% defined only when the notion of negligibility makes sense
% while the relation $\sim_{\zeta,\eta}$ is defined only when
% the inequality $\ge$ makes sense. It is not a trivial task
% compare the two relations. It is possible to assume that 
% the parameters $\zeta$ and $\eta$ are functions of security parameters, but
% to define an inequality $\ge$ on such functions appropriately is not a
% trivial task. 

% Let us assume again that quantum states depend on security parameters,
% and for $f,g:\mathbb{N}\rightarrow [0,1]$, define $f \ge g$ iff $f(n)
% \ge g(n)$ for all $n$. If
% $\con{P}{\rho}\sim_{\zeta, \eta}\con{Q}{\sigma}$ holds for some 
% negligible functions $\zeta, \eta$ with respect to $n$,
% then $\con{P}{\rho} \sim \con{Q}{\sigma}$ holds. We do not know, however,
% whether the converse is true. When we assume $\con{P}{\rho} \sim
% \con{Q}{\sigma}$, by the condition 3 in Lemma \ref{neg:coinduction}, we
% have
% \[
%  \forall \E_{\tilde r}:\mbox{CP map}.\, \exists
%  \eta:\mbox{non-negligible}. \cdots
% \]
% but this does not immediately imply
% \[
%  \exists \eta:\mbox{non-negligible}.\, 
% \forall \E_{\tilde r}:\mbox{CP map}. \cdots.
% \]

Although we use only the relation $\sim$ for
the verification in this thesis, the relation 
$\sim_{\zeta, \eta}$ will be useful when we
evaluate the gap of two configurations quantitatively.
By $\con{P}{\rho}\sim \con{Q}{\sigma}$, the value of the trace distance
is simply
understood to be negligible, but it cannot be evaluated more explicitly.
% Besides, let us focus again on parallel composition.
% For the relation $\sim$, we
% have already discussed Corollary \ref{neg:multisession} but
% the number of parallel composition cannot depend on a security
% parameter.
Using the relation $\sim_{\zeta, \eta}$,
the gap can be evaluated concretely. For example,
if $\con{P_i}{\rho_i \otimes \rho^E_i} \sim_{\zeta,\eta}
\con{Q_i}{\sigma_i \otimes \rho^E_i}$ holds
for all $\rho^E_i \in \D(\H_{\sfqv - \qv{P_i}})$
for all $i \in [1..k]$ and $\qv{P_1} \cap \cdots \cap \qv{P_k} =
\emptyset$ holds, then we have
\[
 \con{P_1||\cdots||P_k}{\rho} \sim_{\zeta', \eta'}
 \con{Q_1||\cdots||Q_k}{\sigma},
\]
where $\zeta' = k\zeta$ and $\eta' = (k-1)(k+2)\zeta + \eta$.

\section{Guarantees of Approximate Bisimulation}
\subsection{Application to Verification of QKD protocols' Security}
We explain about feasibility of the relation $\sim$ for verification
of security of QKD protocols. In the next chapter, we will verify
$\con{P}{\rho} \sim \con{Q}{\sigma}$, where
$\con{P}{\rho}$ and $\con{Q}{\sigma}$ are configurations formalizing the EDP-based protocol
(Section \ref{pre:EDPbased}) and EDP-ideal (Section \ref{fml:formalverif2}).
% a sort of cheating protocol named
% EDP-ideal. In EDP-ideal protocol, 
% Alice and Bob initially share EPR pairs, and
% they execute the same protocol as 
% the EDP-based protocol until the decision of continue or aborting
% by the result of the error checking. Only when they
% decide to continue, they create the secret keys using pre-shared EPR
% pairs instead of pairs obtained after the entanglement distillation
% protocol.
% Therefore, if the secret keys is created, Eve has no information
% of the keys.
Assume $\con{P}{\rho} \sim \con{Q}{\sigma}$ and
\[
 \con{P}{\E_{\tilde r}(\rho)} \xrightarrow{\alpha} 
 \con{P_1}{\E^1_{\tilde r_1}(\rho_1)} \xrightarrow{\alpha_1} \cdots 
 \xrightarrow{\sndq{ska}{k_A}} \con{P'}{\rho'}
\]
and $\tr{}{\rho'}$ is non-negligible, where
the last transition $\xrightarrow{\sndq{ska}{k_A}}$
represents that Alice's key $k_A$ is created\footnote{In Chapter 5, we
actually formalize the protocols as configurations
that do such transitions. This point is discussed in Section \ref{fml:outputtingsks}}.
Then, there exists the following
transition
\[
 \con{Q}{\E_{\tilde r}(\sigma)} \weak{\alpha} 
 \con{Q_1}{\E^1_{\tilde r_1}(\sigma_1)} \weak{\alpha_1} \cdots 
 \weak{\sndq{ska}{k_A}} \con{Q'}{\sigma'}
\]
such that $d(\tr{\qv{P'}}{\rho'}, \tr{\qv{Q'}}{\sigma'})$ is negligible.
By Proposition \ref{par:trdisproperty}, we have that
\begin{align*}
 &|\tr{}{\rho'} - \tr{}{\sigma'}| \mbox{ and }
 |\tr{}{\rho'}\cdot\tr{}{\pi\frac{\tr{\qv{P'}}{\rho'}}{\tr{}{\rho'}}} -
 \tr{}{\sigma'}\cdot\tr{}{\pi\frac{\tr{\qv{Q'}}{\sigma'}}{\tr{}{\sigma'}}} | 
\end{align*}
are negligible for all projector $\pi$. Especially, let $\pi$ be
the projector to the subspace where $i$-th bits of Alice's key and
Eve's key are equal. We can rephrase the above expression as follows.
\begin{align*}
 &|\Pr(A) - \Pr(B)| \mbox{ and }
 |\Pr(A)\Pr(k_{A,i} = k_{E,i}|A) - \Pr(B)\Pr(k'_{A,i} = k'_{E,i}|B)|
\end{align*}
are negligible, where 
\begin{itemize}
 \item $k_{A,i}$ and $k_{E,i}$ are random variables of $i$-th bits of
       Alice's and Eve's keys in the EDP-based protocol, 
 \item $k'_{A,i}$ and $k'_{E,i}$ are those in EDP-ideal, and
 \item $A$ and $B$ are the events that
       $\con{P}{\rho}$ reaches $\con{P'}{\rho'}$ and
       $\con{Q}{\sigma}$ reaches $\con{Q'}{\sigma'}$.
\end{itemize}
Moreover, we have
\begin{align*}
    &|\Pr(A)\Pr(k_{A,i} = k_{E,i}|A) - \Pr(B)\Pr(k'_{A,i} = k'_{E,i}|B)|\\
\le &\Pr(A)|\Pr(k_{A,i} = k_{E,i}|A) - \Pr(k'_{A,i} = k'_{E,i}|B)|\\
 &+ \Pr(k'_{A,i} = k'_{E,i}|B)|\Pr(A) - \Pr(B)|\\
= & \Pr(A)|\Pr(k_{A,i} = k_{E,i}|A) - \frac{1}{2}| + \frac{1}{2}|\Pr(A) - \Pr(B)|.
\end{align*}
The equation $\Pr(k'_{A,i} = k'_{E,i}|B) = \frac{1}{2}$ holds by the
definition
of EDP-ideal. If $\Pr(A)$ is greater than negligible, we have
that $|\Pr(k_{A,i} = k_{E,i}|A) - \frac{1}{2}|$ is negligible.
It seems possible to derive that the mutual information of Alice's and
Eve's keys is negligible. Similarly, we have that Alice's and Bob's
keys are identical with overwhelming probability.

\subsection{On Guarantees in General Cases}
As long as we verify security of an actual protocol $\con{P}{\rho}$ by
proving $\con{P}{\rho} \sim \con{Q}{\sigma}$ for an ideal protocol
$\con{Q}{\sigma}$, we can discuss similarly to the previous
subsection.
Concretely, if $\con{P}{\rho} \sim \con{Q}{\sigma}$, then
for all $\con{P'}{\rho'}$ such that
\[
 \con{P}{\E_{\tilde r}(\rho)} \xrightarrow{\alpha} 
 \con{P_1}{\E^1_{\tilde r_1}(\rho_1)} \xrightarrow{\alpha_1} \cdots 
 \xrightarrow{\alpha_m} \con{P'}{\rho'}
\]
and $\tr{}{\rho'}$ is non-negligible,
there exists $\con{Q'}{\sigma'}$ such that
\[
 \con{Q}{\E_{\tilde r}(\sigma)} \weak{\alpha} 
 \con{Q_1}{\E^1_{\tilde r_1}(\sigma_1)} \weak{\alpha_1} \cdots 
 \weak{\alpha_m} \con{Q'}{\sigma'}
\]
and $\con{P'}{\rho'} \sim \con{Q'}{\sigma'}$. 
The last condition implies $d(\tr{\qv{P'}}{\rho'},
\tr{\qv{Q'}}{\sigma'})$ is negligible.
By proposition \ref{par:trdisproperty}, we have
$$|\tr{}{\rho'} - \tr{}{\sigma'}|
\mbox{ and } \tr{}{\rho'}d(\frac{\tr{\qv{P'}}{\rho'}}{\tr{}{\rho'}},
\frac{\tr{\qv{Q'}}{\sigma'}}{\tr{}{\sigma'}})$$
are negligible. 
 We thus have the following conclusions.
\begin{enumerate}
 \item The probabilitiy to reach $\con{P'}{\rho'}$ from
       $\con{P}{\rho}$ is negligibily colse
       to that to reach $\con{Q'}{\sigma'}$ from $\con{Q}{\sigma}$.
 \item The greater $\tr{}{\rho'}$ we have, the less 
       $d(\frac{\tr{\qv{P'}}{\rho'}}{\tr{}{\rho'}},
       \frac{\tr{\qv{Q'}}{\sigma'}}{\tr{}{\sigma'}})$ we have.
       Especially, if $\tr{}{\rho'}$ is greater than negligible,
       then $d(\frac{\tr{\qv{P'}}{\rho'}}{\tr{}{\rho'}},
       \frac{\tr{\qv{Q'}}{\sigma'}}{\tr{}{\sigma'}})$ is negligible.
\end{enumerate}
When the protocol $\con{P}{\rho}$ is for generation of certain data,
the data will be sent to the outside by the
final transition $\xrightarrow{\alpha_m}$ of the form
$\xrightarrow{\sndq{c}{q}}$,
where $\H_q$ is the state space of the data.
By the condition 2 above, we have that 
whenever the probability to reach $\con{P'}{\rho'}$
from the start point $\con{P}{\rho}$ is greater than negligible,
the data have been almost correctly generated
at $\con{P'}{\rho'}$.

Another important way to examine end guarantees of the approximate
bisimulation is to define \emph{testing equivalence}
\cite{DeNicola1984, Deng2009}
or \emph{observational equivalence} 
\cite{Blanchet2008cryptoverif, Yasuda2014} that includes the
approximate bisimulation. Observationally equivalent processes
become ready to use the same channel with the same probability
under the application of an arbitrary evaluation context.
It is an open problem to define such equivalence that
includes the approximate bisimulation.
Before the approximation,
to define testing or observational equivalence that includes
bisimulation in qCCS is still an open problem.
\emph{Reduction barbed congruence} $\approx_r$ is defined
\cite{DengFeng2012} but it
seems more distinguishing than what is meant by observational
equivalence, because it
requires \emph{reduction closedness}:
if $\con{P}{\rho} \approx_r \con{Q}{\sigma}$ and $\con{P}{\rho}
\Rightarrow \mu$ hold, then there exists $\nu$ satisfying
$\con{Q}{\sigma} \Rightarrow \nu$ and $\mu \approx_r \nu$.
The relation $\approx_r$ in fact coincides with the bisimulation
relation $\approx$.
Yasuda defined two kinds of observational equivalence
$\approx_{\mathit{oe}}$ and $\approx_{\mathit{oe}}^{\mathit{st}}$
on qCCS configurations \cite{Yasuda2014} but neither of them includes
$\approx$.

\section{Automated Verification of Approximate Bisimulation}
\subsection{Extension of the Symbolic Representation and  the Algorithm}
\label{neg:extensionofalgorithm}
We extended Verifier1, which is described in 
Chapter \ref{symqccs}, to verify the approximate bisimilarity.
Let us call the extended verifier \emph{Verifier2}.
We applied Verifier2 to the second part of
Shor-Preskill's security proof. This is described in the next chapter.

Since the outsider's quantum operations are assumed 
to be CP (Definition \ref{neg:defofapproxbisim}),
it was necessary to extend the syntax of symbolic representations of
probability-weighted quantum states. The set $\mathcal{S}$ of the symbolic
representations was defined in Definition \ref{symqccs:symbrep}.
In the definition, we assumed $\mathit{op}$ is an element of 
$S_{\mathit{op}}$, which is a set of TPCP map symbols.
The expressions 
$\mathtt{proj0}\mathtt{[}b\mathtt{](}\rho\mathtt{)}$ and
$\mathtt{proj1}\mathtt{[}b\mathtt{](}\rho\mathtt{)}$ 
represent applications of special CP maps
$\ket{0}\bra{0}_b\braw{\rho}\ket{0}\bra{0}_b$ and
$\ket{1}\bra{1}_b\braw{\rho}\ket{1}\bra{1}_b$
but application of general CP maps
was not allowed in the syntax.

For the extension, we first add a definition
\begin{itemize}
 \item $S_{\mathit{cp}}$ is a set of symbols representing CP maps.
       Symbols $\mathtt{proj0}$ and $\mathtt{proj1}$ are elements of
       $S_{\mathit{cp}}$.
\end{itemize}
We then define the set $\mathcal{S}'$ of extended
symbolic representations. In the definition, 
we do not refer $S_\mathit{op}$ because
TPCP maps are also CP maps.
\begin{defi}
\label{neg:exsymbrep}
\begin{align*}
\mathcal{S}' \ni \rho, \sigma ::=\,\, &X\mathtt{[}\tilde q\mathtt{]} ~|~
\mathit{cp}\mathtt{[}\tilde q\mathtt{](}\rho\mathtt{)} ~|~ \rho \,
  \mathtt{*} \, \sigma
~|~\mathtt{Tr}\mathtt{[}\tilde q\mathtt{](}\rho\mathtt{)}
\end{align*}
where $\mathit{cp} \in S_{\mathit{cp}}$.
\end{defi}

The interpretation $\braw{\cdot}:\mathcal{S}' \rightarrow \Delta(\H)$
is naturally extended.
Verifier2 obeys the same transition rules as the
previous one, and
takes as input two elements in
$\mathcal{P} \times \mathcal{S'}$,
a user-defined set of equations $\mathit{eqs}$
on symbolic quantum states,
and additionally a user-defined set of triples
$\mathit{inds} \subseteq \mathcal{S'} \times \mathcal{S'} \times
S_\mathit{nat}$, which we call {\it indistinguishability expressions}.
An indistinguishability expression $(\rho, \sigma, n)$ intuitively means
the trace distance of $\rho$ and $\sigma$ is negligible with respect to
$n$.

We modified the steps 1, 4, 5, 6, and 7 in the algorithm described in
Section \ref{symqccs:algorithmforbisim}.
The new algorithm of the recursive procedure is as follows.
\begin{enumerate}
 \item The procedure takes as input two configurations
       $\con{P_0}{\rho_0}$,
       $\con{Q_0}{\sigma_0}$ and user-defined equations $\mathit{eqs}$
       and indistinguishability expressions $\mathit{inds}$
       on quantum states.
 \item If $P_0$ and $Q_0$ can perform any $\tau$-transitions of
       TPCP map applications, they are all performed at this point.
       Let $\con{P}{\rho}$ and $\con{Q}{\sigma}$ be obtained
       configurations.
 \item Whether $\qv{P} = \qv{Q}$ is checked. If it does not hold, the
       procedure returns $\mathit{false}$.
 \item Whether $\ptrtt{\qv{P}}{\rho} = \ptrtt{\qv{Q}}{\sigma}$ is
       checked using $\eqs$ and $\inds$.
       If it does not hold, the procedure returns $\mathit{false}$.
 \item A new {\it CP} map symbol $\E\texttt{[}\qv{\rho} - \qv{P}\texttt{]}$
       that stands for an
       arbitrary operation is generated. 
 \item For each $\con{P'}{\rho'}$ such that
       $\con{P}{\E\texttt{[}\qv{\rho} -
       \qv{P}\texttt{](}\rho\texttt{)}} \xrightarrow{\alpha}
       \con{P'}{\rho'}$,
       the procedure checks whether there exists $\con{Q'}{\sigma'}$ such
       that \\
       $\con{Q}{\E\texttt{[}\qv{\sigma} -
       \qv{Q}\texttt{](}\sigma\texttt{)}} \weak{\hat \alpha}
       \con{Q'}{\sigma'}$ and 
       the procedure returns $\mathit{true}$ with input
       $\con{P'}{\rho'}$,
       $\con{Q'}{\sigma'}$, and $\mathit{eqs}$. If there exists, it goes
       to the
       next step 7. Otherwise, it returns $\mathit{false}$.
 \item For each $\con{Q'}{\sigma'}$ such that
       $\con{Q}{\E\texttt{[}\qv{\sigma} -
       \qv{Q}\texttt{](}\sigma\texttt{)}} \xrightarrow{\alpha}
       \con{Q'}{\sigma'}$,
       the procedure checks whether there exists $\con{P'}{\rho'}$ such
       that\\
       $\con{P}{\E\texttt{[}\qv{\rho} -
       \qv{P}\texttt{](}\rho\texttt{)}} \weak{\hat \alpha}
       \con{P'}{\rho'}$ and 
       the procedure returns $\mathit{true}$ with input
       $\con{P'}{\rho'}$ and
       $\con{Q'}{\sigma'}$, and $\mathit{eqs}$. If there exists, it
       returns
       $\mathit{true}$. Otherwise, it returns $\mathit{false}$.
\end{enumerate}
The way to use $\inds$ to test $\ptrtt{\qv{P}}{\rho} =
\ptrtt{\qv{Q}}{\sigma}$
is similar to that of $\eqs$, that is,
for $(\rho, \sigma, n) \in \inds$,
a part in an objective quantum state
that matches to $\rho$ is rewritten to $\sigma$.

\subsection{Correctness of the Extended Verifier}
A user-defined set $\inds$ is said to be valid if for all
element $(\rho, \sigma, n) \in \mathit{inds}$, $d(\braw{\rho},
\braw{\sigma})$ is a negligible function of $\braw{n}$.
Let $\mathit{eqs}$ and $\inds$ be valid.
Let a relation $\R_{\eqs, \inds} \subseteq \C \times \C$ be defined
as follows.
\begin{align*}
 \R_{\eqs, \inds} :=
 \{&(\con{P}{\braw{\rho}},
 \con{Q}{\braw{\sigma}}) \,|\,
 \mbox{ Verifier2 returns } \mathit{true}
 \mbox{ with }\\
 & \con{P}{\rho}, \con{Q}{\sigma} \mbox{ using } \mathit{eqs}
 \mbox{ and } \mathit{inds}.
\}
\end{align*}
The relation $\R_{\eqs, \inds}$ is an approximate bisimulation
relation.

The argument about the correctness of Verifier2
is basically similar to that about the original one.
We focused on the two different points. The first is
that whether the trace distance of objective quantum
states is negligible or not is tested instead of the equality of
the states. The second point is about the simulation condition.

\subsubsection{On the Static Condition of Partial Trace}
It is necessary to check the partial rewriting of quantum states
done by Verifier2 is correct. It rewrites a
symbolic representation of the form $\rho_l \, \mathtt{*}\,
 \rho \,\mathtt{*}\, \rho_r$ to
the symbolic representation $\rho_l \, \mathtt{*}\, \sigma
 \,\mathtt{*}\, \rho_r$ given
$(\rho, \sigma, n) \in \inds$. 
The correctness is guaranteed from the fact that 
$d(X, Y) = d(X_l \otimes X \otimes X_r,
X_l \otimes Y \otimes X_r)$ holds for all $X, Y, X_l, X_r \in
\Delta(\H)$. If $(\rho, \sigma, n)$ is valid, namely 
$d(\braw{\rho}, \braw{\sigma})$ is negligible with respect to $n$, 
then $d(\braw{\rho_l \, \mathtt{*}\,
 \rho \,\mathtt{*}\, \rho_r},
\braw{\rho_l \, \mathtt{*}\,
 \sigma \,\mathtt{*}\, \rho_r})$ is negligible
 with respect to $n$ because
$d(\braw{\rho_l \, \mathtt{*}\,
 \rho \,\mathtt{*}\, \rho_r},
\braw{\rho_l \, \mathtt{*}\,
 \sigma \,\mathtt{*}\, \rho_r}) =
d(\braw{\rho_l} \otimes
 \braw{\rho} \otimes \braw{\rho_r},
\braw{\rho_l} \otimes
 \braw{\sigma} \otimes \braw{\rho_r}
)=
d(\braw{\rho},\braw{\sigma})
$ holds.

\subsubsection{On the Simulation Condition}
The simulation condition of approximate bisimulation is only required
to transitions with non-negligible probability, stating
\begin{itemize}
 \item If $\con{P}{\E_{\tilde r}(\rho)}
       \xrightarrow{\alpha}
       \con{P'}{\rho'}$ holds and $\tr{}{\rho'}$ is
       non-negligible, then\\
       $\con{Q}{\E_{\tilde r}(\sigma)} \weak{\hat \alpha}
       \con{Q'}{\sigma'}$ and 
       $\con{P'}{\rho'} \R \con{Q'}{\sigma'}$ hold
       for some $\con{Q'}{\sigma'}$.
\end{itemize}
However, Verifier2 does not check whether the probability of a
transition is non-negligible or not. It returns $\mathit{false}$
when a transition cannot be simulated even if the probability is
negligible. As for simulation,
the condition that Verifier2 returns $\mathit{true}$
is strictly stronger than the simulation condition of the approximate
bisimulation.

\section{Discussion}
\subsection{Relation between the Verifiers}
Before the extension, the steps 6 and 7 required the 
correspondence of qubit variable $b$ when there is a transition
caused by (Meas0) or (Meas1) rules.
%, in other words, caused by a part of a process of the form
%$\measure{b}{P}$.
The proof of Verifier1's soundness was made easier by this condition.
Verifier2 does not check such correspondence, which straightforwardly
checks the
conditions stated in the definition of the relation $\sim$.
On this point, Verifier1 checks more strict condition.
On the other hand, for the step 5, Verifier2
generates a CP map symbol representing the outsider's operation, which
is not necessarily TPCP, while Verifier1 generates a TPCP map symbol.
On this point, Verifier2 checks more strict condition.

In fact, whether an outsider's operation is TPCP or CP
does not matter in verification in most cases.
Only the rewriting rule (3.2) of
partial traces in Section \ref{symqccs:traceoutalgo} cannot
be applied if $\mathit{op}$ is a CP map symbol. Except for
it, quantum operators are treated equivalently in the both verifiers
whether they are TPCP or CP.
Moreover, even if outsider's operators are TPCP,
the rule (3.2) cannot be applied to them
in most cases, where we assume the outsider has her own
``local memory''.
Assume she does not send a quantum variable $q^E$
to the insider (the process).
It is in the domain of her operation $\E$ in general.
Hence, an arbitrary symbol generated in the step 5 is of the form 
$\E\texttt{[}\cdots, q^E, \cdots\texttt{]}$.
However, she does not send $q^E$ to the process.
Therefore, in any expressions of the 
form 
\[
 \ptrtt{\tilde q}{\cdots\E\texttt{[}\cdots, q^E, \cdots\texttt{]}
 \cdots},
\]
which appear in the test of equality of partial traces, 
$q^E \notin \tilde q$ holds and thus the rule (3.2) cannot be applied.

Let us now consider a new verifier, which we call Verifier3,
that generates {\it CP} maps in the 
step 5 but executes the same algorithm as Verifier1.
We have that Verifier3 verifies more strict condition
than Verifier2, in other words, if Verifier3 returns 
$\mathit{true}$ with configurations $\con{P}{\rho}$, $\con{Q}{\sigma}$,
and user-defined equations $\mathit{eqs}$, Verifier2
returns $\mathit{true}$ with the same input.
