\chapter{Automated Verification of Bisimilarity of qCCS configurations}
\label{symqccs}
\section{qCCS}
We introduce the qCCS formal framework presented by Deng and Feng
\cite{DengFeng2012}.
Three data types $\mathit{Bool}, \mathit{Real}$, and $\mathit{Qbt}$ are
used for booleans, real numbers, and qubits, respectively.
Let $\mathit{cVar}$ be a countably infinite set for
classical variables, and $\sfqv$ be a finite
set\footnote{The set of quantum variables is
countably infinite in the original qCCS and each element
represents a qubit, not a qubit string.} $\sfqv$
for quantum variables. $\mathit{cVar}$ and $\sfqv$ are
ranged over by $x, y, z,...$ and $q, r,...$.
For each $q \in \sfqv$, its qubit-length $|q|$ is defined.
A finite sequence of quantum variables
is written $\tilde q$. When $\tilde q = q_1,q_2,...,q_n$, 
$|\tilde q|$ represents $|q_1| + |q_2| + \cdots + |q_n|$.
A sequence $\tilde q = q_1,q_2,...,q_n$ may be regarded as a set
$\{q_1,q_2,...,q_n\}$ implicitly when there is no fear of confusion.
Let $\mathit{Exp}$ be a set of real expressions,
and $\mathit{BExp}$ be a set of boolean expressions.
$\mathit{Exp}$ is ranged over by $e, e',...$.
$\mathit{BExp}$, ranged over by $b, b',...$, is composed of constants
$\mathtt{true}, \mathtt{false}$, atomic expressions $e\,\mathrm
{rel}\,e'$, and logical connectives $\neg, \wedge, \vee$, and
$\rightarrow$, where $\mathrm{rel} \in \{>, <, \ge, \le, =\}$.

Let $\mathit{cChan}$ be a set of
classical channels, and $\mathit{qChan}$ be
a set of quantum channels.
$\mathit{cChan}$ is ranged over by $c,d,...$, and
$\mathit{qChan}$ is ranged over by $\mathsf{c}, \mathsf{d},...$.

For a Hilbert space $\H$, $\dim{(\H)}$ denotes the dimension of $\H$.
For a linear operator $A:\H \rightarrow \H$, $\dim{(A)}$ denotes
$\dim{(\H)}$. For a TPCP map $\E:\D(\H_A) \rightarrow \D(\H_B)$,
$\dom{\E}$ and $\cod{\E}$ denote its domain $\D(\H_A)$ and codomain
$\D(\H_B)$.
For $e \in \mathit{Exp}$ and $b \in \mathit{BExp}$, $\braw{e}$ and
$\braw{b}$ denote their evaluations.

Let $\mathit{Op}$, ranged
over by $\mathit{op}, \mathit{op}_1,...$, 
be a set of identifiers of TPCP maps.
For each $\mathit{op} \in \mathit{Op}$, 
a corresponding TPCP map $\E^{\mathit{op}}$ satisfying
$\dom{\E^{\mathit{op}}} = \cod{\E^{\mathit{op}}}$ is defined.

\subsection{Syntax}
While the original syntax of qCCS allows
recursive definitions of processes, we restricted them for simplicity.
The 
sub-language is still expressive to describe protocols including 
our target QKD protocols. We also eliminated the constructors of 
choice $+$, tau $\tau.P$ and relabeling $P[f]$
because we do not use them.
\begin{defi}
The syntax of qCCS process is given as follows.
\begin{align*}
 \mathit{Proc} \ni P,Q &::= ~\nil ~|~ c?x.P ~|~ c!e.P ~|~
\rcvq{c}{q}.P ~|~\sndq{c}{q}.P \\
 ~|~ \myif{b&}{P}
 ~|~ \mathit{op}[\tilde q].P ~|~ M[\tilde q;x].P~|~P||Q~|~P\backslash L
\end{align*}
where $M$ is an Hermitian operator and 
$L$ is a set of channels.

The set of quantum free variables in a process $P$, denoted
by $\qv{P}$, is inductively defined as follows. 
\begin{align*}
 &\qv{\nil} = \emptyset
 &&\qv{c!e.P} = \qv{P}\\
 &\qv{c?x.P} = \qv{P}
 &&\qv{\sndq{c}{q}.P} = \{q\} \cup \qv{P}\\
 &\qv{\rcvq{c}{q}.P} = \qv{P} - \{q\}
 &&\qv{\myif{b}{P}} = \qv{P}\\
 &\qv{\mathit{op}[\tilde q].P} = \tilde q \cup \qv{P}
 &&\qv{M[\tilde q;x].P} = \tilde q \cup \qv{P}\\
 &\qv{P||Q} = \qv{P} \cup \qv{Q}
 &&\qv{P \backslash L} = \qv{P}
\end{align*}
The constructors $c?x, M[\tilde q;x]$, and $\rcvq{c}{q}$ bind
a classical variable $x$ and a quantum variable $q$.
Bound quantum variables in $P$ is denoted $\mathrm{qbv}(P)$.

For a process to be legal, the following conditions are required.
\begin{enumerate}
\item $\sndq{c}{q}.P \in \Proc$ only if $q \notin \qv{P}$,
\item $P||Q \in \Proc$ only if $\qv{P} \cap \qv{Q} = \emptyset$.
\end{enumerate}
\end{defi}
We explain intuitive meanings of the constructors. The process $\nil$
does nothing. The process $c?x.P$ receives a value of the type 
$\mathit{Real}$ through the channel $c$, binds it to the variable $x$,
and executes $P$. The process
$c!e.P$ sends a value that is obtained evaluating the
expression $e$ through the channel $c$, and executes $P$. The process
$\rcvq{c}{q}.P$
receives a qubit through the channel $\mathsf{c}$, and executes $P$.
The process $\sndq{c}{q}.P$
sends a qubit indicated by the quantum variable $q$ through the channel
$\mathsf{c}$, and executes $P$. The requirement {\it 1} says that
a qubit string, which is a physical object, becomes inaccessible after one
sends it.
The process $\myif{b}{P}$ executes $P$ iff the evaluation of the 
condition $b$ is $\mathit{true}$. The process
$\mathit{op}[\tilde q].P$ performs the corresponding 
TPCP map $\E^{\mathit{op}}$ to
the Hilbert space indicated by $\tilde q$, and executes $P$.
The process $M[\tilde q;x].P$ measures an observable $M$ of
the quantum state indicated by $\tilde q$, stores the result of the
measurement into a classical variable $x$, and executes $P$. 
The process $P||Q$ executes the process $P$ and $Q$ in parallel.
The requirement {\it 2} means that $P$ and $Q$ do not share quantum
systems.
The process $P \backslash L$ executes the process $P$ with private 
channels in $L$. 

For a classical variable $x$ and a value $v$ of
the type $\mathit{Real}$, $P\subst{v}{x}$ is the process obtained
replacing $x$ with $v$.
For quantum variables $q$ and $r$,
$P\subst{r}{q}$ is the process obtained
replacing $q$ with $r$.

\begin{ex} Examples of the processes are as follows,
\label{symqccs:processex}
 \begin{align*}
  &\sndq{c}{r}. M_1[q;x].\nil\\
  &\mathtt{measure}[r]. \sndq{c}{r}. \nil\\
  &M_1[q;x].M_2[r,s;y].\myif{x + y \le
	4}{(c!(x+y).\sndq{c}{r}.\nil||
	c?z.d!z.\rcvq{d}{t}.\nil)}\backslash\{c\}
 \end{align*}
where $x, y \in \mathit{cVar}$, $q,r,s,t \in \sfqv$, $c, d \in
 \mathit{cChan}$, $\mathsf{c}, \mathsf{d} \in
 \mathit{qChan}$. $M_1 = \ket{1}\bra{1}$ and
 $M_2 = \ket{001}\bra{001} + 2(\ket{010}\bra{010}
 + \ket{011}\bra{011})
 + 6\ket{110}\bra{110}$
 with $|q|=|r|=|t|=1$ and $|s|=2$.
 $\mathtt{measure} \in \mathit{Op}$ corresponds a 
 TPCP map
 $\E^{\mathtt{measure}}(\rho) = \ket{0}\bra{0}\rho\ket{0}\bra{0} +
\ket{1}\bra{1}\rho\ket{1}\bra{1}$.
\end{ex}

\subsection{Semantics}
For each $q \in \sfqv$, there assumed to be a corresponding $2^{|q|}$
dimensional
Hilbert space $\mathcal{H}_q$.
For $\tilde q = q_1,q_2,...,q_l$, let $\H_{\tilde q}$ be $\H_{q_1}
\otimes \H_{q_2} \otimes \cdots \otimes \H_{q_l}$.
Let $\mathcal{H}_S=\bigotimes_{q \in {\it S}} \mathcal{H}_q$ for $S
\subseteq \sfqv$ and let $\H = \H_{\sfqv}$
\footnote{As assumed in Chapter \ref{prel}, we identify $H_1 \otimes
H_2$ with $H_2 \otimes H_1$ for Hilbert spaces $H_1$ and $H_2$.
Therefore, the order of $\H_{q}$ with respect to $\otimes$ for $q \in S$
is not significant here.}.
Let $\mathcal{D}(\mathcal{H})$, ranged over by $\rho, \sigma,...$, be the
set of all density operators on $\H$.
For a process to be legal with respect to 
the semantics, the following conditions are additionally required.
\begin{enumerate}
\setcounter{enumi}{2}
 \item $\mathit{op}[\tilde q].P \in \Proc$
       only if $\dom{\E^{\mathit{op}}} =
       \D(\H_{\tilde q})$,
 \item $M[\tilde q;x].P \in \Proc$ only if $\dom{M} = \H_{\tilde q}$,
\end{enumerate}
For $\E^{\mathit{op}}$ with $\dom{\E^{\mathit{op}}} = \H_{\tilde q}$,
let $\E^{\mathit{op}}_{\tilde q}:\D(\H) \rightarrow \D(\H)$ be
$I_{\D(\H_S)} \otimes \E^{\mathit{op}} \otimes I_{\D(\H_T)}$
for $S \cup T =
\sfqv - \tilde q$, where $I_{\D(\H_S)}$ and $I_{\D(\H_T)}$ are identity
operators on
$\D(\H_{S})$ and $\D(\H_{T})$.
Similarly, for an Hermitian operator $M:\H_{\tilde q} \rightarrow
\H_{\tilde q}$ with spectrum decomposition $M = \sum_{i} \lambda_i E^i$,
$E^i_{\tilde q}:\H \rightarrow \H$ is defined as
$I_{\H_S} \otimes E^i_{\tilde q} \otimes I_{\H_T}$.

Let $\Con = \Proc \times \D(\H)$. An
element of $\Con$ is called a configuration.
A configuration consisting of $P \in \Proc$ and $\rho \in \D(\H)$
is written $\con{P}{\rho}$\footnote{In \cite{FengDuanJiYing2007,
Ying2009, FengDuanYing2011, DengFeng2012,
FengDengYing2012}, a configuration is written
$\langle P, \rho \rangle$ using angle brackets. In this thesis,
we write $\con{P}{\rho}$ since we frequently write density operators using
bra-ket notation.}.

\begin{ex} 
Examples of the configurations are as follows,
  \begin{align*}
  &\con{\sndq{c}{r}. M_1[q;x].\nil}{\mathit{EPR}_{q,r}
   \otimes \ket{01}\bra{01}_s \otimes \ket{-}\bra{-}_t}\\
  &\con{\mathtt{measure}[r]. \sndq{c}{r}. \nil}
   {\mathit{EPR}_{q,r} \otimes \ket{00}\bra{00}_s
   \otimes \ket{+}\bra{+}_t}\\
  &\con{M_1[q;x].M_2[r,s;y].\myif{x + y \le
	4}{(c!(x+y).\sndq{c}{r}.\nil||
	c?z.d!z.\rcvq{d}{t}.\nil)}\backslash\{c\}}
   {\\&\ket{+}\bra{+}_q \otimes \ket{+}\bra{+}_r \otimes \ket{10}\bra{10}_s
   \otimes \ket{0}\bra{0}_t}
 \end{align*}
where the sets $\mathit{cVar}$,
$\mathit{qVar}$, $\mathit{Op}$,
and the Hermitian operators $M_1$ and $M_2$
are defined in Example \ref{symqccs:processex}.
\end{ex}

qCCS has a nondeterministic and 
finite-support probabilistic transition system.
The set of all finite-support 
probability distribution on $\mathit{Con}$ is denoted $D(\mathit{Con})$,
which is ranged over by $\mu, \nu,...$.
Namely,
\begin{align*}
D(\Con)=\{\mu &\,|\, \sum_{\con{P}{\rho} \in \Con}
\mu(\con{P}{\rho}) = 1, \mbox{ and for only finitely many }
\con{P}{\rho},\\
&\mbox{we have } \mu(\con{P}{\rho}) > 0 \}. 
\end{align*}
For $\mu \in D(\Con)$, we write
$\mu = \boxplus_{i \in I} p_i \bullet \con{P_i}{\rho_i}$ if
$\mu(\con{P_i}{\rho_i}) = p_i$ and $\sum_{i \in I} p_i = 1$ hold.
For a point distribution, we may simply write $\con{P}{\rho}$ instead of
$1 \bullet \con{P}{\rho}$.
We also write $\mu = \sum_{i \in I} p_i \mu_i$ if $\mu(\con{P}{\rho}) =
\sum_{i \in
I} p_i \mu_i(\con{P}{\rho})$ for all $\con{P}{\rho} \in \Con$ and $\mu_i
\in D(\Con)$.

Let the set of actions $\mathit{Act}_{\tau}$, ranged over by
$\alpha,...$, be $\{c?v, c!v, \sndq{c}{q}, \rcvq{c}{q} \,|\,
c \in \mathit{cChan}, \mathsf{c} \in \mathit{qChan}, 
v \mbox{ is of the type } \mathit{Real}, q \in \sfqv\} \cup {\tau}$.
Channel name $\mathrm{cn}(\alpha)$ in $\alpha$
is defined as $\mathrm{cn}(c?v) =
\mathrm{cn}(c!v) = \{c\}$, $\mathrm{cn}(\sndq{c}{q}) =
\mathrm{cn}(\rcvq{c}{q}) = \{\mathsf{c}\}$, and
$\mathrm{cn}(\tau) = \emptyset$.
Quantum bound variable $\mathrm{qbv}(\alpha)$ in $\alpha$ is defined as
$\mathrm{qbv}(c!v) = \mathrm{qbv}(c?v) =
\mathrm{qbv}(\sndq{c}{q}) = 
\mathrm{qbv}(\tau) = \emptyset$, and
$\mathrm{qbv}(\rcvq{c}{q}) = \{q\}$.

\begin{defi}
The relation of transitions $\rightarrow \subseteq \Con \times
\mathit{Act}_\tau \times D(\Con)$
is defined by the rules in Figure \ref{fig:qccs-ltr}.
We regard $\xrightarrow{\alpha}$ as the subset of
$\Con \times D(\Con)$ for fixed $\alpha$.
The relation $\xrightarrow{\hat \alpha} \subseteq \Con \times D(\Con)$
is defined as follows.
\[
 \xrightarrow{\hat \alpha} := \begin{cases}
			       \xrightarrow{\tau} \cup \{(\con{P}{\rho},
			       1\bullet
			       \con{P}{\rho})\}~~~&(\alpha
			       \mbox{ is } \tau)\\
			       \xrightarrow{\alpha}~~~&(\mbox{otherwise})
			      \end{cases} 
\]
\end{defi}
\begin{figure}
\begin{align*}
&\frac{ v \mbox{ is of the type } \mathit{Real}
}
{
~\con{c?x.P}{\rho} \xrightarrow{c?v}
 \con{P\brac{v/x}}{\rho}~
}(\mbox{C-Inp})~~~~~
\frac{
 \braw{e} = v
}
{
~\con{c!e.P}{\rho} \xrightarrow{c!v}
 \con{P}{\rho}~
}(\mbox{C-Outp})\\
\\
&~\frac{\con{P_1}{\rho} \xrightarrow{c!v} \con{P_1'}{\rho}~~
\con{P_2}{\rho} \xrightarrow{c?v} \con{P_2'}{\rho}~
}
{
\con{P_1||P_2}{\rho} \xrightarrow{\tau} \con{P_1'||P_2'}{\rho}
}(\mbox{C-Com})
\\
\\
&\frac{
~r \notin \qv{P}\backslash{\{q\}}
}
{
~\con{\rcvq{c}{q}.P}{\rho} \xrightarrow{\rcvq{c}{r}}
 \con{P\brac{r/q}}{\rho}~
}(\mbox{Q-Inp})~~~~~
\frac{
}
{
~\con{\sndq{c}{q}.P}{\rho} \xrightarrow{\sndq{c}{q}}
 \con{P}{\rho}~
}(\mbox{Q-Outp})
\\
\\
&\frac{
}
{
~\con{\mathit{op}[\tilde q].P}{\rho} \xrightarrow{\tau}
 \con{P}{\E^{\mathit{op}}_{\tilde q}(\rho)}~
}(\mbox{Oper})~
\frac{
\con{P_1}{\rho} \xrightarrow{\rcvq{c}{r}} \con{P'_1}{\rho}\,
\con{P_2}{\rho} \xrightarrow{\sndq{c}{r}}
 \con{P'_2}{\rho}
}
{
\con{P_1||P_2}{\rho} \xrightarrow{\tau}
 \con{P'_1||P'_2}{\rho}
}(\mbox{Q-Com})
\\
\\
&\frac{
\con{P}{\rho} \xrightarrow{\alpha} \mu,
\braw{b} = \mathit{true}
}
{
~\con{\myif{b}{P}}{\rho} \xrightarrow{\alpha} \mu
}(\mbox{Cho})
~~~~~
\frac{\con{P}{\rho} \xrightarrow{\alpha}
 \boxplus_i p_i \bullet \con{P_i}{\rho_i}
~~\mathrm{cn}(\alpha) \cap
 L = \emptyset}
{\con{P \backslash L}{\rho} \xrightarrow{\alpha} 
 \boxplus_i p_i \bullet \con{P_i \backslash L}{\rho_i} }(\mbox{Res})
\\
\\
&\frac{
\con{P}{\rho} \xrightarrow{\alpha}
\boxplus_i p_i \bullet\con{P'_i}{\rho_i}~\mathrm{qbv}(\alpha) \cap
 \mathrm{qv}(Q) = \emptyset
}
{
~\con{P||Q}{\rho} \xrightarrow{\alpha}
\boxplus_i p_i \bullet\con{P'_i||Q}{\rho_i}
}(\mbox{IntL})\\
\\
&\frac{
\con{P}{\rho} \xrightarrow{\alpha}
\boxplus_i p_i \bullet\con{P'_i}{\rho_i}~\mathrm{qbv}(\alpha) \cap
 \mathrm{qv}(Q) = \emptyset
}
{
~\con{Q||P}{\rho} \xrightarrow{\alpha}
\boxplus_i p_i \bullet\con{Q||P'_i}{\rho_i}
}(\mbox{IntR})\\
\\
&\frac{
}
{
~\con{M[\tilde r;x].P}{\rho} \xrightarrow{\tau}
 \sum_i p_i \bullet \con{P\brac{\lambda_i/x}}
{E^i_{\tilde r} \rho E^i_{\tilde r}/p_i}~
}(\mbox{Meas})\\
&\mbox{where }M\mbox{ has the spectrum decomposition}\\
&M = \sum_i \lambda_i E^i,~{\rm and}
~p_i = \mathrm{tr}(E^i_{\tilde r}\rho)
\end{align*}
\caption{Labelled Transition Rule}
\label{fig:qccs-ltr}
\end{figure}

\begin{figure}[htbp]
 \begin{center}
  \includegraphics[width=110mm]{transex1.png}
 \end{center}
 \caption{Examples of the Transitions (1)}
 \label{fig:transex1}
\end{figure}
\begin{figure}[htbp]
 \begin{center}
  \includegraphics[width=130mm]{transex2.png}
 \end{center}
 \caption{Examples of the Transitions (2)}
 \label{fig:transex2}
\end{figure}

\begin{ex} Examples of the transitions are described in Figure
 \ref{fig:transex1} and \ref{fig:transex2}.
\end{ex}

\subsection{Lifting Relations}
To define weak
bisimilarity, the relations of transitions $\xrightarrow{\alpha},
\xrightarrow{\hat \alpha}
\subseteq \Con \times D(\Con)$ are lifted to subsets of $D(\Con)
\times
D(\Con)$. We introduce the definitions by
Deng and Feng \cite{DengFeng2012} here with some of the useful
properties.
Some definitions are rephrased in equivalent forms.

\begin{defi}
 For $\R \subseteq \Con \times D(\Con)$, its lifted
 relation $\R^\dagger \subseteq D(\Con) \times
 D(\Con)$ is defined as the smallest relation that satisfies
 \begin{itemize}
  \item $\con{P}{\rho} \R \mu$ implies
	$1 \bullet \con{P}{\rho} \R^\dagger \mu$,
	and
  \item (Linearity) $\mu_i \R^\dagger \nu_i$
	for any $i \in I$ implies
	$\sum_{i \in I}p_i \mu_i \R^\dagger
	\sum_{i \in I}p_i \nu_i$ for any $p_i \in [0,1]$ with 
	$\sum_{i \in I} p_i = 1$, where $I$ is a finite
	index set.
 \end{itemize}
\end{defi}

\begin{prop}
 $\mu \R^\dagger \nu$ if and only if there is a 
 finite set $I$ such that
 \begin{itemize}
  \item $\mu = \sum_{i \in I} p_i \con{P_i}{\rho_i}$,
  \item $\nu = \sum_{i \in I} p_i \nu_i$,
  \item $\con{P_i}{\rho_i} \R \nu_i$ for all
	$i \in I$.
 \end{itemize}
\end{prop}

$\R^\dagger$ may be simply written
 $\R$. We have that
$\con{P}{\rho} \xrightarrow{\alpha} \mu$ implies 
$\con{P}{\rho} (\xrightarrow{\alpha})^\dagger \mu$. 
The converse is not true. Indeed, $\con{c!1.\nil||c!1.\nil}{\rho}
 (\xrightarrow{c!1})^\dagger \frac{1}{2}\con{\nil||c!1.\nil}{\rho} \boxplus
\frac{1}{2}\con{c!1.\nil||\nil}{\rho}$ holds but the statement does not
 hold that is obtained replacing $(\xrightarrow{c!1})^\dagger$ to 
$\xrightarrow{c!1}$.

The internal action is then defined. It represents actions that are
not observed by the outsider and is important to define weak bisimilarity.
\begin{defi}
The internal action $\Rightarrow \subseteq
D(\Con) \times D(\Con)$ is defined as $((\xrightarrow{\hat
 \tau})^\dagger)^{\ast}$.
\end{defi}

\begin{prop}
 The relation 
 $\Rightarrow \xrightarrow{\hat \tau} \Rightarrow$ is equal to
 $\Rightarrow$.
 The relation $\Rightarrow \xrightarrow{\hat \alpha} \Rightarrow$
 is linear for all $\alpha$.
\end{prop}


Relations on $\Con$ is also lifted to those on $D(\Con)$.

\begin{defi}
For $\R \subseteq \Con \times \Con$, $\R^\dagger \subseteq D(\Con) \times
D(\Con)$ is defined as
\begin{align*}
 \{(\mu, \nu) \,|\, & \exists I:\mbox{finite index set}.\,\mu = \sum_{i \in
 I}p_i \con{P_i}{\rho_i}, \nu = \sum_{i \in I} p_i \con{Q_i}{\sigma_i},\\
 &\forall i \in I. \, \con{P_i}{\rho_i} \R \con{Q_i}{\sigma_i}\}.
\end{align*}
\end{defi}

\begin{prop}
  For $\R \subseteq \Con \times \Con$, $\R^\dagger \subseteq D(\Con) \times
D(\Con)$ is linear.
\end{prop}

\subsection{Bisimulation}
The strong and weak open bisimulation relation of qCCS configurations
defined by
Deng and Feng \cite{DengFeng2012} is introduced.
Let $\H_{\overline{\qv{P}}}$ be $\bigotimes_{q \in \mathit{\sfqv}-
\qv{P}}\H_q$.

\begin{defi}
A relation $\R \subseteq \Con \times \Con$ is a strong simulation
 if
 $\con{P}{\rho} \R \con{Q}{\sigma}$
implies $\qv{P}=\qv{Q}$, $\tr{\qv{P}}{\rho}=\tr{\qv{Q}}{\sigma}$ and
for all TPCP map $\E$ that acts on $\H_{\overline{\qv{P}}}$,
\begin{itemize}
\item whenever $\con{P}{\E(\rho)} \xrightarrow{\alpha} \mu$,
      there exists $\nu$ such that
      $\con{Q}{\E(\sigma)}
      \xrightarrow{\alpha} \nu$ and $\mu \R^\dagger
      \nu$.
\end{itemize}
$\R$ is a strong bisimulation if $\R$ and $\R^{-1}$ are strong simulations. 
The relation $\strg$ is defined as the largest strong bisimulation.
 If $\con{P}{\rho} \strg
 \con{Q}{\sigma}$, we say they are strongly bisimilar.
\end{defi}

\begin{defi}
A relation $\R \subseteq \Con \times \Con$ is a weak  
simulation if $\con{P}{\rho} \R \con{Q}{\sigma}$
implies $\qv{P}=\qv{Q}$, $\tr{\qv{P}}{\rho}=\tr{\qv{Q}}{\sigma}$ and
for all TPCP map $\E$ that acts on $\H_{\overline{\qv{P}}}$.
\begin{itemize}
\item whenever $\con{P}{\E(\rho)} \xrightarrow{\alpha} \mu$,
      there exists $\nu$ such that
      $\con{Q}{\E(\sigma)}\Rightarrow
      \xrightarrow{\hat \alpha} \Rightarrow \nu$ and $\mu \R^\dagger
      \nu$
\end{itemize}
$\R$ is a weak bisimulation if $\R$ and $\R^{-1}$ are weak simulations.
The relation $\approx$ is defined as
the largest weak bisimulation. If $\con{P}{\rho} \approx
 \con{Q}{\sigma}$, we may simply say they are bisimilar instead of
weakly bisimilar.
\end{defi}

For the bisimulation relations of qCCS configurations, ownership of
quantum variables, which represent physical objects, is significant.
The first condition $\qv{P}=\qv{Q}$ implies $\sfqv - \qv{P} = 
\sfqv - \qv{Q}$, which means the equality of quantum variables
that the outsider possesses. The second condition
$\tr{\qv{P}}{\rho}=\tr{\qv{Q}}{\sigma}$ means the equality of quantum
states that the outsider can access. In the next condition, 
an arbitrary TPCP map $\E$ that acts on $\sfqv - \qv{P}$ is taken.
This allows the outsider to perform an arbitrary operation to quantum
systems that she can access.

\begin{rem}
 There is another way to define probabilistic bisimulation based on
 equivalence classes \cite{Larsen1991, Goubault2007, Davidson-etal2012}.
 When we define by this way, an equivalent notion is in fact defined.
 Concretely, if $\con{P}{\rho} \R \con{Q}{\sigma}$ holds for some 
 strong bisimulation relation $\R$, then 
 \begin{align*}
  & \qv{P} = \qv{Q}, \tr{\qv{P}}{\rho} = \tr{\qv{Q}}{\sigma}, \mbox{ and }\\
  &\forall \E_{\tilde r}:\D(\H_{\sfqv - \qv{P}})\rightarrow \D(\H_{\sfqv
  - \qv{P}}). ~ \forall S \in \Con/\R.\\
  (&\exists \mu.\,(\con{P}{\E_{\tilde r}(\rho)} \xrightarrow{\alpha} \mu \mbox{ and }
  \sum_{S \R X_i} \mu(X_i)=p)\\
  \Leftrightarrow&
 \exists \nu.\,(\con{Q}{\E_{\tilde r}(\sigma)} \xrightarrow{\alpha} \nu \mbox{ and }
 \sum_{S \R Y_i} \nu(Y_i)=p))
 \end{align*}
hold, and conversely.
\end{rem}

Although we consider a little different formal framework,
$\approx$ has the following properties.
Proposition \ref{symqccs:equivalence} and Theorem 
\ref{symqccs:congruence} are proven similarly to the
original \cite{DengFeng2012}.
Theorem \ref{symqccs:coinduction} is proven similarly to
the previous version \cite{FengDuanYing2011}.

\begin{prop}
\label{symqccs:equivalence}
 $\approx$ is an equivalence relation.
\end{prop}
\begin{thm}
\label{symqccs:coinduction}
$\con{P}{\rho} \approx \con{Q}{\sigma}$
if and only if
$\qv{P}=\qv{Q}$,\\
$\tr{\qv{P}}{\rho}=\tr{\qv{Q}}{\sigma}$ and
for all TPCP map $\E$ that acts on $\H_{\overline{\qv{P}}}$,
\begin{enumerate}
 \item whenever $\con{P}{\E(\rho)} \xrightarrow{\alpha} \mu$,
       there exists $\nu$ such that
       $\con{Q}{\E(\sigma)}\Rightarrow
       \xrightarrow{\hat \alpha} \Rightarrow \nu$ and
       $\mu \approx^\dagger \nu$,
 \item whenever $\con{Q}{\E(\sigma)} \xrightarrow{\alpha} \nu$,
       there exists $\mu$ such that
       $\con{P}{\E(\rho)}\Rightarrow
       \xrightarrow{\hat \alpha} \Rightarrow \mu$ and $\mu
       \approx^\dagger \nu$.
\end{enumerate}
\end{thm}

Especially, the next theorem is useful to examine equivalence of
protocols under the existence of other protocols.
\begin{thm}
\label{symqccs:congruence}
If $\con{P}{\rho} \approx \con{Q}{\sigma}$,
\begin{itemize}
 \item $\con{P \backslash L}{\rho} \approx \con{Q \backslash L}{\sigma}$,
       and
 \item $\con{P||R}{\rho} \approx \con{Q||R}{\sigma}$
\end{itemize}
hold for all set of channels $L$ and process $R$ with
$\qv{P} \cap \qv{Q} = \emptyset$.
\end{thm}

We also use the following properties of {\it strong} bisimulation to
prove the soundness of our verifier. 
The properties are proven
similarly to those of weak bisimulation \cite{DengFeng2012}.
\begin{prop}
\label{symqccs:strgequiv} 
$\strg$ is an equivalence relation.
\end{prop}

\begin{prop}
\label{symqccs:strgopclose}
 If $\con{P}{\rho} \strg \con{Q}{\sigma}$, then
$\con{P}{\E(\rho)} \strg \con{Q}{\E(\sigma)}$ for all
TPCP map acting on $\H_{\overline{\qv{P}}}$.
\end{prop}
\begin{thm}
\label{symqccs:strgcong}
 If $\con{P}{\rho} \strg \con{Q}{\sigma}$, then
\begin{itemize}
 \item $\con{P \backslash L}{\rho} \strg \con{Q \backslash L}{\sigma}$,
       and
 \item $\con{P||R}{\rho} \strg \con{Q||R}{\sigma}$
\end{itemize}
hold for all set of channels $L$ and process $R$ with
$\qv{P} \cap \qv{Q} = \emptyset$.
\end{thm}

We call the properties of $\approx$ and $\strg$ {\it congruence}
that are stated by Theorem \ref{symqccs:congruence} and Theorem
\ref{symqccs:strgcong}, although the relations are {\it not} closed
under application of {\it all} constructors.

Finally, we introduce some example and counter-example of bisimulation.
In the following examples, let $\EPR$ be
$(\frac{\ket{00}+\ket{11}}{\sqrt{2}})
(\frac{\ket{00}+\ket{11}}{\sqrt{2}})^\dagger$.
\begin{ex}
 The following two configurations are bisimilar
for an arbitrary process $P(q^A)$ 
satisfying $q^A \in \qv{P(q^A)}$ and quantum state $\rho^E \in
\D(\H_{\overline{\{q^A, q^B\}}})$.

 \begin{enumerate}
  \item $X \defequiv \con{
	\sndq{c}{q^B}. {\tt measure}[q^A].P(q^A)
	}{
	\EPR_{q^A, q^B} \otimes \rho^E
	}$
  \item $Y \defequiv \con{
        {\tt measure}[q^A]. \sndq{c}{q^B}.P(q^A)
	}{
	\EPR_{q^A, q^B} \otimes \rho^E
	}$
 \end{enumerate}
\end{ex}
 A proof of the bisimilarity using Theorem \ref{symqccs:coinduction} is
 as follows.
 For $X = \con{P}{\rho} \in \Con$, let $\E(X)$ be $\con{P}{\E(\rho)}$
 for a TPCP map $\E$.
 For $X$ and $Y$, the conditions of quantum variables and
 partial traces are easily checked. Without loss of generality, 
we can take $I \otimes \E_1$ as an arbitrary
TPCP map acting on $\H_{\over{\{q_A, q_B\}}}$.
Let $\rho'$ be $\E_1(\rho^E)$.
For the transition
 \[
  (I \otimes \E_1)(X)
 \xrightarrow{\sndq{c}{q^B}} 
 \con{
 {\tt measure}[q^A].P(q^A)
 }{
 \EPR_{q^A, q^B} \otimes \rho'
 },
 \]
 we have
 \[
 (I \otimes \E_1)(Y)
 \Rightarrow \xrightarrow{\sndq{c}{q^B}} 
 \con{P(q^A)}{(\frac{1}{2}\ket{00}\bra{00} +
 \frac{1}{2}\ket{11}\bra{11})_{q^A, q^B} \otimes \rho'}.
 \]
 For the transition
 \[
 (I \otimes \E_1)(Y) \xrightarrow{\tau}
 \con{\sndq{c}{q^B}.P(q^A)}
{(\frac{1}{2}\ket{00}\bra{00}+\frac{1}{2}\ket{11}\bra{11})_{q^A, q^B}
 \otimes \rho'},
 \]
 we have $(I \otimes \E_1) (X) \Rightarrow 
(I \otimes \E_1) (X)$. To prove $X \approx Y$, it is sufficient to
show 
\begin{align*}
 \con{{\tt measure}[q^A].P(q^A)}
 {\EPR_{q^A, q^B} \otimes \rho'}
 &\approx
 \con{P(q^A)}{(\frac{1}{2}\ket{00}\bra{00} +
 \frac{1}{2}\ket{11}\bra{11})_{q^A, q^B} \otimes \rho'}~(\sharp)\\
 \mbox{ and }
 (I \otimes \E_1) (X)
 &\approx
 \con{\sndq{c}{q^B}.P(q^A)}{\EPR_{q^A, q^B} \otimes \rho'}~(\flat).
\end{align*}
 For $(\sharp)$, the condition of partial
 trace holds
 because
 \[
  \tr{q^A}{\EPR_{q^A,q^B}} = 
 \tr{q^A}{(\frac{1}{2}\ket{00}\bra{00} +
 \frac{1}{2}\ket{11}\bra{11})_{q^A,q^B}}
 \]
 holds.
 Let $\E_2$ be an arbitrary TPCP map acting on $\H_{\over{\{q^A\}}}$.
 For the transition
\begin{align*}
 &\con{{\tt measure}[q^A].P(q^A)}
 {\E_2(\EPR_{q^A, q^B} \otimes \rho')}\\
 \xrightarrow{\tau}
 &\con{P(q^A)}
 {\E_2(\frac{1}{2}\ket{00}\bra{00} +
 \frac{1}{2}\ket{11}\bra{11})_{q^A, q^B} \otimes \rho')} \defequiv Z,
\end{align*}
 we have
\begin{align*}
 &\con{P(q^A)}
 {\E_2(\frac{1}{2}\ket{00}\bra{00} +
 \frac{1}{2}\ket{11}\bra{11})_{q^A, q^B} \otimes \rho')}
 \Rightarrow Z
\end{align*}
 and $Z \approx^\dagger Z$.
 Next, for an arbitrary TPCP map $\E_3$ acting on $\H_{\over{\{q^A\}}}$
 and transition,
\[
 \con{P(q^A)}
 {\E_3(\frac{1}{2}\ket{00}\bra{00} +
 \frac{1}{2}\ket{11}\bra{11})_{q^A, q^B} \otimes \rho^E)}
 \xrightarrow{\alpha} \mu,\]
 we have the transition
 \[
\con{{\tt measure}[q^A].P(q^A)}
 {\E_3(\EPR_{q^A, q^B} \otimes \rho^E)} \Rightarrow
 (\xrightarrow{\alpha})^\dagger
 \mu
\]
 and $\mu \approx^\dagger \mu$ holds.
 The case when $P(q^A)$ does not perform any transition is
 easily checked.

The condition $(\flat)$ can
 be similarly checked.

\begin{ex}
 The following two configurations are not bisimilar in general.
 \begin{enumerate}
  \item $X \defequiv \con{
	\sndq{c}{q^B}. \ket{1}\bra{1}[q^A;x].P(q^A)
	}{
	\EPR_{q^A, q^B} \otimes \rho^E
	}$
  \item $Y \defequiv \con{
        \ket{1}\bra{1}[q^A;x]. \sndq{c}{q^B}.P(q^A)
	}{
	\EPR_{q^A, q^B} \otimes \rho^E
	}$
 \end{enumerate}
 We prove one of the necessary conditions of bisimulation cannot be
satisfied. Let $P(q^A)$ do not perform any transition.
 For the transition 
\[
  X \xrightarrow{\sndq{c}{q}} \con{
 \ket{1}\bra{1}[q^A;x].P(q^A)
 }{
 \EPR_{q^A, q^B} \otimes \rho^E
 },
\]
 the only possible action by $Y$ that perform $\Rightarrow
 \xrightarrow{\sndq{c}{q}} \Rightarrow$ is
\begin{align*}
 Y \Rightarrow \xrightarrow{\sndq{c}{q}} 
 &\frac{1}{2} \bullet \con{P(q^A)\subst{0}{x}}{
	\ket{00}\bra{00}_{q^A, q^B} \otimes \rho^E
	} \\
 &\boxplus
 \frac{1}{2} \bullet \con{P(q^A)\subst{1}{x}}{
	\ket{11}\bra{11}_{q^A, q^B} \otimes \rho^E
	}
 \defequiv \nu.
\end{align*}
 Therefore, it is necessary for bisimilarity to 
\[
 \con{
 \ket{1}\bra{1}[q^A;x].P(q^A)
 }{
 \EPR_{q^A, q^B} \otimes \rho^E
 }
 \approx \nu.
\]
However, this does not hold because
 $\tr{q^A}{\EPR_{q^A, q^B}} \neq \tr{q^A}{\ket{ii}\bra{ii}_{q^A, q^B}}$
 for $i \in \{0,1\}$.
\end{ex}

\section{Simplification of qCCS's Syntax}
\subsection{Motivation}
\subsubsection{On Formalization of Measurement}
qCCS's syntax has the constructors of TPCP map
application $\mathit{op}[\tilde q].P$
and quantum measurement $M[\tilde q, x].P$.
Since a quantum measurement can also be formalized as a TPCP map,
we have two ways to formalize a measurement.
For example, quantum measurement of the quantum
state $\ket{+}\bra{+}$
is formalized in the following two ways,
where the TPCP map $\E^{\mathtt{measure}}(\rho)$
that corresponds to $\mathtt{measure}[q]$ is $\ket{0}\bra{0}\rho
\ket{0}\bra{0}+\ket{1}\bra{1}\rho
\ket{1}\bra{1}$, $\rho^E \in \H_{\sfqv -\{q\}}$ is an arbitrary quantum
states, and $P(q)$ is an arbitrary process with $q \in \qv{P(q)}$.
 \begin{align*}
  &1.~~\con{\ket{1}\bra{1}[q;x].P(q)}{\ket{+}\bra{+}_q \otimes \rho^E}
  \xrightarrow{\tau}
  \frac{1}{2} \bullet \con{P(q)}
  {\ket{0}\bra{0}_q \otimes \rho^E} \boxplus\\
  &~~~~~~~~~~~~~~~~~~~~~~~~~~~~~~~~~~~~~~~~~~~~~~~~~~
  \frac{1}{2} \bullet \con{P(q)}
  {\ket{1}\bra{1}_q \otimes \rho^E}\\
 &2.~~\con{\mathtt{measure}[q].P(q)}{\ket{+}\bra{+}_q \otimes \rho^E}
  \xrightarrow{\tau}
  \con{P(q)}
  {1/2(\ket{0}\bra{0}+\ket{1}\bra{1})_q}
\end{align*}
Although the two processes apparently formalize the same deed,
they are not bisimilar.

Indeed, the way to formalize a quantum measurement is important in the
formal verification of Shor and Preskill's security proof using qCCS.
In the transformation step, the EDP-based protocol
is converted to the next protocol based on the
fact that nobody outside cannot distinguish the following 
two processes:
\begin{itemize}
 \item[A.] Alice measures a half of an EPR pair
       and then sends the other half to the outside.
 \item[B.] Alice sends a half of an EPR pair to the outside
       and then measures the other half.
\end{itemize}
First, when the measurement is formalized using the
constructor $M[\tilde q, x].P$, the following two configurations are
obtained formalizing the above two, where
$\EPR =
(\frac{\ket{00}+\ket{11}}{\sqrt{2}})
(\frac{\ket{00}+\ket{11}}{\sqrt{2}})^\dagger$,
$\rho^E \in \H_{\sfqv - \{q^A,q^B\}}$
is an arbitrary quantum states, and $Q(q^A)$ is
the successive process. They are not bisimilar.
\begin{itemize}
 \item[A-1.] $\con{
       \sndq{c}{q^B}. \ket{1}\bra{1}[q^A;x].Q(q^A)
       }{\EPR_{q^A, q^B} \otimes \rho^E}$
 \item[B-1.] $\con{
        \ket{1}\bra{1}[q^A;x]. \sndq{c}{q^B}.Q(q^A)
       }{\EPR_{q^A, q^B} \otimes \rho^E}$
\end{itemize}
Second, when the measurement is formalized as 
a TPCP map, the following two configurations are obtained
formalizing the example. They are bisimilar.
\begin{itemize}
 \item[A-2.] $\con{
       \sndq{c}{q^B}. {\tt measure}[q^A].Q(q^A)
       }{
       \EPR_{q^A, q^B} \otimes \rho^E
       }$,
 \item[B-2.] $\con{
        {\tt measure}[q^A]. \sndq{c}{q^B}.Q(q^A)
       }{
       \EPR_{q^A, q^B} \otimes \rho^E
       }$, where
\end{itemize}
\[
\E^{\tt measure}_{q^A}(\rho) = 
\ket{0}\bra{0}_{q^A}\rho\ket{0}\bra{0}_{q^A} + \ket{1}\bra{1}_{q^A}
\rho \ket{1}\bra{1}_{q^A}.
\]

\subsubsection*{Criteria to Select the Way to Formalize}
\label{symqccs:criteria}
By the definition of weak bisimulation relation,
whether probabilistic branches
evoked by $M[\tilde q;x]$ \emph{exist or not} is
significant in transition trees of qCCS configurations.
Therefore, the two different formalization of a
quantum measurement are considered to be 
different from the view of the outsider.
In general, it is unnatural that the outsider recognize
the existence of probabilistic branches {\it without viewing
configuration's different behaviour} that depends on 
the result of the branch.
Hence, we think that if probabilistic branches are evoked, then
the insider must perform a different labelled transition.
We accordingly propose a criteria to select one way from the two to
formalize a quantum measurement.
\begin{itemize}
 \item If transitions with different labels occur according to
       the result of the measurement, the measurement should be formalized
       using the constructor $M[\tilde q;x].P$;
 \item otherwise, it should be formalized
       as a TPCP map, namely, using the constructor $\op{op}{\tilde
       q}.P$.
\end{itemize}
By our criteria, we should formalize the measurement of $\ket{+}\bra{+}$
in the first example as
\begin{align*}
  &1.~~\con{\ket{1}\bra{1}[q;x].P(q)}{\ket{+}\bra{+}_q \otimes \rho^E}
  \xrightarrow{\tau}
  \frac{1}{2} \bullet \con{P(q)}
  {\ket{0}\bra{0}_q \otimes \rho^E} \boxplus\\
  &~~~~~~~~~~~~~~~~~~~~~~~~~~~~~~~~~~~~~~~~~~~~~~~~~~
  \frac{1}{2} \bullet \con{P(q)}
  {\ket{1}\bra{1}_q \otimes \rho^E}
\end{align*}
if $P(q)$ performs different labeled transitions according to the
result. A typical case is when $P \equiv \myif{x =
1}{\sndq{c}{q}.P'}$ holds for some $\mathsf{c}$ and $P'$.
Otherwise, we should formalize it as
\[
 2.~~\con{\mathtt{measure}[q].P(q)}{\ket{+}\bra{+}_q \otimes \rho^E}
  \xrightarrow{\tau}
  \con{P(q)}
  {1/2(\ket{0}\bra{0}+\ket{1}\bra{1})_q}.
\]
Next, let us consider the processes A and B in the second example.
In fact, it is natural that we assume the successive process $Q(q^A)$
does not perform different labeled transitions according to the result of
the measurement of $q^A$. By the definition of the QKD protocols we
consider, which channels Alice and Bob use does not depend on
the result of the measurement of $q^A$.
Hence, we should formalize them as follows by our
criteria.
\begin{enumerate}
 \item[A-2.] $\con{
       \sndq{c}{q^B}. {\tt measure}[q^A].Q(q^A)
       }{
       \EPR_{q^A, q^B} \otimes \rho^E
       }$
 \item[B-2.] $\con{
        {\tt measure}[q^A]. \sndq{c}{q^B}.Q(q^A)
       }{
       \EPR_{q^A, q^B} \otimes \rho^E
       }$.
\end{enumerate}

We simplified the syntax so that
it reflects these criteria.
We eliminated the constructions $M[\tilde q;x].P$ and 
$\myif{b}{P}$.
Instead, we introduced a new syntax $\measure{q}{P}$, where the
observable $\ket{1}\bra{1}$ on the space corresponding to the 
\emph{qubit}\footnote{Note that the meta variable $b$ stands for a boolean
condition in the original syntax but
the meta variable $b$ stands for a quantum variable with length $1$ in
our simplified syntax.}
 $b$ (i.e. $|b|=1$ must be satisfied.) is measured, and if the
result is 1, then it behaves like $P$, else it terminates.
In the new syntax, the qubit $b$ represents the condition for the branch,
which
is supposed to be computed beforehand by some TPCP map. % The final
% condition we impose is that $P$ must contain at least one $\sndq{c}{q}$
% or
% $\rcvq{c}{q} \in \mathcal{P}$ with non-restricted channel $\mathsf{c}$
% to make a visible labeled transition occur.
Besides, we eliminated classical communications for simplicity.
Since classical data can be represented by quantum data,
the elimination of the use of classical data
does not weaken crucially the expressiveness of the language.
Indeed, a distribution where we have the value $0$ with probability $p$
and $1$ with probability $1-p$ is represented by the diagonal 
density operator $p\ket{0}\bra{0} + (1-p)\ket{1}\bra{1}$.

\subsubsection{On Ownership of Quantum System}
By the definition of bisimulation relation, 
if $\con{P}{\rho} \approx \con{Q}{\sigma}$, then
$\tr{\qv{P}}{\rho}=\tr{\qv{Q}}{\sigma}$. 
Intuitively, $\qv{P}$ is considered as the set of quantum variables of
the process
$P$'s own, and $\sfqv - \qv{P}$ is the outsider's. 
$\tr{\qv{P}}{\rho} \in \D(\H_{\sfqv - \qv{P}})$ is considered as the 
quantum states that the outsider can access. For the bisimulation
relation, ownership of quantum variables is significant. In
the transitions of qCCS processes, the ownership changes by
the communication between the process and the outsider by $\sndq{c}{q}$
and
$\rcvq{c}{q}$. However, there are cases where ownership changes
without communication between a process and it's outsider.
\[
 \con{\mathtt{hadamard}[q].\nil}{\ket{0}\bra{0}_{q}
 \otimes \rho^E} \xrightarrow{\tau}
 \con{\nil}{\ket{0}\bra{0}_{q}
 \otimes \rho^E} 
\]
In the above configurations, $\qv{\mathtt{hadamard}[q].\nil} = \{q\}$ and
$\qv{\nil} = \emptyset$. The process loses in the 
transition the ownership of the variable
$q$ without sending it to the outside. We added the restriction
that $\op{op}{\tilde q}.P$ is defined only if $\tilde q \subseteq
\qv{P}$, and changed $\nil$ to a new constructor $\discard{\tilde q}$
that terminates keeping the quantum variables $\tilde q$ inside.
In the original qCCS, process's termination keeping 
quantum variables $\tilde q$ inside is realized for example by
$\myif{\mathtt{false}}{\op{\mathit{I}}{\tilde
q}.\nil}$, where $\mathit{I}$ is the identity operator.

\subsection{Simplified Syntax}
\label{symqccs:modifiedsyntax}
\begin{defi}
The simplified qCCS syntax is given as follows.
\begin{align*}
\mathcal{P} \ni P,Q ::=\, &\discard{\tilde q} ~|~ \sndq{c}{q}.P ~|~
 \rcvq{c}{q}.P
 ~|~\op{\mathit{op}}{\tilde q}.P\\
&~|~ P||Q ~|~ \measure{b}{P} ~|~ P
 \backslash L
\end{align*}
$\qv{\cdot}$ for the simplified syntax is defined as
$\qv{\discard{\tilde q}} = \tilde q$ and \\
$\qv{\measure{b}{P}} = \qv{P}$.
For a process to be legal, the following conditions are required.
 \begin{enumerate}
 \item $\sndq{c}{q}.P \in \mathcal{P}$ iff $q \notin \qv{P}$.
 \item $\rcvq{c}{q}.P \in \mathcal{P}$ iff $q \in \qv{P}$.
 \item $P||Q \in \mathcal{P}$ iff $\qv{P} \cap \qv{Q} = \emptyset$.
 \item $\op{\mathit{op}}{\tilde q}.P \in \mathcal{P}$ iff $\tilde q \subseteq \qv{P}$.
 \item $\measure{b}{P} \in \mathcal{P}$ iff $b \in \qv{P}$.%  and $P$ contains at least one $\sndq{c}{q}$ or
       % $\rcvq{c}{q} \in \mathcal{P}$ with non-restricted channel $\mathsf{c}$.
 \end{enumerate}
\end{defi}

\section{Simplification of Operational Semantics}
We simplified the operational semantics for convenience of implementation.
Instead of considering a probability distribution on configurations,
we consider a probability-weighted quantum states represented by probability-weighted
density operators. For example, instead of considering
\begin{align*}
 \con{\measure{b}{P}}{\rho} \, \xrightarrow{\tau} \,
 & p \bullet \con{P}{\frac{\ket{1}\bra{1}_b\rho\ket{1}\bra{1}_b}{p}} \boxplus \\
 & (1-p) \bullet
 \con{\discard{\qv{P}}}{\frac{\ket{0}\bra{0}_b\rho\ket{0}\bra{0}_b}{1-p}}, \\
\mbox{where } p = \tr{}{\ket{1}\bra{1}_b\rho},
\end{align*}
we consider
\begin{align*}
 \con{\measure{b}{P}}{\rho} \, \xrightarrow{\tau} \,
 & \con{P}{\ket{1}\bra{1}_b\rho\ket{1}\bra{1}_b}\\
 \con{\measure{b}{P}}{\rho} \, \xrightarrow{\tau} \,
 & \con{\discard{\qv{P}}}{\ket{0}\bra{0}_b\rho\ket{0}\bra{0}_b}.
\end{align*}
For this purpose, we define the set of probability-weighted quantum 
states
$\Delta(\H)$, ranged over by $\rho, \sigma,...$, as
$\{p\rho\,|\, p \in [0,1], \rho \in \D(\H)\}$. Any element $\rho \in
\Delta(\H)$ can be converted to an ordinary density operator
 $\frac{\rho}{\tr{}{\rho}} \in \D(\H)$.
If there is no fear of confusion, we may simply say quantum states
instead of
probability-weighted quantum states. We again call a pair of a process and
a probability-weighted quantum state a configuration.
The set of configurations $\C$ is defined as $\mathcal{P} \times
\Delta(\H)$. For a configuration $\con{P}{\rho} \in \C$, $\tr{}{\rho}$
can be interpreted as the probability of reaching it from another
configuration. By this simplification, 
probability was excluded from the transition system. The simplified
transition system is only nondeterministic, not probabilistic.

\begin{defi}
\label{symqccs:simpletrans}
Let $\A_{\tau} := \{\tau\} \cup
 \{\sndq{c}{q},\rcvq{c}{q} \,|\, \mathsf{c} \in \mathit{qChan}, q \in 
\sfqv \}$ be the set of actions.
The transition $\rightarrow \subseteq \C \times \A_{\tau} \times \C$ is 
defined by the rules in Figure \ref{fig:simplified-semantics}.
The transition $\xrightarrow{\hat \alpha}$ is defined as follows.
\[
 \xrightarrow{\hat \alpha} := \begin{cases}
			       \xrightarrow{\tau} \cup \{(\con{P}{\rho},
			       \con{P}{\rho})\}~~~&(\alpha
			       \mbox{ is } \tau)\\
			       \xrightarrow{\alpha}~~~&(\mbox{otherwise})
			      \end{cases} 
\]
\begin{figure}[htbp]
\begin{minipage}{0.5\hsize}
\begin{align*}
%send
&\frac{}
{\con{\sndq{c}{q}.P}{\rho} \xrightarrow{\sndq{c}{q}}
\con{P}{\rho}}
\,\mbox{(In)}\\
\\
%receive
&\frac{r \in \sfqv - \qv{P}}
{\con{\rcvq{c}{q}.P}{\rho} \xrightarrow{\rcvq{c}{r}}
\con{P\subst{r}{q}}{\rho}}
\,\mbox{(Out)}\\
\\
%operator
&\frac{}
{\con{\op{op}{\tilde q}.P}{\rho} \xrightarrow{\tau} 
\con{P}{\E^\mathit{op}_{\tilde q}(\rho)}}
\,\mbox{(Op)}
\end{align*}
\end{minipage}
\begin{minipage}{0.5\hsize}
 \begin{align*}
%restriction
&\frac{
\con{P}{\rho} \xrightarrow{\alpha} 
\con{P'}{\rho'}~~~
\mathrm{cn}(\alpha) \cap L = \emptyset
}
{
\con{P \backslash L}{\rho} \xrightarrow{\alpha} 
\con{P' \backslash L}{\rho'}
}\,\mbox{(Res)}\\
\\
%interleavingR
&\frac{
\con{Q}{\rho} \xrightarrow{\alpha} 
\con{Q'}{\rho'}~~~
}
{
\con{P||Q}{\rho} \xrightarrow{\alpha} 
\con{P||Q'}{\rho'}~~~
}\,\mbox{(Right)}\\
\\
%interleavingL
&\frac{
\con{P}{\rho} \xrightarrow{\alpha} 
\con{P'}{\rho'}
}
{
\con{P||Q}{\rho} \xrightarrow{\alpha} 
\con{P'||Q}{\rho'}~~~
}\,\mbox{(Left)}
 \end{align*}
\end{minipage}

\begin{align*}
  %communication
 &\frac{
 \con{P}{\rho} \xrightarrow{\sndq{c}{q}} 
 \con{P'}{\rho}~~~
 \con{Q}{\rho} \xrightarrow{\rcvq{c}{q}} 
 \con{Q'}{\rho}
 }
 {
 \con{P||Q}{\rho} \xrightarrow{\tau} 
 \con{P'||Q'}{\rho}
 }\,\mbox{(Comm)}\\
\\
%measure
&~\frac{ \ket{1}\bra{1}^{b} \rho \ket{1}\bra{1}^{b} \neq O
}
{
\con{\measure{b}{P}}{\rho} \xrightarrow{\tau}
\con{P}{\ket{1}\bra{1}^{b} \rho \ket{1}\bra{1}^{b}}
}\,\mbox{(Meas1)}\\
\\
&\frac{ \ket{0}\bra{0}^{b} \rho \ket{0}\bra{0}^{b} \neq O
}
{
\con{\measure{b}{P}}{\rho} \xrightarrow{\tau}
\con{\discard{\qv{P}}}{\ket{0}\bra{0}^{b} \rho \ket{0}\bra{0}^{b}}
}\,\mbox{(Meas0)}
 \end{align*}
\caption{Simplified Semantics}
\label{fig:simplified-semantics}
\end{figure}
\end{defi}
We call a new formal framework {\it nondeterministic qCCS}
whose set of configuration is $\C$ and transition rules are defined in
Definition \ref{symqccs:simpletrans}. Although our verifier, 
called \emph{Verifier1}, is
implemented based on nondeterministic qCCS, it verifies the
relation $\approx$ defined by Deng et al \cite{DengFeng2012}.
We call the property of Verifier1 \emph{soundness}, which is
further discussed in Section \ref{symqccs:sndness}.

\section{Automated Verification of Bisimilarity}
Verifier1 handles the configurations $\con{P}{\rho}$ that consist
of a process $P \in \mathcal{P}$ and a symbolic representation
$\rho \in \mathcal{S}$ of a probability-weighted quantum state $\braw{\rho} \in
\Delta(\H)$. Verifier1 obeys the simplified transition rules
defined in Definition \ref{symqccs:simpletrans} except for (Meas$i$)
for $i = 0,1$: it performs the transition even if 
$ \braw{\opsym{\mathtt{proj}i}{b}{\rho}} = 
\ket{i}\bra{i}_b \braw{\rho}\ket{i}\bra{i}_b
= O$ for $\rho \in \mathcal{S}$.

\subsection{Symbolic Representation of Quantum States}
Since cryptographic protocols are defined with security parameters,
the dimensions of quantum states, which are data in protocols,
are unfixed. In our verifier, quantum states are represented as
symbols. First, finite sets
$S_{\mathit{nat}},
S_{\mathit{stat}},
S_{\mathit{op}}$ of symbols respectively
representing
natural numbers, quantum states, and TPCP maps are assumed. 
\begin{itemize}
 \item $S_{\mathit{nat}}$ is a set of symbols representing
       natural numbers. A symbol $\mathtt{1}$ is
       an element of $S_{\mathit{nat}}$.
 \item $S_{\mathit{stat}}$ is a set of symbols representing quantum
       states.
 \item $S_{\mathit{op}}$ is a set of symbols representing TPCP maps.
 \item A function $\mathrm{len}:\sfqv \rightarrow
       S_{\mathit{nat}}$ carries each quantum variable to its 
       qubit-length. $b \in \sfqv$ is called
       a qubit variable if $\mathrm{len}(b) = \mathtt{1}$.
 \item A function $\mathrm{arg}:S_{\mathit{stat}} \cup
       S_{\mathit{op}} \rightarrow \bigcup_{n \in
       \mathbb{N_{+}}}(S_{\mathit{nat}})^n$ carries
       each symbol of quantum states or TPCP map to
       qubit-lengths of its arguments. For example,
       an EPR pair $(\frac{\ket{00}+\ket{11}}{\sqrt{2}})
       (\frac{\ket{00}+\ket{11}}{\sqrt{2}})_{q,r}^\dagger$
       is represented as
       $\texttt{EPR[}q,r\texttt{]}$, where $\mathtt{EPR} \in
       S_{\mathit{stat}}$,
       $\mathrm{len}(q)=\mathrm{len}(r)=\mathtt{1}$, and
       $\mathrm{arg}(\mathtt{EPR})=(\mathtt{1},\mathtt{1})$.
\end{itemize}

After the sets $S_{\mathit{nat}}$,
$S_{\mathit{stat}}$, $S_{\mathit{op}}$, $\mathrm{len}(\cdot)$, and
$\mathrm{arg}(\cdot)$ are defined,
the syntax of symbolic representations of quantum states are
defined.

\begin{defi}
\label{symqccs:symbrep}
The syntax of symbolic representations of quantum states 
are given as  follows,
\begin{align*}
\mathcal{S} \ni \rho, \sigma ::=\,\, &X\mathtt{[}\tilde q\mathtt{]} ~|~
\mathit{op}\mathtt{[}\tilde q\mathtt{](}\rho\mathtt{)} ~|~ \rho \,
  \mathtt{*} \, \sigma
  ~|~\mathtt{proj0}\mathtt{[}b\mathtt{](}\rho\mathtt{)}
~|~\mathtt{proj1}\mathtt{[}b\mathtt{](}\rho\mathtt{)}
~|~\mathtt{Tr}\mathtt{[}\tilde q\mathtt{](}\rho\mathtt{)}
\end{align*}
where $b$ is a qubit variable, $X \in S_{\mathit{stat}}$, and 
$\mathit{op} \in S_{\mathit{op}}$. The set of quantum variables 
$\qv{\rho}$ in symbolic representations $\rho$ are defined as follows.
\begin{align*}
&\qv{X\mathtt{[}\tilde q\mathtt{]}} = \tilde q,
&&\qv{\mathit{op}\mathtt{[}\tilde
 q\mathtt{](}\rho\mathtt{)}}=\qv{\rho},\\
&\qv{\rho \,\mathtt{*}\, \sigma} = 
 \qv{\rho} \cup \qv{\sigma},
&&\qv{\mathtt{proj}i\mathtt{[}b\mathtt{](}\rho\mathtt{)}}
= \qv{\rho},\\
&\qv{\mathtt{Tr}\mathtt{[}\tilde
 q\mathtt{](}\rho\mathtt{)}}=\qv{\rho} - \tilde q.
\end{align*}
For a symbolic representation to be legal, the following conditions are
required.
\begin{enumerate}
 \item $X\mathtt{[}q_1,q_2,...,q_n\mathtt{]} \in \mathcal{S}$ iff 
       $\mathrm{arg}(X)=(\mathrm{len}(q_1),\mathrm{len}(q_2),...,\mathrm{len}(q_
       n))$.
 \item $\mathit{op}\mathtt{[}q_1,q_2,...,q_n\mathtt{](}\rho\mathtt{)}
       \in \mathcal{S}$ iff
       $\mathrm{arg}(op)=(\mathrm{len}(q_1),\mathrm{len}(q_2),...,
       \mathrm{len}(q_n))$ and \\
       $\{q_1,q_2,...,q_n\} \subseteq \qv{\rho}$.
 \item $\rho \, \mathtt{*} \, \sigma \in \mathcal{S}$ iff $\qv{\rho} \cap
       \qv{\sigma} = \emptyset$.
 \item For $i \in \{\mathtt{0}, \mathtt{1}\}$,
       $\mathtt{proj}i\mathtt{[}b\mathtt{](}\rho\mathtt{)} \in \mathcal{S}$ 
       iff $b \in \qv{\rho}$.
 \item $\mathtt{Tr}\mathtt{[}\tilde q \mathtt{](}\rho\mathtt{)} \in
       \mathcal{S}$ iff $\tilde q \subseteq \qv{\rho}$.
\end{enumerate}
If $\rho, \sigma \in \mathcal{S}$ are syntactically equal, we write
 $\rho \equiv \sigma$.
\end{defi}
Intuitive meanings are as follows.
$X\texttt{[}\tilde q\texttt{]}$ means that $\tilde q$'s quantum state
is $X$.
$\mathit{op}\texttt{[}\tilde q\texttt{](}\rho\texttt{)}$
is a quantum state obtained after application of a TPCP map
$\mathit{op}$ that
acts on $\tilde q$, 
to $\rho$.  $\rho \,  \mathtt{*} \, \sigma$ is the tensor product of
states $\rho$ and $\sigma$. $\mathtt{proj}i\texttt{[}b\texttt{](}\rho\texttt{)}$
means the quantum state $\rho$ obtained after application of the projector
$\ket{i}\bra{i}_b$. $\mathtt{Tr}\texttt{[}\tilde
q\texttt{](}\rho\texttt{)}$ means the partial trace of $\rho$ by
$\tilde q$.

Next, we define the formal interpretation of the symbolic
representations.
To define it, interpretations of the elements of 
$S_{\mathit{nat}}$, $S_{\mathit{stat}}$, and $S_{\mathit{op}}$ must be
defined beforehand. The interpretations depend on a set of
security parameters. Let $\Lambda$ be the product of
the ranges of security parameters.
The types of the interpretations $\braw{\cdot}$, which 
depend on $\Lambda$, are as
follows. \footnote{For $f : \Lambda \rightarrow \mathbb{N}_+$, 
$2^{f} : \Lambda \rightarrow \mathbb{N}_+$ represents
$2^{f}(\lambda) = 2^{f(\lambda)}$ for all $\lambda \in \Lambda$.}
\begin{itemize}
 \item For $n \in S_{\mathit{nat}}$, $\braw{n} :
       \Lambda \rightarrow \mathbb{N}_+$. For $q \in \sfqv$ with 
       $\mathrm{len}(q) = n$, $\H_q$ is $2^{\braw{n}}$-dimensional.
 \item For $X \in S_{\mathit{stat}}$ with
       $\mathrm{arg}(X) = (n_1,n_2,...,n_m)$, $\braw{X}$
       is an element of a Hilbert space with dimension
       $2^{\braw{n_1}+
       \braw{n_2} + \cdots\braw{n_m}}$.
 \item For $\mathit{op} \in S_{\mathit{op}}$ with
       $\mathrm{arg}(\mathit{op}) = (n_1,n_2,...,n_m)$,
       $\braw{\mathit{op}}$
       is a TPCP map on Hilbert space with dimension
       $2^{\braw{n_1} +
       \braw{n_2} + \cdots\braw{n_m}}$.
\end{itemize}
The interpretation of the symbolic representations is then 
defined as follows.
\begin{itemize}
 \item $\braw{X\texttt{[}\tilde q\texttt{]}} =
       \braw{X} \in \Delta(\H_{\tilde q})$
 \item $\braw{\mathit{op}\texttt{[}\tilde
       q\texttt{](}\rho\texttt{)}} =
       \braw{\mathit{op}}_{\tilde q}(\braw{\rho})
       \in \Delta(\H_{\qv{\rho}})$
 \item $\braw{\rho \,\mathtt{*} \, \sigma} = 
       \braw{\rho} \otimes
       \braw{\sigma} \in \Delta(\H_{\qv{\rho}} \otimes
       \H_{\qv{\sigma}})$
 \item $\braw{\mathtt{proj0}\texttt{[}b\texttt{](}\rho\texttt{)}}
       =
       \ket{0}\bra{0}_b \braw{\rho} \ket{0}\bra{0}_b \in
       \Delta(\H_{\qv{\rho}})$
 \item $\braw{\mathtt{proj}1\texttt{[}b\texttt{](}\rho\texttt{)}} 
       =
       \ket{1}\bra{1}_b\braw{\rho}\ket{1}\bra{1}_b \in
       \Delta(\H_{\qv{\rho}})$
 \item $\braw{\mathtt{Tr}\texttt{[}\tilde q\texttt{](}\rho\texttt{)}}
       =
       \tr{\tilde q}{\braw{\rho}} \in
       \Delta(\H_{\qv{\rho} - \tilde q})$
\end{itemize}
\begin{ex}
 Let $\Lambda = \mathbb{N}_+$, $\sfqv = \{\mathtt{q}, \mathtt{r}\}$,
 $\mathrm{len}(\mathtt{q})=
 \mathrm{len}(\mathtt{r})=\mathtt{n}$ and let
 $\braw{\mathtt{n}}(k) = k$, $\braw{\mathtt{EPR}} =
 ((\frac{\ket{00}
 + \ket{11}}{\sqrt{2}})
 (\frac{\ket{00}
 + \ket{11}}{\sqrt{2}})^\dagger)^{\otimes k} \defequiv \mathit{EPR}$,
 $\mathrm{arg}(\mathtt{EPR})=(\mathtt{n},\mathtt{n})$, \\
 $\braw{\mathtt{measure}} (\rho) = \ket{0}\bra{0}^{\otimes k}
 \rho \ket{0}\bra{0}^{\otimes k} + \ket{1}\bra{1}^{\otimes k} \rho
 \ket{1}\bra{1}^{\otimes k}$, and
 $\mathrm{arg}(\mathtt{measure})=(\mathtt{n})$.
 The interpretation of the symbolic representation
 $\mathtt{measure[q](EPR[q,r])}$ is calculated as follows.
 \begin{align*}
  \braw{\mathtt{measure[q](EPR[q,r])}}
  &= \braw{\mathtt{measure}}_{\mathtt{q}} (\mathit{EPR}_{\mathtt{q},
  \mathtt{r}})\\
  &= \braw{\mathtt{measure}} \otimes I_{\H_\mathtt{r}}
   (\mathit{EPR}_{\mathtt{q}, \mathtt{r}})\\
  &= \sum_{j \in \{0,1\}^k} \frac{1}{2^k}\ket{jj}\bra{jj}_{\mathtt{q},
  \mathtt{r}}
 \end{align*}
\end{ex}

\subsection{Equality Test of Partial Traces}
\subsubsection{Calculation of Partial Traces}
\label{symqccs:traceoutalgo}
To verify bisimilarity, the equality of partial traces
must be checked.
In fact, partial traces can be to some extent calculated
quite simply focusing on the structure of the expression of the quantum
states. 
For example,
suppose there are 2 qubits named $q$ and $r$, and the 
the outsider has only $r$. When the quantum state of the total system
is $\E_q(\ket{0}\bra{0}_q\otimes\ket{1}\bra{1}_r)$,
the quantum state that the outsider can access is
$\tr{q}{\E_q(\ket{0}\bra{0}_q\otimes\ket{1}\bra{1}_r)}$.
We have $\tr{q}{\E_q(\ket{0}\bra{0}_q\otimes\ket{1}\bra{1}_r)} =
\ket{1}\bra{1}_r$ for an arbitrary operator $\E_q$ that acts on $q$,
simply eliminating the state $\ket{0}\bra{0}_q$ and
the operator $\E_q$.
This is intuitively interpreted that 
the outsider cannot observe what happens to quantum system that
he or she cannot access.
For the symbolic representations, they can be simplified
focusing on occurrence of quantum variables.
Formulating such calculation, we obtain the following rewriting
rules, where interpretation of the right-hand side of $=$ is equal to
that of the left-hand side regardless of $S_{\mathit{nat}},
S_{\mathit{stat}}, S_{\mathit{op}}$, their interpretations, 
and definitions of $\mathrm{len}(\cdot)$ and $\mathrm{arg}(\cdot)$.
\begin{align}
& \ptr{\tilde q}{\rho}=\ptr{\tilde r}{\ptr{\tilde s}{\rho}}
 \mbox{ if } \tilde q = \tilde r \cup \tilde s\\
& \ptr{\tilde q}{\opsym{op}{\tilde
 r}{\rho}} = \ptr{\tilde q}{\rho} \mbox{ if }
 \tilde r \subseteq \tilde q \\
&\ptr{\tilde q}{\opsym{op}{\tilde r}{\rho}}=
 \opsym{op}{\tilde r}{\ptr{\tilde q}{\rho}}~\mathrm{if}~\tilde q \cap
 \tilde r = \emptyset\\
& \ptr{\tilde q}{\opsym{\mathtt{proj}i}{b}{\rho}}=
\opsym{\mathtt{proj}i}{b}{\ptr{\tilde q}{\rho}}
 \mbox{ if } b \notin \tilde q\\
& \ptr{\tilde q}{\opsym{\mathtt{proj}i}{b}{\rho}}=
\ptr{b}{\opsym{\mathtt{proj}i}{b}{\ptr{\tilde q -
 \{b\}}{\rho}}} \mbox{ if } b \in \tilde q\\
&\ptr{\tilde q}{\rho * \rho_{\tilde q} * \sigma} = \rho *
 \sigma, \mbox{ where } \qv{\rho_{\tilde q}} = \tilde q
\end{align}

\subsubsection{Algorithm to Trace Out}
Verifier1 uses the rewriting rules above.
The procedure goes as follows. Let
$\ptr{\tilde q}{\E_1\texttt{[}\tilde q_1\texttt{]}
\texttt{(}...\texttt{(}\E_n\texttt{[}\tilde 
q_n\texttt{](}\rho_1 \texttt{*} ... \texttt{*} \rho_m \texttt{))}...
\texttt{)}}$
be the objective quantum state, where $\E_i$ is either TPCP map symbol or
$\mathtt{proj}j \, (j \in \{\mathtt{0}, \mathtt{1}\})$ for $0 \le i \le n$.
\begin{enumerate}
 \item A set $S_0$ is initialized to be $\tilde q$.
 \item For each $i\,(1 \leq i \leq n)$, $\E_i$'s are
       successively processed.
       \begin{itemize}
	\item If $\E_i$ is a TPCP
	      map symbol and $\tilde q_i \subseteq S_{i - 1}$ holds,
	      then 
	      $\E_i\texttt{[}\tilde q_i\texttt{]}$ is
	      eliminated by rule (3.2), and $S_i$ is defined to be
	      $S_{i - 1}$.
	\item If $\E_i$ is a TPCP map symbol and $\tilde q_i \subseteq
	      S_{i - 1}$ does not hold, then 
	      $S_i$ is defined to be $S_{i - 1}\backslash \tilde
	      q_i$, which is application of rules (3.1) and (3.3).
	\item If $\E_i\texttt{[}\tilde q\texttt{]}$ is
	      $\mathtt{proj}i\texttt{[}b\texttt{]}$
	      and $b \in S_{i - 1}$ holds, then 
	      $S_i$ is defined to be $S_{i - 1}$,
	      which is application of rules (3.4).
	\item If $\E_i\texttt{[}\tilde q\texttt{]}$ is
	      $\mathtt{proj}i\texttt{[}b\texttt{]}$
	      and $b \notin S_{i - 1}$ holds, then 
	      $S_i$ is defined to be $S_{i - 1}\backslash \{b\}$,
	      which is application of rules (3.5).
       \end{itemize}
 \item A set $T$, recording which quantum variables related to
       the state has been deleted by rule (3.6), is initialized
       to $\emptyset$.
       For each $j\,(1 \leq j \leq m)$, if $\qv{\rho_j} \subseteq S_n$,
       then $\rho_j$ is eliminated and $T$ is updated to 
       $T \cup \qv{\rho_j}$.
 \item $\texttt{Tr[}\tilde q \texttt{]}$ is rewritten to
       $\texttt{Tr[}\tilde q - T \texttt{]}$.
\end{enumerate}
\begin{ex}
 A symbolic representation 
\[
\mathtt{Tr[q,r,b](neg[b](proj0[b](hadamard[r](cnot[q,b](EPR[q,s]*R[r]*R[b])))))}
\]
is simplified by the trace out procedure as follows.
\begin{align*}
&\mathtt{Tr[q,r,b](neg[b](proj0[b](hadamard[r](cnot[q,b](EPR[q,s]*R[r]*R[b])))))}\\
=&\mathtt{Tr[b](proj0[b](Tr[q,r](hadamard[r](cnot[q,b](EPR[q,s]*R[r]*R[b])))))}\\
=&\mathtt{Tr[b](proj0[b](Tr[q,r](cnot[q,b](EPR[q,s]*R[r]*R[b]))))}\\
=&\mathtt{Tr[b](proj0[b](Tr[q](cnot[q,b](Tr[r](EPR[q,s]*R[r]*R[b])))))}\\
=&\mathtt{Tr[b](proj0[b](Tr[q](cnot[q,b](EPR[q,s]*R[b]))))}\\
=&\mathtt{Tr[b,q](proj0[b](cnot[q,b](EPR[q,s]*R[b])))}
\end{align*}
\end{ex}

\subsubsection{User-defined Equations}
Verifier1 also takes user-defined equations to verify equality
of quantum states that are symbolically represented. The equations
are of the form $\rho = \sigma$, where $\rho, \sigma \in \mathcal{S}$.
An equation $\rho = \sigma$ is said to be valid if $\braw{\rho} =
\braw{\sigma}$.

There is a restriction on user-defined equations $\rho = \sigma$:
$\rho$ and $\sigma$ must contain the same number of
$\mathtt{proj}i[b]$ for $i = \mathtt{0}, \mathtt{1}$ and for all
$b \in \sfqv$. This makes the proof
of the soundness (Theorem \ref{symqccs:correspondence}) easier.

\subsubsection{Application of User-defined Equations}
If an objective quantum state has a part that
matches to the left-hand side of a user-defined equation,
the part is rewritten to the right-hand side.
To apply a user-defined equation, Verifier1 automatically
solves commutativity of TPCP maps or partial traces
for disjoint sets of quantum variables.
For example, if the objective quantum
state is $\texttt{Tr[q](hadamard[s]}\texttt{(EPR[q,r]*X[s]))}$ and a user
defines an equation $\texttt{Tr[q](EPR[q,r])=Tr[q](PROB[q,r])}~\mathrm{(E1)}$, 
the application procedure goes as follows.
\begin{align*}
&\texttt{Tr[q](hadamard[s](EPR[q,r]*X[s]))}\\
=~&\texttt{hadamards[s](Tr[q](EPR[q,r]*X[s]))} &&\mbox{(by (3.2))}\\
=~&\texttt{hadamard[s](Tr[q](PROB[q,r]*X[s]))} &&\mbox{(by E1)}
\end{align*}
Since trace-out may have become applicable by application of
user-defined rules, trace-out procedure is applied again.
In each opportunity to test the equality of quantum states, 
each user-defined equation is applied only once.
This guarantees whatever rules a user defines, the equality test
terminates.

\subsubsection{Equality Test after the Rewriting}
After the rewriting by
user-defined equations and trace out,
equality of the two symbolic representations are
checked up to exchange of the order of TPCP map application and
tensor product. For example, symbolic expressions
\begin{align*}
&\texttt{Tr[q](hadamard[s](bitflip[r](EPR[q,r]*X[s])))} \mbox{ and}\\
&\texttt{Tr[q](bitflip[r](hadamard[s](X[s]*EPR[q,r])))}
\end{align*}
must be judged to be equal.
Verifier1 automatically judges the equality by
syntactically checking disjointness of TPCP maps'
arguments and by sorting environment symbols by name, which are
concatenated by ``$\texttt{*}$''.

\subsection{Algorithm to Check Bisimilarity}
\label{symqccs:algorithmforbisim}
The recursive procedure to verify bisimilarity is as follows.
It returns either $\mathit{true}$ or $\mathit{false}$.
\begin{enumerate}
\item The procedure takes as input two configurations $\con{P_0}{\rho_0}$,
      $\con{Q_0}{\sigma_0}$ and user-defined equations $\mathit{eqs}$
      on quantum states.
\item If $P_0$ and $Q_0$ can perform any $\tau$-transitions of
      TPCP map applications, they are all performed at this point.
      Let $\con{P}{\rho}$ and $\con{Q}{\sigma}$ be the configurations
      thus obtained.
\item Whether $\qv{P} = \qv{Q}$ is checked. If it does not hold, the
      procedure returns $\mathit{false}$.
\item Whether $\ptrtt{\qv{P}}{\rho} = \ptrtt{\qv{Q}}{\sigma}$ is 
      checked
      using $\mathit{eqs}$.
      The procedure to check equality of quantum states are described
      in the previous subsection.
      If it does not hold, the procedure returns $\mathit{false}$.
\item A new TPCP map symbol $\E\texttt{[}\qv{\rho} - \qv{P}\texttt{]}$
      that stands for an
      arbitrary operation is generated.
\item \begin{enumerate}
       \item For each $\con{P'}{\rho'}$ such that
	     \begin{itemize}
	      \item $\con{P}{\opsym{\E}{\qv{\rho} - \qv{P}}{\rho}}
		    \xrightarrow{\alpha}
		    \con{P'}{\rho'}$ holds, and
	      \item  neither 
		     \begin{itemize}
		      \item $\rho' \equiv 
			    \opsym{\texttt{proj}0}{b}{
			    \opsym{\E}{\qv{\rho} -\qv{P}}{\rho}}$ nor
		      \item $\rho' \equiv 
			    \opsym{\texttt{proj}1}{b}{
			    \opsym{\E}{\qv{\rho} -\qv{P}}{\rho}}$
		     \end{itemize}
		     holds for any $b \in \qv{P}$,
	     \end{itemize}
	     the procedure checks whether there exists
	     $\con{Q'}{\sigma'}$
	     such that
	     \[
	      \con{Q}{\opsym{\E}{\qv{\sigma} -
	     \qv{Q}}{\sigma}} \weak{\hat \alpha}
	     \con{Q'}{\sigma'}
	     \]
	     holds and 
	     the procedure returns $\mathit{true}$ with the input
	     $\con{P'}{\rho'}$,
	     $\con{Q'}{\sigma'}$, and $\mathit{eqs}$. 
       \item For each pair $(\con{P'}{\rho'}, \con{P''}{\rho''})$
	     such that
	     \begin{itemize}
	      \item $\con{P}{\E\texttt{[}\qv{\rho} -
		    \qv{P}\texttt{](}\rho\texttt{)}} \xrightarrow{\tau}
		    \con{P'}{\rho'}$,
	      \item $\rho' \equiv 
		    \opsym{\texttt{proj0}}{b}{
                    \opsym{\E}{\qv{\rho} -\qv{P}}{\rho}}$,
	      \item $\con{P}{\E\texttt{[}\qv{\rho} -
		    \qv{P}\texttt{](}\rho\texttt{)}} \xrightarrow{\tau}
		    \con{P''}{\rho''}$, and
	      \item $\rho'' \equiv 
		    \opsym{\texttt{proj1}}{b}{
                    \opsym{\E}{\qv{\rho} -\qv{P}}{\rho}}$
	     \end{itemize}
	     hold for some $b \in \qv{P}$,
	     the procedure checks whether there exists a pair
	     $(\con{Q'}{\sigma'}, \con{Q''}{\sigma''})$
	     such that
	     \begin{itemize}
	      \item $\con{Q}{\E\texttt{[}\qv{\sigma} -
		    \qv{Q}\texttt{](}\sigma\texttt{)}}
		    \xrightarrow{\tau\ast}
		    \con{\hat Q}{\hat \sigma}$,
	      \item $\con{\hat Q}{\hat \sigma}
		    \xrightarrow{\tau}
		    \con{\hat Q'}
                        {\opsym{\texttt{proj0}}{b}{\hat \sigma}}
		    \xrightarrow{\tau\ast}
		    \con{Q'}{\sigma'}$, and
	      \item $\con{\hat Q}{\hat \sigma}
		    \xrightarrow{\tau}
		    \con{\hat Q''}
                        {\opsym{\texttt{proj1}}{b}{\hat \sigma}}
		    \xrightarrow{\tau\ast}
		    \con{Q''}{\sigma''}$
	     \end{itemize}
	     hold for some $\hat Q$, $\hat Q'$, and $\hat Q''$, and
	     \begin{itemize}
	      \item Verifier1
		    returns $\mathit{true}$ with $(\con{P'}{\rho'},
		    \con{Q'}{\sigma'})$ and $\mathit{eqs}$, and
	      \item Verifier1
		    returns $\mathit{true}$ with $(\con{P''}{\rho''},
		    \con{Q''}{\sigma''})$ and $\mathit{eqs}$.
	     \end{itemize}
      \end{enumerate}
      If there exists, it goes
      to the next step 7. Otherwise, it returns $\mathit{false}$.
\item For each $\con{Q'}{\sigma'}$ such that
      $\con{Q}{\E\texttt{[}\qv{\sigma} -
      \qv{Q}\texttt{](}\sigma\texttt{)}} \xrightarrow{\alpha}
      \con{Q'}{\sigma'}$,
      the procedure checks the symmetric condition of the step 6.
      If there exists, it returns $\mathit{true}$. Otherwise,
      it returns $\mathit{false}$.
\end{enumerate}
The procedure always terminates.
This is because the transition of the processes is finite and
equality check in the step 4 always terminates.

The step 2 prominently decreases the spaces to search.
This is based on the fact that 
$\con{\mathit{op}_1[\tilde q].P||\mathit{op}_2[\tilde
r].Q}{\rho}$ and $ 
\con{P||Q}{\F_{op_2}^{\tilde r}(\E_{op_1}^{\tilde q}(\rho))}$ are
bisimilar, and \\
 $\F_{op_2}^{\tilde r}(\E_{op_1}^{\tilde q}(\rho)) =
\E_{op_1}^{\tilde q}(\F_{op_2}^{\tilde r}(\rho)) $ holds because $\tilde
q \cap \tilde r = \emptyset$ and $\qv{P}\cap\qv{Q}=\emptyset$ hold.

In Section \ref{symqccs:sndness}, when we prove soundness of
Verifier1, we apply the fact that the numbers of $\mathtt{proj}i$'s
are equal in $\rho$ and $\sigma$ if Verifier1 returns $\mathit{true}$
with $\con{P}{\rho}$ and $\con{Q}{\sigma}$. The reason is as follows.
There is the restriction that the both sides of an equation contains
the same number of $\mathtt{proj}i$'s.
This implies that rewriting by an arbitrary user-defined equation
does not change the number of $\mathtt{proj}i$'s in a symbolic
representation. Besides, the trace out procedure does not
eliminate $\mathtt{proj}i$'s. Therefore, the numbers of 
$\mathtt{proj}i$'s must be equal to have passed the equality test
in the step 4. 

We make here a remark about the step 6 (b).
Suppose the following transitions are performed.
\begin{itemize}
 \item $ \con{P}{\E\texttt{[}\qv{\rho} -
		    \qv{P}\texttt{](}\rho\texttt{)}}
       \xrightarrow{\tau}
       \con{P'}{\opsym{\mathtt{proj0}}{b}{\E\texttt{[}\qv{\rho} -
       \qv{P}\texttt{](}\rho\texttt{)}}}$
 \item $ \con{P}{\E\texttt{[}\qv{\rho} -
		    \qv{P}\texttt{](}\rho\texttt{)}}
       \xrightarrow{\tau}
       \con{P''}{\opsym{\mathtt{proj1}}{b}{\E\texttt{[}\qv{\rho} -
       \qv{P}\texttt{](}\rho\texttt{)}}}$
\end{itemize}
To return $\mathit{true}$ with $\con{P}{\rho},
\con{Q}{\sigma}$, and $\mathit{eqs}$, 
it requires the existence of $\con{Q'}{\sigma'}$ and
$\con{Q''}{\sigma''}$ satisfying the conditions mentioned in 6
(b), \emph{even if} $\braw{\opsym{\mathtt{proj0}}{b}{\E\texttt{[}\qv{\rho} -
       \qv{P}\texttt{](}\rho\texttt{)}}} = O$ \emph{holds}, which
means the probability of this transition is $0$.
As for this case, in fact, only the 
existence of $\con{Q''}{\sigma''}$ is necessary 
in the proof of the soundness but that of
$\con{Q'}{\sigma'}$ is not. Therefore, the condition
that Verifier1 returns $\mathit{true}$ with $\con{P}{\rho}$ and
$\con{Q}{\sigma}$ is stronger than the condition that
the two qCCS configurations corresponding to them are bisimilar.

\subsubsection{Memoization}
We also employ a memoization technique. Let $\con{P}{\rho}$ and
$\con{Q}{\sigma}$
have the transitions $\con{P}{\rho} \xrightarrow{\alpha} \con{P'}{\rho'}$ and 
$\con{Q}{\sigma} \xrightarrow{\alpha} \con{Q'}{\sigma'}$, and assume $\con{P'}{\rho'} \approx \con{Q'}{\sigma'}$.
When checking whether $\con{P}{\rho} \approx \con{Q}{\sigma}$ holds, Verifier1 
first checks $\con{Q}{\sigma}$ simulates $\con{P}{\rho}$'s transition. For $\con{P}{\rho}
\xrightarrow{\alpha} \con{P'}{\rho'}$, Verifier1 finds $\con{Q}{\sigma} \xrightarrow{\alpha}
\con{Q'}{\sigma'}$, and then recursively checks $\con{P'}{\rho'} \approx \con{Q'}{\sigma'}$.
Verifier1 then checks $\con{P}{\rho}$ simulates $\con{Q}{\sigma}$'s transition. For the transition $\con{Q}{\sigma} \xrightarrow{\alpha} \con{Q'}{\sigma'}$, it finds
the transition $\con{P}{\rho} \xrightarrow{\alpha} \con{P'}{\rho'}$, and next
checks $\con{Q'}{\sigma'} \approx \con{P'}{\rho'}$. Since $\approx$ is a 
symmetric relation, the last condition has been
already obtained when checking $\con{P'}{\rho'} \approx
\con{Q'}{\sigma'}$.
Verifier1 reuses the result.

\section{Soundness of Verifier1}
\label{symqccs:sndness}
Verifier1 is designed to be sound in the following sense.
If Verifier1 returns $\mathit{true}$ for two configurations
$\con{P}{\rho},
\con{Q}{\sigma} \in \mathcal{P} \times \mathcal{S}$,
and some valid user-defined equations, then
the corresponding two configurations that are elements of $\Con$
are bisimilar in the original qCCS. Our goal is to prove Theorem
 \ref{symqccs:correspondence} that states the correspondence formally.
In this section, we prepare lemmas to prove it.

Suppose $\con{P}{\rho} \xrightarrow{\alpha} \con{P'}{\rho'}$ holds. 
We say the transition $\xrightarrow{\alpha}$ is caused by {\it rule's
name}, where {\it rule's name} is either $\mathrm{(In)},
\mathrm{(Out)}, \mathrm{(Op)}, \mathrm{(Meas1)}$, or
$\mathrm{(Meas0)}$, if the derivation tree begins with the application
of the rule and
$\mathrm{(Comm)}$ rule is not used. If $\mathrm{(Comm)}$ rule is used,
we say the transition is caused by $\mathrm{(Comm)}$.
We first prepare the notation to focus on a part of a process that causes
a transition.

\begin{defi} The evaluation contexts are defined as follows.
 \[
  C[\_] ::= \_ ~~|~~  C[\_]\lVert P ~~|~~ P\lVert C[\_] ~~|~~ C[\_]
 \backslash L \]
\end{defi}
\begin{lem}
\label{symqccs:redex}
 If $\mathcal{P} \times \mathcal{S}
 \ni \con{P}{\rho} \xrightarrow{\sndq{c}{q}} \con{P'}{\rho'}$,
 then
 $P = C[\sndq{c}{q}.P_0]$ and $P' = C[P_0]$
 hold for some process $P_0$ and evaluation context $C[\_]$ that does 
 not restrict $\mathsf{c}$.
\end{lem}

\begin{proof}
 We prove it by induction of the number $n$ of application
 of the transition rules. \\
 (Case 1) Assume $n = 1$. The only rule to derive
 $\xrightarrow{\sndq{c}{q}}$ with one time
 application is $\mathrm{(Out)}$. Therefore, $P = \sndq{c}{q}.P_0$
 holds.\\
 (Case 2) Assume $n > 1$. The last rule applied is either 
 $\mathrm{(Res)}, \mathrm{(Right)}$, or $\mathrm{(Left)}$.
 We prove the case of $\mathrm{(Res)}$ as other cases are similar.
 The last derivation is
\[
 \frac{~\con{P_1}{\rho} \xrightarrow{\sndq{c}{q}} \con{P'_1}{\rho}~~~
  \mathsf{c} \notin L~} 
 {\con{P_1 \backslash L}{\rho} \xrightarrow{\sndq{c}{q}}
  \con{P'_1 \backslash L}{\rho}},
\]
where $P = P_1 \backslash L$ and $P' = P'_1 \backslash L$ 
hold, for some $L$.
By I.H., $P_1 = C[\sndq{c}{q}.P_2]$ and $P'_1 = C[P_2]$ for 
some $C[\_]$ and $P_2$.
We take an evaluation context $C'[\_] = C[\_] \backslash L$.
As $\mathsf{c} \notin L$, $C'[\_]$ does not restrict $\mathsf{c}$.
We then have $P = C'[\sndq{c}{q}.P_2]$ and $P' = C'[P_2]$.
\end{proof}

Lemma \ref{symqccs:redex} is for transitions caused by (Out).
Transitions caused by (In), (Op), (Meas$i$), and (Comm) have
a similar property since the derivations start from those and
proceed by applying (Left), (Right), (Res) rules. The original
qCCS also has a similar property.

Next, we define the correspondence of processes in $\mathcal{P}$ and
those in $\Proc$, where $\mathcal{P}$ is the set of 
the processes of nondeterministic qCCS
and $\Proc$ is the set of the processes of original qCCS.

\begin{defi}
 The function $\mathrm{cnv}: \mathcal{P} \rightarrow \Proc$
 is inductively defined as follows.
 \begin{align*}
  &\cnv{\mathtt{discard[}\tilde q\mathtt{]}} =
  \myif{\mathtt{false}}{I[\tilde q].\nil}\\
  &\cnv{\sndq{c}{q}.P} = \sndq{c}{q}.\cnv{P} \\
  &\cnv{\rcvq{c}{q}.P} = \rcvq{c}{q}.\cnv{P}\\
  &\cnv{\mathit{op}\mathtt{[}\tilde q\mathtt{]}.P} =
  \mathit{op}\mathtt{[}\tilde q\mathtt{]}.\cnv{P} \\
  &\cnv{\measure{b}{P}} = \ket{1}\bra{1}[b;x].\myif{x = 1}{\cnv{P}}\\
  &\cnv{P||Q} = \cnv{P}||\cnv{Q}\\
  &\cnv{P \backslash L} = \cnv{P}\backslash L
 \end{align*}
For an evaluation context $C[\_]$, $\mathrm{cnv}(C)[\_]$ is the
context of original qCCS process obtained applying $\mathrm{cnv}$ to
all processes in $C[\_]$. 
\end{defi}
By the definition, we have the following proposition.
\begin{prop}
\label{symqccs:cnvinj}
 $\qv{P}=\qv{\mathrm{cnv(P)}}$ holds.
 If $\cnv{P} = \cnv{Q}$, then $P = Q$. 
\end{prop}
We then prove lemmas that state correspondence of original and
Verifier1's frameworks.
\begin{lem}
 $\mathcal{P} \times \mathcal{S} \ni
 \con{P}{\rho} \xrightarrow{\alpha} \con{P'}{\rho'}$ and
 $\tr{}{\braw{\rho}} =
 \tr{}{\braw{\rho'}}$ hold, then 
 \[
   \con{\cnv{P}}{\frac{\braw{\rho}}{\tr{}{\braw{\rho}}}} 
 \xrightarrow{\alpha} \mu \mbox{ and }
\mu (\strg)^\dagger 1\bullet
 \con{\cnv{P'}}{\frac{\braw{\rho'}}{\tr{}{\braw{\rho'}}}}
 \]
 hold for some $\mu \in D(\Con)$.
\end{lem}
\begin{proof}
%%%%
 ($\alpha$ is $\sndq{c}{q}$)
 By lemma \ref{symqccs:redex}, $P = C[\sndq{c}{q}.P_0]$ and 
 $P' = C[P_0]$ holds for some evaluation
 context $C[\_]$ and process $P_0$.
 Since 
 \[
 \cnv{P} = \cnv{C[\sndq{c}{q}.P_0]} =
 \cnv{C}[\sndq{c}{q}.\cnv{P_0}] \mbox{ and }
 \cnv{P'} = \cnv{C[P_0]}
 \]
 hold,
 $\con{\cnv{P}}
      {\frac{\braw{\rho}}{\tr{}{\braw{\rho}}}}
 \xrightarrow{\sndq{c}{q}}
 \con{\cnv{P'}}{\frac{\braw{\rho}}{\tr{}{\braw{\rho}}}}$ holds. 
 The conclusion of the lemma holds because identity is a 
strong bisimulation.
 \\
%%%%
 ($\alpha$ is $\rcvq{c}{q}$ or $\tau$ caused by
 $\mathrm{Comm}$) Similar to the above case.\\
%%%%
 ($\alpha$ is $\tau$ caused by $\mathrm{Op}$) Similar to the above
 cases
 except that the quantum state changes. The correctness of 
 the statement is checked observing that $\rho' =
 \op{\mathit{op}}{\tilde r}{\mathtt{(}\rho\mathtt{)}}$ for some
 $\mathit{op}[\tilde q]$and
 $\E^{op}_{\tilde r}(\frac{\braw{\rho}}{\tr{}{\braw{\rho}}}) =
 \frac{\braw{\rho'}}{\tr{}{\braw{\rho'}}}$ holds because $\E^{op}$ is
 trace-preserving.\\
%%%%
 ($\alpha$ is $\tau$ caused by $\mathrm{Meas1}$)
 Similarly to lemma \ref{symqccs:redex}, $P = C[\measure{b}{P_0}]$ and 
 $P' = C[P_0]$ holds for some evaluation
 context $C[\_]$ and process $P_0$.
 By $\tr{}{\braw{\rho}} = \tr{}{\braw{\rho'}} = 
 \tr{}{\braw{\mathtt{proj1}[b](\rho)}} = 
 \tr{}{\ket{1}\bra{1}_b\braw{\rho}}$,
\begin{align*}
 &\con{\cnv{C}[\ket{1}\bra{1}[b;x].\myif{x = 1}{\cnv{P_0}}]}
 {\frac{\braw{\rho}}{\tr{}{\braw{\rho}}}} \xrightarrow{\tau}\\
 &1 \bullet \con{\cnv{C}[\myif{1=1}{\cnv{P_0}}]}
 {\frac{\ket{1}\bra{1}_b\braw{\rho}}
  {\tr{}{\ket{1}\bra{1}_b\braw{\rho}}}}
\end{align*}
 holds.
 Since
 \[
\con{\myif{1=1}{\cnv{P_0}}}
{\frac{\ket{1}\bra{1}_b\braw{\rho}\ket{1}\bra{1}_b}
      {\tr{}{\ket{1}\bra{1}_b\braw{\rho}}}}
 (\strg)^\dagger
\con{\cnv{P_0}}{\frac{\ket{1}\bra{1}_b\braw{\rho}\ket{1}\bra{1}_b}
{\tr{}{\ket{1}\bra{1}_b\braw{\rho}}}}
\]
 and $\ket{1}\bra{1}_b\braw{\rho}\ket{1}\bra{1}_b = \braw{\rho'}$
 hold, 
\[
 1 \bullet \con{\cnv{C}[\myif{1 = 1}{\cnv{P_0}}]}
 {\frac{\ket{1}\bra{1}_b\braw{\rho}\ket{1}\bra{1}_b}
       {\tr{}{\ket{1}\bra{1}_b\braw{\rho}}}} (\strg)^\dagger
1 \bullet \con{\cnv{P'}}{\frac{\braw{\rho'}}{\tr{}{\braw{\rho'}}}}
.
\]
holds by the congruence of $\strg$ (Proposition \ref{symqccs:strgcong}).
\\
%%%%
 ($\alpha$ is $\tau$ caused by $\mathrm{Meas0}$) Similar to the
 case of $\mathrm{Meas1}$.
\end{proof}

\begin{lem}
\label{symqccs:weak}
  $ \mathcal{P} \times \mathcal{S} \ni
\con{P}{\rho} \weak{\hat \alpha} \con{P'}{\rho'}$ and
 $\tr{}{\braw{\rho}} =
 \tr{}{\braw{\rho'}}$ holds, then
 $\con{\cnv{P}}{\frac{\braw{\rho}}{\tr{}{\braw{\rho}}}} 
 \Rightarrow (\xrightarrow{\hat \alpha})^\dagger \Rightarrow 
 \mu$ and 
 $\mu (\strg)^\dagger
 \con{\cnv{P'}}{\frac{\braw{\rho'}}{\tr{}{\braw{\rho'}}}}$
 holds
 for some $\mu \in D(\Con)$.
\end{lem}
\begin{proof}
\label{symqccs:prfofweak}
By assumption, we have
\begin{itemize}
 \item $\con{P}{\rho} \xrightarrow{\tau} \con{P_1}{\rho_1}
       \xrightarrow{\tau} \cdots \xrightarrow{\tau} \con{P_k}{\rho_k}
       \xrightarrow{\hat \alpha} 
       \con{\hat P} {\hat \rho}
       \xrightarrow{\tau} \con{P'_1}{\rho'_1}
       \xrightarrow{\tau} \cdots \xrightarrow{\tau} \con{P'_m}{\rho'_m},$
 \item $\tr{}{\braw{\rho}} = \tr{}{\braw{\rho_1}} = 
       \cdots = \tr{}{\braw{\rho'}}$, and
 \item $\con{P'_m}{\rho'_m} = \con{P'}{\rho'}$
\end{itemize}
for some $k$, $m$,
$P_1$,...,$P_k$,$\hat P$,$P'_1$,...,$P'_m$
$\rho_1$,...,$\rho_k$,$\hat \rho$,$\rho'_1$,...,$\rho'_m$.
By $\con{P}{\rho} \xrightarrow{\tau} \con{P_1}{\rho_1}$ and
$\tr{}{\braw{\rho}} = \tr{}{\braw{\rho_1}}$
and the previous lemma, 
\[
 \con{\cnv{P}}{\frac{\braw{\rho}}{\tr{}{\braw{\rho}}}} 
 \xrightarrow{\tau} \mu_1 (\strg)^\dagger 
 1 \bullet \con{\cnv{P_1}}{\frac{\braw{\rho_1}}{\tr{}{\braw{\rho_1}}}} 
\]
holds for some $\mu_1 \in D(\Con)$. Next, we prove for all 
$i \, (1 \le i \le k-1)$ that 
$\con{P_i}{\rho_i}
\xrightarrow{\tau} \con{P_{i+1}}{\rho_{i+1}}$ and
$\tr{}{\braw{\rho_i}} = \tr{}{\braw{\rho_{i+1}}}$
and
\[
 \con{\cnv{P}}{\frac{\braw{\rho}}{\tr{}{\braw{\rho}}}} 
 \Rightarrow \mu_i (\strg)^\dagger 
 1 \bullet \con{\cnv{P_i}}{\frac{\braw{\rho_i}}{\tr{}{\braw{\rho_i}}}} 
 \mbox{ for some } \mu_i
\]
imply
\[
 \con{\cnv{P}}{\frac{\braw{\rho}}{\tr{}{\braw{\rho}}}} 
 \Rightarrow
 \mu_{i+1} (\strg)^\dagger 
 1 \bullet
 \con{\cnv{P_{i+1}}}{\frac{\braw{\rho_{i+1}}}{\tr{}{\braw{\rho_{i+1}}}}} 
 \mbox{ for some } \mu_{i+1}
 .
\]
By $\con{P_i}{\rho_i}
\xrightarrow{\tau} \con{P_{i+1}}{\rho_{i+1}}$ and
$\tr{}{\braw{\rho_i}} = \tr{}{\braw{\rho_{i+1}}}$ and the previous
lemma, we have 
\[
 \con{\cnv{P_i}}{\frac{\braw{\rho_i}}{\tr{}{\braw{\rho_i}}}} 
 \Rightarrow
 \mu'_{i+1} (\strg)^\dagger 
 1 \bullet
 \con{\cnv{P_{i+1}}}{\frac{\braw{\rho_{i+1}}}{\tr{}{\braw{\rho_{i+1}}}}} 
 \mbox{ for some } \mu'_{i+1}.
\]
By $\mu_i (\strg)^\dagger 1 \bullet
 \con{\cnv{P_i}}{\frac{\braw{\rho_i}}{\tr{}{\braw{\rho_i}}}}$ and
 $\con{\cnv{P_i}}{\frac{\braw{\rho_i}}{\tr{}{\braw{\rho_i}}}}
 \Rightarrow
 \mu'_{i+1}$, $\mu_i \Rightarrow \mu_{i+1}$ and\\
 $\mu_{i+1} (\strg)^\dagger \mu'_{i+1}$ holds
 for some $\mu_{i+1}$. We then have
 \[
 \con{\cnv{P}}{\frac{\braw{\rho}}{\tr{}{\braw{\rho}}}} 
 \Rightarrow
 \mu_i
 \Rightarrow
 \mu_{i+1} (\strg)^\dagger \mu'_{i+1} 
 (\strg)^\dagger
 1 \bullet 
\con{\cnv{P_{i+1}}}
    {\frac{\braw{\rho_{i+1}}}{\tr{}{\braw{\rho_{i+1}}}}},
 \]
 namely,
 \[
 \con{\cnv{P}}{\frac{\braw{\rho}}{\tr{}{\braw{\rho}}}} 
 \Rightarrow
 \mu_{i+1} 
 (\strg)^\dagger
 1 \bullet
 \con{\cnv{P_{i+1}}}
     {\frac{\braw{\rho_{i+1}}}{\tr{}{\braw{\rho_{i+1}}}}}
 \mbox{ for some } \mu_{i+1}.
 \]
 Applying this argument repeatedly, we have
 \[
 \con{\cnv{P}}{\frac{\braw{\rho}}{\tr{}{\braw{\rho}}}} 
 \Rightarrow
 \mu_{k} 
 (\strg)^\dagger
 1 \bullet
 \con{\cnv{P_{k}}}
     {\frac{\braw{\rho_{k}}}{\tr{}{\braw{\rho_{k}}}}}
 \mbox{ for some } \mu_{k}.
 \]
By the similar argument, we have
 \[
 \con{\cnv{P}}{\frac{\braw{\rho}}{\tr{}{\braw{\rho}}}} 
 \Rightarrow
 (\xrightarrow{\hat \alpha})^\dagger
 \hat \mu 
 (\strg)^\dagger
 1 \bullet
 \con{\cnv{\hat P}}
     {\frac{\braw{\hat \rho}}{\tr{}{\braw{\hat \rho}}}}
 \mbox{ for some } \hat \mu.
 \]
Furthermore, we have
 \[
 \con{\cnv{P}}{\frac{\braw{\rho}}{\tr{}{\braw{\rho}}}} 
 \Rightarrow
 (\xrightarrow{\hat \alpha})^\dagger
 \Rightarrow
 \mu
 (\strg)^\dagger
 1 \bullet
 \con{\cnv{P'}}
     {\frac{\braw{\rho'}}{\tr{}{\braw{\rho'}}}}
 \mbox{ for some } \mu.
 \]
\end{proof}

\begin{lem}
 If 
 \begin{itemize}
  \item $ \mathcal{P} \times \mathcal{S} \ni
	\con{P}{\rho} = \con{C[\measure{b}{P_0}]}{\rho}
	\xrightarrow{\tau} 
	\con{C[P_0]}{\mathtt{proj1}[b](\rho)} \defeq \con{P'}{\rho'}$ and
  \item $\con{P}{\rho} 
	\xrightarrow{\tau} 
	\con{C[\mathtt{discard}(\qv{P_0})]}{\mathtt{proj0}[b](\rho)} \defeq
	\con{P''}{\rho''}$
 \end{itemize}
 hold, then 
\begin{itemize}
 \item  $\con{\cnv{P}}{\frac{\braw{\rho}}{\tr{}{\braw{\rho}}}} 
	\xrightarrow{\tau} \mu$ and
 \item  $\mu (\strg)^\dagger
	\frac{\tr{}{\braw{\rho'}}}{\tr{}{\braw{\rho}}} \bullet
	\con{\cnv{P'}}{\frac{\braw{\rho'}}{\tr{}{\braw{\rho'}}}} +
	\frac{\tr{}{\braw{\rho''}}}{\tr{}{\braw{\rho}}} \bullet
	\con{\cnv{P''}}{\frac{\braw{\rho''}}{\tr{}{\braw{\rho''}}}}$
\end{itemize}
 hold for some $\mu \in D(\Con)$.
\end{lem}
\begin{proof}
 We have
 \begin{align*}
 &\con{\cnv{P}}{\frac{\braw{\rho}}{\tr{}{\braw{\rho}}}}=
 \con{\cnv{C}[\ket{1}\bra{1}[b;x].\myif{x =
 1}{\cnv{P_0}}]}{\frac{\braw{\rho}}{\tr{}{\braw{\rho}}}}\\
\xrightarrow{\tau}
 &\frac{\tr{}{\ket{0}\bra{0}_b\braw{\rho}}}{\tr{}{\braw{\rho}}} \bullet
 \con{\cnv{C}[\myif{0 =
 1}{\cnv{P_0}}]}{\frac{\ket{0}\bra{0}_b\braw{\rho}\ket{0}\bra{0}_b}
                    {\tr{}{\ket{0}\bra{0}_b\braw{\rho}}}}\\
 &+\frac{\tr{}{\ket{1}\bra{1}_b\braw{\rho}}}{\tr{}{\braw{\rho}}} \bullet
 \con{\cnv{C}[\myif{1 = 1}{\cnv{P_0}}]}{\frac{\ket{1}\bra{1}_b\braw{\rho}
  \ket{1}\bra{1}_b}{
  \tr{}{\ket{1}\bra{1}_b\braw{\rho}}}} \defeq \mu.
\end{align*}
Besides, we have
\begin{align*}
&\con{\cnv{C}[\myif{0 =1}{\cnv{P_0}}]}
 {\frac{\ket{0}\bra{0}_b\braw{\rho}\ket{0}\bra{0}_b}
                    {\tr{}{\ket{0}\bra{0}_b\braw{\rho}}}}
\\
 \strg &\con{\cnv{C}[\myif{0 = 1}{I[\qv{P_0}].\nil}]}
              {\frac{\ket{0}\bra{0}_b\braw{\rho}\ket{0}\bra{0}_b}
                    {\tr{}{\ket{0}\bra{0}_b\braw{\rho}}}} =
        \con{\cnv{P''}}{\frac{\braw{\rho''}}{\tr{}{\braw{\rho''}}}},
\\
&\mbox{and}\\
&\con{\cnv{C}[\myif{1 = 1}{\cnv{P_0}}]}
     {\frac{\ket{1}\bra{1}_b\braw{\rho}\ket{1}\bra{1}_b}
                    {\tr{}{\ket{1}\bra{1}_b\braw{\rho}}}}
\\
 \strg &\con{\cnv{C}[\cnv{P_0}]}
              {\frac{\ket{1}\bra{1}_b\braw{\rho}\ket{1}\bra{1}_b}
                    {\tr{}{\ket{1}\bra{1}_b\braw{\rho}}}} = 
        \con{\cnv{P'}}{\frac{\braw{\rho'}}{\tr{}{\braw{\rho'}}}}.
\end{align*}
We have the conclusion of the lemma by the linearity of
$(\strg)^\dagger$.
\end{proof}

\begin{lem}
\label{symqccs:prfofweakmeas}
If
\begin{itemize}
 \item  $\mathcal{P} \times \mathcal{S} \ni
	\con{P}{\rho} \xrightarrow{\tau\ast}
	\con{C[\measure{b}{P'}]}{\rho'} 
	\xrightarrow{\tau}
	\con{C[P']}{\mathtt{proj1}[b](\rho')} 
	\xrightarrow{\tau\ast} \con{P_1}{\rho_1}$,
 \item  $\con{C[\measure{b}{P'}]}{\rho'} \xrightarrow{\tau}
	\con{C[\mathtt{discard}(\qv{P'})]}{\mathtt{proj0}[b](\rho')}
	\xrightarrow{\tau\ast}\con{P_0}{\rho_0}$, and
 \item  $\tr{}{\braw{\rho}} = \tr{}{\braw{\rho'}} = 
	\tr{}{\braw{\rho_0}} + \tr{}{\braw{\rho_1}}$
\end{itemize}
hold, then
\begin{itemize}
 \item $\con{\cnv{P}}{\frac{\braw{\rho}}{\tr{}{\braw{\rho}}}}
       \Rightarrow \mu$ and
 \item $\mu (\strg)^\dagger
       \frac{\tr{}{\braw{\rho_0}}}{\tr{}{\braw{\rho}}} \bullet 
       \con{\cnv{P_0}}{\frac{\braw{\rho_0}}{\tr{}{\braw{\rho_0}}}}
       +\frac{\tr{}{\braw{\rho_1}}}{\tr{}{\braw{\rho}}} \bullet
       \con{\cnv{P_1}}{\frac{\braw{\rho_1}}{\tr{}{\braw{\rho_1}}}}$
\end{itemize}
hold for some $\mu \in D(\Con)$.
\end{lem}
\begin{proof}
 By the same argument as the proof of Lemma \ref{symqccs:weak}, we have
\begin{align*}
\con{\cnv{P}}{\frac{\braw{\rho}}{\tr{}{\braw{\rho}}}}
 \Rightarrow 
\mu_0 (\strg)^\dagger
1 \bullet \con{\cnv{C[\measure{b}{P'}]}}{\frac{\braw{\rho'}}{\tr{}{\braw{\rho'}}}}
\end{align*}
for some $\mu_0$.
By the previous lemma, we have
\begin{align*}
 &1 \bullet \con{\cnv{C[\measure{b}{P'}]}}
      {\frac{\braw{\rho'}}{\tr{}{\braw{\rho'}}}} \Rightarrow \mu_1,
\mbox{ and }\\
\mu_1 (\strg)^\dagger
 &\frac{\tr{}{\braw{\mathtt{proj0}[b](\rho')}}}{\tr{}{\braw{\rho}}}
 \bullet 
\con{\cnv{C[\mathtt{discard}(\qv{P'})]}}
{\frac{\braw{\mathtt{proj0}[b](\rho')}}
      {\tr{}{\braw{\mathtt{proj0}[b](\rho')}}}}\\
&+\frac{\tr{}{\braw{\mathtt{proj1}[b](\rho')}}}{\tr{}{\braw{\rho}}}
 \bullet
\con{\cnv{C[P']}}{\frac{\braw{\mathtt{proj1}[b](\rho')}}{\tr{}{\braw{\mathtt{proj1}[b](\rho')}}}}.
\end{align*}
for some $\mu_1$.
By
\begin{align*}
 &\tr{}{\braw{\rho'}} = \tr{}{\braw{\mathtt{proj0}[b](\rho')}} +
 \tr{}{\braw{\mathtt{proj1}[b](\rho')}},\\
 &\tr{}{\braw{\mathtt{proj0}[b](\rho')}} \ge \tr{}{\braw{\rho_0}},
 \mbox{ and}\\
 &\tr{}{\braw{\mathtt{proj1}[b](\rho')}} \ge \tr{}{\braw{\rho_1}},
\end{align*}
we have
$\tr{}{\braw{\mathtt{proj0}[b](\rho')}} = \tr{}{\braw{\rho_0}}$
and
$\tr{}{\braw{\mathtt{proj1}[b](\rho')}} = \tr{}{\braw{\rho_1}}$.
Let $\frac{\tr{}{\braw{\rho_0}}}{\tr{}{\braw{\rho}}} \defeq
p_0$ and 
$\frac{\tr{}{\braw{\rho_1}}}{\tr{}{\braw{\rho}}} \defeq
p_1$.
Now, we apply the same argument as the proof of 
Lemma \ref{symqccs:weak}
to each configuration. We have
\begin{align*}
 X \defeq &\con{\cnv{C[\mathtt{discard}(\qv{P'})]}}
{\frac{\braw{\mathtt{proj0}[b](\rho')}}
      {\tr{}{\braw{\mathtt{proj0}[b](\rho')}}}}
\Rightarrow \mu_2 (\strg)^\dagger 
\con{\cnv{P_0}}{\frac{\braw{\rho_0}}{\tr{}{\braw{\rho_0}}}}\\
\mbox{and}&\\
Y \defeq &\con{\cnv{C[P']}}
 {\frac{\braw{\mathtt{proj1}[b](\rho')}}
 {\tr{}{\braw{\mathtt{proj1}[b](\rho')}}}} 
\Rightarrow \mu_3 (\strg)^\dagger 
\con{\cnv{P_1}}{\frac{\braw{\rho_1}}{\tr{}{\braw{\rho_1}}}}
\end{align*}
for some $\mu_2$ and $\mu_3$. We then have
\begin{align*}
\mu_1 (\strg)^\dagger &
 p_0 \bullet X + p_1 \bullet Y
 \Rightarrow
 p_0 \mu_2 +  p_1 \mu_3 ~~ (\sharp)~~ \mbox{ and }\\
 p_0 \mu_2 +  p_1 \mu_3
 (\strg)^\dagger &
 p_0 \bullet \con{\cnv{P_0}}{\frac{\braw{\rho_0}}{\tr{}{\braw{\rho_0}}}}
+
 p_1 \bullet \con{\cnv{P_1}}{\frac{\braw{\rho_1}}{\tr{}{\braw{\rho_1}}}}.
\end{align*}
By $(\sharp)$, we have $\mu_1 \Rightarrow \mu$ and $\mu (\strg)^\dagger
p_0 \mu_2 + p_1 \mu_3$ for some $\mu$.
Therefore,
 \begin{align*}
  &\con{\cnv{P}}{\frac{\braw{\rho}}{\tr{}{\braw{\rho}}}}
  \Rightarrow \mu_0 \Rightarrow \mu_1 \Rightarrow \mu \mbox{ and}\\
  &\mu  (\strg)^\dagger 
  p_0 \bullet \con{\cnv{P_0}}{\frac{\braw{\rho_0}}
  {\tr{}{\braw{\rho_0}}}}
  +
  p_1 \bullet
  \con{\cnv{P_1}}{\frac{\braw{\rho_1}}{\tr{}{\braw{\rho_1}}}}
 \end{align*}
hold.
\end{proof}

\begin{lem}
 \label{symqccs:csimscon_point}
 If
 $\Proc \times \D(\H) \ni
 \con{\cnv{P}}{\frac{\braw{\rho}}{\tr{}{\braw{\rho}}}}
 \xrightarrow{\alpha}
 \mu$ holds and $\mu$ is a point distribution, then
 $\mathcal{P} \times \mathcal{S} \ni 
 \con{P}{\rho} \xrightarrow{\alpha} \con{P'}{\rho'}$ and
 $\mu (\strg)^\dagger
 1 \bullet \con{\cnv{P'}}{\frac{\braw{\rho'}}{\tr{}{\braw{\rho'}}}}$
 for some $\con{P'}{\rho'}$.
\end{lem}
\begin{proof}
%%
 ($\alpha$ is $\sndq{c}{q}$) There exist a qCCS's
evaluation context $D[\_]$
that does not restrict $\mathsf{c}$ and process $\tilde P \in
\Proc$ such that
$\cnv{P} = D[\sndq{c}{q}.\tilde P]$. There exist
an evaluation context $C[\_]$ of simplified processes not 
restricting $\mathsf{c}$ and
$P_0 \in \mathcal{P}$ such that 
$D[\_] = \cnv{C}[\_]$ and $\tilde P = \cnv{P_0}$.
Therefore, $\cnv{P} = \cnv{C[\sndq{c}{q}.P_0]}$ holds.
By Proposition \ref{symqccs:cnvinj}, $P = C[\sndq{c}{q}.P_0]$ holds.
We then have $\con{P}{\rho} \xrightarrow{\sndq{c}{q}} 
\con{C[P_0]}{\rho}$.
We also have
\[
 \con{\cnv{P}}{\frac{\braw{\rho}}{\tr{}{\braw{\rho}}}}
 \xrightarrow{\sndq{c}{q}}
 \con{D[\tilde P]}{\frac{\braw{\rho}}{\tr{}{\braw{\rho}}}}
 =
 \con{\cnv{C[P_0]}}{\frac{\braw{\rho}}{\tr{}{\braw{\rho}}}}
\]
%%
 ($\alpha$ is $\rcvq{c}{q}$) This case is similar to the above case.\\
%%
 ($\alpha$ is $\tau$ caused by application of a TPCP map or
communication) These cases are
also similar to the case of $\sndq{c}{q}$.\\
%%
($\alpha$ is $\tau$ caused by measurement) We assume 
the result of the measurement is $1$ with probability $1$.
The argument of the other case is similar. We omit the
similar argument as that in the case of $\sndq{c}{q}$.
We have
\begin{itemize}
 \item $P = C[\measure{b}{P_0}]$, 
 \item $\con{\cnv{P}}{\frac{\braw{\rho}}{\tr{}{\braw{\rho}}}}
       \xrightarrow{\tau}
       \con{D[\myif{1 = 1}{\tilde P}]}
       {\frac{\ket{1}\bra{1}_b\braw{\rho}\ket{1}\bra{1}_b}
       {\tr{}{\ket{1}\bra{1}_b\braw{\rho}}}}$
 \item $\con{D[\myif{1 = 1}{\tilde P}]}
       {\frac{\ket{1}\bra{1}_b\braw{\rho}\ket{1}\bra{1}_b}
       {\tr{}{\ket{1}\bra{1}_b\braw{\rho}}}}
       (\strg)^\dagger
       \con{D[\tilde P]}
       {\frac{\ket{1}\bra{1}_b\braw{\rho}\ket{1}\bra{1}_b}
       {\tr{}{\ket{1}\bra{1}_b\braw{\rho}}}}$, \mbox{ and}
 \item $D[\tilde P] = \cnv{C[P_0]}$
\end{itemize}
for some $C[\_], b$, $P_0$, $D[\_]$, and $\tilde P$.
\end{proof}
\begin{lem}
 \label{symqccs:csimscon_notpoint}
 If
 $\con{\cnv{P}}{\frac{\braw{\rho}}{\tr{}{\braw{\rho}}}}
 \xrightarrow{\alpha}
 \mu$ holds and $\mu$ is not a point distribution, then
\begin{enumerate}
 \item $\alpha$ is $\tau$,
 \item $\con{P}{\rho} = \con{C[\measure{b}{P'}]}{\rho}$
       for some evaluation context $C[\_]$, qubit variable $b$, process
       $P'$,
 \item $\con{P}{\rho} 
       \xrightarrow{\tau}
       \con{C[\discard{\qv{P'}}]}{\rho_1}
       \defeq \con{P_1}{\rho_1}$,
 \item $\con{P}{\rho} 
       \xrightarrow{\tau}
       \con{C[P']}{\rho_2} \defeq \con{P_2}{\rho_2}$, and
 \item $\mu (\strg)^\dagger
       \frac{\tr{}{\braw{\rho_1}}}{\tr{}{\braw{\rho}}}
       \bullet 
       \con{\cnv{P_1}}
           {\frac{\braw{\rho_1}}{\tr{}{\braw{\rho_1}}}} + 
       \frac{\tr{}{\braw{\rho_2}}}{\tr{}{\braw{\rho}}}
       \bullet 
       \con{\cnv{P_2}}
           {\frac{\braw{\rho_2}}{\tr{}{\braw{\rho_2}}}}$
\end{enumerate}
\end{lem}
\begin{proof}
 Since $\mu$ is not a point distribution, 
 the transition is caused by measurement (1).
 Therefore, we have 
 $P = C[\measure{b}{P'}]$ for some evaluation context
 $C[\_]$, qubit variable $b$, and process $P'$ (2).
 We have (3) and (4) immediately.
 We also have
\begin{align*}
 \mu =
 &\frac{\tr{}{\braw{\rho_1}}}{\tr{}{\braw{\rho}}}
 \bullet \con{\cnv{C}[\myif{0 = 1}{\cnv{P'}}]}
              {\frac{\braw{\rho_1}}{\tr{}{\braw{\rho_1}}}}\\
 + 
 &\frac{\tr{}{\braw{\rho_1}}}{\tr{}{\braw{\rho}}}
 \bullet \con{\cnv{C}[\myif{1 = 1}{\cnv{P'}}]}
             {\frac{\braw{\rho_2}}{\tr{}{\braw{\rho_2}}}}\\
(\strg)^\dagger &
\frac{\tr{}{\braw{\rho_1}}}{\tr{}{\braw{\rho}}}
 \bullet \con{\cnv{P_1}}
              {\frac{\braw{\rho_1}}{\tr{}{\braw{\rho_1}}}}\\
 + 
 &\frac{\tr{}{\braw{\rho_2}}}{\tr{}{\braw{\rho}}}
 \bullet \con{\cnv{P_2}}
             {\frac{\braw{\rho_2}}{\tr{}{\braw{\rho_2}}}}.
\end{align*}
\end{proof}

\subsection{The Correspondence}
The following theorem states the soundness of Verifier1.
\begin{thm}
\label{symqccs:correspondence}
 If Verifier1 returns $\mathit{true}$ with the input $\con{P}{\rho},
 \con{Q}{\sigma} \in \mathcal{P} \times \mathcal{S}$ satisfying
 $\tr{}{\braw{\rho}}=\tr{}{\braw{\sigma}}=1$,
 and a set of valid equations $\mathit{eqs}$, then
 $\con{\cnv{P}}{\braw{\rho}}
 \approx \con{\cnv{Q}}{\braw{\sigma}}$ holds.
\end{thm}
\begin{proof}
We assume 
that Verifier1 uses a simplified algorithm without the step 2 in
which TPCP maps are performed without any transition.
The theorem is still proven to hold with the step 2 extending the
proof.

 Assume all equations in $\mathit{eqs}$ are valid. We define
\[
 \R_{\mathit{eqs}} :=
 \Set{(X, Y) |
 \begin{array}{l}
  \text{$X \strg \con{\cnv{P}}{\frac{\braw{\rho}}{\tr{}{\braw{\rho}}}},
   Y \strg \con{\cnv{Q}}{\frac{\braw{\sigma}}{\tr{}{\braw{\sigma}}}}$,
   and}\\
  \text{Verifier1 returns $\mathit{true}$ with 
 $\con{P}{\rho}$ and $\con{Q}{\sigma}$ using $\mathit{eqs}$.}
 \end{array}
 }.
\]
We then have $\con{\cnv{P}
 }{\braw{\rho}} \R_{\mathit{eqs}} \con{\cnv{Q}}{\braw{\sigma}}$ if
Verifier1 returns $\mathit{true}$ with the input $\con{P}{\rho},
 \con{Q}{\sigma} \in \C$, $\tr{}{\braw{\rho}}=\tr{}{\braw{\sigma}}=1$,
 and $\mathit{eqs}$. It is sufficient to show that $\R_{\mathit{eqs}}$
 is a weak bisimulation relation. Let
 $(X, Y)$ be
 an arbitrary element in $\R_{\mathit{eqs}}$. The condition of
 quantum variable is satisfied by the definition of $\cnv{}$.
 The condition of partial trace is checked as follows.
 \begin{align*}
  \tr{\qv{\cnv{P}}}{\frac{\braw{\rho}}{\tr{}{\braw{\rho}}}} 
  &= 
  \frac{1}{\tr{}{\braw{\rho}}}\braw{\mathtt{Tr}[\qv{P}](\rho)}
  && (\mbox{by the definition of } \braw{})\\
  &= \frac{1}{\tr{}{\braw{\sigma}}}\braw{\mathtt{Tr}[\qv{Q}](\sigma)}
  && (\mbox{by the validity of } \mathit{eqs})\\
  &= \tr{\qv{\cnv{Q}}}{\frac{\braw{\sigma}}{\tr{}{\braw{\sigma}}}} 
  && (\mbox{by the definition of } \braw{})
 \end{align*}
Next, we check the condition of simulation. Let $\E_{\tilde r}$ be
an arbitrary TPCP map acting on $\tilde r \subseteq \sfqv - \qv{P}$.
Assume $X \xrightarrow{\alpha} \mu$.
By strong bisimulation,
$\con{\cnv{P}}{\E_{\tilde r}(\frac{\braw{\rho}}{\tr{}{\braw{\rho}}})}
\xrightarrow{\alpha} \mu'$ and $\mu (\strg)^\dagger \mu'$ hold.
\\
{\bf (Case 1)} Assume $\mu'$ is a point distribution.
 By lemma \ref{symqccs:csimscon_point},
 $\con{P}{\op{\bar \E}{\tilde r}(\rho)} \xrightarrow{\alpha}
 \con{P'}{\rho'}$ interpreting $\bar \E$ as $\E_{\tilde r}$, 
 and $\mu' (\strg)^\dagger
 \con{\cnv{P'}}{\frac{\braw{\rho'}}{\tr{}{\braw{\rho'}}}}$
 hold for some $\con{P'}{\rho'}$. 
Since Verifier1 returns $\mathit{true}$, 
 there exists $\con{Q'}{\sigma'}$ such that 
$\con{Q}{\op{\bar \E}{\tilde
 r}(\sigma)} \weak{\hat \alpha} \con{Q'}{\sigma'}$ holds and
 Verifier1
 returns $\mathit{true}$ with $\con{P'}{\rho'}$. This implies 
 $\tr{}{\braw{\sigma'}} = \tr{}{\braw{\rho'}} = 
\tr{}{\braw{\op{\bar \E}{\tilde r}(\rho)}} = \tr{}{\braw{\bar \E[\tilde
 r](\sigma)}}$. Now, we can apply Lemma \ref{symqccs:weak}.
 We have 
\[
 \con{\cnv{Q}}{\E_{\tilde r}
               (\frac{\braw{\sigma}}{\tr{}{\braw{\sigma}}})} 
 \Rightarrow
 \xrightarrow{\hat \alpha} 
 \Rightarrow 
 \nu'
 (\strg)^\dagger
 1 \bullet
 \con{\cnv{Q'}}{\frac{\braw{\sigma'}}{\tr{}{\braw{\sigma'}}}}
\]
 for some $\nu$.
 Next, by strong bisimulation,
 $Y 
  \Rightarrow 
  \xrightarrow{\hat \alpha} 
  \Rightarrow 
  \nu
 $ and
 $\nu' (\strg)^\dagger \nu$.
 We then have
\begin{align*}
&\mu 
 (\strg)^\dagger 
 \con{\cnv{P'}}{\frac{\braw{\rho'}}{\tr{}{\braw{\rho'}}}}
 \mbox{  and  }
\nu 
 (\strg)^\dagger 
 \con{\cnv{Q'}}{\frac{\braw{\sigma'}}{\tr{}{\braw{\sigma'}}}}.
\end{align*}
By the definition of $(\cdot)^\dagger$, $\mu$ can be
written as $\sum_i p_i X'_i$ and
$X'_i \strg  \con{\cnv{P'}}{\frac{\braw{\rho'}}{\tr{}{\braw{\rho'}}}}$
for all $i$.
Similarly, $\nu$ can be written as $\sum_j q_j Y'_j$ and
$Y'_j \strg  \con{\cnv{Q'}}
                 {\frac{\braw{\sigma'}}{\tr{}{\braw{\sigma'}}}}$
for all $j$.
Since Verifier1 returns $\mathit{true}$ with $\con{P'}{\rho'}$ and 
$\con{Q'}{\sigma'}$, $X_i \R_{\mathit{eqs}} Y_j$ for all $i, j$.
 holds. Therefore, $\sum_{i,j} p_i q_j X_i \R_{\mathit{eqs}}^\dagger
 \sum_{i,j} p_i q_j Y_j$ holds. This is equivalent to
 $\mu \R_{\mathit{eqs}}^\dagger \nu$.
 \\
{\bf (Case 2)}
 Assume $\mu'$ is not a point distribution. 
 By lemma \ref{symqccs:csimscon_notpoint},
 $\con{P}{\op{\bar \E}{\tilde r}(\rho)} 
  \xrightarrow{\tau}
  \con{P_1}{\rho_1}$ and
 $\con{P}{\op{\bar \E}{\tilde r}(\rho)}
  \xrightarrow{\tau}
  \con{P_2}{\rho_2}$ interpreting $\bar \E$ as
 $\E_{\tilde r}$, and
\[
 \mu' (\strg)^\dagger 
 \frac{\tr{}{\braw{\rho_1}}}{\tr{}{\braw{\rho}}}
 \bullet 
 \con{\cnv{P_1}}
 {\frac{\braw{\rho_1}}{\tr{}{\braw{\rho_1}}}} + 
 \frac{\tr{}{\braw{\rho_2}}}{\tr{}{\braw{\rho}}}
 \bullet 
 \con{\cnv{P_2}}
 {\frac{\braw{\rho_2}}{\tr{}{\braw{\rho_2}}}}
\]
hold.
 Since Verifier1 returns $\mathit{true}$, there exists 
 configurations $\con{Q_1}{\sigma_1}$ and $\con{Q_2}{\sigma_2}$ 
 such that
 \begin{itemize}
  \item $\con{Q}{\op{\bar \E}{\tilde r}(\sigma)} \xrightarrow{\tau\ast} 
	\con{D[\measure{b}{Q'}]}{\sigma'}$,
  \item $\con{D[\measure{b}{Q'}]}{\sigma'} \xrightarrow{\tau}
	\con{D[Q']}{\mathtt{proj1}[b](\sigma')} \xrightarrow{\tau\ast}
	\con{Q_1}{\sigma_1}$,
  \item $\con{D[\measure{b}{Q'}]}{\sigma'} \xrightarrow{\tau}
	\con{D[\mathtt{discard}(\qv{Q'})]}{\mathtt{proj0}[b](\sigma')}\\
	\xrightarrow{\tau\ast} \con{Q_2}{\sigma_2}$,
 \end{itemize}
 hold for some $D, b, Q'$ and $\sigma'$, and
\begin{itemize}
 \item Verifier1 returns $\mathit{true}$ with $\con{P_1}{\rho_1}$,
       $\con{Q_1}{\sigma_1}$, and $\mathit{eqs}$.
 \item Verifier1 returns $\mathit{true}$ with $\con{P_2}{\rho_2}$,
       $\con{Q_2}{\sigma_2}$, and $\mathit{eqs}$.
\end{itemize}
 Moreover, $\tr{}{\braw{\sigma}} = \tr{}{\braw{\sigma'}}$ holds; 
 Otherwise, $\tr{}{\braw{\sigma}} > \tr{}{\braw{\sigma'}}$ holds.
 Since Verifier1 returns $\mathit{true}$ with two pairs
 $\con{P}{\rho}$ and $\con{Q}{\sigma})$,
 the numbers of
 $\mathtt{proj}i$'s occurring in $\rho$ and $\sigma$ are equal
 (Section \ref{symqccs:algorithmforbisim}).
 Let the number be $N$.
 In $\rho_1$, there are $N + 1$ $\mathtt{proj}i$'s.
 By the transition, there are more than $N + 2$ $\mathtt{proj}i$'s or
 $N + 2$ $\mathtt{proj}i$'s in $\sigma_1$. This contradicts that 
 Verifier1 returned $\mathit{true}$ with 
 $\con{P_1}{\rho_1}$ and $\con{Q_1}{\sigma_1}$, and thus
 the numbers of $\mathtt{proj}i$'s in $\rho_1$ and $\sigma_1$ are
 equal.

 Next, by the validity of $\mathit{eqs}$,
 $\tr{}{\braw{\rho_1}} = \tr{}{\braw{\sigma_1}}$ and
 $\tr{}{\braw{\rho_2}} = \tr{}{\braw{\sigma_2}}$ hold.
 Thus we have $\tr{}{\braw{\sigma_1}} + \tr{}{\braw{\sigma_2}} = 
 \tr{}{\braw{\rho_1}} + \tr{}{\braw{\rho_2}} = 
 \tr{}{\braw{\rho}} = \tr{}{\braw{\sigma}} = \tr{}{\braw{\sigma'}}$.
 Now, we can apply the Lemma \ref{symqccs:prfofweakmeas} to have
\begin{align*}
&\con{\cnv{Q}}{\frac{\braw{\sigma}}{\tr{}{\braw{\sigma}}}}
 \Rightarrow
 \nu'\\
&\nu'
 (\strg)^\dagger
 \frac{\tr{}{\braw{\rho_1}}}{\tr{}{\braw{\rho}}} \bullet
 \con{\cnv{Q_1}}{\frac{\braw{\sigma_1}}{\tr{}{\braw{\sigma_1}}}}
 +\frac{\tr{}{\braw{\rho_2}}}{\tr{}{\braw{\rho}}} \bullet
 \con{\cnv{Q_2}}{\frac{\braw{\sigma_2}}{\tr{}{\braw{\sigma_2}}}}
\end{align*}
for some $\nu'$.
By strong bisimulation, 
$Y \Rightarrow \nu$ and $\nu' (\strg)^\dagger \nu$ hold for some
$\nu$.  We then have
\begin{align*}
 &\mu 
 (\strg)^\dagger 
 \frac{\tr{}{\braw{\rho_1}}}{\tr{}{\braw{\rho}}}
 \bullet 
 \con{\cnv{P_1}}
 {\frac{\braw{\rho_1}}{\tr{}{\braw{\rho_1}}}} + 
 \frac{\tr{}{\braw{\rho_1}}}{\tr{}{\braw{\rho}}}
 \bullet 
 \con{\cnv{P_2}}
 {\frac{\braw{\rho_2}}{\tr{}{\braw{\rho_2}}}} \defeq U\\
 &\nu
 (\strg)^\dagger
 \frac{\tr{}{\braw{\rho_1}}}{\tr{}{\braw{\rho}}} \bullet
 \con{\cnv{Q_1}}{\frac{\braw{\sigma_1}}{\tr{}{\braw{\sigma_1}}}}
 +\frac{\tr{}{\braw{\rho_2}}}{\tr{}{\braw{\rho}}} \bullet
 \con{\cnv{Q_2}}{\frac{\braw{\sigma_2}}{\tr{}{\braw{\sigma_2}}}}
 \defeq V
\end{align*}
Let $A, B, C$ and $D$ be
$\con{\cnv{P_1}}{\frac{\braw{\rho_1}}{\tr{}{\braw{\rho_1}}}}$,
$\con{\cnv{P_2}}{\frac{\braw{\rho_2}}{\tr{}{\braw{\rho_2}}}}$,
$\con{\cnv{Q_1}}{\frac{\braw{\sigma_1}}{\tr{}{\braw{\sigma_1}}}}$,
and
$\con{\cnv{Q_2}}{\frac{\braw{\sigma_2}}{\tr{}{\braw{\sigma_2}}}}$,
respectively.
By the definition of $(\cdot)^\dagger$, $\mu$ is written as\\
$\sum_{i \in I}p_i X_i + \sum_{j \in J}q_j X_j$ with 
$I \cap J = \emptyset$, and
$U$ is written as $\sum_{i \in I}p_i A + \sum_{j \in J}q_j B$, and
$X_i \strg A$ for all $i \in I$, and
$X_j \strg B$ for all $j \in J$. 
$\sum_{i \in I} p_i = \frac{\tr{}{\braw{\rho_1}}}{\tr{}{\braw{\rho}}}$
 and
$\sum_{j \in J} q_j =
\frac{\tr{}{\braw{\rho_2}}}{\tr{}{\braw{\rho}}}$ hold.
Similarly,
$\nu$ is written as\\
$\sum_{k \in K}p'_k Y_k + \sum_{l \in L}q'_l Y_l$ with 
$K \cap L = \emptyset$, and
$V$ is written as $\sum_{k \in K}p'_k C + \sum_{l \in L}q'_l D$, and
$Y_k \strg C$ for all $k \in K$, and
$Y_l \strg D$ for all $l \in L$. 
$\sum_{k \in K} p'_k = \frac{\tr{}{\braw{\rho_1}}}{\tr{}{\braw{\rho}}}$
 and
$\sum_{l \in L} q'_l =
\frac{\tr{}{\braw{\rho_2}}}{\tr{}{\braw{\rho}}}$ hold.
Since Verifier1 returns $\mathit{true}$ with \\
$(\con{P_m}{\rho_m}, \con{Q_m}{\sigma_m})\,(m = 1,2)$,
$X_i \R_{\mathit{eqs}} Y_k$ for all $i \in I, k \in K$, and
$X_j \R_{\mathit{eqs}} Y_l$ for all $j \in J, l \in L$.
We then have the conclusion $\mu (\R_{\mathit{eqs}})^\dagger \nu$
observing 
\begin{align*}
 &\mu = \frac{\tr{}{\braw{\rho}}}{\tr{}{\braw{\rho_1}}}
         \sum_{i,k}p_i p'_k X_i +
        \frac{\tr{}{\braw{\rho}}}{\tr{}{\braw{\rho_2}}}
         \sum_{j,l}q_j q'_l X_j \mbox{ and }\\
 &\nu = \frac{\tr{}{\braw{\rho}}}{\tr{}{\braw{\rho_1}}}
         \sum_{i,k}p_i p'_k Y_k +
        \frac{\tr{}{\braw{\rho}}}{\tr{}{\braw{\rho_2}}}
         \sum_{j,l}q_j q'_l Y_l.
\end{align*}
By definition of $\R$, $\R$ is symmetric and thus $\R^{-1}$ also
satisfies the conditions.
\end{proof}

\section{Discussion}
\subsection{On Completeness}
Let us consider an ``ideal'' verifier that can test
equality of partial traces perfectly.
It takes two configurations
$\con{P}{\rho}$ and $\con{Q}{\sigma}$, but does not take a
set of user-defined equations. In the step 4 of the algorithm,
it goes to the next step 
if and only if $\braw{\ptrtt{\qv{P}}{\rho}} =
\braw{\ptrtt{\qv{Q}}{\sigma}}$.
Let us then consider the following statement.
For $\con{P}{\rho},\con{Q}{\sigma} \in \mathcal{P} \times \mathcal{S}$,
if $\tr{}{\braw{\rho}}=\tr{}{\braw{\sigma}}=1$ and
$\con{\cnv{P}}{\braw{\rho}} \approx
\con{\cnv{Q}}{\braw{\sigma}}$ hold, then
the ideal verifier returns $\mathit{true}$ with
the input $\con{P}{\rho},\con{Q}{\sigma}$.
We call this
\emph{the completeness of nondeterministic qCCS with respect to qCCS}.
In fact, this statement is not true.
A counter example is as follows.
\begin{align*}
 \con{P}{\braw{\rho}} &\defeq \con{\measure{b}{\discard{b,q}}}
 {\ket{\psi}\bra{\psi}_b \otimes \ket{0}\bra{0}_q \otimes \rho^E} \mbox{ and}
\\
 \con{Q}{\braw{\sigma}} &\defeq \con{\discard{b,q}}
 {\ket{\psi}\bra{\psi}_b \otimes \ket{0}\bra{0}_q \otimes \rho^E},
 \mbox{ where}\\
 \ket{\psi} &= \sqrt{\frac{1}{3}}\ket{0} + \sqrt{\frac{2}{3}}\ket{1}.
\end{align*}
The two transitions of $\con{P}{\braw{\rho}}$ are
\begin{align*}
 \con{P}{\braw{\rho}} &\xrightarrow{\tau} \con{\discard{b,q}}
 {\frac{1}{3}\ket{0}\bra{0}_b \otimes \ket{0}\bra{0}_q \otimes \rho^E}
 \mbox{ and}\\
 \con{P}{\braw{\rho}} &\xrightarrow{\tau} \con{\discard{b,q}}
 {\frac{2}{3}\ket{1}\bra{1}_b \otimes \ket{0}\bra{0}_q \otimes \rho^E}
\end{align*}
but partial traces of them ($\frac{1}{3}\rho^E$ and 
$\frac{2}{3}\rho^E$) are not equal to that of $\con{Q}{\braw{\sigma}}$ (
namely $\rho^E$). In our simplified formal framework,
two configurations $\con{P_0}{\braw{\rho_0}}$ and $\con{Q_0}{\braw{\sigma_0}}$
with $\tr{}{\rho_0} \neq \tr{}{\sigma_0}$ are always separated even if
$\con{P_0}{\frac{\braw{\rho_0}}{\tr{}{\braw{\rho_0}}}}$ and
$\con{Q_0}{\frac{\braw{\sigma_0}}{\tr{}{\braw{\sigma_0}}}}$ are identified.
To identify such configurations, we must withdraw the simplification of 
operational semantics.

However, there seem to be a number of cases where we can assume that
processes behave differently according to the result of quantum
measurements.
In general, we can formalize a QKD protocol of the form 
\[
 ... \mathtt{abort\_flag[...,}b\mathtt{,...].}\measure{b}{P'} ...,
\]
where an operator $\mathtt{abort\_flag}$ calculates the number of errors
in check bits and sets a bit $b$ representing whether to abort the
protocol. Alice and Bob continue to communicate
only if the protocol has not aborted, which the outsider can
recognize. 

Besides, with our criteria discussed in Section \ref{symqccs:criteria},
we expect that there is a transition of $R$
with a label $\sndq{c}{q}$ or $\rcvq{c}{q}$, when a process
$\measure{b}{R}$ is considered.
Only for processes satisfying the condition, it is still possible that
the completeness holds. It needs to be discussed precisely.

% However, it seems not to be usual to consider a process like $P$
% which behaves equivalently whichever result of a measurement 
% it obtains. 
% With our criteria, which we discussed in Section \ref{symqccs:criteria},
% we expect that there is a transition of $R$
% with a label $\sndq{c}{q}$ or $\rcvq{c}{q}$, when a process
% $\measure{b}{R}$ is considered.
% Only for processes satisfying the condition, it is still possible that
% the statement be true. It needs to be discussed precisely.

% Before simplifying the operational semantics, 
% quantum states are always normalized (namely, trace is $1$) and thus
% $\con{P}{\braw{\rho}} \approx \con{Q}{\braw{\sigma}}$ holds.
% The key point is that
% \begin{align*}
%  \con{P}{\braw{\rho}} \xrightarrow{\tau}& \frac{1}{3}\con{\discard{b,q}}
%  {\ket{0}\bra{0}_b \otimes \ket{0}\bra{0}_q \otimes \rho^E}\\
%  &\boxplus \frac{2}{3}\con{\discard{b,q}}
%  {\ket{1}\bra{1}_b \otimes \ket{0}\bra{0}_q \otimes \rho^E} \defeq \mu,\\
%  \con{\discard{b,q}}{&\ket{0}\bra{0}_b \otimes \ket{0}\bra{0}_q \otimes \rho^E}
%  \approx \con{Q}{\braw{\sigma}}, \mbox{ and}\\
%  \con{\discard{b,q}}{&\ket{1}\bra{1}_b \otimes \ket{0}\bra{0}_q \otimes \rho^E}
%  \approx \con{Q}{\braw{\sigma}}
% \end{align*}
% hold and implies $\mu \approx^\dagger 1 \bullet \con{Q}{\braw{\sigma}}$.
