%\usepackage{graphics}
\chapter{Approximate Bisimulation for Quantum Processes}
\label{prob_bisim}
Two notions of approximate bisimulation
are defined in our simplified formal framework named 
nondeterministic qCCS, which is 
defined in the previous chapter, namely, the 
relations are on $\C = \mathcal{P} \times \Delta(\H)$
with simplified operational semantics.
The relation $\sim_{\zeta, \eta}$ is defined in Section \ref{par} and
the relation $\sim$ is in Section \ref{neg}.
Before the definitions, we introduce some preliminaries.

\section{Preliminaries}
\subsection{Negligible Functions}
\begin{defi}
 A function $f:\mathbb{N}\rightarrow [0, 1]$ is \emph{negligible} if and
 only if for all polynomial $p(\cdot)$, there exists a natural number
 $N$ such that $f(n) \le \frac{1}{p(n)}$ holds
 for all $n \ge N$. $f$ is
 \emph{non-negligible} if $f$ is not negligible.
\end{defi}
\begin{rem}
 A function $\frac{f}{g}(n) \defeq \frac{f(n)}{g(n)}$ can be
 non-negligible even if $f$ is negligible and $g$ is non-negligible.
 An example is $f(n) = \frac{1}{2^n}$ and
\begin{displaymath}
 g(n) = \left\{
\begin{array}{l}
\frac{1}{2^n}~~~(n \mbox{ is even})\\
\frac{1}{n^2}~~~(\mbox{otherwise}).
\end{array}
\right.
\end{displaymath}
 In this thesis, we say $g$ is \emph{greater than negligible} if the
 function
 $\frac{f}{g}$ is negligible for all
 negligible function $f$.
\end{rem}

\begin{prop}
\label{par:propneg}
 If $f$ and $g$ are negligible, then $f + g$ and $cf$ is negligible for
 all $c \ge 0$.
\end{prop}

\subsection{Trace Distance of Probability-Weighted Quantum States}
Trace distance is a metric on a set of
linear operators.
To compare quantum states, trace distance on $\D(\H)$
is usually considered \cite[Chapter
9]{NielsenChuang-Kimura2004}.
Since we consider probability-weighted quantum states, 
we consider trace distance on $\Delta(\H)$, and discuss an
interpretation of it with respect to our transition system.

Let $\rho, \sigma \in \Delta(\H)$. $\rho - \sigma$ is an Hermitian
operator. Let $\sqrt{A}$ be $\sum_i \sqrt{\lambda_i}P_i$ for
an Hermitian operator $A$ with the spectrum decomposition
$\sum_i \lambda_i P_i$. Let $|A|$ be $\sqrt{A^\dagger A}$.

\begin{defi}
 Trace distance $d: \Delta(\H) \times \Delta(\H) \rightarrow [0,1]$ is
 defined as 
\[
 d(\rho, \sigma) = \frac{1}{2}\mathrm{tr}|\rho - \sigma|.
\]
\end{defi}
If $\tr{}{\rho} = \tr{}{\sigma} = 1$, trace distance can be thought as
a generalization of Kolmogorov distance.
Indeed, if $\rho$ and $\sigma$ are diagonal with respect to
orthonormal basis
$\{\ket{i}\}_{i}$, then $\rho = \sum_i p_i \ket{i}\bra{i}$,
$\rho = \sum_i q_i \ket{i}\bra{i}$, and $d(\rho,\sigma) = 
\frac{1}{2}\sum_i |p_i - q_i|$ hold for some unique $p_i$'s and $q_i$'s
satisfying $\sum_i p_i = \sum_i q_i = 1$.
We introduce some useful properties of trace distance as follows.

\begin{prop}
\label{par:trdismax}
 If $\tr{}{\rho} = \tr{}{\sigma} = 1$ holds, then 
\[
 d(\rho, \sigma) = 
 \max\{\tr{}{P\rho} - \tr{}{P\sigma}\,|\, P \mbox{ is a projector.}\}
\]
holds.
\end{prop}

\begin{prop}
\label{par:tpcpnonincrease}
If $\tr{}{\rho} = \tr{}{\sigma} = 1$ holds, then
 $d(\E(\rho), \E(\sigma))
\le d(\rho, \sigma)$ holds for all TPCP map $\E$.
\end{prop}
Proposition \ref{par:tpcpnonincrease}
can be extended to trace non-increasing cases.
In the proof, we use the fact that $\mathrm{tr}|\cdot|$ is a
norm on a set of linear operators.

\begin{prop}
\label{par:positivenonincrease}
 $d(\E(\rho), \E(\sigma)) \le d(\rho, \sigma)$ for all
 trace non-increasing positive map $\E$.
\end{prop}

\begin{proof}
 Let $\sum_i \lambda_i \ket{i}\bra{i}$ be an arbitrary eigenvalue
 decomposition of $\rho - \sigma$. 
 We obtain the proposition by the following calculation.
 \begin{align*}
  \mathrm{tr}|\E(\rho - \sigma)|
  &= \mathrm{tr}|\sum_i \lambda_i \E(\ket{i}\bra{i})| \\
  &\le \sum_i |\lambda_i| \cdot \mathrm{tr}|\E(\ket{i}\bra{i})| 
  &&\mbox{(by the axiom of trace norm)}\\
  &\le \sum_i |\lambda_i| \cdot \mathrm{tr}(\ket{i}\bra{i})
  &&\mbox{(} \E \mbox{ is trace non-increasing and positive.)}\\
  &= \sum_i |\lambda_i| = \mathrm{tr}|\rho - \sigma|
 \end{align*}
\end{proof}

For a configuration $\con{P}{\rho} \in \C$, $\tr{}{\rho}$ is 
interpreted
as the probability to reach $\con{P}{\rho}$, and the quantum state  
is $\frac{\rho}{\tr{}{\rho}}$.
The next proposition 
gives a way to interpret that $d(\rho, \sigma)$ is 
small for two configurations $\con{P}{\rho}, \con{Q}{\sigma} \in \C$.
\begin{prop}
\label{par:trdisproperty}
 Let $\rho, \sigma : \mathbb{N} \rightarrow \Delta(\H)$ be functions
 and regard $d(\rho, \sigma):\mathbb{N} \rightarrow [0,1]$ 
 as a function. $d(\rho, \sigma)$
 is negligible iff $|\tr{}{\rho} - \tr{}{\sigma}|$ and
 $\tr{}{\rho}\cdot d(\frac{\rho}{\tr{}{\rho}},
 \frac{\sigma}{\tr{}{\sigma}})$ are negligible.
\end{prop}

\begin{proof}
 $\mathrm{(\Rightarrow)}$ As $\tr{}{\cdot}$ is a TPCP map, $d(\rho,
 \sigma) \ge d(\tr{}{\rho}, \tr{}{\sigma}) = \frac{1}{2}|\tr{}{\rho} - 
 \tr{}{\sigma}|$. $|\tr{}{\rho} -  \tr{}{\sigma}|$ is negligible by
 Proposition \ref{par:propneg}.
 $\tr{}{\rho}\cdot d(\frac{\rho}{\tr{}{\rho}}, \frac{\sigma}{\tr{}{\sigma}})$
 is shown to be negligible by the following calculation.
 \begin{align*}
  \tr{}{\rho}\cdot d(\frac{\rho}{\tr{}{\rho}}, \frac{\sigma}{\tr{}{\sigma}})
  &\le \tr{}{\rho}\cdot (d(\frac{\rho}{\tr{}{\rho}},
  \frac{\sigma}{\tr{}{\rho}}) + d(\frac{\sigma}{\tr{}{\rho}},
  \frac{\sigma}{\tr{}{\sigma}}))\\
  &= d(\rho, \sigma) + |\tr{}{\rho} -
  \tr{}{\sigma}| \cdot \mathrm{tr}|\frac{\sigma}{\tr{}{\sigma}}|\\
  &= d(\rho, \sigma) + |\tr{}{\rho} - \tr{}{\sigma}|
 \end{align*}
$\mathrm{(\Leftarrow)}$ By triangle inequality, we have
$d(\rho, \frac{\tr{}{\rho}}{\tr{}{\sigma}}\sigma) + 
 d(\frac{\tr{}{\rho}}{\tr{}{\sigma}}\sigma, \sigma)
 \ge d(\rho, \sigma)$. This is equivalent to
$\tr{}{\rho} \cdot d(\frac{\rho}{\tr{}{\rho}},
 \frac{\sigma}{\tr{}{\sigma}}) + 
 d(\frac{\tr{}{\rho}}{\tr{}{\sigma}}\sigma, \sigma)
 \ge d(\rho, \sigma)$. The left-hand side is shown to be negligible
 by the calculation $d(\frac{\tr{}{\rho}}{\tr{}{\sigma}}\sigma, \sigma)
 = 
|\tr{}{\rho} -
  \tr{}{\sigma}| \cdot \mathrm{tr}|\frac{\sigma}{\tr{}{\sigma}}|$.
\end{proof}

For $\con{P}{\rho}, \con{Q}{\sigma}$, assume
$d(\rho, \sigma)$ is negligible. It is equivalent to that
$|\tr{}{\rho} - \tr{}{\sigma}|$ and $\tr{}{\rho} \cdot
d(\frac{\rho}{\tr{}{\rho}}, \frac{\sigma}{\tr{}{\sigma}})$ are
negligible. By Proposition \ref{par:trdismax}, 
we have that $\tr{}{\rho} \cdot \tr{}{P\frac{\rho}{\tr{}{\rho}}} - 
\tr{}{\rho} \cdot \tr{}{P\frac{\sigma}{\tr{}{\sigma}}}$ is negligible
for all projector $P$. As $|\tr{}{\rho}$ - $\tr{}{\sigma}|$ is
negligible, we have that
$\tr{}{\rho} \cdot \tr{}{P\frac{\rho}{\tr{}{\rho}}} - 
\tr{}{\sigma} \cdot \tr{}{P\frac{\sigma}{\tr{}{\sigma}}}$ is negligible.
$\tr{}{\rho} \cdot \tr{}{P\frac{\rho}{\tr{}{\rho}}}$ is the joint
probability that a process reaches $\con{P}{\rho}$ and observes
the measurement result corresponding to the projector $P$.
Therefore, that $d(\tr{}{\rho}, \tr{}{\sigma})$ is negligible implies
that the difference of the joint probabilities is negligible for an
arbitrary measurement.

\section{Approximate Bisimulation up to Parameters}
\label{par}
We define the first approximate bisimulation relation.
\begin{defi}
\label{par:defofzetaetarel}
 Let $0 \le \zeta, \eta \le 1$. A symmetric relation $\R \subseteq \C \times \C$
 is called an $(\zeta,\eta)$-bisimulation if for all
 $\con{P}{\rho},\con{Q}{\sigma}$,$\con{P}{\rho} \R \con{Q}{\sigma}$
 implies 
\begin{enumerate}
 \item $\qv{P}=\qv{Q} \defeq \tilde q$,
 \item $d(\tr{\tilde q}{\rho}, \tr{\tilde q}{\sigma}) \le \zeta$, and
 \item for all CP map $\E_{\tilde r}$ acting on 
       $\tilde r \subseteq \sfqv - \tilde q$,
       if $\con{P}{\E_{\tilde r}(\rho)} \xrightarrow{\alpha}
       \con{P'}{\rho'}$ and $\tr{}{\rho'} \ge \eta$ hold,
       then $\con{Q}{\E_{\tilde r}(\sigma)} \xrightarrow{\tau*}
       \xrightarrow{\hat \alpha}
       \xrightarrow{\tau*} \con{Q'}{\sigma'}$ and
       $\con{P'}{\rho'}\mathcal{R} \con{Q'}{\sigma'}$ hold for some
       $\con{Q'}{\sigma'}$
\end{enumerate}
We call the conditions 1 and 2 the static conditions, and
the condition 3 the simulation condition.
\end{defi}
Precisely in the condition 3, it is possible that $\E_{\tilde r}(\rho) = O
\notin \Delta(\H)$ holds for some $\E_{\tilde r}$. We exclude such
CP maps in our discussions.

\begin{defi}
\label{par:defofzetaeta}
 We define 
 \[
 \sim_{\zeta, \eta} := \{(\C, \D) \in \Con \times \Con~|~
 \C\R\D \mbox{ holds for some } (\zeta,\eta)\mbox{-bisimulation }
 \R\}.
 \]
 We say $\con{P}{\rho}$ and $\con{Q}{\sigma}$
 are $(\zeta, \eta)$-bisimilar if
 $\con{P}{\rho} \sim_{\zeta, \eta} \con{Q}{\sigma}$.
\end{defi}
The relation $\sim_{\zeta, \eta}$ has the similar properties
as the bisimulation relations defined in qCCS or other process 
calculi have \cite{Milner1999}.

\begin{lem}
\label{par:alsobisimulation}
$\sim_{\zeta,\eta}$ is a ($\zeta,\eta$)-bisimulation.
\end{lem}
\begin{proof}
By definition of $\sim_{\zeta, \eta}$, it is symmetric.
By $\con{P}{\rho}\sim_{\zeta, \eta}\con{Q}{\sigma}$,
there exists a $(\zeta, \eta)$-bisimulation
$\R$ satisfying $\con{P}{\rho}\R\con{Q}{\sigma}$.
The static conditions are easily checked.
Next, we have that for all CP map $\E_{\tilde r}$ that acts on 
$\tilde r \subseteq \sfqv - \qv{P}$, 
 if $\con{P}{\E_{\tilde r}(\rho)} \xrightarrow{\alpha} \con{P'}{\rho'}$
 and $\tr{}{\rho'} \ge \eta$ hold,
 then there exists
 $\con{Q'}{\sigma'}$ satisfying
 $\con{Q}{\E_{\tilde r}(\sigma)} \weak{\hat \alpha} \con{Q'}{\sigma'}$
 and 
 $\con{P'}{\rho'}\R\con{Q'}{\sigma'}$.
 This implies
 $\con{P'}{\rho'}\sim_{\zeta, \eta}\con{Q'}{\sigma'}$.
\end{proof}

\begin{lem}
\label{prob_bisim:parcoinduction}
 $\con{P}{\rho} \sim_{\zeta,\eta} \con{Q}{\sigma}$ if and
 only if 
\begin{enumerate}
 \item $\qv{P}=\qv{Q} \defeq \tilde q$,
 \item $d(\tr{\tilde q}{\rho}, \tr{\tilde q}{\sigma})  \le \zeta$,
 \item and for all CP map $\E_{\tilde r}$ acting on 
       $\tilde r \subseteq \sfqv - \tilde q$,
       \begin{itemize}
	\item
	     if $\con{P}{\E_{\tilde r}(\rho)} \xrightarrow{\alpha}
	     \con{P'}{\rho'}$ and $\tr{}{\rho'} \ge \eta$ hold,
	     then
	     $\con{Q}{\E_{\tilde r}(\sigma)} 
	     \xrightarrow{\tau*}
	     \xrightarrow{\hat \alpha}
	     \xrightarrow{\tau*} \con{Q'}{\sigma'}$ and
	     $\con{P'}{\rho'} \sim_{\zeta, \eta}
	     \con{Q'}{\sigma'}$ hold
	     for some  $\con{Q'}{\sigma'}$
	\item
	     if $\con{Q}{\E_{\tilde r}(\sigma)} \xrightarrow{\alpha}
	     \con{Q'}{\sigma'}$ and $\tr{}{\sigma'} \ge \eta$ hold,
	     then
	     $\con{P}{\E_{\tilde r}(\rho)} \xrightarrow{\tau*}
	     \xrightarrow{\hat \alpha}
	     \xrightarrow{\tau*} \con{P'}{\rho'}$ and
	     $\con{P'}{\rho'} \sim_{\zeta, \eta} \con{Q'}{\sigma'}$
	     hold for some
	     $\con{P'}{\rho'}$
       \end{itemize}
\end{enumerate}
\end{lem}
\begin{proof}
 ($\Rightarrow$) proven as the previous lemma.\\
 ($\Leftarrow$) We define $\hat \R\,:=\,\sim_{\zeta, \eta}\cup 
\{(\con{P}{\rho}, \con{Q}{\sigma} ) \} \cup \{(\con{Q}{\sigma},
 \con{P}{\rho}) \}$. $\hat \R$ is symmetric by the definition.
 It is sufficient to show $\hat
 \R$ is a ($\zeta, \eta$)-bisimulation relation. Let
 $(\con{P_0}{\rho_0},
 \con{Q_0}{\sigma_0})$
 be an arbitrary element of $\hat \R$.
 \begin{enumerate}
  \item  Suppose $(\con{P_0}{\rho_0}, \con{Q_0}{\sigma_0}) \in
	 \sim_{\zeta, \eta}$. Since $\sim_{\zeta, \eta}$
	 is a $(\zeta, \eta)$-bisimulation, the static
	 conditions are satisfied. Next, let
	 $\E_{\tilde r}$ be an arbitrary CP map acting on 	
	 $\tilde r \subseteq \sfqv - \qv{P_0}$, and assume
	 $\con{P_0}{\E_{\tilde r}(\rho_0)} 
	 \xrightarrow{\alpha}
	 \con{P'}{\rho'}$ and $\tr{}{\rho'} \ge \eta$ hold.
	 By the previous lemma,
	 we have
	 $\con{Q_0}{\E_{\tilde r}(\sigma_0)}
	 \xrightarrow{\tau*}
	 \xrightarrow{\hat \alpha}
	 \xrightarrow{\tau*} \con{Q'}{\sigma'}$ and
	 $\con{P'}{\rho'}\sim_{\zeta, \eta}
	 \con{Q'}{\sigma'}$
	 for some
	 $\con{Q'}{\sigma'}$.
	 This implies $\con{Q_0}{\E_{\tilde r}(\sigma_0)}
	 \xrightarrow{\tau*}
	 \xrightarrow{\hat \alpha}
	 \xrightarrow{\tau*} \con{Q'}{\sigma'}$ and
	 $\con{P'}{\rho'}\hat \R
	 \con{Q'}{\sigma'}$
	 for some $\con{Q'}{\sigma'}$ since 
	 $\sim_{\zeta, \eta} \subseteq \hat \R$.
  \item Suppose $(\con{P_0}{\rho_0},
	\con{Q_0}{\sigma_0}) = (\con{P}{\rho}, \con{Q}{\sigma})$.
	The static conditions are easily checked.
	The simulation condition holds by the assumption and
	$\sim_{\zeta, \eta} \subseteq \hat \R$.
  \item Suppose $(\con{P_0}{\rho_0},
	\con{Q_0}{\sigma_0}) = (\con{Q}{\sigma}, \con{P}{\rho})$.
	The proof is similar to the previous case.
 \end{enumerate}
\end{proof}

\begin{prop}
\label{par:sym-by-par}
 $\con{P||Q}{\rho} \sim_{0, 0} \con{Q||P}{\sigma}$.
\end{prop}
\begin{proof}
 This is proved by the definition of the transition rules, where
 (Left) and (Right) rules are symmetric.
\end{proof}

\begin{lem}
 If $\con{P}{\rho} \sim_{\zeta, \eta} \con{Q}{\sigma}$ and 
$\con{P}{\rho} \xrightarrow{\tau\ast} \con{P'}{\rho'}$ and
 $\tr{}{\rho'} \ge \eta$, then
$\con{Q}{\sigma} \xrightarrow{\tau\ast}
 \con{Q'}{\sigma'}$ and $\con{P'}{\rho'} \sim_{\zeta, \eta} \con{Q'}{\sigma'}$ hold
 for some $\con{Q'}{\sigma'}$.
\end{lem}
\begin{proof}
 Assume $\con{P}{\rho} \xrightarrow{\tau}^n \con{P'}{\rho'}$
 and let $\con{P_i}{\rho_i}$ be $i$-th configuration with $0 \le i \le
 n$. The case when $n = 0$ is trivial. Let $n > 0$.
 Since $\rho = \rho_0 \ge \rho_1 \ge \cdots \ge \rho_n =
 \rho'$ and $\tr{}{\rho'} \ge \eta$ hold, 
 $\tr{}{\rho_i} \ge \eta$ holds for all $i$. Therefore,
 $\con{Q}{\sigma} \xrightarrow{\tau\ast}^n
 \con{Q'}{\sigma'}$ and $\con{P'}{\rho'} \sim_{\zeta, \eta} \con{Q'}{\sigma'}$ hold
 for some $\con{Q'}{\sigma'}$.
\end{proof}

\begin{lem}
\label{par:weaksimulated}
 If $\con{P}{\rho} \sim_{\zeta, \eta} \con{Q}{\sigma}$
 and $\con{P}{\rho} \weak{\hat \alpha} \con{P'}{\rho'}$ and
 $\tr{}{\rho'} \ge \eta$ hold, then 
 $\con{Q}{\sigma} \weak{\hat \alpha} \con{Q'}{\sigma'}$ and
 $\con{P'}{\rho'} \sim_{\zeta, \eta} \con{Q'}{\sigma'}$ hold for some
 $\con{Q'}{\sigma'}$.
\end{lem}
\begin{proof}
 It is proven similarly to the previous lemma.
\end{proof}

The relation $\sim_{\zeta, \eta}$ is closed under application
of an arbitrary CP map by the outsider,
namely, one acts on $\sfqv - \tilde q$.
This is one of the 
similar properties as the original qCCS's largest 
bisimulation relation $\approx$ has \cite{DengFeng2012}.
\begin{lem}
\label{par:cpclosed}
 If $\con{P}{\rho} \sim_{\zeta, \eta} \con{Q}{\sigma}$,
 then $\con{P}{\E_{\tilde r}(\rho)} \sim_{\zeta, \eta}
 \con{Q}{\E_{\tilde r}(\sigma)}$ holds for all 
 CP map $\E$ acting on $\tilde r \subseteq \sfqv - \qv{P}$.
\end{lem}
\begin{proof}
 We use $(\Leftarrow)$ implication of Lemma
 \ref{prob_bisim:parcoinduction}.
 By Lemma \ref{par:positivenonincrease},
 $d(\E_{\tilde r}(\rho), \E_{\tilde
 r}(\sigma)) \leq d(\rho, \sigma)$ holds for all
 CP map $\E_{\tilde r}$ and thus the static condition on partial trace holds.
 Since $\E_{\hat r}$ ranges over arbitrary CP map in the definition,
 the simulation condition holds.
\end{proof}

The relation $\sim_{\zeta, \eta}$ is reflexive and symmetric but
not transitive because of the condition of trace distance. Instead,
it has the following properties.
\begin{prop}
\label{par:transitivitylike}
 \begin{enumerate}
  \item If $\con{P}{\rho} \sim_{\zeta, \eta} \con{Q}{\sigma}$,
	$\zeta \le \zeta'$, and $\eta \le \eta'$ hold, 
	then $\con{P}{\rho} \sim_{\zeta', \eta'} \con{Q}{\sigma}$ holds.
  \item If $\con{P}{\rho} \sim_{\zeta, \eta} \con{Q}{\sigma}$ and
	$\con{Q}{\sigma} \sim_{\zeta', \eta'} \con{R}{\theta}$ hold,
	then
	\[
	 \con{P}{\rho} \sim_{\zeta + \zeta',\,\mathrm{max}\{\eta,
	\eta'\} + 2(\zeta + \zeta')} \con{R}{\theta}
	\] 
	holds.
 \end{enumerate}
\end{prop}
\begin{proof}
 \begin{enumerate}
  \item It is sufficient to prove that $\sim_{\zeta, \eta}$ is 
	a $(\zeta', \eta')$-bisimulation. 
	The relation $\sim_{\zeta, \eta}$ is symmetric by Lemma \ref{par:alsobisimulation}.
	The static conditions is checked observing
	$d(\tr{\tilde q}{\rho}, \tr{\tilde q}{\sigma}) \le
	\zeta \le \zeta'$. For the simulation condition,
	assume 
	$\con{P}{\E_{\tilde r}(\rho)} \xrightarrow{\alpha}
	\con{P'}{\rho'}$ and $\tr{}{\rho'} \ge \eta'$
	for a CP map $\E_{\tilde r}$. Since $\tr{}{\rho'} \ge \eta' \ge
	\eta$, there exists a configuration $\con{Q'}{\sigma'}$
	satisfying $\con{Q}{\E_{\tilde r}(\sigma)} \weak{\hat \alpha} 
	\con{Q'}{\sigma'}$ and $\con{P'}{\rho'} \sim_{\zeta, \eta}
	\con{Q'}{\sigma'}$.
  \item We define a relation $\R \subseteq \C \times \C$ as follows.
	\begin{align*}
	 \R' :=& \{(\con{P}{\rho}, \con{R}{\theta})\,|\,
	            \con{P}{\rho} \sim_{\zeta, \eta} \circ
	 \sim_{\zeta', \eta'} \con{R}{\theta}\}\\
	 \R := & \R' \cup \R'^{-1}
	\end{align*}
	The assumption implies $\con{P}{\rho} \R \con{R}{\theta}$.
	It is sufficient to prove that $\R$ is a $(\max\{\eta,
	\eta'\}+2(\zeta + \zeta'))$-bisimulation. By definition,
	$\R$ is symmetric. \\
	{\bf (Case 1)} Let an arbitrary element
	$(\con{P}{\rho},\con{R}{\theta}) \in \R'$.
	There exists $\con{Q}{\sigma}$ satisfying
	$\con{P}{\rho} \sim_{\zeta, \eta} \con{Q}{\sigma}$ and
	$\con{Q}{\sigma} \sim_{\zeta', \eta'} \con{R}{\theta}$.
	The static condition is checked observing
	$d(\tr{\tilde q}{\rho}, \tr{\tilde q}{\theta}) \le
	d(\tr{\tilde q}{\rho}, \tr{\tilde q}{\sigma}) +
	d(\tr{\tilde q}{\sigma}, \tr{\tilde q}{\theta}) \le
	\zeta + \zeta'$. For the simulation condition, assume
	$\con{P}{\E_{\tilde r}(\rho)} \xrightarrow{\alpha}
	\con{P'}{\rho'}$ and $\tr{}{\rho'} \ge \max\{\eta,
	\eta'\}+2(\zeta + \zeta')$ for a CP map $\E_{\tilde r}$.
	Since $\tr{}{\rho'} \ge \eta$ and
 	$\con{P}{\rho} \sim_{\zeta, \eta} \con{Q}{\sigma}$
	hold, there exists
	$\con{Q'}{\sigma'}$ satisfying
	$\con{Q}{\E_{\tilde r}(\sigma)} \weak{\hat \alpha}
	\con{Q'}{\sigma'}$ and
	$\con{P'}{\rho'} \sim_{\zeta, \eta} \con{Q'}{\sigma'}$.
	Applying Lemma \ref{par:cpclosed} to
	$\con{Q}{\sigma} \sim_{\zeta', \eta'} \con{R}{\theta}$, we
	have
	$\con{Q}{\E_{\tilde r}(\sigma)} \sim_{\zeta', \eta'}
	\con{R}{\E_{\tilde r}(\theta)}$. We also have
	$\tr{}{\sigma'} \ge \tr{}{\rho'} - 2\zeta \ge \eta'$.
	By Lemma \ref{par:weaksimulated}, there exists
	$\con{R'}{\theta'}$
	satisfying $\con{R}{\E_{\tilde r}(\theta)} \weak{\hat \alpha}
	\con{R'}{\theta'}$ and $\con{Q'}{\sigma'} \sim_{\zeta', \eta'}
	\con{R'}{\theta'}$. We also have $\con{P'}{\rho'} \R' \con{R'}{\theta'}$.
	\\
	{\bf (Case 2)} Let an arbitrary element
	$(\con{R}{\theta}, \con{P}{\rho}) \in \R'^{-1}$.
	The proof is similar to the previous case.
 \end{enumerate}
\end{proof}

Examples of the relation are as follows.
\begin{ex}
\label{par:ex1}
  \begin{align*}
   (1)~~&\con{\measure{b}{\sndq{c}{b}.\discard{}}}
   {\ket{+}\bra{+}_b \otimes \ket{+}\bra{+}_q}\\
   \sim_{\frac{1}{2}, \frac{1}{2}}
   &\con{\measure{b}{\sndq{c}{b}.\discard{}}}
   {(\frac{1}{2}\ket{00}\bra{00} +
   \frac{1}{2}\ket{11}\bra{11})_{b,q}} \mbox{ holds.}\\
   (2)~~&\con{\measure{b}{\sndq{c}{q}.\discard{}}}
   {\ket{\psi}\bra{\psi}_b \otimes \ket{0}\bra{0}_q}\\
   \sim_{0, \frac{1}{4}}
   &\con{\measure{b}{\sndq{c}{q}.\discard{}}}
    {\ket{\psi}\bra{\psi}_b \otimes \ket{1}\bra{1}_q} \mbox{ holds,
   where}\\
   & \ket{\psi} = \frac{\sqrt{3}\ket{0} + \ket{1}}{2}
  \end{align*}
\end{ex}
We prove that the relation $\sim_{\zeta, \eta}$ is
closed under application of evaluation contexts.
We first show that $\sim_{\zeta, \eta}$ is
closed under restriction.
\begin{lem}
\label{par:resclosed}
 If $\con{P}{\rho} \sim_{\zeta, \eta} \con{Q}{\sigma}$ holds, then
 $\con{P \backslash L}{\rho} \sim_{\zeta, \eta}
 \con{Q \backslash L}{\sigma}$ holds.
\end{lem}
\begin{proof}
\label{par:resclosed}
 Let $\R := \{(\con{P \backslash L}{\rho}, \con{Q \backslash
 L}{\sigma}) \,|\, \con{P}{\rho} \sim_{\zeta, \eta} \con{Q}{\sigma}\}$.
 It is
 sufficient to show that $\R$ is an approximate bisimulation relation.
 $\R$ is symmetric by the definition.
 Let $(\con{P \backslash L}{\rho}, \con{Q \backslash
 L}{\sigma})$ be an arbitrary element of $\R$. The
 static conditions are easily checked. Assume $\con{P \backslash
 L}{\E_{\tilde r}(\rho)} \xrightarrow{\alpha} \con{P' \backslash
 L}{\rho'}$ and $\tr{}{\rho'} \ge \eta$.
 This implies $\con{P}{\E_{\tilde r}(\rho)} \xrightarrow{\alpha}
 \con{P'}{\rho'}$ and $\mathrm{cn}(\alpha) \cap L = \emptyset$.
 We have
 $\con{P}{\E_{\tilde r}(\rho)} \sim \con{Q}{\E_{\tilde r}(\sigma)}$
 from $\con{P}{\rho} \sim_{\zeta, \eta} \con{Q}{\sigma}$. We then have 
 $\con{Q}{\E_{\tilde r}(\sigma)} \weak{\hat \alpha} \con{Q'}{\sigma'}$ 
 and $\con{P'}{\rho'} \sim_{\zeta, \eta} \con{Q'}{\sigma'}$ for some
 $\con{Q'}{\sigma'}$. As $\mathrm{cn}(\hat \alpha) \cap L = \emptyset$
 and $\mathrm{cn}(\tau) = \emptyset$, we have 
 $\con{Q \backslash L}{\E_{\tilde r}(\sigma)} \weak{\hat \alpha}
 \con{Q' \backslash L}{\sigma'}$ and 
 $\con{P' \backslash L}{\rho'} \R \con{Q' \backslash L}{\sigma'}$. 
\end{proof}

The relation $\sim_{\zeta, \eta}$ is
closed under parallel composition of the processes.

\begin{thm}
\label{par:pallaclosed}
 If $\con{P}{\rho} \sim_{\zeta, \eta} \con{Q}{\sigma}$ holds, 
 $\con{P||R}{\rho} \sim_{\zeta, \eta} \con{Q||R}{\sigma}$ holds for
 all process $R$.
\end{thm}
\begin{proof}
\label{par:prfof-pallaclosed}
 We define 
\[
 \hat \R := \{(\con{P||R}{\rho}, \con{Q||R}{\sigma})\,|\,
 \con{P}{\rho} \sim_{\zeta, \eta} \con{Q}{\sigma}, R \in \mathcal{P}
 \}. 
\]
 As $\sim_{\zeta, \eta}$ is symmetric, $\hat \R$ is symmetric.
 It is sufficient to show $\hat \R$ is a 
$(\zeta, \eta)$-bisimulation.
 Let $(\con{P||R}{\rho}, \con{Q||R}{\sigma})$ be an arbitrary element
 in $\hat \R$.
 The static conditions are checked as follows. By the definition
 of $\qv{\cdot}$ and the condition $\qv{P} = \qv{Q}$
 obtained from $\con{P}{\rho} \sim_{\zeta, \eta} \con{Q}{\sigma}$,
 $\tilde q \defeq 
 \qv{P||R}=\qv{P}\cup\qv{R}=\qv{Q}\cup\qv{R}=\qv{Q||R}$ holds.
 Next, 
 \begin{align*}
 d(\tr{\tilde q}{\rho},
 \tr{\tilde q}{\sigma})
 = \, &
 d(\tr{\qv{R}}{\tr{\qv{P}}{\rho}}, \tr{\qv{R}}{\tr{\qv{Q}}{\sigma}})\\
 \le \, & d(\tr{\qv{P}}{\rho}, \tr{\qv{Q}}{\sigma}) \le \zeta.
 \end{align*}
 holds.
 We then show that the simulation condition is satisfied.
 Let $\E_{\tilde r}$ be an arbitrary TPCP map acting on $\tilde r
 \subseteq \sfqv - \tilde q$.
 A transition of a parallely-composed process is either the 
 3 cases by the transition rules.\\
 {\bf (Case 1)}The transition is performed only by
       $P$.
       Assume $\con{P}{\E_{\tilde r}(\rho)} \xrightarrow{\alpha}
       \con{P'}{\rho'}$ and $\tr{}{\rho'} \ge \eta$ hold.
       By $\con{P}{\rho} \sim_{\zeta, \eta} \con{Q}{\sigma}$,
       there exists $\con{Q'}{\sigma'}$ satisfying
       $\con{Q}{\E_{\tilde r}(\sigma)} 
       \weak{\hat \alpha} \con{Q'}{\sigma'}$ and
       $\con{P'}{\rho'}\sim_{\zeta, \eta}\con{Q'}{\sigma'}$.
       Therefore, $\con{Q||R}{\sigma}
       \weak{\hat \alpha} \con{Q'||R}{\sigma'}$ holds by (Left) rule
       and
       $\con{P'||R}{\rho'} \hat \R \con{Q'||R}{\sigma'}$ holds
       by the definition of $\hat \R$.\\
 {\bf (Case 2)} The transition is performed only by
       $\con{R}{\rho}$.
       Assume $\con{R}{\E_{\tilde r}(\rho)} \xrightarrow{\alpha}
       \con{R'}{\rho'}$ and $\tr{}{\rho'} \ge \eta$ hold.
       Because $R$ has a redex that
       causes the transition $\xrightarrow{\alpha}$,
       $\con{Q||R}{\E_{\tilde r}(\sigma)}
       \xrightarrow{\alpha} \con{Q||R'}{\sigma'}$ holds for some
       $\sigma'$. It is sufficient to show
       $\con{P}{\rho'} \sim_{\zeta, \eta} \con{Q}{\sigma'}$ by the
       definition of $\hat \R$. 
       $\rho' = \F_{\tilde s} \circ \E_{\tilde r}(\rho)$ and
       $\sigma' = \F_{\tilde s} \circ \E_{\tilde r}(\sigma)$ hold
       for some CP map $\F_{\tilde s}$ acting on $\tilde s \subseteq
       \qv{R}$. As $\con{P}{\rho} \sim_{\zeta, \eta}
       \con{Q}{\sigma}$ and $\tilde s, \tilde r \subseteq \sfqv - 
       \qv{P}$ hold, $\con{P}{\rho'} \sim_{\zeta, \eta}
       \con{Q}{\sigma'}$ holds by Lemma \ref{par:cpclosed}.
       This implies $\con{P||R'}{\rho'} \hat \R
       \con{Q||R'}{\sigma'}$.\\
 {\bf (Case 3)} The transition is performed by 
       communication of $P$ and $R$. As the communication rule is 
       applied, the $P||R$ can be written as
       $\con{C_1[\sndq{c}{q}.P']||C_2[\rcvq{c}{r}.R']}
       {\E_{\tilde r}(\rho)}$
       for some evaluation contexts $C_1[\_], C_2[\_]$, processes
       $P', R'$, and
       non-restricted channel $\mathsf{c}$.
       The transition to consider is
       \[
       \con{C_1[\sndq{c}{q}.P']||C_2[\rcvq{c}{r}.R']}
       {\E_{\tilde r}(\rho)}
       \xrightarrow{\tau}
       \con{C_1[P']||C_2[R']}{\E_{\tilde r}(\rho)}.
       \]
       Assume $\tr{}{\rho} \ge \eta$.
       By $\con{C_1[\sndq{c}{q}.P']}{\rho}
       \sim_{\zeta, \eta} 
       \con{Q}{\sigma}$
       and
       $\con{C_1[\sndq{c}{q}.P']}{\E_{\tilde r}(\rho)}
       \xrightarrow{\sndq{c}{q}}
       \con{C_1[P']}{\E_{\tilde r}(\rho)}$,
       $\con{Q}{\E_{\tilde r}(\sigma)} \weak{\sndq{c}{q}}
       \con{Q'}{\sigma'}$ and
       $\con{C_1[P']}{\E_{\tilde r}(\rho)} \sim_{\zeta, \eta}
       \con{Q'}{\sigma'}$ hold for some $\con{Q'}{\sigma'}$.
       This implies $\mathsf{c}$ is not restricted in $Q$
       and thus receive redex
       $\rcvq{c}{r}$ in $C_2[\rcvq{c}{r}.R']$
       can be react.
       The reaction does not influence the quantum state $\sigma'$. 
       Therefore,
       \[
         \con{Q||R}{\E_{\tilde r}(\sigma)} \xrightarrow{\tau \ast}
         \con{Q'||C_2[R']}{\sigma'} \mbox{ and }
         \con{C_1[P']||C_2[R']}{\E_{\tilde r}(\rho)} 
         \hat \R
         \con{Q'||C_2[R']}{\sigma'}
       \]
       hold by the definition of $\hat \R$.       
\end{proof}
By the two previous lemmas and Proposition \ref{par:sym-by-par},
we have the following corollary.
\begin{col}
\label{par:congruence}
 If $\con{P}{\rho} \sim_{\zeta, \eta} \con{Q}{\sigma}$ holds, 
 $\con{C[P]}{\rho} \sim_{\zeta, \eta} \con{C[Q]}{\sigma}$ holds for
 all non-aborting evaluation context $C[\_]$.
\end{col}
